\subsection{Avec des arguments élémentaires} Commençons par une preuve explicative qui malheureusement ne nous permet pas de voir d'où vient l'astuce \emph{(nous explorerons ceci dans la sous-section suivante)}. 

\showstepnovfill{Calculs faits dans les deux phases.}{tikz/why/twophases-focus}


\bigskip


Par construction, nous avons $a = qb + r$ et $X = qY + Z$. Ceci nous donne :

\vspace{-1em}

\begin{flalign*}
	d &= aY - bX               & \\
	  &= (qb + r)Y - b(qY + Z) & \\
	  &= rY - bZ               & \\
	  &= -e                    & \\
\end{flalign*}

\vspace{-1em}


Donc si l'on fait \myquote{glisser} des carrés sur les deux colonnes de gauche, les produits en croix dans ces carrés ne différeront que de leur signe. 


\medskip


Grâce à la représentation finale ci-dessous, nous obtenons $aY - bX = \pm \pgcd(a ; b)$ car le dernier reste non nul de l'algorithme d'Euclide est $\pgcd(a ; b)$. Ceci prouve la validité de la méthode dans le cas général. On comprend au passage l'ajout initial du $0$ et du $1$ dans la 3\ieme{} colonne \emph{(bien entendu, $(-1)$ aurait aussi pu convenir)}.

\showstepnovfill{Une représentation symbolique au complet.}{tikz/why/twophases-all}
	

% --------------- %


\subsection{Avec des matrices pour aller plus loin}

Reprenons le cas de base suivant mais an l'analysant à l'aune des matrices.

\showstepnovfill{Calculs faits dans les deux phases.}{tikz/why/twophases-focus}


\medskip


Notant
$M 
 =
 \begin{pmatrix}
	a & X \\ 
	b & Y
 \end{pmatrix}$,
et
$N 
 =
 \begin{pmatrix}
	b & Y \\ 
	r & Z
 \end{pmatrix}$,
nous avons $d = \det M$ et $e = \det N$.


\medskip


Comme 