\documentclass[12pt]{article}

\usepackage{tnsmath}

\setlength\parindent{0pt}

\begin{document}

Démontrons rapidement que $n \divides \big( 2^n + 1 \big)$ dès que $n = 3^k$ par récurrence sur $k \in \NNs$ car le cas $k = 0$ est trivial.

\bigskip

\textbf{Initialisation pour $k = 1$.}

\smallskip

Clairement, $3 \divides \big( 2^3 + 1 \big)$ .

\bigskip

\textbf{Étape de récurrence.}

\smallskip

On a les implications logiques suivantes.

\medskip

\begin{stepcalc}[style=ar*, ope=\implies]
	\big( 3^k \big) \divides \Big( 2^{\big( 3^k \big)} + 1 \Big)
\explnext{}
	\exists m \in \ZZ \,.\, \Big[ 2^{\big( 3^k \big)} + 1 = m \cdot 3^k  \Big]
\explnext{}
	\exists m \in \ZZ \,.\, \Big[ 2^{\big( 3^k \big)} = - 1 + m \cdot 3^k  \Big]
\explnext{}
	\exists m \in \ZZ \,.\, \Big[ \left( 2^{\big( 3^k \big)} \right)^3 = \big( - 1 + m \cdot 3^k \big)^3  \Big]
\explnext{}
	\exists m \in \ZZ \,.\, \Big[ 2^{\big( 3^{k+1} \big)} = - 1 + 3 \cdot m \cdot 3^k - 3 \cdot \big( m \cdot 3^k \big)^2 + \big( m \cdot 3^k \big)^3  \Big]
\explnext*{Besoin de \\ $k \neq 0$ ici.}{}
	2^{\big( 3^{k+1} \big)} \equiv - 1 \mod\!\big( 3^{k+1} \big)
\end{stepcalc}

\smallskip

En résumé, 
$\big( 3^k \big) \divides \Big( 2^{\big( 3^k \big)} + 1 \Big)$ 
implique
$\big( 3^{k+1} \big) \divides \Big( 2^{\big( 3^{k+1} \big)} + 1 \Big)$  .

\bigskip

\textbf{Conclusion :} \dots

\bigskip

\textbf{Seul truc intéressant dans cette affaire de niveau TS de l'ancien temps :} calcul d'une limite de suite dans l'anneau des entiers $3$-adique.


\bigskip

La preuve passe en fait à l'échelle pour démontrer que si 
$d \divides \big( 2^d + 1 \big)$
alors
$d^k \divides \big( 2^{(d^k)} + 1 \big)$ pour $k \in \NNs$ ,
et aussi que si
$a \divides \big( 2^a + 1 \big)$ et $b \divides \big( 2^b + 1 \big)$
alors
$a^k \cdot b \divides \big( 2^{(a^k \cdot b)} + 1 \big)$ pour $k \in \NNs$ .


\bigskip

Une recherche brutale de solutions du type $p_1 \cdot p_2 \cdot p_3$, avec $p_i$ premier, nous donne les solutions suivantes.

\end{document}
