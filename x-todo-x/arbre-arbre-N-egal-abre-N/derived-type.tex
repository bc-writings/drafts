\begin{definition}
	Le type $arbre(arbre(\alpha))$ est défini comme suit où $\alpha$ est un type \emph{(nous appliquons juste au type $arbre(\alpha)$ la définition \ref{basic-type})}.

	\medskip

	\begin{center}
		\begin{prooftree}
    		\hypo{a : arbre(\alpha)}
    		\infer1[\footnotesize\itshape (feuille)]{%
    			Leaf(a) : arbre(arbre(\alpha))%
			}
		\end{prooftree}
		\quad\quad\quad
		\begin{prooftree}
    		\hypo{t_1 : arbre(arbre(\alpha))}
    		\hypo{t_2 : arbre(arbre(\alpha))}
    		\infer2[\footnotesize\itshape (noeud)]{%
    			Node(t_1, t_2) : arbre(arbre(\alpha))%
			}
		\end{prooftree}
	\end{center}
\end{definition}


\begin{example}\label{exa-tree-tree}
	Dans cet exemple, nous noterons \verb+Leaf''+ et \verb+Node''+ les applications feuille et noeud pour le type $arbre(arbre(\NN))$.
	Le plus \emph{\og simple \fg} des arbres de type $arbre(arbre(\NN))$ est le suivant avec $k \in \NN$ quelconque.
	\begin{pseudocode}
Leaf''(Leaf(k))
	\end{pseudocode}
	
	\medskip
	
	On peut alors fabriquer un arbre similaire, mais pas identique, au 1\ier{} arbre de l'exemple \ref{exa-tree}.
	\begin{pseudocode}
Node''(
	Node''(
		Leaf''(Leaf(1)),
		Node''(
			Leaf''(Leaf(2)),
			Leaf''(Leaf(3))
		)
	),
	Leaf''(Leaf(4))
)
	\end{pseudocode}
\end{example}


