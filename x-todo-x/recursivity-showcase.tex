\documentclass[12pt]{amsart}
\usepackage[T1]{fontenc}
\usepackage[utf8]{inputenc}

\usepackage[top=1.95cm, bottom=1.95cm, left=2.35cm, right=2.35cm]{geometry}

\usepackage{hyperref}
\usepackage{enumitem}
\usepackage{tcolorbox}
\usepackage{multicol}
\usepackage{fancyvrb}
\usepackage{amsmath}
\usepackage[french]{babel}
\usepackage[
    type={CC},
    modifier={by-nc-sa},
	version={4.0},
]{doclicense}
\usepackage{textcomp}
\usepackage{tcolorbox}
\usepackage{lymath}

    
\newtheorem{fact}{Fait}%[section]
\newtheorem*{theorem}{Théorème}
\newtheorem{example}{Exemple}[section]
\newtheorem{remark}{Remarque}[section]
\newtheorem*{proof*}{Preuve}

\setlength\parindent{0pt}


\DeclareMathOperator{\taille}{\text{\normalfont\texttt{taille}}}

\newcommand\sqseq[2]{\fbox{$#1$}_{\,\,#2}}

\newcommand\floor[1]{\left\lfloor #1 \right\rfloorx}



\DefineVerbatimEnvironment{rawcode}%
	{Verbatim}%
	{tabsize=4,%
	 frame=lines, framerule=0.3mm, framesep=2.5mm}
	 

\newcommand\word[1]{\emph{\og #1 \fg}}
	 
\begin{document}

\title{BROUILLON - EN VRAC - La récursivité, quésaco ?}
\author{Christophe BAL}
\date{20 Mars 2019}

\maketitle

\begin{center}
	\itshape
	Document, avec son source \LaTeX, disponible sur la page
	
	\url{https://github.com/bc-writing/drafts}.
\end{center}


\bigskip


\begin{center}
	\hrule\vspace{.3em}
	{
		\fontsize{1.35em}{1em}\selectfont
		\textbf{Mentions \og légales \fg}
	}
			
	\vspace{0.45em}
	\doclicenseThis
	\hrule
\end{center}

	
\setcounter{tocdepth}{2}
\tableofcontents



\section{Étymologie}

Le mot \word{récursif} vient du mot anglais \word{recursive}, introduit dans les années 1960. Quelque chose de récursif se définit en s'utilisant soi-même, directement ou indirectement
\footnote{
	Source : \url{https://fr.wiktionary.org} , le 20 mars 2019.
}.
Le mot vient de \word{recursus} qui désigne \word{retour en courant}
\footnote{
	Source : \url{https://en.wiktionary.org} , le 20 mars 2019.
}.
Gardons bien ceci en tête tout au long de ce modeste document.





\section{Une description courte d'un damier}


\section{Les quintés gagnants dans le désordre ou non}


\section{Les tours d'Hanoï}







%
%\section{Suites récursives}
%
%Commençons par une situation mathématique très simple.
%
%
%
%
%
%
%
%
%
%
%
%\section{Exponentiation rapide}
%
%Commençons par une situation mathématique très simple.
%
%
%
%\section{Les plantes de Lindenmayer}
%
%Commençons par une situation mathématique très simple.
%
%
%
%\section{Systèmes itératifs fractales}
%
%Commençons par une situation mathématique très simple.
%
%
%
%\section{Les structures informatiques récursives}
%
%Commençons par une situation mathématique très simple.
%
\end{document}
