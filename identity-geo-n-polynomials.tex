\documentclass[12pt]{amsart}
\usepackage[T1]{fontenc}
\usepackage[utf8]{inputenc}

\usepackage[top=1.95cm, bottom=1.95cm, left=2.35cm, right=2.35cm]{geometry}


\usepackage{wrapfig} 

\usepackage{hyperref}
\usepackage{enumitem}
\usepackage{tcolorbox}
\usepackage{float}
\usepackage{cleveref}
\usepackage{multicol}
\usepackage{fancyvrb}
\usepackage{enumitem}
\usepackage{amsmath}
\usepackage{textcomp}
\usepackage{numprint}
\usepackage[french]{babel}
\usepackage[
    type={CC},
    modifier={by-nc-sa},
	version={4.0},
]{doclicense}

\newcommand\floor[1]{\left\lfloor #1 \right\rfloor}

\usepackage{lymath}


\newtheorem{fact}{Fait}
\newtheorem*{fact*}{Fait}

\newtheorem{example}{Exemple}

\newtheorem{remark}{Remarque}
\newtheorem*{remark*}{Remarque}

\newtheorem*{proof*}{Preuve}

\setlength\parindent{0pt}

\floatstyle{boxed} 
\restylefloat{figure}


\DeclareMathOperator{\taille}{\text{\normalfont\texttt{taille}}}

\newcommand\sqseq[2]{\fbox{$#1$}_{\,\,#2}}


\DefineVerbatimEnvironment{rawcode}%
	{Verbatim}%
	{tabsize=4,%
	 frame=lines, framerule=0.3mm, framesep=2.5mm}
	 
	 
	 
\begin{document}

\title{BROUILLON - Identités remarquables via des calculs d'aires, est-ce rigoureux ?}
\author{Christophe BAL}
\date{16 Juillet 2019}

\maketitle

\begin{center}
	\itshape
	Document, avec son source \LaTeX, disponible sur la page
	
	\url{https://github.com/bc-writing/drafts}.
\end{center}


\bigskip


\begin{center}
	\hrule\vspace{.3em}
	{
		\fontsize{1.35em}{1em}\selectfont
		\textbf{Mentions \og légales \fg}
	}
			
	\vspace{0.45em}
	\doclicenseThis
	\hrule
\end{center}


%\setcounter{tocdepth}{1}
%\tableofcontents

\vspace{1.75em}


\begin{wrapfigure}{r}{0.45\textwidth} 
	\vspace{-.5em}
	\begin{center}
		\fbox{\includegraphics[scale = .7]{identity-geo-n-polynomials/(a+b)^2.png}}
	\end{center}
	\vspace{-2.25em}
\end{wrapfigure} 

Les identités remarquables peuvent se découvrir facilement via de simples calculs d'aires.


\medskip

Par exemple, considérons le dessin ci-contre où $ABCD$ , $AEGF$ et $GHCK$ sont des carrés de côtés respectifs $(a + b)$ , $a$ et $b$ .
Il est évident que nous avons alors $(a + b)^2 = a^2 + b^2 + 2 ab$ mais n'oublions que $a > 0$ et $b > 0$ \emph{(ce sont des contraintes géométriques concrètes)}.


\medskip

Comment passer à $(a + b)^2 = a^2 + b^2 + 2 ab$ pour $a$ et $b$ deux réels de signes quelconques ?
Une première idée est tout simplement de passer par une vérification via un calcul algébrique. En résumé, on conjecture géométriquement puis on valide algébriquement.


\medskip

\begin{wrapfigure}{l}{0.45\textwidth} 
	\vspace{-.5em}
	\begin{center}
		\fbox{\includegraphics[scale = .7]{identity-geo-n-polynomials/(a+b)^3.png}}
	\end{center}
	\vspace{-1.25em}
\end{wrapfigure} 

Bien que rigoureuse, la démarche précédente est peu satisfaisante car elle balaye d'un revers de main l'approche géométrique dont le rôle est réduit à la découverte d'une formule.


\medskip

Si l'on considère le dessin ci-contre, il est tout de même dommage de devoir passer par du calcul algébrique, un peu pénible ici, pour avoir l'identité $(a + b + c)^2 = a^2 + b^2 + c^2 + 2 ab + 2 ac + 2 bc$ avec $a$ , $b$ et $c$ des réels de signes quelconques.


\medskip

Ce qui serait bien ce serait de pouvoir dire que puisque $(a + b + c)^2 = a^2 + b^2 + c^2 + 2 ab + 2 ac + 2 bc$ est vraie si $a > 0$ , $b > 0$ et $c > 0$ , alors l'identité est automatiquement vérifiée par $a$ , $b$ et $c$ des réels de signes quelconques.


% ----------- %


\medskip

Ce passage automatique est bien licite car nous avons le fait suivant que l'on peut appliquer à $P(a ; b) = (a + b)^2 - a^2 - b^2 - 2 ab$ et $P(a ; b : c) = (a + b + c)^2 - a^2 - b^2 - c^2 - 2 ab - 2 ac - 2 bc$ .

\begin{fact*}
	Soit $P \in \polyset{\RR}{X_1 | ... | X_n}$ un polynôme réel à $n$ variables où $n \in \NNs$ .
	Si $P$ s'annule sur $\left( \RRsp \right)^n$ alors $P$ s'annule sur $\RR^n$ tout entier. 
\end{fact*}


\begin{proof}
	Faisons une preuve par récurrence sur $n \in \NNs$ pour démontrer la validité de la propriété $\probaset{P}(n)$ définie comme suit :
	\emph{\og 
		Pour tout polynôme réel $P \in \polyset{\RR}{X_1 | ... | X_n}$ à $n$ variables,
		si $P$ s'annule sur $\left( \RRsp \right)^n$ alors $P$ s'annule sur $\RR^n$ tout entier. 
	\fg}.

	\begin{itemize}[label=\small\textbullet]
		\item \emph{Cas de base.}
	
		\noindent
		$\probaset{P}(1)$ signifie qu'un polynôme réel à une variable s'annulant sur $\RRsp$ est identiquement nul sur $\RR$ tout entier.
		
		\smallskip
		\noindent
		Comme un polynôme réel non nul n'a qu'un nombre fini de racines, nous avons la validité de $\probaset{P}(1)$ .


		\medskip
		\item \emph{Hérédité.}
	
		\noindent
		Supposons $\probaset{P}(n)$ valide pour un naturel $n$ fixé, mais quelconque, puis considérons un polynôme $P$ de $(n + 1)$ variables qui vérifie les conditions de la propriété $\probaset{P}(n + 1)$ .
	
		\smallskip
		\noindent
		Fixons $x \in \RRsp$ et considérons le polynôme $P_x(X_1 ; ... ; X_n) = P(X_1 ; ... ; X_n ; x)$ .
		Comme $P_x$  vérifie les conditions de la propriété $\probaset{P}(n)$ ,
		nous avons par hypothèse de récurrence 
		$P_x(x_1 ; ... ; x_n) = 0$ soit $P(x_1 ; ... ; x_n ; x) = 0$ pour tous réels $x_1$ , ... , $x_n$ .
	
		\smallskip
		\noindent
		Fixons maintenant des réels $x_1$ , ... , $x_n$ de signes quelconques et considérons le polynôme $p(X) = P(x_1 ; ... ; x_n ; X)$ .
		Comme $p$ vérifie $\probaset{P}(1)$ , nous avons $p(x) = 0$ soit $P(x_1 ; ... ; x_n ; x) = 0$ pour tout réel $x$ .
	
		\smallskip
		\noindent
		Finalement $P(x_1 ; ... ; x_n ; x) = 0$ pour tous réels $x_1$ , ... , $x_n$ , $x$ .
		Nous avons bien déduit la validité de $\probaset{P}(n+1)$ à partir de celle de $\probaset{P}(n)$ .


	\medskip
	\item \emph{Hérédité.}
	
	\smallskip
	\noindent
	Par récurrence sur $n \in \NNs$ , la propriété $\probaset{P}(n)$ est vraie pour tout naturel non nul $n$ .
	\end{itemize} 

\end{proof}


% ----------- %


\begin{remark*}
	En utilisant une approche géométrique, il devient évident, mais aussi rigoureux maintenant, d'affirmer que pour tous réels $a_1$ , ... , $a_n$ , où $n \in \NNs$ , l'on a :
\[
	\left( \sum_{k=1}^{n}a_k \right)^2
	=
	\sum_{k=1}^{n} \left( a_k \right)^2
	+
	2 \sum_{1 \leq i < j \leq n} a_i a_j
\]
\end{remark*}

\end{document}
