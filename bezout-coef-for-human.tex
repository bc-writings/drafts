\documentclass[12pt]{amsart}
\usepackage[T1]{fontenc}
\usepackage[utf8]{inputenc}

\usepackage[top=1.95cm, bottom=1.95cm, left=2.35cm, right=2.35cm]{geometry}


\usepackage{wrapfig} 

\usepackage{hyperref}
\usepackage{enumitem}
\usepackage{tcolorbox}
\usepackage{float}
\usepackage{cleveref}
\usepackage{multicol}
\usepackage{fancyvrb}
\usepackage{enumitem}
\usepackage{amsmath}
\usepackage{textcomp}
\usepackage{numprint}
\usepackage[french]{babel}
\usepackage[
    type={CC},
    modifier={by-nc-sa},
    version={4.0},
]{doclicense}

\newcommand\floor[1]{\left\lfloor #1 \right\rfloor}

\usepackage{lymath}


\newtheorem{fact}{Fait}
\newtheorem*{fact*}{Fait}

\newtheorem{example}{Exemple}

\newtheorem{remark}{Remarque}
\newtheorem*{remark*}{Remarque}

\newtheorem*{proof*}{Preuve}

\setlength\parindent{0pt}

\floatstyle{boxed} 
\restylefloat{figure}


\DeclareMathOperator{\taille}{\text{\normalfont\texttt{taille}}}

\newcommand\sqseq[2]{\fbox{$#1$}_{\,\,#2}}


\DefineVerbatimEnvironment{rawcode}%
    {Verbatim}%
    {tabsize=4,%
     frame=lines, framerule=0.3mm, framesep=2.5mm}
     
     
     
\begin{document}

\title{BROUILLON - CANDIDAT - Algorithme d'Euclide et coefficients de Bachet-Bézout pour les humains}
\author{Christophe BAL}
\date{6 Septembre 2019}

\maketitle

\begin{center}
    \itshape
    Document, avec son source \LaTeX, disponible sur la page
    
    \url{https://github.com/bc-writing/drafts}.
\end{center}


\bigskip


\begin{center}
    \hrule\vspace{.3em}
    {
        \fontsize{1.35em}{1em}\selectfont
        \textbf{Mentions \og légales \fg}
    }
            
    \vspace{0.45em}
    \doclicenseThis
    \hrule
\end{center}


\setcounter{tocdepth}{2}
\tableofcontents


\section{????}

????


\begin{tikzpicture}
    \matrix[
        row sep    = .5em,
        column sep = 1.5em,
    block/.style={
      rectangle,
      draw=blue,
      thick,
      fill=blue!20,
      text width=5em,
      align=center,
      rounded corners,
      minimum height=2em
    },
    ]{
                     & \node[block] {$9665$};      &                \\
        \node {$2$}; & \node[block] {$488888887$}; & \node {$3$};   \\
                     & \node {$777$};              & \node {$11$};  \\
                     &                             & \node {$99$};  \\
      };
\end{tikzpicture}


\section{????}

????


\begin{tikzpicture}
    \matrix[
        row sep    = 1em,
        column sep = 1.5em,
    block/.style={
      rectangle,
      draw=blue,
      thick,
      fill=blue!20,
      text width=5em,
      align=center,
      rounded corners,
      minimum height=2em
    },
    ]{
                            & \node {$9665$};      &                \\
        \node[block] {$2$}; & \node {$488888887$}; & \node[block] {$3$};   \\
                            & \node {$777$};       & \node[block] {$11$};  \\
                            &                      & \node {$99$};  \\
      };
\end{tikzpicture}

\end{document}
