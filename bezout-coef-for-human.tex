\documentclass[12pt]{amsart}
\usepackage[T1]{fontenc}
\usepackage[utf8]{inputenc}

\usepackage[top=1.95cm, bottom=1.95cm, left=2.35cm, right=2.35cm]{geometry}


\usepackage{hyperref}
\usepackage{enumitem}
\usepackage{tcolorbox}
\usepackage{cleveref}
\usepackage{multicol}
\usepackage{amsmath}
\usepackage{nicematrix}
\usepackage[french]{babel}
\usepackage[
    type={CC},
    modifier={by-nc-sa},
    version={4.0},
]{doclicense}

\usepackage{tikz}
\usetikzlibrary{arrows, matrix, positioning,fit}


\usepackage{lymath}
\usepackage{lyalgo}


\newtheorem{fact}{Fait}
\newtheorem*{fact*}{Fait}

\newtheorem{example}{Exemple}

\newtheorem{remark}{Remarque}
\newtheorem*{remark*}{Remarque}

\newtheorem*{proof*}{Preuve}

\setlength\parindent{0pt}


% -- NEW ENVIRONMENTS -- %




% -- NEW MACROS -- %

\newcommand\ph{\phantom{x}}

\newcommand\myquote[1]{\emph{\og #1 \fg}}

\newcommand\bb{Bachet-Bézout}

% Arguments
%    #1 Caption
%    #2 TiKZ file

\newcommand\showstep[2]{
    \begin{center}
        \input{bezout-coef-for-human/#2.tkz}
    
        \vfill
        
        \small \itshape #1
    \end{center}    
}


\newcommand\showstepnovfill[2]{
    \begin{center}
        \input{bezout-coef-for-human/#2.tkz}
  
        \small \itshape #1
    \end{center}    
}


% -- SETTINGS -- %

\tikzset{
    % Good spacing hack (the ghost mode)
     gs/.style={
        rectangle,
        thick,
        text width=3.5em,
        align=center,
        draw=white,
        rounded corners,
        minimum height=2em
    }, 
	% Common
    rc/.style={rectangle,
        thick,rounded corners,draw,
        minimum height=2em},
    % Operator in a circle
    oc/.style={
        circle,
        draw,
        fill=white,
        inner sep=1pt, 
        outer sep=1pt
    },
    % Blue frame
    bf/.style={
        rc,
        align=center,
        draw=blue,
        fill=blue!20,
        text width=3.5em,
    },
    % Explain remained
    er/.style={
        rc,
        align=center,
        text width=3.5em,
        draw=gray,
        text=darkgray
    },
    % Explain activated
    ea/.style={
        rc,
        align=center,
        text width=3.5em,
        draw=gray,
        text=darkgray
    },
    % Long explain activated
    le/.style={
        rc,
        align=center,
        text width=8em,
        draw=gray,
        text=darkgray
    },
    % Blur effect
    be/.style={
        rectangle,
        thick,rounded corners,
        minimum height=2em,
        align=center,
        opacity = 0.3,
        text width=3.5em,
    },
    % Focus effect
    fe/.style={
        rc,
        align=center,
        text width=8em,
        draw=red,
        text=magenta
    },
    % Focus effect Bis
    feb/.style={
        rc,
        align=center,
        text width=8em,
        draw=blue,
        text=violet
    },
    % Long focus effect
    lfe/.style={
        rc,
        align=center,
        text width=12em,
        draw=red,
        text=magenta
    },
    % Long focus effect bis
    lfeb/.style={
        rc,
        align=center,
        text width=12em,
        draw=blue,
        text=violet
    },
}
      

     
     
\begin{document}

\title{BROUILLON - Algorithme d'Euclide et coefficients de Bachet-Bézout pour les humains}
\author{Christophe BAL}
\date{10 Sept. 2019 - 13  Sept. 2019}

\maketitle

\begin{center}
    \itshape
    Document, avec son source \LaTeX, disponible sur la page
    
    \url{https://github.com/bc-writing/drafts}.
\end{center}


\bigskip


\begin{center}
    \hrule\vspace{.3em}
    {
        \fontsize{1.35em}{1em}\selectfont
        \textbf{Mentions \og légales \fg}
    }
            
    \vspace{0.45em}
    \doclicenseThis
    \hrule
\end{center}


\setcounter{tocdepth}{2}
\tableofcontents


% --------------- %


\section{Où allons-nous ?}

Le but de ce document est de tenter de dégager une méthodologie pour la mise en place d'algorithmes concurrents.
Comme l'algorithmique sans programmation est une rime sans poème, chaque algorithme sera implémenté en langage \java. 
Le choix retenu est de travailler avec le minimum d'outils, le but étant de voir comment tout fonctionne à un niveau intermédiaire de la machine sans passer par des bibliothèques qu'il serait plus utile dans la vraie vie d'un programmeur.


% --------------- %


\section{L'algorithme \og human friendly \fg{} appliqué de façon magique}

\subsection{Un exemple complet façon \myquote{diaporama}}

Sur le lieu de téléchargement de ce document se trluve un fichier \verb+PDF+ de chemin relatif \verb+bezout-coef-for-human/slide-version.pdf+ présentant la méthode sous la forme d'un diaporama. Nous vous conseillons de le regarder avant de lire les explications suivantes.


% --------------- %


\subsection{Phase 1 -- Au début était l'algorithme d'Euclide}

Pour chercher des coefficients de \bb{} pour $(a ; b) = (27 ; 141)$, on commence par appliquer l'algorithme d'Euclide \myquote{verticalement} comme suit.

\begin{multicols}{2}
	\showstep{1}{tikz/27-141[all]/down-1}{le plus grand naturel est mis au-dessus.}

	\showstep{2}{tikz/27-141[all]/down-2}{deux naturels à diviser.}
\end{multicols}

\begin{multicols}{2}
	\showstep{3}{tikz/27-141[all]/down-3}{première division euclidienne.}

	\showstep{4}{tikz/27-141[all]/down-4}{on passe aux deux naturels suivants.}
\end{multicols}


\medskip


En répétant ce processus, nous arrivons à la représentation suivante.

\showstepnovfill{finale (1\iere{} phase)}{tikz/27-141[main]/down}{l'algorithme d'Euclide \myquote{vertical}.}


% --------------- %


\subsection{Phase 2 -- Remonter facile des étapes}

La méthode classique consiste à remonter les calculs. Mais comment faire cette remontée tout en évitant un claquage neuronal ? L'astuce est la suivante.

\begin{multicols}{2}
	\showstep{1}{tikz/27-141[all]/up-1}{ajout d'une nouvelle colonne.}

	\showstep{2}{tikz/27-141[all]/up-2}{on n'utilise pas la colonne centrale.}
\end{multicols}

\showstepnovfill{3}{tikz/27-141[all]/up-3}{on fait une sorte de division \myquote{inversée}.}

\showstepnovfill{4}{tikz/27-141[all]/up-4}{on passe aux trois naturels suivants.}


\medskip


En répétant ce processus, nous arrivons à la représentation suivante.

\showstepnovfill{finale (2\ieme{} phase)}{tikz/27-141[main]/up}{remontée à mains nues des calculs.}


% --------------- %


\subsection{Et voilà comment conclure !}

\showstepnovfill{finale (la vraie)}{tikz/27-141[main]/last}{on finit avec un produit en croix.}


\medskip


Des coefficients de \bb{} s'obtiennent sans souci via l'équivalence suivante où nous avons $3 = \mathrm{pgcd}(27 , 141)$.
\[141 \times 4 - 27 \times 21 = -3 \,\Longleftrightarrow\, 27 \times 21 - 141 \times 4 = 3\]


\medskip


Nous allons voir, dans la section qui suit, que l'on obtient forcément à la fin $\pm \mathrm{pgcd}(27 , 141)$. 


\medskip


En pratique, nous n'avons pas besoin de détailler les calculs comme nous l'avons fait à certains moments afin d'expliquer comment procéder.
Avec ceci en tête, on comprend toute l'efficacité de la méthode présentée, mais pas encore justifiée, car il suffit de garder une trace minimale, mais complète, des étapes tout en ayant à chaque étape des opérations assez simples à effectuer.
Il reste à démontrer que notre méthode marche à tous les coups. Ceci est le propos de la section suivante.
	


% --------------- %


\section{Pourquoi cela marche-t-il ?}

\subsection{Avec des arguments élémentaires} Commençons par une preuve explicative qui malheureusement ne nous permet pas de voir d'où vient l'astuce \emph{(nous explorerons ceci dans la sous-section suivante)}. 

\showstepnovfill{Calculs faits dans les deux phases.}{tikz/why/twophases-focus}


\bigskip


Par construction, nous avons $a = qb + r$ et $X = qY + Z$. Ceci nous donne :

\vspace{-1em}

\begin{flalign*}
	d &= aY - bX               & \\
	  &= (qb + r)Y - b(qY + Z) & \\
	  &= rY - bZ               & \\
	  &= -e                    & \\
\end{flalign*}

\vspace{-1em}


Donc si l'on fait \myquote{glisser} des carrés sur les deux colonnes de gauche, les produits en croix dans ces carrés ne différeront que de leur signe. 


\medskip


Grâce à la représentation finale ci-dessous, nous obtenons $aY - bX = \pm \pgcd(a ; b)$ car le dernier reste non nul de l'algorithme d'Euclide est $\pgcd(a ; b)$. Ceci prouve la validité de la méthode dans le cas général. On comprend au passage l'ajout initial du $0$ et du $1$ dans la 3\ieme{} colonne \emph{(bien entendu, $(-1)$ aurait aussi pu convenir)}.

\showstepnovfill{Une représentation symbolique au complet.}{tikz/why/twophases-all}
	

% --------------- %


\subsection{Avec des matrices pour aller plus loin}

Reprenons le cas de base suivant mais an l'analysant à l'aune des matrices.

\showstepnovfill{Calculs faits dans les deux phases.}{tikz/why/twophases-focus}


\medskip


Notant
$M 
 =
 \begin{pmatrix}
	a & X \\ 
	b & Y
 \end{pmatrix}$,
et
$N 
 =
 \begin{pmatrix}
	b & Y \\ 
	r & Z
 \end{pmatrix}$,
nous avons $d = \det M$ et $e = \det N$.


\medskip


Comme 


% --------------- %


%\section{Des algorithmes \myquote{computer friendly}, de grands classiques}

%\input{bezout-coef-for-human/computer}

\bigskip

\hrule

\section{AFFAIRE À SUIVRE...}

\bigskip

\hrule
\end{document}
