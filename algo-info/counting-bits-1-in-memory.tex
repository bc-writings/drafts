\documentclass[12pt]{amsart}
\usepackage[T1]{fontenc}
\usepackage[utf8]{inputenc}

\usepackage[top=1.95cm, bottom=1.95cm, left=2.35cm, right=2.35cm]{geometry}

\usepackage{hyperref}
\usepackage{enumitem}
\usepackage{tcolorbox}
\usepackage{multicol}
\usepackage{fancyvrb}
\usepackage{xstring}
\usepackage{amsmath}
\usepackage[french]{babel}
\usepackage[
    type={CC},
    modifier={by-nc-sa},
	version={4.0},
]{doclicense}
\usepackage{textcomp}
\usepackage{xcolor}
\usepackage{tcolorbox}

\usepackage{pgffor}
      
\usepackage{tnsmath}

    
\newtheorem{fact}{Fait}%[section]
\newtheorem*{fact*}{Fait}%[section]


\newtheorem{definition}{Définition}%[section]
\newtheorem{theorem}{Théorème}%[section]
\newtheorem{example}{Exemple}%[section]
\newtheorem{exercise}{Exercice}%[section]
\newtheorem{remark}{Remarque}%[section]
\newtheorem*{notations}{Notations}

\setlength\parindent{0pt}


\DeclareMathOperator{\ediv}{\operatorname{div}}


\newcommand\centerit[1]{%%
	\smallskip
	
	\begin{center}
		#1
	\end{center}
}



% Source pour le \@tfor :
% 	* https://tex.stackexchange.com/a/253205/6880
\makeatletter
	\newcommand\equality{\@ifstar{\@equality@star@}{\@equality@no@star@}}

	\newcommand\@equality@star@[1]{
		\smallskip
	
		$=$ {} \qquad  $\leftarrow$ #1
	
		\smallskip
	}

	\newcommand\@equality@no@star@{
		\smallskip
	
		$=$

		\smallskip
	}

% Nothing special for the cell...	
	\newcommand\elasticbox[1]{%
		\fcolorbox{black}{white}{\normalfont #1}%
	}

	\newcommand\fixedboxcolored[3]{%
		\fcolorbox{#1}{#2}{\makebox[1em]{\normalfont\vphantom{?}#3}}%
	}

	\newcommand\fixedbox[1]{%
		\fixedboxcolored{black}{white}{\makebox[1em]{\normalfont\vphantom{?}#1}}%
	}

% ... will move
	\newcommand\elasticboxwill[1]{%
		\fcolorbox{black}{red!15}{\normalfont #1}%
	}

	\newcommand\fixedboxwill[1]{%
		\fixedboxcolored{black}{red!15}{\normalfont #1}%
	}

% ... has moved
	\newcommand\elasticboxhas[1]{%
		\fcolorbox{black}{black!10}{\normalfont #1}%
	}
	
	\newcommand\fixedboxhas[1]{%
		\fixedboxcolored{black}{black!10}{\normalfont #1}%
	}

% Let's go !
	\newcommand\onefortwocomplemen{\fixedboxcolored{black}{green!15}{\bfseries 1}}
	\newcommand\twocomplement{\fixedboxwill{$\bullet$}}

	\newcommand\@empty@content{\phantom{?}}
	\newcommand\emptycell{\fixedbox{\@empty@content}}
	\newcommand\one{\fixedbox{\bfseries 1}}
	\newcommand\zero{\fixedbox{\bfseries 0}}
	
	\newcommand\@ellipsis@{\,\,\vphantom{?}$\cdots$\,\,}
	\newcommand\myellipsis{\elasticbox{\@ellipsis@}}

% ... will move
	\newcommand\emptycellwill{\fixedboxwill{\@empty@content}}
	\newcommand\onewill{\fixedboxwill{\bfseries 1}}
	\newcommand\zerowill{\fixedboxwill{\bfseries 0}}
	\newcommand\myellipsiswill{\elasticboxwill{\@ellipsis@}}
	
% ... has moved
	\newcommand\emptycellhas{\fixedboxhas{\@empty@content}}
	\newcommand\onehas{\fixedboxhas{\bfseries 1}}
	\newcommand\zerohas{\fixedboxhas{\bfseries 0}}
	\newcommand\myellipsishas{\elasticboxhas{\@ellipsis@}}


% 0 : 0 normal
% 1 : 1 normal
% . : vide normal
% - : ellipsis
%
% Z : 0 qui va bouger
% U : 1 qui va bouger
% v : vide qui va bouger
% = : ellipsis qui va bouger
%
% z : 0 qui vient de bouger
% u : 1 qui vient de bouger
% a : vide qui vient de bouger
% + : ellipsis
%
%
% * : 1 pour complément à 2 + 1
% o : complément à 2 + 1
%
% < Ouvre un cadre de surlignement de plusieurs cellules
% < Ferme un cadre de surlignement de plusieurs cellules
	\newcommand\binary[1]{%
		\@tfor\next:=#1\do{%
			\if\next 0\zero{}\else%
		    \if\next 1\one{}\else%
			\if\next .\emptycell{}\else%
			\if\next -\myellipsis{}\else%
			%
			\if\next Z\zerowill{}\else%
		    \if\next U\onewill{}\else%
		    \if\next v\emptycellwill{}\else%
			\if\next =\myellipsiswill{}\else%
		    %
			\if\next z\zerohas{}\else%
		    \if\next u\onehas{}\else%
		    \if\next a\emptycellhas{}\else%
			\if\next +\myellipsishas{}\else%
			%
			\if\next *\onefortwocomplemen{}\else%
			\if\next o\twocomplement{}\else%
		    %
\quad {\bfseries [[ illegal character : \, {\normalfont \next{}} \, ]]} \quad % BUG
			\fi%  :
			\fi%  a
			\fi%  u
			\fi%  p
			%
			\fi%  =
			\fi%  v
			\fi%  n
			\fi%  b
			%
			\fi%  -
			\fi%  .
			\fi%  N
			\fi%  B
			%
			\fi%  *
			\fi%  o
		}%
	}
	

	\newcommand\autobox[1]{%
      \foreach \i in {1, ..., #1} {%
        \fixedbox{\i}%
      }
	}
	

	\newcommand\binaryplus{\@ifstar{\@binaryplus@star@}{\@binaryplus@no@star@}}

	\newcommand\@binaryplus@no@star@[2]{
		\binary{#1}
		\setbox0=\hbox{#2\unskip}\ifdim\wd0=0pt \else {\small \ $\leftarrow$ #2}\fi
	}

	\newcommand\@binaryplus@star@[2]{
		\@binaryplus@no@star@{#1}{#2}
		
		\StrLen{#1}[\@length@]
		\noindent  
		\autobox{\@length@}
	}

	\newcommand\listbox[1]{%
		\@tfor\next:=#1\do{\fixedbox{\next}}%
	}
\makeatother


 
\begin{document}

\title{BROUILLON - candidat - Compter en place le nombre de bits non nuls d'une écriture binaire}
\author{Christophe BAL}
\date{31 Mars 2019}

\maketitle

\begin{center}
	\itshape
	Document, avec son source \LaTeX, disponible sur la page
	
	\url{https://github.com/bc-writing/drafts}.
\end{center}


\bigskip


\begin{center}
	\hrule\vspace{.3em}
	{
		\fontsize{1.35em}{1em}\selectfont
		\textbf{Mentions \og légales \fg}
	}
			
	\vspace{0.45em}
	\doclicenseThis
	\hrule
\end{center}


\bigskip


En informatique, il existe une méthode astucieuse pour compter le nombre de \verb+1+ dans une écriture binaire.
Expliquons comment faire avec l'écriture binaire \verb+1101.1001.0110.1111+ qui contient $16 = 2^4$ chiffres. Ci-dessous, toutes les opérations se font en base $2$. 
\textbf{Notez bien que l'on note les résultats dans une écriture binaire  de taille 16 et c'est là la belle astuce du point de vue informatique.}

\begin{itemize}[label=\small\textbullet]
	\item On additionne les chiffres voisins puis on conserve le résultat sur deux chiffres.
	
	\begin{center}
	\verb-[1+1 | 0+1 | 1+0 | 0+1 | 0+1 | 1+0 | 1+1 | 1+1]-

	\verb-[10  | 01  | 01  | 01  | 01  | 01  | 10  | 10 ]-
	\end{center}


	\item On additionne les paires voisines puis on conserve le résultat sur quatre chiffres.
	
	\begin{center}
	\verb-[10  +  01 | 01  +  01 | 01  +  01 | 10  +  10]-

	\verb-[  0011    |   0010    |   0010    |   0100   ]-
	\end{center}


	\item On additionne les quadruplets voisins puis on conserve le résultat sur huit chiffres.

	\begin{center}
	\verb-[  0011    +   0010    |   0010    +   0100   ]-

	\verb-[      00000101        |       00000110       ]-
	\end{center}


	\item On additionne les octuplets voisins puis on conserve le résultat sur seize chiffres et on s'arrête pour analyser le résultat.

	\begin{center}
	\verb-[      00000101        +       00000110       ]-

	\verb-[              0000000000001011               ]-
	\end{center}
	
	
	\item Le dernier nombre a pour valeur décimale $11$ qui est exactement le nombre de \verb+1+ contenus dans \verb+1101.1001.0110.1111+.
\end{itemize}




Très joli ! Non ? Très bien ! Mais pourquoi, ou comment cela marche-t-il ? En fait c'est tout simple. Pour le voir reprenons les étapes en expliquant l'information qu'elles construisent.

\begin{itemize}[label=\small\textbullet]
	\item La 1\iere étape redonnée ci-dessous compte le nombre de \verb+1+ par paire de voisins. 
	Il est important de noter qu'il n'y a pas de perte d'information car au maximum il y aura deux \verb+1+, or deux s'écrit \verb+10+ en base $2$. 

	\begin{center}
	\verb-[1+1 | 0+1 | 1+0 | 0+1 | 0+1 | 1+0 | 1+1 | 1+1]-

	\verb-[10  | 01  | 01  | 01  | 01  | 01  | 10  | 10 ]-
	\end{center}


	\item La 2\ieme étape redonnée ci-dessous compte le nombre de \verb+1+ par quadruplets de voisins sans perdre d'information car au maximum il y aura quatre \verb+1+, or $4$ s'écrit \verb+0100+ en base $2$. 

	\begin{center}
	\verb-[10  +  01 | 01  +  01 | 01  +  01 | 10  +  10]-

	\verb-[  0011    |   0010    |   0010    |   0100   ]-
	\end{center}


	\item La 3\ieme étape redonnée ci-dessous compte le nombre de \verb+1+ par octuplets de voisins sans perdre d'information. 

	\begin{center}
	\verb-[  0011    +   0010    |   0010    +   0100   ]-

	\verb-[      00000101        |       00000110       ]-
	\end{center}


	\item La 4\ieme étape redonnée ci-dessous compte le nombre de \verb+1+ pour $16$ voisins, c'est à dire tous les chiffres, sans perdre d'information. Simple mais efficace !

	\begin{center}
	\verb-[      00000101        +       00000110       ]-

	\verb-[              0000000000001011               ]-
	\end{center}
\end{itemize}

Vous noterez le caractère général des explications précédentes.
Ainsi il est immédiat d'étendre la méthode à toute écriture binaire contenant $2^p$ chiffres où $p\in\NNs$.

\end{document}
