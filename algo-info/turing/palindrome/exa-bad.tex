Voici les grandes étapes présentées sur deux colonnes.


\begin{multicols}{2}

% ????

\phantom{\emptybox\emptybox}%
	\deah

\emptybox\emptybox%
	\boxit{a}\boxit{b}\boxit{b}\boxit{c}\boxit{a}%
\emptybox\emptybox

\emptybox\emptybox%
	\emptybox\emptybox\emptybox\emptybox\emptybox%
\emptybox\emptybox

\phantom{\emptybox\emptybox}%
	\head


\medskip % ????

\phantom{\emptybox\emptybox\emptybox}%
	\deah

\emptybox\emptybox%
	\boxit{a}\boxit{b}\boxit{b}\boxit{c}\boxit{a}%
\emptybox\emptybox

\emptybox\emptybox%
	\nboxit{a}\emptybox\emptybox\emptybox\emptybox%
\emptybox\emptybox

\phantom{\emptybox\emptybox\emptybox}%
	\head


\medskip % ????

\phantom{\emptybox\emptybox\emptybox\emptybox}%
	\deah

\emptybox\emptybox%
	\boxit{a}\boxit{b}\boxit{b}\boxit{c}\boxit{a}%
\emptybox\emptybox

\emptybox\emptybox%
	\nboxit{a}\nboxit{b}\emptybox\emptybox\emptybox%
\emptybox\emptybox

\phantom{\emptybox\emptybox\emptybox\emptybox}%
	\head


\bigskip % ????

\dots

\bigskip % ????

\phantom{\emptybox\emptybox\emptybox\emptybox\emptybox\emptybox\emptybox}%
	\deah

\emptybox\emptybox%
	\boxit{a}\boxit{b}\boxit{b}\boxit{c}\boxit{a}%
\emptybox\emptybox

\emptybox\emptybox%
	\nboxit{a}\nboxit{b}\nboxit{b}\nboxit{c}\nboxit{a}%
\emptybox\emptybox

\phantom{\emptybox\emptybox\emptybox\emptybox\emptybox\emptybox\emptybox}%
	\head


\vfill\null
\columnbreak  % ????

\phantom{\emptybox}%
	\deah

\emptybox\emptybox%
	\boxit{a}\boxit{b}\boxit{b}\boxit{c}\boxit{a}%
\emptybox\emptybox

\emptybox\emptybox%
	\nboxit{a}\nboxit{b}\nboxit{b}\nboxit{c}\nboxit{a}%
\emptybox\emptybox

\phantom{\emptybox\emptybox\emptybox\emptybox\emptybox\emptybox\emptybox}%
	\head


\medskip % ????

\phantom{\emptybox\emptybox}%
	\deah

\emptybox\emptybox%
	\fboxit{a}\boxit{b}\boxit{b}\boxit{c}\boxit{a}%
\emptybox\emptybox

\emptybox\emptybox%
	\nboxit{a}\nboxit{b}\nboxit{b}\nboxit{c}\fboxit{a}%
\emptybox\emptybox

\phantom{\emptybox\emptybox\emptybox\emptybox\emptybox\emptybox}%
	\head


\medskip % ????

\phantom{\emptybox\emptybox\emptybox}%
	\deah

\emptybox\emptybox%
	\fboxit{a}\wboxit{b}\boxit{b}\boxit{c}\boxit{a}%
\emptybox\emptybox

\emptybox\emptybox%
	\nboxit{a}\nboxit{b}\nboxit{b}\wboxit{c}\fboxit{a}%
\emptybox\emptybox

\phantom{\emptybox\emptybox\emptybox\emptybox\emptybox}%
	\head

\vfill\null
\end{multicols}




\vspace{-1em}


Qu'a-t-on fait ?
\begin{enumerate}
	\item On note la lettre pointée par la tête de lecture.
	      Commence alors une phase de recherche d'une lettre connue.

	\item On avance tant que l'on ne rencontre par une case vide. Une fois celle-ci repérée on revient d'une case en arrière.
	
	\item On compare alors la lettre pointée par la tête de lecture avec celle de la phase de recherche en cours. Deux cas sont possibles.
	\begin{enumerate}
		\item Si les lettres sont différentes alors on ne fait plus rien. 
		      On a un état bloquant et le mot n'est pas validé.

		\item Si les lettres sont identiques alors on efface la lettre en cours puis on va vers la gauche jusqu'à la prochaine case vide. 
		      Une fois celle-ci trouvée, on avance d'une case vers la droite pour effacer son contenu, puis on avance d'une autre case vers la droite pour recommencer les actions à partir du point 1.
	\end{enumerate}
\end{enumerate}


La méthode ci-dessus est en fait incomplète comme nous allons le voir dans la section suivante avec un exemple de palindrome à repérer.

