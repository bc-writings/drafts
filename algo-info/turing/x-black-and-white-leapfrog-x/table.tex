Comme les machines de Turing utilisent des cases vides, nous allons coder la case vide de la devinette par une case contenant la croix $\times$.
Reprenons la configuration donnée en exemple dans la section précédente. 

\begin{center}
	\boxedB\boxedB\boxedB%
	\emptybox%
	\boxedW\boxedW%
\end{center}

Pour la machine de Turing, nous utiliserons la représentation initiale suivante.

\begin{center}
\emptybox\emptybox%
	\boxedB\boxedB\boxedB%
	\boxit{$\times$}%
	\boxedW\boxedW%
\emptybox\emptybox
\end{center}


Afin de vous laisser chercher un peu, nous ne donnons que dans la page suivante une résolution possible de la devinette pour la configuration ci-dessus. Nous utilisons les conventions suivantes.

\begin{enumerate}
	\item Une case jaune indique un mouton qui va bouger.

	\item Une case grise indique un mouton qui vient de se déplacer.

	\item Une case rouge indique un mouton qui est arrivé à sa place finale.
\end{enumerate}



\newpage

\begin{center}
\emptybox\emptybox%
	\boxedB%
	\boxedB%
	\boxedB%
	\boxit{$\times$}%
	\boxedW%
	\boxedW%
\emptybox\emptybox


\vspace{.75em} %

\emptybox\emptybox%
	\boxedB%
	\boxedB%
	\boxedB%
	\boxit{$\times$}%
	\fboxedW%
	\boxedW%
\emptybox\emptybox


\vspace{.75em} %

\emptybox\emptybox%
	\boxedB%
	\boxedB%
	\fboxedB%
	\nboxedW%
	\boxit{$\times$}%
	\boxedW%
\emptybox\emptybox


\vspace{.75em} %

\emptybox\emptybox%
	\boxedB%
	\fboxedB%
	\boxit{$\times$}%
	\boxedW%
	\nboxedB%
	\boxedW%
\emptybox\emptybox


\vspace{.75em} %

\emptybox\emptybox%
	\boxedB%
	\boxit{$\times$}%
	\nboxedB%
	\fboxedW%
	\boxedB%
	\boxedW%
\emptybox\emptybox


\vspace{.75em} %

\emptybox\emptybox%
	\boxedB%
	\nboxedW%
	\boxedB%
	\boxit{$\times$}%
	\boxedB%
	\fboxedW%
\emptybox\emptybox


\vspace{.75em} %

\emptybox\emptybox%
	\boxedB%
	\boxedW%
	\boxedB%
	\nboxedW%
	\fboxedB%
	\boxit{$\times$}%
\emptybox\emptybox


\vspace{.75em} %

\emptybox\emptybox%
	\boxedB%
	\boxedW%
	\fboxedB%
	\boxedW%
	\boxit{$\times$}%
	\nboxedB%
\emptybox\emptybox


\vspace{.75em} %

\emptybox\emptybox%
	\fboxedB%
	\boxedW%
	\boxit{$\times$}%
	\boxedW%
	\nboxedB%
	\wboxedB%
\emptybox\emptybox


\vspace{.75em} %

\emptybox\emptybox%
	\boxit{$\times$}%
	\fboxedW%
	\nboxedB%
	\boxedW%
	\wboxedB%
	\wboxedB%
\emptybox\emptybox


\vspace{.75em} %

\emptybox\emptybox%
	\nboxedW%
	\boxit{$\times$}%
	\boxedB%
	\fboxedW%
	\wboxedB%
	\wboxedB%
\emptybox\emptybox


\vspace{.75em} %

\emptybox\emptybox%
	\wboxedW%
	\nboxedW%
	\fboxedB%
	\boxit{$\times$}%
	\wboxedB%
	\wboxedB%
\emptybox\emptybox


\vspace{.75em} %

\emptybox\emptybox%
	\wboxedW%
	\wboxedW%
	\boxit{$\times$}%
	\nboxedB%
	\wboxedB%
	\wboxedB%
\emptybox\emptybox


\vspace{.75em} %

\emptybox\emptybox%
	\wboxedW%
	\wboxedW%
	\boxit{$\times$}%
	\wboxedB%
	\wboxedB%
	\wboxedB%
\emptybox\emptybox

\end{center}


Les mouvements faits suivent la stratégie gloutonne consistant à bouger le plus de moutons d'une couleur donnée sans se mettre dans une position bloquante autre que la position gagnante, puis à changer de couleur si aucun mouvement n'est possible et si la partie n'est pas finie. Il ne reste plus qu'à traduire cela via une machine de Turing\dots
Pas si vite ! En effet, il nous faut repérer les configurations permettant d'avancer, celles impliquant un changement de couleur, et enfin celle indiquant que la partie est finie.
En marquant les moutons bien placés, la tache devient assez facile. Voici les points essentiels qui vont nous donner une machine de Turing résolvant la devinette \emph{(on garde la convention précédente pour le fond rouge)}.

\begin{enumerate}
	\item Supposons que ce soit la couleur noire qui joue.
	\begin{enumerate}
		\item \boxedB\boxit{$\times$}\emptybox{}
		      devient
		      \boxit{$\times$}\wboxedB\emptybox{}
		      par simple déplacement.


		\item \boxedB\boxit{$\times$}\wboxedB{}
		      devient
	    	  \boxit{$\times$}\wboxedB\wboxedB{}
		      par simple déplacement.


		\item \boxedB\boxit{$\times$}\boxedB{} indique un changement de couleur 
		      \footnote{
		      		En effet, le mouton à droite n'est pas encore bien placé.
			  }.


		\item \boxedB\boxit{$\times$}\boxedW{}
		      devient
		      \boxit{$\times$}\boxedB\boxedW{}
		      par simple déplacement.


		\item \boxedB\boxedW\boxit{$\times$}{}
		      devient
	    	  \boxit{$\times$}\boxedW\boxedB{}
		      via un bond.
	\end{enumerate}
	
	
	\item Pour la couleur blanche, nous avons les mouvements similaires suivants.
	\begin{enumerate}
		\item \emptybox\boxit{$\times$}\boxedW{}
		      devient
		      \emptybox\wboxedW\boxit{$\times$} .


		\item \wboxedW\boxit{$\times$}\boxedW{}
		      devient
	    	  \wboxedW\wboxedW\boxit{$\times$} .


		\item \boxedW\boxit{$\times$}\boxedW{} indique un changement de couleur.


		\item \boxedB\boxit{$\times$}\boxedW{}
		      devient
		      \boxedB\boxedW\boxit{$\times$} .


		\item \boxit{$\times$}\boxedB\boxedW{}
		      devient
	    	  \boxedW\boxedB\boxit{$\times$} .
	\end{enumerate}
	
	
	\item \wboxedW\boxit{$\times$}\wboxedB{} indique que la partie est finie.
\end{enumerate}


Nous admettrons que les mouvements précédents résolvent à tous les coups la devinette
\footnote{
	Les preuves arriveront plus tard car l'auteur manque de temps libre.
}.
Ceci nous donne la table des transitions suivante où nous utilisons des lettres minuscules pour indiquer les moutons bien placés.
De plus, nous supposons que ce sont les blancs qui commencent, la couleur jouant étant codée sur une 2\ieme{} bande.


\begin{center}
	\emph{\small Phase 1 : commencer par se positionner sur la croix.}
	
	\smallskip
	\renewcommand{\arraystretch}{1.25}
	\begin{tabular}{|c||c|c|c|}
		\hline
		$\delta$
			& $\twocoord{K}{B}$
			& $\twocoord{W}{B}$
			& $\twocoord{\times}{B}$ \\
		\hline
		\hline
		$q_0$
			& \transition{r_{cross}}{\twocoord{K}{W}}{\twocoord{D}{D}}
			&
			&                                                          \\
		\hline
		$r_{cross}$
			& \transition{r_{cross}}{\twocoord{K}{B}     }{\twocoord{D}{I}}
			& \transition{r_{cross}}{\twocoord{W}{B}     }{\twocoord{D}{I}}
			& \transition{s_{ana}  }{\twocoord{\times}{B}}{\twocoord{G}{I}} \\
		\hline
	\end{tabular}
	\renewcommand{\arraystretch}{1}
\end{center}


\begin{center}
	\emph{\small Phase 2-a : analyser la configuration locale \emph{(partie 1)}.}
	
	\smallskip
	\renewcommand{\arraystretch}{1.25}
	\begin{tabular}{|c||c|c|c|c|c|}
		\hline
		$\delta$
			& $\twocoord{K}{B}$
			& $\twocoord{W}{B}$
			& $\twocoord{k}{B}$
			& $\twocoord{w}{B}$
			& $\twocoord{B}{B}$ \\
		\hline
		\hline
		$s_{ana}$
			& \transition{c_{K}}{\twocoord{K}{B}}{\twocoord{D}{I}}
			& \transition{c_{W}}{\twocoord{W}{B}}{\twocoord{D}{I}}
			& \transition{c_{k}}{\twocoord{k}{B}}{\twocoord{D}{I}}
			& \transition{c_{w}}{\twocoord{w}{B}}{\twocoord{D}{I}}
			& \transition{c_{B}}{\twocoord{w}{B}}{\twocoord{D}{I}} \\
		\hline
	\end{tabular}
	\renewcommand{\arraystretch}{1}
\end{center}


\begin{center}
	\emph{\small Phase 2-b : analyser la configuration locale \emph{(partie 2)}.}
	
	\smallskip
	\renewcommand{\arraystretch}{1.25}
	\begin{tabular}{|c||c|}
		\hline
		$\delta$
			& $\twocoord{\times}{B}$ \\
		\hline
		\hline
		$c_{K}$
			& \transition{c_{K\times}}{\twocoord{\times}{B}}{\twocoord{D}{I}} \\
		\hline
		$c_{W}$
			& \transition{c_{W\times}}{\twocoord{\times}{B}}{\twocoord{D}{I}} \\
		\hline
		$c_{k}$
			& \transition{c_{k\times}}{\twocoord{\times}{B}}{\twocoord{D}{I}} \\
		\hline
		$c_{w}$
			& \transition{c_{w\times}}{\twocoord{\times}{B}}{\twocoord{D}{I}} \\
		\hline
		$c_{B}$
			& \transition{c_{B\times}}{\twocoord{\times}{B}}{\twocoord{D}{I}} \\
		\hline
	\end{tabular}
	\renewcommand{\arraystretch}{1}
\end{center}


\newpage


\begin{center}
	\emph{\small Phase 2-c : analyser la configuration locale \emph{(partie 3)}.}
	
	\smallskip
	\renewcommand{\arraystretch}{1.25}
	\begin{tabular}{|c||c|c|c|c|c|}
		\hline
		$\delta$
			& $\twocoord{K}{B}$
			& $\twocoord{W}{B}$
			& $\twocoord{k}{B}$
			& $\twocoord{w}{B}$
			& $\twocoord{B}{B}$ \\
		\hline
		\hline
		$c_{K\times}$
			& \transition{c_{K\times K}}{\twocoord{K}{B}     }{\twocoord{G}{G}}
			& \transition{c_{K\times W}}{\twocoord{W}{B}     }{\twocoord{G}{G}}
			& \transition{c_{K\times k}}{\twocoord{k}{B}     }{\twocoord{G}{G}}
			& \transition{c_{K\times w}}{\twocoord{w}{B}     }{\twocoord{G}{G}}
			& \transition{c_{K\times B}}{\twocoord{\times}{B}}{\twocoord{G}{G}} \\
		\hline
		$c_{W\times}$
			& \transition{c_{W\times K}}{\twocoord{K}{B}     }{\twocoord{G}{G}}
			& \transition{c_{W\times W}}{\twocoord{W}{B}     }{\twocoord{G}{G}}
			& \transition{c_{W\times k}}{\twocoord{k}{B}     }{\twocoord{G}{G}}
			& \transition{c_{W\times w}}{\twocoord{w}{B}     }{\twocoord{G}{G}}
			& \transition{c_{W\times B}}{\twocoord{\times}{B}}{\twocoord{G}{G}} \\
		\hline
		$c_{k\times}$
			& \transition{c_{k\times K}}{\twocoord{K}{B}     }{\twocoord{G}{G}}
			& \transition{c_{k\times W}}{\twocoord{W}{B}     }{\twocoord{G}{G}}
			& \transition{c_{k\times k}}{\twocoord{k}{B}     }{\twocoord{G}{G}}
			& \transition{c_{k\times w}}{\twocoord{w}{B}     }{\twocoord{G}{G}}
			& \transition{c_{k\times B}}{\twocoord{\times}{B}}{\twocoord{G}{G}} \\
		\hline
		$c_{w\times}$
			& \transition{c_{w\times K}}{\twocoord{K}{B}     }{\twocoord{G}{G}}
			& \transition{c_{w\times W}}{\twocoord{W}{B}     }{\twocoord{G}{G}}
			& \transition{c_{w\times k}}{\twocoord{k}{B}     }{\twocoord{G}{G}}
			& \transition{c_{w\times w}}{\twocoord{w}{B}     }{\twocoord{G}{G}}
			& \transition{c_{w\times B}}{\twocoord{\times}{B}}{\twocoord{G}{G}} \\
		\hline
		$c_{B\times}$
			& \transition{c_{B\times K}}{\twocoord{K}{B}     }{\twocoord{G}{G}}
			& \transition{c_{B\times W}}{\twocoord{W}{B}     }{\twocoord{G}{G}}
			& \transition{c_{B\times k}}{\twocoord{k}{B}     }{\twocoord{G}{G}}
			& \transition{c_{B\times w}}{\twocoord{w}{B}     }{\twocoord{G}{G}}
			& \transition{c_{B\times B}}{\twocoord{\times}{B}}{\twocoord{G}{G}} \\
		\hline
	\end{tabular}
	\renewcommand{\arraystretch}{1}
\end{center}


\begin{center}
	\emph{\small Phase 3 : c'est \myquote{blanc qui joue}.}
	
	\smallskip
	\renewcommand{\arraystretch}{1.25}
	\begin{tabular}{|c||c|c|c|c|c|}
		\hline
		$\delta$
			& $\twocoord{K}{B}$
			& $\twocoord{W}{B}$
			& $\twocoord{k}{B}$
			& $\twocoord{w}{B}$
			& $\twocoord{B}{B}$ \\
		\hline
		\hline
		$c_{B\times W}$
			&
			&
			&
			&
			& \\
		\hline
		$c_{w\times W}$
			&
			&
			&
			&
			& \\
		\hline
		$c_{W\times W}$
			&
			&
			&
			&
			& \\
		\hline
		$c_{K\times W}$
			&
			&
			&
			&
			& \\
		\hline
		$c_{K\times W}$
			&
			&
			&
			&
			& \\
		\hline
	\end{tabular}
	\renewcommand{\arraystretch}{1}
\end{center}
