Voici une petite devinette bien sympathique.

\begin{enumerate}
	\item Dans des cases sont disposés à gauche uniquement des moutons noirs \textbf{K}
	      \footnote{
	      	   Penser à la fin de \myquote{black}.
	      }
	      et à droite que des blancs \textbf{W}
	      \footnote{
	      	   Penser au début de \myquote{white}.
	      }
	      avec une case vide entre chaque groupe. Voici un exemple.
		  \begin{center}
			\boxedB\boxedB\boxedB%
			\emptybox%
			\boxedW\boxedW%
		  \end{center}


	\item Les moutons noirs ne se déplacent que vers la droite, et les blancs uniquement vers la gauche. Aucun retour en arrière n'est possible !


	\item Pour avancer, un mouton ne peut faire que deux choses dans sa direction de déplacement.
	\begin{enumerate}
		\item Si la case vide est devant lui, un mouton peut avancer dans cette case.

		\item Un mouton peut sauter au-dessus d'un seul mouton d'une autre couleur pour arriver dans la case vide.
	\end{enumerate}
\end{enumerate}


La question est de savoir s'il est possible de faire toujours passer tous les moutons blancs à gauche les uns à côté des autres, et tous les noirs à droite avec une case vide entre ces deux groupes de moutons. Nous allons répondre par l'affirmative en fournissant une machine de Turing résolvant le problème.
