La méthodologie est ici assez simple. On reprend l'une des tables vues précédemment.
Vous avez dû noter que sont utilisés des états passerelles pour passer la main au 2\ieme{} traitement. On va donc rendre tous ces états finaux.
Ceci ne suffit pas car il faut aussi que les états bloquants
\footnote{
	Il faut bien entendu penser à traiter les états bloquants non indiqués sur la table initiale !
}
de la 1\iere{} machine donne la main à la seconde en faisant tout de même attention pour les machines à une seule bande à bien replacer la tête de lecture \emph{(dans notre cas, ce problème ne se pose pas, ce qui nous enlève un peu de travail)}.
Nous avons alors les trois tables des transitions suivantes.



\begin{center}
	\emph{\bfseries Table à une bande \myquote{automatique}}
	
	\smallskip
	\begin{tabular}{|c||c|c|c|}
% MUTIPLE OF 3
		\hline
		$\delta$ 
			& $0$ 
			& $1$
			& $B$ \\
		\hline
		\hline
		$q_0$ 
			& $(d , 0, D)$ 
			& $(d , 1, D)$
			&  \\
		\hline
		$d$ 
			& $(d , 0, D)$ 
			& $(d , 1, D)$
			& $(sp_0, B, G)$ \\
		\hline
		\hline
		$sp_0$ 
			& $(si_0 , 0, G)$ 
			& $(si_1 , 1, G)$
			& $(f    , B, I)$ \\
		\hline
		$sp_1$ 
			& $(si_1 , 0, G)$ 
			& $(si_2 , 1, G)$
			& $(m    , B, I)$ \\
		\hline
		$sp_2$ 
			& $(si_2 , 0, G)$ 
			& $(si_0 , 1, G)$
			& $(m    , B, I)$ \\
		\hline
		\hline
		$si_0$ 
			& $(sp_0 , 0, G)$ 
			& $(sp_2 , 1, G)$
			& $(f    , B, I)$ \\
		\hline
		$si_1$ 
			& $(sp_1 , 0, G)$ 
			& $(sp_0 , 1, G)$
			& $(m    , B, I)$ \\
		\hline
		$si_2$ 
			& $(sp_2 , 0, G)$ 
			& $(sp_1 , 1, G)$
			& $(m    , B, I)$ \\
		\hline
% PARITY
		\hline
		$m$
			& $(\ell_0 , 0 , D)$
			& $(\ell_1 , 1 , D)$
			& $(m      , B , D)$ \\
		\hline
		\hline
		$\ell_0$
			& $(\ell_0 , 0 , D)$
			& $(\ell_1 , 1 , D)$
			& $(f      , B , I)$ \\
		\hline
		$\ell_1$
			& $(\ell_0 , 0 , D)$
			& $(\ell_1 , 1 , D)$
			&                    \\
		\hline
	\end{tabular}
\end{center}



\begin{center}
	\emph{\bfseries Table \myquote{optimisée}}
	
	\smallskip
	\begin{tabular}{|c||c|c|c|}
% MUTIPLE OF 3
		\hline
		$\delta$ 
			& $0$ 
			& $1$
			& $B$ \\
		\hline
		\hline
		$q_0$ 
			& $(\ell_0 , 0, D)$ 
			& $(\ell_1 , 1, D)$
			&  \\
		\hline
		$\ell_0$
			& $(\ell_0 , 0 , D)$
			& $(\ell_1 , 1 , D)$
			& $(f      , B , G)$ \\
		\hline
		$\ell_1$
			& $(\ell_0 , 0 , D)$
			& $(\ell_1 , 1 , D)$
			& $(sp_0   , B , G)$ \\
		\hline
		\hline
		$sp_0$ 
			& $(si_0 , 0, G)$ 
			& $(si_1 , 1, G)$
			& $(f    , B, I)$ \\
		\hline
		$sp_1$ 
			& $(si_1 , 0, G)$ 
			& $(si_2 , 1, G)$
			&                 \\
		\hline
		$sp_2$ 
			& $(si_2 , 0, G)$ 
			& $(si_0 , 1, G)$
			&                 \\
		\hline
		\hline
		$si_0$ 
			& $(sp_0 , 0, G)$ 
			& $(sp_2 , 1, G)$
			& $(f    , B, I)$ \\
		\hline
		$si_1$ 
			& $(sp_1 , 0, G)$ 
			& $(sp_0 , 1, G)$
			&                 \\
		\hline
		$si_2$ 
			& $(sp_2 , 0, G)$ 
			& $(sp_1 , 1, G)$
			&                 \\
		\hline
	\end{tabular}
\end{center}




\begin{center}
	\emph{\bfseries Table à deux bandes}
	
	\smallskip
	\emph{\small Phase 1 : copie de l'entrée.}
	
	\smallskip
	\renewcommand{\arraystretch}{1.25}
	\begin{tabular}{|c||c|c|c|c|c|}
		\hline
		$\delta$ 
			& $\twocoord{0}{B}$ 
			& $\twocoord{1}{B}$ 
			& $\twocoord{B}{B}$ 
			& $\twocoord{0}{0}$ 
			& $\twocoord{1}{1}$ \\
		\hline
		\hline
		$q_0$ 
			& $\left( c, \twocoord{0}{0}, \twocoord{D}{D} \right)$ 
			& $\left( c, \twocoord{1}{1}, \twocoord{D}{D} \right)$
			&                   
			&                   
			&                                                      \\
		\hline
		$c$ 
			& $\left( c, \twocoord{0}{0}, \twocoord{D}{D} \right)$ 
			& $\left( c, \twocoord{1}{1}, \twocoord{D}{D} \right)$
			& $\left( g, \twocoord{B}{B}, \twocoord{G}{G} \right)$
			&                   
			&                                                      \\
		\hline
		$g$ 
			&                     
			&                   
			& $\left( t_m, \twocoord{B}{B}, \twocoord{D}{D} \right)$
			& $\left( g  , \twocoord{0}{0}, \twocoord{G}{G} \right)$ 
			& $\left( g  , \twocoord{1}{1}, \twocoord{G}{G} \right)$ \\
		\hline
	\end{tabular}
	\renewcommand{\arraystretch}{1}
\end{center}



\begin{center}
	\emph{\small Phase 2 : a-t-on un multiple de $3$ via la bande du haut ?}
	
	\smallskip
	\renewcommand{\arraystretch}{1.25}
	\begin{tabular}{|c||c|c|c||}
		\hline
		$\delta$ 
			& $\twocoord{0}{\bullet}$ 
			& $\twocoord{1}{\bullet}$ 
			& $\twocoord{B}{\bullet}$  \\
		\hline
		\hline
		$t_m$ 
			& $\left( d , \twocoord{0}{\bullet}, \twocoord{D}{I} \right)$ 
			& $\left( d , \twocoord{1}{\bullet}, \twocoord{D}{I} \right)$
			&  \\
		\hline
		$d$ 
			& $\left( d   , \twocoord{0}{\bullet}, \twocoord{D}{I} \right)$ 
			& $\left( d   , \twocoord{1}{\bullet}, \twocoord{D}{I} \right)$
			& $\left( sp_0, \twocoord{B}{\bullet}, \twocoord{G}{I} \right)$ \\
		\hline
		\hline
		$sp_0$ 
			& $\left( si_0 , \twocoord{0}{\bullet}, \twocoord{G}{I} \right)$ 
			& $\left( si_1 , \twocoord{1}{\bullet}, \twocoord{G}{I} \right)$
			& $\left( f    , \twocoord{B}{\bullet}, \twocoord{I}{I} \right)$ \\
		\hline
		$sp_1$ 
			& $\left( si_1 , \twocoord{0}{\bullet}, \twocoord{G}{I} \right)$ 
			& $\left( si_2 , \twocoord{1}{\bullet}, \twocoord{G}{I} \right)$
			& $\left( t_p  , \twocoord{B}{\bullet}, \twocoord{I}{I} \right)$ \\
		\hline
		$sp_2$ 
			& $\left( si_2 , \twocoord{0}{\bullet}, \twocoord{G}{I} \right)$ 
			& $\left( si_0 , \twocoord{1}{\bullet}, \twocoord{G}{I} \right)$
			& $\left( t_p  , \twocoord{B}{\bullet}, \twocoord{I}{I} \right)$ \\
		\hline
		\hline
		$si_0$ 
			& $\left( sp_0 , \twocoord{0}{\bullet}, \twocoord{G}{I} \right)$ 
			& $\left( sp_2 , \twocoord{1}{\bullet}, \twocoord{G}{I} \right)$
			& $\left( f    , \twocoord{B}{\bullet}, \twocoord{I}{I} \right)$ \\
		\hline
		$si_1$ 
			& $\left( sp_1 , \twocoord{0}{\bullet}, \twocoord{G}{I} \right)$ 
			& $\left( sp_0 , \twocoord{1}{\bullet}, \twocoord{G}{I} \right)$
			& $\left( t_p  , \twocoord{B}{\bullet}, \twocoord{I}{I} \right)$ \\
		\hline
		$si_2$ 
			& $\left( sp_2 , \twocoord{0}{\bullet}, \twocoord{G}{I} \right)$ 
			& $\left( sp_1 , \twocoord{1}{\bullet}, \twocoord{G}{I} \right)$
			& $\left( t_p  , \twocoord{B}{\bullet}, \twocoord{I}{I} \right)$ \\
		\hline
	\end{tabular}
	\renewcommand{\arraystretch}{1}
\end{center}




\begin{center}
	\emph{\small Phase 3 : a-t-on un multiple de $3$ qui est aussi pair via la bande du bas ?}
	
	\smallskip
	\renewcommand{\arraystretch}{1.25}
	\begin{tabular}{|c||c|c|c||}
		\hline
		$\delta$ 
			& $\twocoord{\bullet}{0}$ 
			& $\twocoord{\bullet}{1}$ 
			& $\twocoord{\bullet}{B}$  \\
		\hline
		\hline
		$t_p$
			& $\left( \ell_0 , \twocoord{\bullet}{0} , \twocoord{I}{D} \right)$
			& $\left( \ell_1 , \twocoord{\bullet}{1} , \twocoord{I}{D} \right)$
			&                                                                   \\
		\hline
		\hline
		$\ell_0$
			& $\left( \ell_0 , \twocoord{\bullet}{0} , \twocoord{I}{D} \right)$
			& $\left( \ell_1 , \twocoord{\bullet}{1} , \twocoord{I}{D} \right)$
			& $\left( f      , \twocoord{\bullet}{B} , \twocoord{I}{I} \right)$ \\
		\hline
		$\ell_1$
			& $\left( \ell_0 , \twocoord{\bullet}{0} , \twocoord{I}{D} \right)$
			& $\left( \ell_1 , \twocoord{\bullet}{1} , \twocoord{I}{D} \right)$
			&                                                                   \\
		\hline
	\end{tabular}
	\renewcommand{\arraystretch}{1}
\end{center}