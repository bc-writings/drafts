Si l'on sort des méthodes généralistes et automatisables, on peut faire une table plus simple.
En effet, pour tester si l'on a un multiple de $3$, nous partons de la droite.
Or nous savons aussi que le dernier chiffre lu à droite nous permet de savoir si l'on a ou non un nombre pair.
Il suffit donc lors du déplacement à droite de garder la trace de ce dernier chiffre, puis de continuer le travail si l'on a bien un zéro final.
Ceci nous donne ci-après une table des transitions plus courte que les précédentes.

\begin{center}
	\begin{tabular}{|c||c|c|c|}
% MUTIPLE OF 3
		\hline
		$\delta$ 
			& $0$ 
			& $1$
			& $B$ \\
		\hline
		\hline
		$q_0$ 
			& $(\ell_0 , 0, D)$ 
			& $(\ell_1 , 1, D)$
			&  \\
		\hline
		$\ell_0$
			& $(\ell_0 , 0 , D)$
			& $(\ell_1 , 1 , D)$
			& $(sp_0   , B , G)$ \\
		\hline
		$\ell_1$
			& $(\ell_0 , 0 , D)$
			& $(\ell_1 , 1 , D)$
			&                    \\
		\hline
		\hline
		$sp_0$ 
			& $(si_0 , 0, G)$ 
			& $(si_1 , 1, G)$
			& $(f    , B, I)$ \\
		\hline
		$sp_1$ 
			& $(si_1 , 0, G)$ 
			& $(si_2 , 1, G)$
			&                 \\
		\hline
		$sp_2$ 
			& $(si_2 , 0, G)$ 
			& $(si_0 , 1, G)$
			&                 \\
		\hline
		\hline
		$si_0$ 
			& $(sp_0 , 0, G)$ 
			& $(sp_2 , 1, G)$
			& $(f    , B, I)$ \\
		\hline
		$si_1$ 
			& $(sp_1 , 0, G)$ 
			& $(sp_0 , 1, G)$
			&                 \\
		\hline
		$si_2$ 
			& $(sp_2 , 0, G)$ 
			& $(sp_1 , 1, G)$
			&                 \\
		\hline
	\end{tabular}
\end{center}