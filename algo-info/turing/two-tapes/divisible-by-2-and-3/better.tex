Si l'on sort des méthodes généralistes et automatisables, on peut faire une table plus simple.
En effet, pour tester si l'on a un multiple de $3$, nous partons de la droite.
Or nous savons aussi que le dernier chiffre lu à droite nous permet de savoir si l'on a ou non un nombre pair.
Il suffit donc lors du déplacement à droite de garder la trace de ce dernier chiffre, puis de continuer le travail si l'on a bien un zéro final.
Ceci nous donne ci-après une table des transitions plus courte que les précédentes.

\begin{center}
	\begin{tabular}{|c||c|c|c|}
% MUTIPLE OF 3
		\hline
		$\delta$ 
			& $0$ 
			& $1$
			& $B$ \\
		\hline
		\hline
		$q_0$ 
			& \transition{\ell_0}{0}{D} 
			& \transition{\ell_1}{1}{D}
			&                           \\
		\hline
		$\ell_0$
			& \transition{\ell_0}{0}{D}
			& \transition{\ell_1}{1}{D}
			& \transition{sp_0  }{B}{G} \\
		\hline
		$\ell_1$
			& \transition{\ell_0}{0}{D}
			& \transition{\ell_1}{1}{D}
			&                           \\
		\hline
		\hline
		$sp_0$ 
			& \transition{si_0}{0}{G} 
			& \transition{si_1}{1}{G}
			& \transition{f   }{B}{I} \\
		\hline
		$sp_1$ 
			& \transition{si_1}{0}{G} 
			& \transition{si_2}{1}{G}
			&                         \\
		\hline
		$sp_2$ 
			& \transition{si_2}{0}{G} 
			& \transition{si_0}{1}{G}
			&                         \\
		\hline
		\hline
		$si_0$ 
			& \transition{sp_0}{0}{G} 
			& \transition{sp_2}{1}{G}
			& \transition{f   }{B}{I} \\
		\hline
		$si_1$ 
			& \transition{sp_1}{0}{G} 
			& \transition{sp_0}{1}{G}
			&                         \\
		\hline
		$si_2$ 
			& \transition{sp_2}{0}{G} 
			& \transition{sp_1}{1}{G}
			&                         \\
		\hline
	\end{tabular}
\end{center}