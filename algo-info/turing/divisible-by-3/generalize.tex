Le raisonnement précédent s'est appuyé sur $4 = 3 + 1$ soit $3 = 2^2 - 1$. Il est en fait facile de généraliser ce qui a été fait pour repérér les nombres divisibles par $2^k - 1$ où $k \in \NN_{\geq 2}$.
Ceci étant dit, les tables des transitions vont croître de façon exponentielle !


\medskip


\textbf{Remarque :} on peut en fait construire un automate d'état déterministe fini qui reconnait les multiples de $3$, et plus généralement de $d \in \NN_{\geq 2}$, à partir de l'écriture d'un nombre dans une base quelconque $b \in \NN_{\geq 2}$.
Ceci prouve que l'on peut construire des expressions rationnelles pour savoir si une écriture correspond à un multiple.