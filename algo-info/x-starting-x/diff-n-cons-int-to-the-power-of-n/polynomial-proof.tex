Voici une légère modification de l'algorithme \ref{nat-algo} où l'on manipule des polynômes dans $\setpoly{\RR}{X}$.
Notons au passage que nous passons d'un algorithme a priori indéterministe, car on fait un choix de naturels consécutifs, à un autre complètement déterministe.

\begin{algo}[frame]
	\caption{Version polynomiale} \label{nat-2-poly-algo}

	\In{$n \in \NNs$}
	\Out{?}
	
	\BlankLine
	
	\Actions{
		$L \Store [ X^n , (X+1)^n , \dots , (X+n)^n ]$
		\\
		\BlankLine
		\While{$taille(L) \neq 1$}{
			$newL \Store \EmptyList$
			\\
			\ForRange*{i}{1}{taille(L) - 1}{
				\Append{$newL$}{$(L[i + 1] - L[i])$}
			}
			$L \Store newL$
		}
		\BlankLine
		\Return{$L[1]$}
	}
\end{algo}


Il est aisée de traduire cet algorithme en Python. Le code suivant ne révèle aucun test raté.

\medskip

\inputcode{python}{diff-n-cons-int-to-the-power-of-n/exploring-poly-version.py}


\medskip


Il est clair que si l'on prouve que l'algorithme \ref{nat-2-poly-algo} renvoie $n!$, il en sera de même pour \ref{nat-algo}. Ceci découle directement de la validité de l'algorithme ci-dessous où $\setpoly{\RR_{n}}{X}$ désigne l'ensemble des polynômes réels de degré $n \in \NNs$.

\medskip

\begin{algo}[frame]
	\caption{Version polynomiale élargie} \label{gene-poly-algo}

	\In{$P \in \setpoly{\RR_{n}}{X}$ de coefficient dominant $a_n$}
	\Out{$n! \, a_n$}
	
	\BlankLine
	
	\Actions{
		$L \Store \List{P(X) , P(X+1) , \dots , P(X+n)}$
		\\
		\BlankLine
		\While{$taille(L) \neq 1$}{
			$newL \Store \EmptyList$
			\\
			\ForRange*{i}{1}{taille(L) - 1}{
				\Append{$newL$}{$(L[i + 1] - L[i])$}
			}
			$L \Store newL$
		}
		\BlankLine
		\Return{$L[1]$}
	}
\end{algo}


\medskip


Pourquoi cet algorithme est-il valide ? Pour la suite, nous posons $P(x) = \dsum_{k=0}^{n} a_k X^k$.
\begin{enumerate}
	\item Il est immédiat que $\forall k \in \NNs$, $(X+1)^k - X^k = kX^{k-1} + R(X)$ où $\deg R < k-1$ avec la convention $\deg 0 = - \infty$.

% ----------------- %

	\item Le point précédent donne sans effort $P(X+1) - P(X) = n a_n X^{n-1} + S(X)$ où le polynôme $S$ vérifie $\deg S < n-1$. Tout est dit comme nous allons le voir.

% ----------------- %

	\item A la 1\iere{} itération de la boucle, la liste 
			  $\List{P(X) , P(X+1) , \dots , P(X+n)}$
	 	  est transformée en 
		      $\List{P(X+1) - P(X) , P(X+2) - P(X+1) , \dots , P(X+n) - P(X+n-1)}$
		  soit $\List{Q(X) , Q(X+1) , \dots , Q(X+n-1)}$
		  en posant $Q(X) = P(X+1) - P(X)$, un polynôme qui a pour coefficient dominant $n a_n$ et pour degré $\deg Q = n - 1$.

% ----------------- %

	\item Il est alors facile de faire une récurrence pour prouver que la boucle se finit en fournissant ce qui est annoncé.  
\end{enumerate}

Joli, efficace et éclairant sur le pourquoi du comment. Il se trouve que l'algorithme \ref{nat-algo} cache aussi une formule combinatoire très intéressante comme nous allons le voir dans les sections à venir.
L'exposé qui suit reprend la démarche proposée sur le site \href{https://mathlesstraveled.com}{The Math Less Traveled}
\footnote{
	Chercher les articles \emph{\og A combinatorial proof \fg} publiés courant septembre 2019.
} 
