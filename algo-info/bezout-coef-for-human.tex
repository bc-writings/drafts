\documentclass{amsart}

\usepackage[T1]{fontenc}
\usepackage[utf8]{inputenc}

\usepackage[fontsize=13]{fontsize}
 
\usepackage[top=1.95cm, bottom=1.95cm, left=2.35cm, right=2.35cm]{geometry}


\usepackage{hyperref}
\usepackage{enumitem}
\usepackage{tcolorbox}
\usepackage{cleveref}
\usepackage{multicol}

\usepackage[french]{babel}
\usepackage[
    type={CC},
    modifier={by-nc-sa},
    version={4.0},
]{doclicense}

\usepackage{tikz}
\usetikzlibrary{arrows, matrix, positioning,fit}


\usepackage{tnsmath}
\usepackage{tnsalgo}


\newtheorem{fact}{Fait}
\newtheorem*{fact*}{Fait}

\newtheorem{example}{Exemple}

\newtheorem{remark}{Remarque}
\newtheorem*{remark*}{Remarque}

\newtheorem*{proof*}{Preuve}

\setlength\parindent{0pt}


% -- NEW ENVIRONMENTS -- %




% -- NEW MACROS -- %

\newcommand\ph{\phantom{x}}

\newcommand\myquote[1]{\emph{\og #1 \fg}}


\newcommand\bb{Bachet-Bézout}
\newcommand\algoeucl{algorithme d'Euclide}


% Arguments
%    #1 Caption
%    #2 TiKZ file

\newcommand\showstep[2]{
    \begin{center}
        \input{bezout-coef-for-human/#2.tkz}
    
        \vfill
        
        \small \itshape #1
    \end{center}    
}


\newcommand\showstepnovfill[2]{
    \begin{center}
        \input{bezout-coef-for-human/#2.tkz}
  
        \small \itshape #1
    \end{center}    
}


% -- SETTINGS -- %

\tikzset{
    % Good spacing hack (the ghost mode)
     gs/.style={
        rectangle,
        thick,
        text width=3.5em,
        align=center,
        draw=white,
        rounded corners,
        minimum height=2em
    }, 
	% Common
    rc/.style={rectangle,
        thick,rounded corners,draw,
        minimum height=2em},
    % Operator in a circle
    oc/.style={
        circle,
        draw,
        fill=white,
        inner sep=1pt, 
        outer sep=1pt
    },
    % Blue frame
    bf/.style={
        rc,
        align=center,
        draw=blue,
        fill=blue!20,
        text width=3.5em,
    },
    % Explain remained
    er/.style={
        rc,
        align=center,
        text width=3.5em,
        draw=gray,
        text=darkgray
    },
    % Explain activated
    ea/.style={
        rc,
        align=center,
        text width=3.5em,
        draw=gray,
        text=darkgray
    },
    % Long explain activated
    le/.style={
        rc,
        align=center,
        text width=8em,
        draw=gray,
        text=darkgray
    },
    % Blur effect
    be/.style={
        rectangle,
        thick,rounded corners,
        minimum height=2em,
        align=center,
        opacity = 0.3,
        text width=3.5em,
    },
    % Focus effect
    fe/.style={
        rc,
        align=center,
        text width=8em,
        draw=red,
        text=magenta
    },
    % Focus effect Bis
    feb/.style={
        rc,
        align=center,
        text width=8em,
        draw=blue,
        text=violet
    },
    % Long focus effect
    lfe/.style={
        rc,
        align=center,
        text width=12em,
        draw=red,
        text=magenta
    },
    % Long focus effect bis
    lfeb/.style={
        rc,
        align=center,
        text width=12em,
        draw=blue,
        text=violet
    },
}
      

     
     
\begin{document}

\title{BROUILLON - Des coefficients de Bachet-Bézout pour les humains}
\author{Christophe BAL}
\date{10 Sept. 2019 - 21 Sept. 2019}

\maketitle

\begin{center}
    \itshape
    Document, avec son source \LaTeX, disponible sur la page
    
    \url{https://github.com/bc-writing/drafts}.
\end{center}


\bigskip


\begin{center}
    \hrule\vspace{.3em}
    {
        \fontsize{1.35em}{1em}\selectfont
        \textbf{Mentions \og légales \fg}
    }
            
    \vspace{0.45em}
    \doclicenseThis
    \hrule
\end{center}


\vspace{4em}


\setcounter{tocdepth}{2}
\tableofcontents


% --------------- %


\newpage
\section{Où allons-nous ?}

%\newpage
\section{????}

????


% --------------- %


\newpage
\section{L'algorithme \og human friendly \fg{} appliqué de façon magique}

Dans la section \ref{elementary-proof}, nous avons vu que la clé de la réussite de l'algorithme de descente et remontée est l'égalité $aY - bX = bZ - rY$ dans la représentation ci-dessous où $a = qb + r$ est la division euclidienne standard et $X = qY + Z$. 

\showstepnovfill{Calculs faits dans les deux phases.}{tikz/why/twophases-focus-short}


\medskip


Nous avons donc exhibé un invariant et dès que l'on arrive à $r = 0$, c'est à dire à la fin de la phase de descente, nous pouvons avoir $bZ - rY = \pgcd(a ; b)$ grâce au choix $Z = 1$, et du coup en remontant les calculs nous arrivons à nos fins \emph{(au signe près)}.


\medskip


La méthode précédente est peu efficace à cause de la nécessité de mémoriser certains calculs pour la phase de remontée. Ceci est une contrainte forte !
Nous allons essayer de nous passer de cette nécessité de mémoriser des choses.
Pour cela repartons de la représentation symbolique \myquote{complète} ci-dessous où $r_{n+1} = 0$ et $r_n = \pgcd(a ; b)$.

\showstepnovfill{Représentation symbolique au complet où $r_{n+1} = 0$.}{tikz/why/twophases-all-no-Z0}


\medskip


Ce qui fait fonctionner l'algorithme de descente puis remontée c'est que les $r_k$ et les $Z_k$ vérifient la même relation de récurrence.

\begin{enumerate}
	\item $r_{k+2} = r_k - q_{k+1} r_{k+1}$ car $r_k = q_{k+1} r_{k+1} + r_{k+2}$ est la division euclidienne standard.

	\item $Z_{k+2} = Z_k - q_{k+1} Z_{k+1}$ soit $Z_k = q_{k+1} Z_{k+1} + Z_{k+2}$ par définition.
\end{enumerate}


Nous allons essayer de construire deux suites $(u_k)$ et $(v_k)$ telles que $a u_k + b v_k = r_k$ car nous aurons alors la relation de \bb{} $a u_n + b v_n = r_n = \pgcd(a;b)$.
Étant donné ce qui précède, il est maintenant naturel de supposer que $u_{k+2} = u_k - q_{k+1} u_{k+1}$ et $v_{k+2} = v_k - q_{k+1} v_{k+1}$.
En effet, ceci nous donne :
\begin{flalign*}
	a u_{k+2} + b v_{k+2} 
		&= a(u_k - q_{k+1} u_{k+1}) + b(v_k - q_{k+1} v_{k+1}) &\\
		&= a u_k + b v_k  - q_{k+1} (a u_{k+1} + b v_{k+1})    &\\
		&= r_k  - q_{k+1} r_{k+1}    &\\
		&= r_{k+2}    
\end{flalign*}


Il nous reste à trouver les valeurs initiales. Ceci est immédiat puisque nous avons :

\begin{enumerate}
	\item $a u_0 + b v_0 = r_0 = a$ donc $(u_0 ; v_0) = (1 ; 0)$ s'impose.

	\item $a u_1 + b v_1 = r_1 = b$ donc $(u_1 ; v_1) = (0 ; 1)$ s'impose.
\end{enumerate}


Notons que nécessairement $n \geq 1$. Nous voilà prêts à proposer un algorithme classique et efficace pour déterminer des coefficients de \bb.


{\small
\begin{algo}[frame]
	\caption{Classique et efficace} \label{algo-efficient}
	%%%
    \Data{$(a ; b) \in \NNs \!\times \NNs$ avec $a \geq b$}
    \Result{$(u ; v) \in \ZZ \!\times \ZZ$ tel que $au + bv = \pgcd(a ; b)$}
	\BlankLine
    \Actions{
		$u^{\prime} \Store 1$
		\\
		$u^{\prime\prime} \Store 0$
		\\
    	$v^{\prime} \Store 0$
		\\
		$v^{\prime\prime} \Store 1$
		\\
		\BlankLine
        \While{$b \neq 0$}{
			$a = q b + r$ est la division euclidienne standard.
			\\
			$temp_u \Store u^{\prime} - q u^{\prime\prime}$
			\\
			$u^{\prime} \Store u^{\prime\prime}$
			\\
			$u^{\prime\prime} \Store temp_u$
			\\
			$temp_v \Store v^{\prime} - q v^{\prime\prime}$
			\\
			$v^{\prime} \Store v^{\prime\prime}$
			\\
			$v^{\prime\prime} \Store temp_v$
		}
		\Return{$(u^{\prime} ; v^{\prime})$}
	}
\end{algo}
}



% --------------- %


\newpage
\section{Pourquoi cela marche-t-il ?}

\input{bezout-coef-for-human/why}


% --------------- %


\newpage
\section{Des coefficients via des algorithmes programmables}

\subsection{La version \og humaine \fg{} à la main} 

Il n'est pas dur de coder directement la méthode humaine par descente puis remontée
\footnote{
	Sur le lieu de téléchargement du document que vous lisez, se trouvent les fichiers \texttt{down.py} et \texttt{up.py} dans le dossier \texttt{bezout-coef-for-human/euclid2tikz}.
	Ces codes traduisent directement la méthode à la main par descente puis remontée.
}.
Voici un algorithme, peu efficace mais instructif, où $\star$ est un symbole à part, $R[-1]$ le dernier élément de la liste $R$ et $R[-2]$ l'avant-dernier, et enfin $[x, y] + [r, s, t] \eq[def] [x, y, r, s, t]$ \emph{(additionner des listes c'est les concaténer et donc $R + [r]$ est un raccourci pour \emph{\og on ajoute l'élément $r$ à droite de la liste $R$ \fg})}.


{\small
\begin{algo}[frame]
	\caption{Descente et remontée avec du papier et un crayon} \label{algo-human-paper}
	%%%
    \Data{$(a ; b) \in \NNs \!\times \NNs$ avec $a \geq b$}
    \Result{$(u ; v) \in \ZZ \!\times \ZZ$ tel que $au + bv = \pgcd(a ; b)$}
	\BlankLine
    \Actions{
    	\Comment{Phase de descente}
        \Comment{$Q$ est une liste qui va stocker les quotients entiers $q_k$.}
        \Comment{$R$ est une liste qui va stocker les restes $r_k$ (rappelons que}
        \Comment{$r_0 = a$ et $r_1 = b$).}
        \BlankLine
        $Q \Store [\star]$
        \\
        $R \Store [a , b]$
        \\
        \BlankLine
        \While{$R[-1] \neq 0$}{
        	$\alpha \Store R[-2]$
			\\
        	$\beta \Store R[-1]$
			\\
			\BlankLine
			$\alpha = q \beta + r$ est la division euclidienne standard.
			\\
			\BlankLine
			$Q \Store Q + [q]$
			\\
			$R \Store R + [r]$
		}
		\BlankLine
    	\Comment{Phase de remontée}
        \Comment{$Z$ est une liste qui va stocker les entiers tout à droite.}
        \BlankLine
		$\varepsilon \Store 1$
		\\
		$Z \Store [1 , 0]$
		\\
		$c \Store (-1)$
		\\
        \BlankLine
        \While{$Q[c] \neq \star$}{
        	$z \Store Q[c] \cdot Z[-2] + Z[-1]$
			\\
			$Z \Store Z + [z]$
			\\
			\BlankLine
        	$\varepsilon \Store (- \varepsilon)$
			\\
			$c \Store c - 1$
		}
		\BlankLine
    	\Comment{On gère le signe devant le $\pgcd$ grâce à $\varepsilon$.}
        \BlankLine
		$u \Store \varepsilon \cdot Z[-2]$
		\\
		$v \Store (- \varepsilon \cdot Z[-1])$
		\\
		\BlankLine
		\Return{$(u ; v)$}
	}
\end{algo}
}

Nous avons traduit brutalement ce que l'on fait humainement mais à bien y regarder, la seule liste dont nous avons réellement besoin est $Q$. 
On peut donc proposer la variante suivante programmable qui est à la fois proche de la version de descente et remontée tout en limitant l'impact sur la mémoire.


{\small
\begin{algo}[frame]
	\caption{Descente et remontée moins mémophage} \label{algo-human-paper-bis}
	%%%
    \Data{$(a ; b) \in \NNs \!\times \NNs$ avec $a \geq b$}
    \Result{$(u ; v) \in \ZZ \!\times \ZZ$ tel que $au + bv = \pgcd(a ; b)$}
	\BlankLine
    \Actions{
    	\Comment{Phase de descente}
        \BlankLine
        $Q \Store [\star]$
        \\
        \BlankLine
        \While{$b \neq 0$}{
			$a = q b + r$ est la division euclidienne standard.
			\\
			$Q \Store Q + [q]$
			\\
			$a \Store b$
			\\
			$b \Store r$
		}
		\BlankLine
    	\Comment{Phase de remontée}
        \BlankLine
        $u \Store 1$
		\\
        $v \Store 0$
        \\
		$\varepsilon \Store 1$
		\\
		$c \Store (-1)$
		\\
        \BlankLine
        \While{$Q[c] \neq \star$}{
        	$temp \Store Q[c] v + u$
			\\
			$u \Store v$
			\\
			$v \Store temp$
			\\
			\BlankLine
        	$\varepsilon \Store (- \varepsilon)$
			\\
        	$c \Store c - 1$
		}
		\BlankLine
		$u \Store \varepsilon \cdot u$
		\\
		$v \Store (- \varepsilon \cdot v)$
		\\
		\BlankLine
		\Return{$(u ; v)$}
	}
\end{algo}
}



% --------------- %


\newpage
\subsection{La version \og humaine \fg{} via les matrices} 

Voici la version matricielle de l'algorithme de remontée et descente où de nouveau on limite l'impact sur la mémoire. La matrice $R$ correspond au produit cumulé des matrices $R_k$.


{\small
\begin{algo}[frame]
	\caption{Descente et remontée via les matrices} \label{algo-human-matrix}
	%%%
    \Data{$(a ; b) \in \NNs \!\times \NNs$ avec $a \geq b$}
    \Result{$(u ; v) \in \ZZ \!\times \ZZ$ tel que $au + bv = \pgcd(a ; b)$}
	\BlankLine
    \Actions{
		$a_0 \Store a$
		\\
		$\varepsilon \Store 1$
		\\
    	$R \Store \begin{pmatrix}
					1 & 0 \\
					0 & 1
		  		  \end{pmatrix}$
        \\
        \BlankLine
        \While{$b \neq 0$}{
			$a = q b + r$ est la division euclidienne standard.
			\\
			$R \Store R \cdot \begin{pmatrix}
								 q & 1 \\
								 1 & 0
		  		  			  \end{pmatrix}$
			\\
			$a \Store b$
			\\
			$b \Store r$
			\\
			\BlankLine
        	$\varepsilon \Store (- \varepsilon)$
		}
		$R \Store R \cdot \begin{pmatrix}
							 a_0 & 0           \\
							 0   & \varepsilon
		  		  		  \end{pmatrix}$
		\\
		$u \Store R_{22}$
		\\
		$v \Store (- R_{12})$
		\\
		\BlankLine
		\Return{$(u ; v)$}
	}
\end{algo}
}




% --------------- %


\subsection{Tailles des coefficients lors de la remontée} \label{human-size}

Nous redonnons la représentation symbolique complète, ceci afin de rappeler les notations utilisées. 

\showstepnovfill{Représentation symbolique au complet.}{tikz/why/twophases-all}


\medskip


Pour $k \in \ZintervalC{1}{n}$, nous savons que $Z_{k-1} = q_k Z_k + Z_{k+1}$ avec $(Z_{n} ; Z_{n+1}) = (0 ; 1)$, et aussi que $r_{k-1} = q_k r_k + r_{k+1}$ avec $(r_{n} ; r_{n+1}) = (\pgcd(a;b) ; 0)$.


\vspace{-.25em}
\begin{itemize}[label = \small\textbullet]
	\item Le cas minimal est $n = 1$ puisque $1 \leq[hyp] b \leq[hyp] a$ \emph{(en fait $n = 1$ correspond au cas où $b \;|\, a$)}.
	Nous devons donc calculer au moins un quotient $q_k$.


	\item Nous avons clairement $0 \leq Z_n < r_n$.


	\item
	Comme $Z_{n-1} = q_n Z_n + Z_{n+1} = 1$, $r_{n-1} = q_n r_n + r_{n+1} = q_n r_n$ et $r_n < r_{n-1}$ par définition de la division euclidienne standard, nous avons $1 \leq Z_{n-1} < r_{n-1}$. Notons au passage que $q_n \geq 2$.
	
	\item Supposons maintenant que $n > 1$. Comme $Z_{n-2} = q_{n-1} Z_{n-1} + Z_{n}$, il est clair que $Z_{n-2} \geq 1$.
	De plus, nous avons :
	
	\smallskip
	
	\noindent
	$Z_{n-2}
	 = q_{n-1} Z_{n-1} + Z_{n}$

	\noindent
	$\phantom{Z_{n-2}}
	 < q_{n-1} r_{n-1} + r_n$

	\noindent
	$\phantom{Z_{n-2}}
	 = r_{n-2}$
	 
	
	\item Une récurrence descendante finie nous donne que $\forall k \in \ZintervalC{0}{n}$, $0 \leq Z_k < r_k$.
\end{itemize}


\medskip


D'après ce qui précède et comme de plus la suite finie $(r_k)_{0 \leq k \leq n+1}$ est décroissante, nous savons que $\forall k \in \ZintervalC{1}{n}$, $0 \leq Z_k < r_1 = b$ et $0 \leq Z_0 < r_0 = a$.
En particulier $(u ; v) = \pm (Z_1 ; -Z_0)$ qui est tel que $au + bv = \pgcd(a;b)$ vérifie aussi $0 \leq \abs{u} < b$ et $0 \leq \abs{v} < a$.


\begin{remark}
	Le résultat précédent empêche toute explosion en taille des calculs intermédiaires. Ceci est une très bonne chose !
\end{remark}

\begin{remark}
	Il n'est pas dur de vérifier que la suite $(Z_k)_{0 \leq k \leq n}$ est strictement décroisante.
\end{remark}


\begin{remark}
	Notant $d = \pgcd(a ; b)$, nous avons en fait $0 \leq \abs{u} < \frac{b}{2d}$ et $0 \leq \abs{v} < \frac{a}{2d}$, et plus généralement  $\forall k \in \ZintervalC{1}{n}$, $0 \leq Z_k < \frac{b}{2d}$.
	Ceci vient des deux constatations suivantes.
	
	\begin{enumerate}
		\item Tout d'abord en notant que $r_n \leq \frac{1}{2} r_{n-1}$, nous avons $0 \leq \abs{u} < \frac{b}{2}$ et $0 \leq \abs{v} < \frac{a}{2}$.


		\item Posons $a^\prime = \frac{a}{d}$ et $b^\prime = \frac{b}{d}$.
		Les suites $(q^\prime_k)_k$ et $(r^\prime_k)_k$ associées à $a^\prime$ et $b^\prime$ sont tout simplement $(q_k)_k$ et $\left( \frac{r_k}{d} \right)_k$, la deuxième suite n'étant pas utilisée pour la phase de remontée.
		Pour comprendre il faut commencer par noter que si $a = bq + r$ désigne la division euclidienne standard alors $\frac{a}{d} = \frac{b}{d}q + \frac{r}{d}$ en est aussi une.
	\end{enumerate}
\end{remark}





% --------------- %


\subsection{Pas terribles\dots{}} 

\input{bezout-coef-for-human/algo-human/not-so-good}




% --------------- %


\newpage
\section{Un algorithme classique bien plus efficace}

\input{bezout-coef-for-human/algo-efficient}


% --------------- %


\newpage
\section{Une infinité de coefficients de \bb}

\input{bezout-coef-for-human/coef-infinity}


% --------------- %


\newpage
\section{Nombre d'étapes de l'algorithme d'Euclide} \label{euclide-complexity}

\input{bezout-coef-for-human/euclide-complexity}


\end{document}
