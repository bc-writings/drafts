En fait, il est assez facile d'améliorer l'estimation précédente. Prenons le schéma suivant où les deux colonnes de droite donnent des minorants évidents des restes fournis par l'\algoeucl. Les deux colonnes de droite utilisent la construction via des divisions euclidiennes \myquote{inversées} avec des quotients les plus petits possibles, à savoir tous égaux à $1$ \emph{(voir la colonne tout à droite)}. 

\showstepnovfill{Estimer au mieux le nombre d'étapes de l'\algoeucl{}.}{tikz/why/algo-euclide-fibonacci}


\medskip


Pour les fans de Nicolas B.
\footnote{
	Alias Nicolas Bourbaki.
},
voici une démonstration formelle de toutes les minorations. Tout d'abord par définition, la suite $(f_k)_{0 \leq k \leq n}$ est définie par la condition initiale $(f_0 ; f_1) = (0 ; 1)$ puis $f_2 = 2 f_1 + f_0$ et la relation de récurrence $f_{k+2} = f_{k+1} + f_k$ pour $k \in \ZintervalC{3}{n-2}$. Démontrons par récurrence sur $k \in \ZintervalC{0}{n}$ que $f_k \leq r_{n+1-k}$ en nous souvenant que $n \geq 1$. L'hypothèse de récurrence sera que l'inégalité est vérifiée pour tous les indices $i$ tels que $i \leq k$.

\begin{itemize}[label=\small\textbullet]
	\item \emph{Cas de base pour $k \leq 2$.}
		  Il est clair que $f_0 \leq r_{n+1}$ et  $f_1 \leq r_{n}$.
		  Ensuite $r_{n-1} = q_n r_n + r_{n+1}$ avec $q_n \geq 2$ donne $r_{n-1} \geq 2 r_n + r_{n+1} \geq 2 f_1 + f_0 = f_2$. Ceci achève la preuve des cas de base.
	
	
	\item \emph{Hérédité pour $k \in \ZintervalC{3}{n-3}$.} 
		  Nous avons $r_{n+1 - (k+1)} = r_{n-k} = q_{n-k+1} r_{n-k+1} + r_{n-k+2}$ avec $q_n \geq 1$.
		  Or $n-k+1 = n+1 - k$ et $n-k+2 = n+1 - (k-1)$ donc l'hypothèse de récurrence et $q_n \geq 1$ donnent $r_{n+1 - (k+1)} \geq r_{n-k+1} + r_{n-k+2} \geq f_k + f_{k-1} = f_{k+1}$, la dernière égalité venant de $k \geq 3$.
		  Ceci établit bien l'inégalité au rang $(k+1)$ et donc pour tous les rangs $i$ tels que $i \leq k+1$.
\end{itemize}


\medskip


Poursuivons en notant que le 1\ier{} terme $f_0$ de la suite $f$ ne jouera pas un rôle particulier dans l'évaluation du nombre d'étapes.
Ceci permet de remplacer la suite $f$ par la bien connue et très classique suite de Fibonacci
\footnote{
	On entend souvent à tors dire que la suite de Fibonacci donne la complexité au pire de l'algorithme d'Euclide. Ce n'est pas exactement vrai à cause du tout dernier reste nul.
}
$F$ définie par les conditions initiales $F_0 = F_1 = 1$ et la relation de récurrence $F_{k+2} = F_{k+1} + F_k$ puisque $\forall k \in \ZintervalC{1}{n}$, $F_k = f_k$.


\medskip


Nous pouvons donc affirmer que $n \leq \max \setgene{k \in \NNs \,|\, F_k \leq b}$. Le cas où $a = f_n + f_{n-1}$ et $b = f_n$ montre que l'on peut pas espérer faire mieux ! Nous allons estimer ce maximum de deux façons.


% --------------- %


\bigskip


\emph{\bfseries Méthode 1.}
%%
Considérons les suites non nulles du type $(q^k)_{k \in \NN}$ telles que $q^{n+2} = q^{n+1} + q^n$.
Il est facile de montrer qu'il n'y en a que de deux types, à savoir les suites $(\phi^k)_{k \in \NN}$ et $(\psi^k)_{k \in \NN}$ où $\psi = \frac{1 - \sqrt{5}}{2}$ et $\phi = \frac{1 + \sqrt{5}}{2}$, le nombre d'or, sont les deux solutions de l'équation $x^2 = x +1$. Nous avons alors les faits suivants.

\begin{enumerate}
	\item $F_0 = F_1 = 1 = \phi^0$

	\item $F_2 = 2 = \frac{1 + \sqrt{9}}{2} > \phi$

	\item $F_3 = F_2 + F_1$ > $\phi^1 + \phi^0 = \phi^2$

	\item $F_4 = F_3 + F_2$ > $\phi^2 + \phi^1 = \phi^3$ \dots
\end{enumerate}


Une récurrence immédiate à faire nous donne $\forall k \in \NNs$, $F_k > \phi^{k-1}$ de sorte que $F_n \leq b$ implique $\phi^{n-1} < b$ puis $n < 1 + \frac{\log b}{\log \phi}$.
Notant $d$ le nombre de chiffres décimaux de $b$, nous avons $n < 1 + \frac{d}{\log \phi}$. 
Ensuite $\frac{1}{\log \phi} \approx 4,78$ donne $n < 1 + 5d$ puis $n \leq 5d$ ce qui est un peu mieux que la première estimation $n < 7d$.


\begin{remark}
	On peut démontrer que $\forall k \in \NN$, $F_k = \frac{\phi^{k+1} - \psi^{k+1}}{\sqrt{5}}$.
	Pour cela, on prouve l'existence de $(m ;p) \in \RR^2$ tel que la suite $u$ de terme générale $u_k = m \phi^k  + p \psi^k$ vérifie $u_0 = u_1 = 1$. Il est alors facile de conclure.
\end{remark}


% --------------- %


\bigskip


\emph{\bfseries Méthode 2.}
%%
Pauvres de nous qui ne connaissons pas le nombre d'or.
Que faire ? Examinons les $24$ premières valeurs
\footnote{
	Les valeurs ont été fournies par le fichier \texttt{explore.py} disponible  sur le lieu de téléchargement du document que vous lisez : voir le dossier \texttt{bezout-coef-for-human/fibo}.
}
de la suite $F$.

\begin{multicols}{3}
    $F_{0} = 1$

    $F_{1} = 1$

    $F_{2} = 2$

    $F_{3} = 3$

    $F_{4} = 5$

    $F_{5} = 8$

    $F_{6} = 13$

    $F_{7} = 21$

    $F_{8} = 34$

    $F_{9} = 55$

    $F_{10} = 89$

    $F_{11} = 144$

    $F_{12} = 233$

    $F_{13} = 377$

    $F_{14} = 610$

    $F_{15} = 987$

    $F_{16} = 1597$

    $F_{17} = 2584$

    $F_{18} = 4181$

    $F_{19} = 6765$

    $F_{20} = 10946$

    $F_{21} = 17711$

    $F_{22} = 28657$

    $F_{23} = 46368$
\end{multicols}


Un peu d'observation montre que $\forall k \in \NNs$, $F_{k + 5} > 10 F_k$ semble être vraie. Une telle inégalité a l'utilité de nous donner une information sur la taille décimale des $F_k$.
Cette conjecture se consolide facilement via un programme informatique
\footnote{
	Voir le fichier \texttt{conjecture.py} dans le dossier \texttt{bezout-coef-for-human/fibo} présent sur le lieu de téléchargement du document que vous lisez.
}.
Il nous reste à la vérifier à l'aide d'un raisonnement direct.

\begin{itemize}[label=\small\textbullet]
	\item \emph{1\ier{} cas : $k = 1$.}
	Nous avons bien $10 F_1 = 10 < 13 = F_6$.
	
	
	\item \emph{2\ieme{} cas : $k \geq 2$.}
	Nous avons ici :
	
	\noindent
	$F_{k + 5} = F_{k + 4} + F_{k + 3}$
	
	\noindent
	$\phantom{F_{k + 5}} = 2 F_{k + 3} + F_{k + 2}$

	\noindent
	$\phantom{F_{k + 5}} = 3 F_{k + 2} + 2 F_{k + 1}$

	\noindent
	$\phantom{F_{k + 5}} = 5 F_{k + 1} + 3 F_k$

	\noindent
	$\phantom{F_{k + 5}} = 8 F_k + 5 F_{k - 1}$ \quad \emph{\small(en se souvenant que $k \geq 2$)}
	
	\noindent
	Par décroissance de la suite $F$ et comme $k \geq 2$, nous avons $F_k = F_{k - 1} + F_{k - 2} \leq 2 F_{k - 1}$.
	Comme de plus $F_{k - 1} > 0$ puisque $k \geq 2$, nous obtenons :

	\noindent
	$F_{k + 5} = 8 F_k + 4 F_{k - 1} + F_{k - 1}$

	\noindent
	$\phantom{F_{k + 5}} > 8 F_k + 4 F_{k - 1}$

	\noindent
	$\phantom{F_{k + 5}} \geq 8 F_k + 2 F_k$

	\noindent
	$\phantom{F_{k + 5}} = 10 F_k$
\end{itemize}


Comme $F_1 = 1$, le résultat précédent nous donne que $\forall j \in \NN$, $\forall i \in \ZintervalC{1}{5}$, $10^j \leq F_{i + 5j}$.
Autrement dit, $F_{i + 5j}$ s'écrit avec au moins $(j + 1)$ chiffres décimaux.
Ceci se démontre via la récurrence facile suivante sur $j$ avec $i$ fixé.

\begin{itemize}[label=\small\textbullet]
	\item \emph{Cas de base pour $j = 0$.}
		  Nous avons bien $10^0 = 1 \leq F_{i}$ si $i \in \ZintervalC{1}{5}$.
	
	
	\item \emph{Hérédité pour $j \in \NN$.} 
		  Supposons avoir $10^j \leq F_{i + 5j}$, nous avons alors comme souhaité :
		  $F_{i + 5(j+1)} = F_{(i + 5j) + 5} > 10 F_{i + 5j} \geq 10^{j+1}$.
\end{itemize}


Notant $d$ le nombre de chiffres décimaux de $b$, comme $b < 10^d$ et $F_n \leq b$, nous avons $F_n < 10^d$.
Soit $n = i + 5j$ la division euclidienne de $n$ par $5$, nous avons alors $10^j < 10^d$ puis $j < d$ et $n < i + 5d \leq 5(d+1)$ d'où $n \leq 5d + 4$. C'est moins bien que le résultat de la méthode 1.


\medskip


En fait, on peut améliorer l'estimation en raisonnant comme suit
\footnote{
	L'auteur a fait le choix de laisser l'estimation un peu grossière $n \leq 5d + 4$ en raisonnant comme s'il ne connaissait pas le résultat de la méthode 1 afin de rendre la méthode 2 auto-suffisante.
	Le fait que l'on puisse améliorer l'estimation grossière peut se conjecturer par des expériences numériques informatiques.
}.
Si nous avions $n > 5d$, soit $n \geq 5d+1$, alors par croissance de la suite $F$, nous aurions $F_n \geq F_{5d+1} \geq 10^d$ qui contredirait $F_n < 10^d$.
