Il n'est pas dur de coder directement la méthode humaine par descente puis remontée
\footnote{
	Sur le lieu de téléchargement du document que vous lisez, se trouvent les fichiers \texttt{down.py} et \texttt{up.py} dans le dossier \texttt{bezout-coef-for-human/euclid2tikz}.
	Ces codes traduisent directement la méthode à la main par descente puis remontée.
}.
Voici un algorithme, peu efficace mais instructif, où $\star$ est un symbole à part, $R[-1]$ le dernier élément de la liste $R$ et $R[-2]$ l'avant-dernier, et enfin $[x, y] + [r, s, t] \eq[def] [x, y, r, s, t]$ \emph{(additionner des listes c'est les concaténer et donc $R + [r]$ est un raccourci pour \emph{\og on ajoute l'élément $r$ à droite de la liste $R$ \fg})}.


{\small
\begin{algo}[frame]
	\caption{Descente et remontée avec du papier et un crayon} \label{algo-human-paper}
	%%%
    \Data{$(a ; b) \in \NNs \!\times \NNs$ avec $a \geq b$}
    \Result{$(u ; v) \in \ZZ \!\times \ZZ$ tel que $au + bv = \pgcd(a ; b)$}
	\BlankLine
    \Actions{
    	\Comment{Phase de descente}
        \Comment{$Q$ est une liste qui va stocker les quotients entiers $q_k$.}
        \Comment{$R$ est une liste qui va stocker les restes $r_k$ (rappelons que}
        \Comment{$r_0 = a$ et $r_1 = b$).}
        \BlankLine
        $Q \Store [\star]$
        \\
        $R \Store [a , b]$
        \\
        \BlankLine
        \While{$R[-1] \neq 0$}{
        	$\alpha \Store R[-2]$
			\\
        	$\beta \Store R[-1]$
			\\
			\BlankLine
			$\alpha = q \beta + r$ est la division euclidienne standard.
			\\
			\BlankLine
			$Q \Store Q + [q]$
			\\
			$R \Store R + [r]$
		}
		\BlankLine
    	\Comment{Phase de remontée}
        \Comment{$Z$ est une liste qui va stocker les entiers tout à droite.}
        \BlankLine
		$\varepsilon \Store 1$
		\\
		$Z \Store [1 , 0]$
		\\
		$c \Store (-1)$
		\\
        \BlankLine
        \While{$Q[c] \neq \star$}{
        	$z \Store Q[c] \cdot Z[-2] + Z[-1]$
			\\
			$Z \Store Z + [z]$
			\\
			\BlankLine
        	$\varepsilon \Store (- \varepsilon)$
			\\
			$c \Store c - 1$
		}
		\BlankLine
    	\Comment{On gère le signe devant le $\pgcd$ grâce à $\varepsilon$.}
        \BlankLine
		$u \Store \varepsilon \cdot Z[-2]$
		\\
		$v \Store (- \varepsilon \cdot Z[-1])$
		\\
		\BlankLine
		\Return{$(u ; v)$}
	}
\end{algo}
}

Nous avons traduit brutalement ce que l'on fait humainement mais à bien y regarder, la seule liste dont nous avons réellement besoin est $Q$. 
On peut donc proposer la variante suivante programmable qui est à la fois proche de la version de descente et remontée tout en limitant l'impact sur la mémoire.


{\small
\begin{algo}[frame]
	\caption{Descente et remontée moins mémophage} \label{algo-human-paper-bis}
	%%%
    \Data{$(a ; b) \in \NNs \!\times \NNs$ avec $a \geq b$}
    \Result{$(u ; v) \in \ZZ \!\times \ZZ$ tel que $au + bv = \pgcd(a ; b)$}
	\BlankLine
    \Actions{
    	\Comment{Phase de descente}
        \BlankLine
        $Q \Store [\star]$
        \\
        \BlankLine
        \While{$b \neq 0$}{
			$a = q b + r$ est la division euclidienne standard.
			\\
			$Q \Store Q + [q]$
			\\
			$a \Store b$
			\\
			$b \Store r$
		}
		\BlankLine
    	\Comment{Phase de remontée}
        \BlankLine
        $u \Store 1$
		\\
        $v \Store 0$
        \\
		$\varepsilon \Store 1$
		\\
		$c \Store (-1)$
		\\
        \BlankLine
        \While{$Q[c] \neq \star$}{
        	$temp \Store Q[c] v + u$
			\\
			$u \Store v$
			\\
			$v \Store temp$
			\\
			\BlankLine
        	$\varepsilon \Store (- \varepsilon)$
			\\
        	$c \Store c - 1$
		}
		\BlankLine
		$u \Store \varepsilon \cdot u$
		\\
		$v \Store (- \varepsilon \cdot v)$
		\\
		\BlankLine
		\Return{$(u ; v)$}
	}
\end{algo}
}
