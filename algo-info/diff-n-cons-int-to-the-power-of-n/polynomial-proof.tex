Voici une légère modification de l'algorithme \ref{nat-algo} où l'on manipule des polynômes dans $\polyset{\RR}{X}$.
Notons au passage que nous passons d'un algorithme a priori indéterministe, car on fait un choix de naturels consécutifs, à un autre complètement déterministe.

\begin{algo}
	\caption{Version polynomiale} \label{nat-2-poly-algo}

	\In{$n \in \NNs$}
	\Out{?}
	
	\addalgoblank
	
	\Actions{
		$L \Store [ X^n , (X+1)^n , \dots , (X+n)^n ]$
		\\
		\addalgoblank
		\While{$taille(L) \neq 1$}{
			$newL \Store \EmptyList$
			\\
			\ForRange*{i}{1}{taille(L) - 1}{
				Ajouter $(L[i + 1] - L[i])$ à droite de $newL$.
			}
			$L \Store newL$
		}
		\addalgoblank
		\Return{$L[1]$}
	}
\end{algo}


Il est aisée d'implémenter cet algorithme en Python. Le code suivant ne révèle aucun test raté.

\medskip

\inputcode{python}{diff-n-cons-int-to-the-power-of-n/exploring-poly-version.py}


\medskip


Il est clair que si l'on prouve que l'algorithme \ref{nat-2-poly-algo} renvoie $n!$, il en sera de même pour \ref{nat-algo}. Nous allons en fait prouver la validité de l'algorithme ci-dessous où $\polyset{\RR_{n}}{X}$ désigne l'ensemble des polynômes réels de degré $n \in \NNs$.

\medskip

\begin{algo}
	\caption{Version polynomiale élargie} \label{gene-poly-algo}

	\In{$P \in \polyset{\RR_{n}}{X}$ de coefficient dominant $a_n$}
	\Out{$a_n \, n!$}
	
	\addalgoblank
	
	\Actions{
		$L \Store [ P(X) , P(X+1) , \dots , P(X+n) ]$
		\\
		\addalgoblank
		\While{$taille(L) \neq 1$}{
			$newL \Store \EmptyList$
			\\
			\ForRange*{i}{1}{taille(L) - 1}{
				Ajouter $(L[i + 1] - L[i])$ à droite de $newL$.
			}
			$L \Store newL$
		}
		\addalgoblank
		\Return{$L[1]$}
	}
\end{algo}


{\Huge PREUVE: TODO}


%
%We start with the list $[P(X) , P(X+1) , P(X+2) , ... , P(X+n)]$ with a polynomial $P$.
%
%The next list is $[P(X+1) - P(X) , P(X+2) - P(X+1) , P(X+3) - P(X+2) , ... , P(X+n) - P(X+n-1)]$ that is $[Q(X) , Q(X+1) , Q(X+2) , ... , Q(X+n-1)]$ with $Q(X) \eqdef P(X+1) - P(X)$.
%
%If $P(X) = a_n X^n + ...$ then $P(X+1) - P(X) = n a_n X^{n-1} + ...$ because $(X+1)^k - X^k = k X^{k-1} + ...$
%
%An easy recurrence gives the result by noting the decrease of the degree of the polynomials involved at each iteration.
%
%The advantage of this proof is that it show where the trick comes from.
%
