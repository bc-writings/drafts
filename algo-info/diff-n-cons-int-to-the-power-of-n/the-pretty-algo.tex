Considérons l'algorithme suivant dont nous allons donner des cas d'application juste après.

\begin{algo}
	\caption{Version naturelle} \label{nat-algo}

	\In{$n \in \NNs$}
	\Out{?}
	
	\addalgoblank
	
	\Actions{
		Choisir $(n+1)$ naturels consécutifs : $k_1 < k_2 < \dots < k_{n+1}$.
		\\
		$L \Store [ k_1^n , k_2^n , \dots , k_{n+1}^n ]$
		\\
		\addalgoblank
		\While{$taille(L) \neq 1$}{
			$newL \Store \EmptyList$
			\\
			\ForRange*{i}{1}{taille(L) - 1}{
				\Append{$newL$}{$(L[i + 1] - L[i])$}
			}
			$L \Store newL$
		}
		\addalgoblank
		\Return{$L[1]$}
	}
\end{algo}


Pour $n = 2$ avec $k_1 = 3$ , $k_2 = 4$ et $k_3 = 5$, nous avons les valeurs suivantes de la liste $L$.

\begin{enumerate}
	\item $L = [3^2 , 4^2 , 5^2] = [9 , 16 , 25]$

	\item $L = [16 - 9 , 25 - 16] = [7, 9]$

	\item $L = [9 - 7] = [2]$
\end{enumerate}


L'algorithme renvoie donc $2$ ici mais que se passe-t-il si l'on choisit d'autres naturels consécutifs ? Avec $k_1 = 10$ , $k_2 = 11$ et $k_3 = 12$, nous obtenons :

\begin{enumerate}
	\item $L = [10^2 , 11^2 , 12^2] = [100 , 121 , 144]$

	\item $L = [121 - 100 , 144 - 121] = [21 , 23]$

	\item $L = [23 - 21] = [2]$
\end{enumerate}


L'algorithme renvoie de nouveau $2$. Que se passerait-il pour d'autres triplets de naturels consécutifs ? Pour se faire une bonne idée, il va falloir utiliser un programme. Ceci étant dit nous allons toute de suite faire l'hypothèse audacieuse que le choix des naturels consécutifs n'est pas important. Regardons alors ce que renvoie l'algorithme pour $n = 3$.

\begin{enumerate}
	\item $L = [0^3 , 1^3 , 2^3 , 3^3] = [0 , 1 , 8 , 27]$

	\item $L = [1 , 7 , 19]$

	\item $L = [6, 12]$

	\item $L = [6]$
\end{enumerate}


L'algorithme renvoie $6$ pour $n = 3$, et pour $n= 4$ ce qui suit nous donne que $24$ est renvoyé. Ceci nous fait alors penser à $n !$ et donc nous amène à conjecturer, un peu rapidement c'est vrai, que l'algorithme va toujours renvoyé $n !$ .

\begin{enumerate}
	\item $L = [0^4 , 1^4 , 2^4 , 3^4, 4^4] = [0 , 1 , 16 , 81 ,256]$

	\item $L = [1 , 15 , 65 , 175]$

	\item $L = [14, 50 , 110]$

	\item $L = [36 , 60]$

	\item $L = [24]$
\end{enumerate}


Il est temps de passer aux choses un plus sérieuses via une expérimentaion informatique bien plus poussée.
