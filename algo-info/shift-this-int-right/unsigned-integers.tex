Dans ce document, nous allons travailler uniquement avec des entiers non signés, c'est à dire des naturels dont l'écriture en base 2 tient sur 8 bits même si les raisonnements tenus se généralisent à 16, 32 ou 64 bits. Ce choix tient juste à des contraintes de lisibilité du document.

\medskip

Voici des exemples de représentations de type little-endian : on part à gauche de la puissance la plus haute, pour nous ce sera toujours $2^7$, puis on va vers la plus basse \emph{(c'est le même principe que l'écriture des naturels en base $10$)}.

\medskip

\binaryplus{10000101}{$133 = 128 + 4 + 1 = 2^7 + 2^2 + 2^0$}

\medskip

\binaryplus{00100110}{$38 = 32 + 4 + 2 = 2^5 + 2^2 + 2^1$}

\medskip

Avec 8 bits on peut représenter les naturels de $0$ à $1 + 2 + 2^2 + \cdots + 2^7 = 2^8 - 1 = 255$.