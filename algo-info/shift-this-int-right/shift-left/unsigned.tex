Examinons ce qu'il se passe lorsque l'on calcule  $2 n$ si $n < 0$ est un naturel signé.
Voici un exemple sans dépassement de capacité.


\medskip

\binaryplus{U1010111}{$n_0 = -128 + 64 + 16 + 4 + 2 + 1 = -41$}

\medskip

\binaryplus{U010111z}{$n_1 = 2n_0 = -82 = -128 + 32 + 8 + 4 + 2$}

\medskip


On effectue un décalage à gauche de tous les bits, y compris celui du signe, comme avec les entiers non signés.
Mais est-ce toujours vrai ? Que se passe-t-il si le 2\ieme{} bit à gauche est nul ?
En fait, sous l'hypothèse de non dépassement de capacité, tout fonctionne sans souci comme nous allons l'expliquer proprement tout de suite.

\medskip

Le cas $n \in \intervalC{0}{127}$ étant évident, considérons $n \in \intervalC{-128}{-1}$ .
La représentation de $n$ s'obtient alors via
$\displaystyle n = -2^7 + \sum_{0 \leq k \leq 6} b_k \, 2^k$ avec chaque $b_k$ dans $\setgene{0 ; 1}$ .
Nous avons alors :

\medskip

$\displaystyle 2 n 
	= -2^8 + \sum_{0 \leq k \leq 6} b_k \, 2^{k+1}$

\smallskip

$\displaystyle 2 n 
	= -2^8 + b_6 \, 2^7 + \sum_{1 \leq k \leq 6} b_{k-1} \, 2^k$

\smallskip

$\displaystyle 2 n 
	= (b_6 - 2) \, 2^7 + \sum_{1 \leq k \leq 6} b_{k-1} \, 2^k$

\medskip

Si $b_6 = 1$ , nous avons bien le décalage annoncé. Supposons donc que $b_6 = 0$ . Dans ce cas, en notant $n = -2^7 + S$, nous obtenons :

\medskip

$\displaystyle S = \sum_{0 \leq k \leq 6} b_k \, 2^k$

\smallskip

$\displaystyle S = \sum_{0 \leq k \leq 5} b_k \, 2^k$

\smallskip

$\displaystyle S \leq \sum_{0 \leq k \leq 5} 2^k$

\smallskip

$\displaystyle S \leq 2^6 - 1$

\medskip

Nous en déduisons que $n = -2^7 + S \leq -2^7 + 2^6 - 1 = -2^6 - 1$ , puis ensuite $2n \leq -2^7 - 2 \, \text{<} \, -2^7$ ce qui est bien un cas de dépassement de capacité.
