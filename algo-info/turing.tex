% !TEX encoding = UTF-8 Unicode
\documentclass[a4paper, 12pt]{scrartcl}

\usepackage[utf8]{inputenc}
\usepackage[T1]{fontenc}

\usepackage[top=2.5cm, bottom=2.5cm, left=1.95cm, right=1.95cm]{geometry}

\usepackage[french]{babel}
\usepackage{listings}

\usepackage[fr, apmep]{lyxam}
\usepackage{lymath}
\usepackage{fancyvrb}
\usepackage{bera}
\usepackage{enumitem}
\usepackage{multicol}

\usepackage{lastpage}
\usepackage{xcolor}
\usepackage{graphicx}
\usepackage{setspace}

\usepackage{tikz}
\usetikzlibrary{arrows,positioning, calc} 


\usepackage[
    type={CC},
    modifier={by-nc-sa},
	version={4.0},
]{doclicense}


\newcommand\boxit[1]{\fbox{\makebox[.85em]{#1}\vphantom{$pX^M$}}}
\newcommand\fboxit[1]{\fcolorbox{black}{yellow}{\makebox[.85em]{#1}\vphantom{$pX^M$}}}
\newcommand\nboxit[1]{\fcolorbox{black}{lightgray}{\makebox[.85em]{#1}\vphantom{$pX^M$}}}
\newcommand\wboxit[1]{\fcolorbox{black}{red}{\makebox[.85em]{#1}\vphantom{$pX^M$}}}
\newcommand\noboxit[1]{\fcolorbox{white}{white}{\makebox[.85em]{#1}\vphantom{$pX^M$}}}

\newcommand\emptybox{\boxit{\phantom{A}}}

\newcommand\head{\noboxit{$\uparrow$}}


\begin{document}

\title{BROUILLON - Quelques machines de Turing}
\author{Christophe BAL}
\date{7 Février 2020}

\maketitle

\begin{center}
	\itshape
	Document, avec son source \LaTeX, disponible sur la page
	
	\url{https://github.com/bc-writing/drafts}.
\end{center}


\bigskip


\begin{center}
	\hrule\vspace{.3em}
	{
		\fontsize{1.35em}{1em}\selectfont
		\textbf{Mentions \og légales \fg}
	}
			
	\vspace{0.45em}
	\doclicenseThis
	\hrule
\end{center}


\bigskip
\setcounter{tocdepth}{2}
\tableofcontents



\newpage
\section{Détecter les palindromes}

	\subsection{Un exemple avec le mot abbca}

		Voici les grandes étapes présentées sur deux colonnes.


\begin{multicols}{2}

% GESTION DE LA LETTRE LA PLUS À DROITE

\emptybox\emptybox%
	\boxit{a}\boxit{b}\boxit{b}\boxit{c}\boxit{a}%
\emptybox\emptybox

\phantom{\emptybox\emptybox}%
	\head


\medskip % RECHERCHE DE LA DERNIÈRE LETTRE
\emptybox\emptybox%
	\fboxit{a}\boxit{b}\boxit{b}\boxit{c}\boxit{a}%
\emptybox\emptybox

\phantom{\emptybox\emptybox%
	\emptybox\emptybox\emptybox\emptybox}%
	\head


\medskip % BONNE LETTRE
\emptybox\emptybox%
	\fboxit{a}\boxit{b}\boxit{b}\boxit{c}\fboxit{a}%
\emptybox\emptybox

\phantom{\emptybox\emptybox%
	\emptybox\emptybox\emptybox\emptybox}%
	\head


\medskip % EFFACEMENT DES LETTRES
\emptybox\emptybox%
	\emptybox\boxit{b}\boxit{b}\boxit{c}\emptybox%
\emptybox\emptybox

\phantom{\emptybox\emptybox%
	\emptybox}%
	\head

\vfill\null
\columnbreak

\medskip % GESTION DE LA NOUVELLE LETTRE LA PLUS À DROITE
\emptybox\emptybox%
	\emptybox\fboxit{b}\boxit{b}\boxit{c}\emptybox%
\emptybox\emptybox

\phantom{\emptybox\emptybox%
	\emptybox}%
	\head


\medskip % RECHERCHE DE LA DERNIÈRE LETTRE
\emptybox\emptybox%
	\emptybox\fboxit{b}\boxit{b}\boxit{c}\emptybox%
\emptybox\emptybox

\phantom{\emptybox\emptybox%
	\emptybox\emptybox\emptybox}%
	\head


\medskip % MAUVAISE LETTRE
\emptybox\emptybox%
	\emptybox\fboxit{b}\boxit{b}\wboxit{c}\emptybox%
\emptybox\emptybox

\phantom{\emptybox\emptybox%
	\emptybox\emptybox\emptybox}%
	\head

\end{multicols}


\vspace{-1em}

Qu'a-t-on fait ?
\begin{enumerate}
	\item On note la lettre pointée par la tête de lecture.
	      Commence alors une phase de recherche d'une lettre connue.

	\item On avance tant que l'on ne rencontre par une case vide. Une fois celle-ci repérée on revient d'une case en arrière.
	
	\item On compare alors la lettre pointée par la tête de lecture avec celle de la phase de recherche en cours. Deux cas sont possibles.
	\begin{enumerate}
		\item Si les lettres sont différentes alors on ne fait plus rien. 
		      On a un état bloquant et le mot n'est pas validé.

		\item Si les lettres sont identiques alors on efface la lettre en cours puis on va vers la gauche jusqu'à la prochaine case vide. 
		      Une fois celle-ci trouvée, on avance d'une case vers la droite pour effacer son contenu, puis on avance d'une autre case vers la droite pour recommencer les actions à partir du point 1.
	\end{enumerate}
\end{enumerate}


La méthode ci-dessus est en fait incomplète comme nous allons le voir dans la section suivante avec un exemple de palindrome à repérer.




	\subsection{Un exemple avec le mot abbcbba}

		On fait comme précédemment en devant tout parcourir !
Voici les grandes étapes.


\begin{multicols}{2}

\emptybox\emptybox%
	\boxit{a}\boxit{b}\boxit{b}\boxit{c}\boxit{b}\boxit{b}\boxit{a}%
\emptybox\emptybox

\phantom{%
	\emptybox\emptybox}%
	\head
	

\medskip % 1ER EFFACEEMENT

\emptybox\emptybox%
	\fboxit{a}\boxit{b}\boxit{b}\boxit{c}\boxit{b}\boxit{b}\fboxit{a}%
\emptybox\emptybox

\phantom{%
	\emptybox\emptybox%
	\emptybox\emptybox\emptybox\emptybox\emptybox\emptybox}%
	\head
	

\vfill\null
\columnbreak


\emptybox\emptybox%
	\emptybox\boxit{b}\boxit{b}\boxit{c}\boxit{b}\boxit{b}\emptybox%
\emptybox\emptybox

\phantom{%
	\emptybox\emptybox\emptybox}%
	\head
	

\medskip % 2IÈME EFFACEEMENT

\emptybox\emptybox%
	\emptybox\fboxit{b}\boxit{b}\boxit{c}\boxit{b}\fboxit{b}\emptybox%
\emptybox\emptybox

\phantom{%
	\emptybox\emptybox%
	\emptybox\emptybox\emptybox\emptybox\emptybox}%
	\head

\vfill\null
\end{multicols}


\begin{multicols}{2}

\emptybox\emptybox%
	\emptybox\emptybox\boxit{b}\boxit{c}\boxit{b}\emptybox\emptybox%
\emptybox\emptybox

\phantom{%
	\emptybox\emptybox\emptybox\emptybox}%
	\head


\medskip % 3IÈME EFFACEEMENT

\emptybox\emptybox%
	\emptybox\emptybox\fboxit{b}\boxit{c}\fboxit{b}\emptybox\emptybox%
\emptybox\emptybox

\phantom{%
	\emptybox\emptybox%
	\emptybox\emptybox\emptybox\emptybox}%
	\head
	

\medskip

\emptybox\emptybox%
	\emptybox\emptybox\emptybox\boxit{c}\emptybox\emptybox\emptybox%
\emptybox\emptybox

\phantom{%
	\emptybox\emptybox\emptybox\emptybox\emptybox}%
	\head
	

\vfill\null
\columnbreak


% DERNIER EFFACEEMENT

\emptybox\emptybox%
	\emptybox\emptybox\emptybox\fboxit{c}\emptybox\emptybox\emptybox%
\emptybox\emptybox

\phantom{%
	\emptybox\emptybox%
	\emptybox\emptybox\emptybox}%
	\head
	

\medskip

\emptybox\emptybox%
	\emptybox\emptybox\emptybox\emptybox\emptybox\emptybox\emptybox%
\emptybox\emptybox

\phantom{%
	\emptybox\emptybox\emptybox\emptybox\emptybox}%
	\head

\vfill\null
\end{multicols}




	\subsection{La table de transition} \label{duplicate-table}

		On cherche ici à faire une table pour un \emph{\og ou exclusif \fg}.
En utilisant l'identité booléenne $A \boolope{ OUEX } B \eq[id] (\boolope{NON } A \boolope{ ET } B) \boolope{ OU } (A \boolope{ ET NON } B)$, nous pouvons appliquer ce que nous avons utilisé précédemment pour traduire les opérateurs logiques $\boolope{ET}$, $\boolope{OU}$ et $\boolope{NON}$.
Bien que théoriquement correct, ce raisonnement automatique va nous conduire à une trop \myquote{grosse} table des transitions.


\medskip


Pour obtenir une table de taille raisonnable, nous allons ajouter une nouvelle bande.
Expliquons comment faire automatiquement via la table \myquote{optimisée} de la section \ref{2-or-3} \emph{(la méthode présentée est généralisable aux tables à plusieurs bandes)}.
\begin{enumerate}
	\item La bande supplémentaire va juste nous servir à \myquote{compter} les cas gagnants.

	\item Nous allons court-circuiter le 1\ier{} cas final pour aller à l'état $sp_0$  tout en gardant la trace du succès du 1\ier{} test.

	\item Il reste à gérer les états bloquants de la 2\ieme{} machine lorsque la 1\iere{} a fonctionné.
	      Dans l'optique d'utilisation séquentielle de machines, nous remettons à zéro la bande supplémentaire.
\end{enumerate}


\medskip

Nous utilisons $\twocoord{0}{\bullet}$ pour simplifier la table. Dans cette écriture, {\scriptsize$\bullet$} indique un caractère quelconque. Dans une transition, il indique le caractère initial qui est donc inchangé.


\begin{center}
	\emph{\small Phase 1 : a-t-on un pair via la bande du bas ?}
	
	\smallskip
	\begin{tabular}{|c||c|c|c|}
% MUTIPLE OF 3
		\hline
		$\delta$ 
			& $\twocoord{0}{B}$ 
			& $\twocoord{1}{B}$
			& $\twocoord{B}{B}$ \\
		\hline
		\hline
		$q_0$ 
			& \transition{\ell_0}{\twocoord{0}{B}}{\twocoord{D}{I}} 
			& \transition{\ell_1}{\twocoord{1}{B}}{\twocoord{D}{I}}
			&                                                       \\
		\hline
		$\ell_0$
			& \transition{\ell_0 }{\twocoord{0}{B}}{\twocoord{D}{I}} 
			& \transition{\ell_1 }{\twocoord{1}{B}}{\twocoord{D}{I}}
			& \transition{sp_0   }{\twocoord{B}{1}}{\twocoord{G}{I}} \\
		\hline
		$\ell_1$
			& \transition{\ell_0}{\twocoord{0}{B}}{\twocoord{D}{I}} 
			& \transition{\ell_1}{\twocoord{1}{B}}{\twocoord{D}{I}}
			& \transition{sp_0  }{\twocoord{B}{B}}{\twocoord{G}{I}} \\
		\hline
	\end{tabular}
\end{center}



\begin{center}
	\emph{\small Phase 2 : a-t-on un multiple de $3$ via la bande du bas ?}
	
	\smallskip
	\begin{tabular}{|c||c|c|c|c|}
% MUTIPLE OF 3
		\hline
		$\delta$ 
			& $\twocoord{0}{\bullet}$ 
			& $\twocoord{1}{\bullet}$
			& $\twocoord{B}{B}$       
			& $\twocoord{B}{1}$ \\
		\hline
		\hline
		$sp_0$ 
			& \transition{si_0}{\twocoord{0}{\bullet}}{\twocoord{G}{I}} 
			& \transition{si_1}{\twocoord{1}{\bullet}}{\twocoord{G}{I}}
			& \transition{f   }{\twocoord{B}{B}      }{\twocoord{I}{I}}
			&                                                           \\
		\hline
		$sp_1$ 
			& \transition{si_1}{\twocoord{0}{\bullet}}{\twocoord{G}{I}} 
			& \transition{si_2}{\twocoord{1}{\bullet}}{\twocoord{G}{I}}
			&
			& \transition{f   }{\twocoord{B}{B}      }{\twocoord{I}{I}} \\
		\hline
		$sp_2$ 
			& \transition{si_2}{\twocoord{0}{\bullet}}{\twocoord{G}{I}} 
			& \transition{si_0}{\twocoord{1}{\bullet}}{\twocoord{G}{I}}
			&
			& \transition{f   }{\twocoord{B}{B}}{\twocoord{I}{I}}       \\
		\hline
		\hline
		$si_0$ 
			& \transition{sp_0}{\twocoord{0}{\bullet}}{\twocoord{G}{I}} 
			& \transition{sp_2}{\twocoord{1}{\bullet}}{\twocoord{G}{I}}
			& \transition{f   }{\twocoord{B}{B}      }{\twocoord{I}{I}}
			&                                                           \\
		\hline
		$si_1$ 
			& \transition{sp_1}{\twocoord{0}{\bullet}}{\twocoord{G}{I}} 
			& \transition{sp_0}{\twocoord{1}{\bullet}}{\twocoord{G}{I}}
			&
			& \transition{f   }{\twocoord{B}{B}      }{\twocoord{I}{I}} \\
		\hline
		$si_2$ 
			& \transition{sp_2}{\twocoord{0}{\bullet}}{\twocoord{G}{I}} 
			& \transition{sp_1}{\twocoord{1}{\bullet}}{\twocoord{G}{I}}
			&
			& \transition{f   }{\twocoord{B}{B}      }{\twocoord{I}{I}} \\
		\hline
	\end{tabular}
\end{center}




	\subsection{Coder pour tester}

		Sur le site de téléchargement de ce document, dans le sous-dossier \verb+turing/palindrome+, se trouve le fichier \verb+palindrome.txt+ contenant un code utilisable
pour des tests manuels sur le site \url{https://turingmachinesimulator.com}.



\end{document}
