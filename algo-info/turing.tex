  % !TEX encoding = UTF-8 Unicode
\documentclass[a4paper, 12pt]{scrartcl}

\usepackage[utf8]{inputenc}
\usepackage[T1]{fontenc}

\usepackage[top=2.5cm, bottom=2.5cm, left=1.95cm, right=1.95cm]{geometry}

\usepackage[french]{babel}
\usepackage{listings}

\usepackage[hidelinks]{hyperref}

\usepackage[fr, apmep]{lyxam}
\usepackage{tnsmath}
\usepackage{fancyvrb}
\usepackage{bera}
\usepackage{enumitem}
\usepackage{multicol}

\usepackage{lastpage}
\usepackage{xcolor}
\usepackage{graphicx}
\usepackage{setspace}

\usepackage{tikz}
\usetikzlibrary{arrows,positioning, calc} 

\usepackage{fancyvrb}


\usepackage[
    type={CC},
    modifier={by-nc-sa},
	version={4.0},
]{doclicense}


\newcommand\boxit[1]{\fbox{\makebox[.85em]{#1}\vphantom{$pX^M$}}}
\newcommand\fboxit[1]{\fcolorbox{black}{yellow}{\makebox[.85em]{#1}\vphantom{$pX^M$}}}
\newcommand\nboxit[1]{\fcolorbox{black}{lightgray}{\makebox[.85em]{#1}\vphantom{$pX^M$}}}
\newcommand\wboxit[1]{\fcolorbox{black}{red!50}{\makebox[.85em]{#1}\vphantom{$pX^M$}}}
\newcommand\noboxit[1]{\fcolorbox{white}{white}{\makebox[.85em]{#1}\vphantom{$pX^M$}}}

\newcommand\emptybox{\boxit{\phantom{A}}}
\newcommand\nemptybox{\nboxit{\phantom{A}}}
\newcommand\wemptybox{\wboxit{\phantom{A}}}

\newcommand\head{\noboxit{$\uparrow$}}
\newcommand\deah{\noboxit{$\downarrow$}}


\newcommand\boxedU{\boxit{\bfseries ?}}
\newcommand\boxedB{\boxit{\bfseries K}}
\newcommand\boxedW{\boxit{\bfseries W}}

\newcommand\fboxedB{\fboxit{\bfseries K}}
\newcommand\fboxedW{\fboxit{\bfseries W}}

\newcommand\nboxedB{\nboxit{\bfseries K}}
\newcommand\nboxedW{\nboxit{\bfseries W}}

\newcommand\wboxedB{\wboxit{\bfseries K}}
\newcommand\wboxedW{\wboxit{\bfseries W}}



\newcommand\twocoord[2]{{\scriptsize\begin{matrix}#1\\#2\end{matrix}}}
\newcommand\threecoord[3]{{\scriptsize\begin{matrix}#1\\#2\\#3\end{matrix}}}


\newcommand\transition[3]{%
	$\left(\, #1 \,, #2 \,, #3 \,\right)^{\vphantom{4}}$
}

\newcommand\myquote[1]{\emph{\og #1 \fg}}


\newcommand\boolope[1]{\text{\bfseries#1}}


\begin{document}

\title{BROUILLON - Quelques machines de Turing déterministes}
\author{Christophe BAL}
\date{7 Février 2020 -- 20 Mars 2020}

\maketitle

\begin{center}
	\itshape
	Document, avec son source \LaTeX, disponible sur la page
	
	\url{https://github.com/bc-writing/drafts}.
\end{center}


\bigskip


\begin{center}
	\hrule\vspace{.3em}
	{
		\fontsize{1.35em}{1em}\selectfont
		\textbf{Mentions \og légales \fg}
	}
			
	\vspace{0.45em}
	\doclicenseThis
	\hrule
\end{center}


\bigskip
\setcounter{tocdepth}{2}
\tableofcontents



\newpage
\section{Écriture binaire des naturels pairs}

	\subsection{La méthode}

	Commençons par un exemple simple qui va nous permettre de fixer les notations que nous allons utiliser.
Voyons comment repérer un entier naturel pair à partir de son écriture binaire.
La réponse est évidente : un naturel est pair si et seulement si son écriture binaire se finit par un zéro.
Il suffit donc de parcourir cette écriture binaire et d'analyser le chiffre le plus à droite. Ceci se schématise comme suit où \head{} indique la tête de lecture.

\begin{multicols}{2}

%

\emptybox\emptybox%
	\boxit{1}\boxit{0}\boxit{0}\boxit{1}\boxit{1}%
\emptybox\emptybox

\phantom{%
	\emptybox\emptybox}%
	\head


\medskip %

\emptybox\emptybox%
	\boxit{1}\boxit{0}\boxit{0}\boxit{1}\boxit{1}%
\emptybox\emptybox

\phantom{%
	\emptybox\emptybox
	\emptybox}%
	\head


\medskip %

\emptybox\emptybox%
	\boxit{1}\boxit{0}\boxit{0}\boxit{1}\boxit{1}%
\emptybox\emptybox

\phantom{%
	\emptybox\emptybox
	\emptybox\emptybox}%
	\head


\vfill\null
\columnbreak

\medskip %

\emptybox\emptybox%
	\boxit{1}\boxit{0}\boxit{0}\boxit{1}\boxit{1}%
\emptybox\emptybox

\phantom{%
	\emptybox\emptybox
	\emptybox\emptybox\emptybox}%
	\head


\medskip %

\emptybox\emptybox%
	\boxit{1}\boxit{0}\boxit{0}\boxit{1}\boxit{1}%
\emptybox\emptybox

\phantom{%
	\emptybox\emptybox
	\emptybox\emptybox\emptybox\emptybox}%
	\head


\medskip %

\emptybox\emptybox%
	\boxit{1}\boxit{0}\boxit{0}\boxit{1}\boxit{1}%
\emptybox\emptybox

\phantom{%
	\emptybox\emptybox
	\emptybox\emptybox\emptybox\emptybox\emptybox}%
	\head

\vfill\null
\end{multicols}

\vspace{-1em}

Pourquoi s'arrêter à la première case vide ? L'idée va être de garder en mémoire la valeur de la dernière case visitée.
Dans notre exemple, lors du dernier mouvement, on sait que la case précédente était un $1$ et donc que l'écriture binaire n'est pas celle d'un entier naturel pair.



\subsection{Une table des transitions}

	On cherche ici à faire une table pour un \emph{\og ou exclusif \fg}.
En utilisant l'identité booléenne $A \boolope{ OUEX } B \eq[id] (\boolope{NON } A \boolope{ ET } B) \boolope{ OU } (A \boolope{ ET NON } B)$, nous pouvons appliquer ce que nous avons utilisé précédemment pour traduire les opérateurs logiques $\boolope{ET}$, $\boolope{OU}$ et $\boolope{NON}$.
Bien que théoriquement correct, ce raisonnement automatique va nous conduire à une trop \myquote{grosse} table des transitions.


\medskip


Pour obtenir une table de taille raisonnable, nous allons ajouter une nouvelle bande.
Expliquons comment faire automatiquement via la table \myquote{optimisée} de la section \ref{2-or-3} \emph{(la méthode présentée est généralisable aux tables à plusieurs bandes)}.
\begin{enumerate}
	\item La bande supplémentaire va juste nous servir à \myquote{compter} les cas gagnants.

	\item Nous allons court-circuiter le 1\ier{} cas final pour aller à l'état $sp_0$  tout en gardant la trace du succès du 1\ier{} test.

	\item Il reste à gérer les états bloquants de la 2\ieme{} machine lorsque la 1\iere{} a fonctionné.
	      Dans l'optique d'utilisation séquentielle de machines, nous remettons à zéro la bande supplémentaire.
\end{enumerate}


\medskip

Nous utilisons $\twocoord{0}{\bullet}$ pour simplifier la table. Dans cette écriture, {\scriptsize$\bullet$} indique un caractère quelconque. Dans une transition, il indique le caractère initial qui est donc inchangé.


\begin{center}
	\emph{\small Phase 1 : a-t-on un pair via la bande du bas ?}
	
	\smallskip
	\begin{tabular}{|c||c|c|c|}
% MUTIPLE OF 3
		\hline
		$\delta$ 
			& $\twocoord{0}{B}$ 
			& $\twocoord{1}{B}$
			& $\twocoord{B}{B}$ \\
		\hline
		\hline
		$q_0$ 
			& \transition{\ell_0}{\twocoord{0}{B}}{\twocoord{D}{I}} 
			& \transition{\ell_1}{\twocoord{1}{B}}{\twocoord{D}{I}}
			&                                                       \\
		\hline
		$\ell_0$
			& \transition{\ell_0 }{\twocoord{0}{B}}{\twocoord{D}{I}} 
			& \transition{\ell_1 }{\twocoord{1}{B}}{\twocoord{D}{I}}
			& \transition{sp_0   }{\twocoord{B}{1}}{\twocoord{G}{I}} \\
		\hline
		$\ell_1$
			& \transition{\ell_0}{\twocoord{0}{B}}{\twocoord{D}{I}} 
			& \transition{\ell_1}{\twocoord{1}{B}}{\twocoord{D}{I}}
			& \transition{sp_0  }{\twocoord{B}{B}}{\twocoord{G}{I}} \\
		\hline
	\end{tabular}
\end{center}



\begin{center}
	\emph{\small Phase 2 : a-t-on un multiple de $3$ via la bande du bas ?}
	
	\smallskip
	\begin{tabular}{|c||c|c|c|c|}
% MUTIPLE OF 3
		\hline
		$\delta$ 
			& $\twocoord{0}{\bullet}$ 
			& $\twocoord{1}{\bullet}$
			& $\twocoord{B}{B}$       
			& $\twocoord{B}{1}$ \\
		\hline
		\hline
		$sp_0$ 
			& \transition{si_0}{\twocoord{0}{\bullet}}{\twocoord{G}{I}} 
			& \transition{si_1}{\twocoord{1}{\bullet}}{\twocoord{G}{I}}
			& \transition{f   }{\twocoord{B}{B}      }{\twocoord{I}{I}}
			&                                                           \\
		\hline
		$sp_1$ 
			& \transition{si_1}{\twocoord{0}{\bullet}}{\twocoord{G}{I}} 
			& \transition{si_2}{\twocoord{1}{\bullet}}{\twocoord{G}{I}}
			&
			& \transition{f   }{\twocoord{B}{B}      }{\twocoord{I}{I}} \\
		\hline
		$sp_2$ 
			& \transition{si_2}{\twocoord{0}{\bullet}}{\twocoord{G}{I}} 
			& \transition{si_0}{\twocoord{1}{\bullet}}{\twocoord{G}{I}}
			&
			& \transition{f   }{\twocoord{B}{B}}{\twocoord{I}{I}}       \\
		\hline
		\hline
		$si_0$ 
			& \transition{sp_0}{\twocoord{0}{\bullet}}{\twocoord{G}{I}} 
			& \transition{sp_2}{\twocoord{1}{\bullet}}{\twocoord{G}{I}}
			& \transition{f   }{\twocoord{B}{B}      }{\twocoord{I}{I}}
			&                                                           \\
		\hline
		$si_1$ 
			& \transition{sp_1}{\twocoord{0}{\bullet}}{\twocoord{G}{I}} 
			& \transition{sp_0}{\twocoord{1}{\bullet}}{\twocoord{G}{I}}
			&
			& \transition{f   }{\twocoord{B}{B}      }{\twocoord{I}{I}} \\
		\hline
		$si_2$ 
			& \transition{sp_2}{\twocoord{0}{\bullet}}{\twocoord{G}{I}} 
			& \transition{sp_1}{\twocoord{1}{\bullet}}{\twocoord{G}{I}}
			&
			& \transition{f   }{\twocoord{B}{B}      }{\twocoord{I}{I}} \\
		\hline
	\end{tabular}
\end{center}




\subsection{Coder pour tester}

	Sur le site de téléchargement de ce document, dans le sous-dossier \verb+turing/palindrome+, se trouve le fichier \verb+palindrome.txt+ contenant un code utilisable
pour des tests manuels sur le site \url{https://turingmachinesimulator.com}.



\newpage
\section{Écriture binaire des multiples de 3} \label{divisibility-by-3}

	\subsection{La méthode}

	Commençons par un exemple simple qui va nous permettre de fixer les notations que nous allons utiliser.
Voyons comment repérer un entier naturel pair à partir de son écriture binaire.
La réponse est évidente : un naturel est pair si et seulement si son écriture binaire se finit par un zéro.
Il suffit donc de parcourir cette écriture binaire et d'analyser le chiffre le plus à droite. Ceci se schématise comme suit où \head{} indique la tête de lecture.

\begin{multicols}{2}

%

\emptybox\emptybox%
	\boxit{1}\boxit{0}\boxit{0}\boxit{1}\boxit{1}%
\emptybox\emptybox

\phantom{%
	\emptybox\emptybox}%
	\head


\medskip %

\emptybox\emptybox%
	\boxit{1}\boxit{0}\boxit{0}\boxit{1}\boxit{1}%
\emptybox\emptybox

\phantom{%
	\emptybox\emptybox
	\emptybox}%
	\head


\medskip %

\emptybox\emptybox%
	\boxit{1}\boxit{0}\boxit{0}\boxit{1}\boxit{1}%
\emptybox\emptybox

\phantom{%
	\emptybox\emptybox
	\emptybox\emptybox}%
	\head


\vfill\null
\columnbreak

\medskip %

\emptybox\emptybox%
	\boxit{1}\boxit{0}\boxit{0}\boxit{1}\boxit{1}%
\emptybox\emptybox

\phantom{%
	\emptybox\emptybox
	\emptybox\emptybox\emptybox}%
	\head


\medskip %

\emptybox\emptybox%
	\boxit{1}\boxit{0}\boxit{0}\boxit{1}\boxit{1}%
\emptybox\emptybox

\phantom{%
	\emptybox\emptybox
	\emptybox\emptybox\emptybox\emptybox}%
	\head


\medskip %

\emptybox\emptybox%
	\boxit{1}\boxit{0}\boxit{0}\boxit{1}\boxit{1}%
\emptybox\emptybox

\phantom{%
	\emptybox\emptybox
	\emptybox\emptybox\emptybox\emptybox\emptybox}%
	\head

\vfill\null
\end{multicols}

\vspace{-1em}

Pourquoi s'arrêter à la première case vide ? L'idée va être de garder en mémoire la valeur de la dernière case visitée.
Dans notre exemple, lors du dernier mouvement, on sait que la case précédente était un $1$ et donc que l'écriture binaire n'est pas celle d'un entier naturel pair.



\subsection{Une table des transitions}

	On cherche ici à faire une table pour un \emph{\og ou exclusif \fg}.
En utilisant l'identité booléenne $A \boolope{ OUEX } B \eq[id] (\boolope{NON } A \boolope{ ET } B) \boolope{ OU } (A \boolope{ ET NON } B)$, nous pouvons appliquer ce que nous avons utilisé précédemment pour traduire les opérateurs logiques $\boolope{ET}$, $\boolope{OU}$ et $\boolope{NON}$.
Bien que théoriquement correct, ce raisonnement automatique va nous conduire à une trop \myquote{grosse} table des transitions.


\medskip


Pour obtenir une table de taille raisonnable, nous allons ajouter une nouvelle bande.
Expliquons comment faire automatiquement via la table \myquote{optimisée} de la section \ref{2-or-3} \emph{(la méthode présentée est généralisable aux tables à plusieurs bandes)}.
\begin{enumerate}
	\item La bande supplémentaire va juste nous servir à \myquote{compter} les cas gagnants.

	\item Nous allons court-circuiter le 1\ier{} cas final pour aller à l'état $sp_0$  tout en gardant la trace du succès du 1\ier{} test.

	\item Il reste à gérer les états bloquants de la 2\ieme{} machine lorsque la 1\iere{} a fonctionné.
	      Dans l'optique d'utilisation séquentielle de machines, nous remettons à zéro la bande supplémentaire.
\end{enumerate}


\medskip

Nous utilisons $\twocoord{0}{\bullet}$ pour simplifier la table. Dans cette écriture, {\scriptsize$\bullet$} indique un caractère quelconque. Dans une transition, il indique le caractère initial qui est donc inchangé.


\begin{center}
	\emph{\small Phase 1 : a-t-on un pair via la bande du bas ?}
	
	\smallskip
	\begin{tabular}{|c||c|c|c|}
% MUTIPLE OF 3
		\hline
		$\delta$ 
			& $\twocoord{0}{B}$ 
			& $\twocoord{1}{B}$
			& $\twocoord{B}{B}$ \\
		\hline
		\hline
		$q_0$ 
			& \transition{\ell_0}{\twocoord{0}{B}}{\twocoord{D}{I}} 
			& \transition{\ell_1}{\twocoord{1}{B}}{\twocoord{D}{I}}
			&                                                       \\
		\hline
		$\ell_0$
			& \transition{\ell_0 }{\twocoord{0}{B}}{\twocoord{D}{I}} 
			& \transition{\ell_1 }{\twocoord{1}{B}}{\twocoord{D}{I}}
			& \transition{sp_0   }{\twocoord{B}{1}}{\twocoord{G}{I}} \\
		\hline
		$\ell_1$
			& \transition{\ell_0}{\twocoord{0}{B}}{\twocoord{D}{I}} 
			& \transition{\ell_1}{\twocoord{1}{B}}{\twocoord{D}{I}}
			& \transition{sp_0  }{\twocoord{B}{B}}{\twocoord{G}{I}} \\
		\hline
	\end{tabular}
\end{center}



\begin{center}
	\emph{\small Phase 2 : a-t-on un multiple de $3$ via la bande du bas ?}
	
	\smallskip
	\begin{tabular}{|c||c|c|c|c|}
% MUTIPLE OF 3
		\hline
		$\delta$ 
			& $\twocoord{0}{\bullet}$ 
			& $\twocoord{1}{\bullet}$
			& $\twocoord{B}{B}$       
			& $\twocoord{B}{1}$ \\
		\hline
		\hline
		$sp_0$ 
			& \transition{si_0}{\twocoord{0}{\bullet}}{\twocoord{G}{I}} 
			& \transition{si_1}{\twocoord{1}{\bullet}}{\twocoord{G}{I}}
			& \transition{f   }{\twocoord{B}{B}      }{\twocoord{I}{I}}
			&                                                           \\
		\hline
		$sp_1$ 
			& \transition{si_1}{\twocoord{0}{\bullet}}{\twocoord{G}{I}} 
			& \transition{si_2}{\twocoord{1}{\bullet}}{\twocoord{G}{I}}
			&
			& \transition{f   }{\twocoord{B}{B}      }{\twocoord{I}{I}} \\
		\hline
		$sp_2$ 
			& \transition{si_2}{\twocoord{0}{\bullet}}{\twocoord{G}{I}} 
			& \transition{si_0}{\twocoord{1}{\bullet}}{\twocoord{G}{I}}
			&
			& \transition{f   }{\twocoord{B}{B}}{\twocoord{I}{I}}       \\
		\hline
		\hline
		$si_0$ 
			& \transition{sp_0}{\twocoord{0}{\bullet}}{\twocoord{G}{I}} 
			& \transition{sp_2}{\twocoord{1}{\bullet}}{\twocoord{G}{I}}
			& \transition{f   }{\twocoord{B}{B}      }{\twocoord{I}{I}}
			&                                                           \\
		\hline
		$si_1$ 
			& \transition{sp_1}{\twocoord{0}{\bullet}}{\twocoord{G}{I}} 
			& \transition{sp_0}{\twocoord{1}{\bullet}}{\twocoord{G}{I}}
			&
			& \transition{f   }{\twocoord{B}{B}      }{\twocoord{I}{I}} \\
		\hline
		$si_2$ 
			& \transition{sp_2}{\twocoord{0}{\bullet}}{\twocoord{G}{I}} 
			& \transition{sp_1}{\twocoord{1}{\bullet}}{\twocoord{G}{I}}
			&
			& \transition{f   }{\twocoord{B}{B}      }{\twocoord{I}{I}} \\
		\hline
	\end{tabular}
\end{center}




\subsection{Coder pour tester}

	Sur le site de téléchargement de ce document, dans le sous-dossier \verb+turing/palindrome+, se trouve le fichier \verb+palindrome.txt+ contenant un code utilisable
pour des tests manuels sur le site \url{https://turingmachinesimulator.com}.


\subsection{Généralisations}

	Le raisonnement précédent s'est appuyé sur $4 = 3 + 1$ soit $3 = 2^2 - 1$. Il est en fait facile de généraliser ce qui a été fait pour repérér les nombres divisibles par $2^k - 1$ où $k \in \NN_{\geq 2}$.
Ceci étant dit, les tables des transitions vont croître de façon exponentielle !


\medskip


\textbf{Remarque :} on peut en fait construire un automate d'état déterministe fini qui reconnait les multiples de $3$, et plus généralement de $d \in \NN_{\geq 2}$, à partir de l'écriture d'un nombre dans une base quelconque $b \in \NN_{\geq 2}$.
Ceci prouve que l'on peut construire des expressions rationnelles pour savoir si une écriture correspond à un multiple.



\newpage
\section{Écriture binaire des naturels impairs ou de ceux non multiples de 3}

	\subsection{Des tables des transitions}

	On cherche ici à faire une table pour un \emph{\og ou exclusif \fg}.
En utilisant l'identité booléenne $A \boolope{ OUEX } B \eq[id] (\boolope{NON } A \boolope{ ET } B) \boolope{ OU } (A \boolope{ ET NON } B)$, nous pouvons appliquer ce que nous avons utilisé précédemment pour traduire les opérateurs logiques $\boolope{ET}$, $\boolope{OU}$ et $\boolope{NON}$.
Bien que théoriquement correct, ce raisonnement automatique va nous conduire à une trop \myquote{grosse} table des transitions.


\medskip


Pour obtenir une table de taille raisonnable, nous allons ajouter une nouvelle bande.
Expliquons comment faire automatiquement via la table \myquote{optimisée} de la section \ref{2-or-3} \emph{(la méthode présentée est généralisable aux tables à plusieurs bandes)}.
\begin{enumerate}
	\item La bande supplémentaire va juste nous servir à \myquote{compter} les cas gagnants.

	\item Nous allons court-circuiter le 1\ier{} cas final pour aller à l'état $sp_0$  tout en gardant la trace du succès du 1\ier{} test.

	\item Il reste à gérer les états bloquants de la 2\ieme{} machine lorsque la 1\iere{} a fonctionné.
	      Dans l'optique d'utilisation séquentielle de machines, nous remettons à zéro la bande supplémentaire.
\end{enumerate}


\medskip

Nous utilisons $\twocoord{0}{\bullet}$ pour simplifier la table. Dans cette écriture, {\scriptsize$\bullet$} indique un caractère quelconque. Dans une transition, il indique le caractère initial qui est donc inchangé.


\begin{center}
	\emph{\small Phase 1 : a-t-on un pair via la bande du bas ?}
	
	\smallskip
	\begin{tabular}{|c||c|c|c|}
% MUTIPLE OF 3
		\hline
		$\delta$ 
			& $\twocoord{0}{B}$ 
			& $\twocoord{1}{B}$
			& $\twocoord{B}{B}$ \\
		\hline
		\hline
		$q_0$ 
			& \transition{\ell_0}{\twocoord{0}{B}}{\twocoord{D}{I}} 
			& \transition{\ell_1}{\twocoord{1}{B}}{\twocoord{D}{I}}
			&                                                       \\
		\hline
		$\ell_0$
			& \transition{\ell_0 }{\twocoord{0}{B}}{\twocoord{D}{I}} 
			& \transition{\ell_1 }{\twocoord{1}{B}}{\twocoord{D}{I}}
			& \transition{sp_0   }{\twocoord{B}{1}}{\twocoord{G}{I}} \\
		\hline
		$\ell_1$
			& \transition{\ell_0}{\twocoord{0}{B}}{\twocoord{D}{I}} 
			& \transition{\ell_1}{\twocoord{1}{B}}{\twocoord{D}{I}}
			& \transition{sp_0  }{\twocoord{B}{B}}{\twocoord{G}{I}} \\
		\hline
	\end{tabular}
\end{center}



\begin{center}
	\emph{\small Phase 2 : a-t-on un multiple de $3$ via la bande du bas ?}
	
	\smallskip
	\begin{tabular}{|c||c|c|c|c|}
% MUTIPLE OF 3
		\hline
		$\delta$ 
			& $\twocoord{0}{\bullet}$ 
			& $\twocoord{1}{\bullet}$
			& $\twocoord{B}{B}$       
			& $\twocoord{B}{1}$ \\
		\hline
		\hline
		$sp_0$ 
			& \transition{si_0}{\twocoord{0}{\bullet}}{\twocoord{G}{I}} 
			& \transition{si_1}{\twocoord{1}{\bullet}}{\twocoord{G}{I}}
			& \transition{f   }{\twocoord{B}{B}      }{\twocoord{I}{I}}
			&                                                           \\
		\hline
		$sp_1$ 
			& \transition{si_1}{\twocoord{0}{\bullet}}{\twocoord{G}{I}} 
			& \transition{si_2}{\twocoord{1}{\bullet}}{\twocoord{G}{I}}
			&
			& \transition{f   }{\twocoord{B}{B}      }{\twocoord{I}{I}} \\
		\hline
		$sp_2$ 
			& \transition{si_2}{\twocoord{0}{\bullet}}{\twocoord{G}{I}} 
			& \transition{si_0}{\twocoord{1}{\bullet}}{\twocoord{G}{I}}
			&
			& \transition{f   }{\twocoord{B}{B}}{\twocoord{I}{I}}       \\
		\hline
		\hline
		$si_0$ 
			& \transition{sp_0}{\twocoord{0}{\bullet}}{\twocoord{G}{I}} 
			& \transition{sp_2}{\twocoord{1}{\bullet}}{\twocoord{G}{I}}
			& \transition{f   }{\twocoord{B}{B}      }{\twocoord{I}{I}}
			&                                                           \\
		\hline
		$si_1$ 
			& \transition{sp_1}{\twocoord{0}{\bullet}}{\twocoord{G}{I}} 
			& \transition{sp_0}{\twocoord{1}{\bullet}}{\twocoord{G}{I}}
			&
			& \transition{f   }{\twocoord{B}{B}      }{\twocoord{I}{I}} \\
		\hline
		$si_2$ 
			& \transition{sp_2}{\twocoord{0}{\bullet}}{\twocoord{G}{I}} 
			& \transition{sp_1}{\twocoord{1}{\bullet}}{\twocoord{G}{I}}
			&
			& \transition{f   }{\twocoord{B}{B}      }{\twocoord{I}{I}} \\
		\hline
	\end{tabular}
\end{center}




\subsection{Coder pour tester}

	Sur le site de téléchargement de ce document, dans le sous-dossier \verb+turing/palindrome+, se trouve le fichier \verb+palindrome.txt+ contenant un code utilisable
pour des tests manuels sur le site \url{https://turingmachinesimulator.com}.



\newpage
\section{Avec deux bandes}

	Dans cette section, \textbf{nous cherchons à proposer des méthodes généralistes automatiques mais non nécessairement optimales}
	\emph{(voir à ce sujet la section \ref{better-2-and-3})}.

	\subsection{Écriture binaire d'un multiple de 2 et 3}

	\subsubsection{Des tables des transitions}

	On cherche ici à faire une table pour un \emph{\og ou exclusif \fg}.
En utilisant l'identité booléenne $A \boolope{ OUEX } B \eq[id] (\boolope{NON } A \boolope{ ET } B) \boolope{ OU } (A \boolope{ ET NON } B)$, nous pouvons appliquer ce que nous avons utilisé précédemment pour traduire les opérateurs logiques $\boolope{ET}$, $\boolope{OU}$ et $\boolope{NON}$.
Bien que théoriquement correct, ce raisonnement automatique va nous conduire à une trop \myquote{grosse} table des transitions.


\medskip


Pour obtenir une table de taille raisonnable, nous allons ajouter une nouvelle bande.
Expliquons comment faire automatiquement via la table \myquote{optimisée} de la section \ref{2-or-3} \emph{(la méthode présentée est généralisable aux tables à plusieurs bandes)}.
\begin{enumerate}
	\item La bande supplémentaire va juste nous servir à \myquote{compter} les cas gagnants.

	\item Nous allons court-circuiter le 1\ier{} cas final pour aller à l'état $sp_0$  tout en gardant la trace du succès du 1\ier{} test.

	\item Il reste à gérer les états bloquants de la 2\ieme{} machine lorsque la 1\iere{} a fonctionné.
	      Dans l'optique d'utilisation séquentielle de machines, nous remettons à zéro la bande supplémentaire.
\end{enumerate}


\medskip

Nous utilisons $\twocoord{0}{\bullet}$ pour simplifier la table. Dans cette écriture, {\scriptsize$\bullet$} indique un caractère quelconque. Dans une transition, il indique le caractère initial qui est donc inchangé.


\begin{center}
	\emph{\small Phase 1 : a-t-on un pair via la bande du bas ?}
	
	\smallskip
	\begin{tabular}{|c||c|c|c|}
% MUTIPLE OF 3
		\hline
		$\delta$ 
			& $\twocoord{0}{B}$ 
			& $\twocoord{1}{B}$
			& $\twocoord{B}{B}$ \\
		\hline
		\hline
		$q_0$ 
			& \transition{\ell_0}{\twocoord{0}{B}}{\twocoord{D}{I}} 
			& \transition{\ell_1}{\twocoord{1}{B}}{\twocoord{D}{I}}
			&                                                       \\
		\hline
		$\ell_0$
			& \transition{\ell_0 }{\twocoord{0}{B}}{\twocoord{D}{I}} 
			& \transition{\ell_1 }{\twocoord{1}{B}}{\twocoord{D}{I}}
			& \transition{sp_0   }{\twocoord{B}{1}}{\twocoord{G}{I}} \\
		\hline
		$\ell_1$
			& \transition{\ell_0}{\twocoord{0}{B}}{\twocoord{D}{I}} 
			& \transition{\ell_1}{\twocoord{1}{B}}{\twocoord{D}{I}}
			& \transition{sp_0  }{\twocoord{B}{B}}{\twocoord{G}{I}} \\
		\hline
	\end{tabular}
\end{center}



\begin{center}
	\emph{\small Phase 2 : a-t-on un multiple de $3$ via la bande du bas ?}
	
	\smallskip
	\begin{tabular}{|c||c|c|c|c|}
% MUTIPLE OF 3
		\hline
		$\delta$ 
			& $\twocoord{0}{\bullet}$ 
			& $\twocoord{1}{\bullet}$
			& $\twocoord{B}{B}$       
			& $\twocoord{B}{1}$ \\
		\hline
		\hline
		$sp_0$ 
			& \transition{si_0}{\twocoord{0}{\bullet}}{\twocoord{G}{I}} 
			& \transition{si_1}{\twocoord{1}{\bullet}}{\twocoord{G}{I}}
			& \transition{f   }{\twocoord{B}{B}      }{\twocoord{I}{I}}
			&                                                           \\
		\hline
		$sp_1$ 
			& \transition{si_1}{\twocoord{0}{\bullet}}{\twocoord{G}{I}} 
			& \transition{si_2}{\twocoord{1}{\bullet}}{\twocoord{G}{I}}
			&
			& \transition{f   }{\twocoord{B}{B}      }{\twocoord{I}{I}} \\
		\hline
		$sp_2$ 
			& \transition{si_2}{\twocoord{0}{\bullet}}{\twocoord{G}{I}} 
			& \transition{si_0}{\twocoord{1}{\bullet}}{\twocoord{G}{I}}
			&
			& \transition{f   }{\twocoord{B}{B}}{\twocoord{I}{I}}       \\
		\hline
		\hline
		$si_0$ 
			& \transition{sp_0}{\twocoord{0}{\bullet}}{\twocoord{G}{I}} 
			& \transition{sp_2}{\twocoord{1}{\bullet}}{\twocoord{G}{I}}
			& \transition{f   }{\twocoord{B}{B}      }{\twocoord{I}{I}}
			&                                                           \\
		\hline
		$si_1$ 
			& \transition{sp_1}{\twocoord{0}{\bullet}}{\twocoord{G}{I}} 
			& \transition{sp_0}{\twocoord{1}{\bullet}}{\twocoord{G}{I}}
			&
			& \transition{f   }{\twocoord{B}{B}      }{\twocoord{I}{I}} \\
		\hline
		$si_2$ 
			& \transition{sp_2}{\twocoord{0}{\bullet}}{\twocoord{G}{I}} 
			& \transition{sp_1}{\twocoord{1}{\bullet}}{\twocoord{G}{I}}
			&
			& \transition{f   }{\twocoord{B}{B}      }{\twocoord{I}{I}} \\
		\hline
	\end{tabular}
\end{center}




\subsubsection{Une table efficace} \label{better-2-and-3}

	Si l'on sort des méthodes généralistes et automatisables, on peut faire une table plus simple.
En effet, pour tester si l'on a un multiple de $3$, nous partons de la droite.
Or nous savons aussi que le dernier chiffre lu à droite nous permet de savoir si l'on a ou non un nombre pair.
Il suffit donc lors du déplacement à droite de garder la trace de ce dernier chiffre, puis de continuer le travail si l'on a bien un zéro final.
Ceci nous donne ci-après une table des transitions plus courte que les précédentes.

\begin{center}
	\begin{tabular}{|c||c|c|c|}
% MUTIPLE OF 3
		\hline
		$\delta$ 
			& $0$ 
			& $1$
			& $B$ \\
		\hline
		\hline
		$q_0$ 
			& $(\ell_0 , 0, D)$ 
			& $(\ell_1 , 1, D)$
			&  \\
		\hline
		$\ell_0$
			& $(\ell_0 , 0 , D)$
			& $(\ell_1 , 1 , D)$
			& $(sp_0   , B , G)$ \\
		\hline
		$\ell_1$
			& $(\ell_0 , 0 , D)$
			& $(\ell_1 , 1 , D)$
			&                    \\
		\hline
		\hline
		$sp_0$ 
			& $(si_0 , 0, G)$ 
			& $(si_1 , 1, G)$
			& $(f    , B, I)$ \\
		\hline
		$sp_1$ 
			& $(si_1 , 0, G)$ 
			& $(si_2 , 1, G)$
			&                 \\
		\hline
		$sp_2$ 
			& $(si_2 , 0, G)$ 
			& $(si_0 , 1, G)$
			&                 \\
		\hline
		\hline
		$si_0$ 
			& $(sp_0 , 0, G)$ 
			& $(sp_2 , 1, G)$
			& $(f    , B, I)$ \\
		\hline
		$si_1$ 
			& $(sp_1 , 0, G)$ 
			& $(sp_0 , 1, G)$
			&                 \\
		\hline
		$si_2$ 
			& $(sp_2 , 0, G)$ 
			& $(sp_1 , 1, G)$
			&                 \\
		\hline
	\end{tabular}
\end{center}


\subsubsection{Coder pour tester}

	Sur le site de téléchargement de ce document, dans le sous-dossier \verb+turing/palindrome+, se trouve le fichier \verb+palindrome.txt+ contenant un code utilisable
pour des tests manuels sur le site \url{https://turingmachinesimulator.com}.



\newpage
\subsection{Écriture binaire d'un multiple de 2 ou\,/\,et de 3}

	\subsubsection{Des tables des transitions} \label{2-or-3}

	On cherche ici à faire une table pour un \emph{\og ou exclusif \fg}.
En utilisant l'identité booléenne $A \boolope{ OUEX } B \eq[id] (\boolope{NON } A \boolope{ ET } B) \boolope{ OU } (A \boolope{ ET NON } B)$, nous pouvons appliquer ce que nous avons utilisé précédemment pour traduire les opérateurs logiques $\boolope{ET}$, $\boolope{OU}$ et $\boolope{NON}$.
Bien que théoriquement correct, ce raisonnement automatique va nous conduire à une trop \myquote{grosse} table des transitions.


\medskip


Pour obtenir une table de taille raisonnable, nous allons ajouter une nouvelle bande.
Expliquons comment faire automatiquement via la table \myquote{optimisée} de la section \ref{2-or-3} \emph{(la méthode présentée est généralisable aux tables à plusieurs bandes)}.
\begin{enumerate}
	\item La bande supplémentaire va juste nous servir à \myquote{compter} les cas gagnants.

	\item Nous allons court-circuiter le 1\ier{} cas final pour aller à l'état $sp_0$  tout en gardant la trace du succès du 1\ier{} test.

	\item Il reste à gérer les états bloquants de la 2\ieme{} machine lorsque la 1\iere{} a fonctionné.
	      Dans l'optique d'utilisation séquentielle de machines, nous remettons à zéro la bande supplémentaire.
\end{enumerate}


\medskip

Nous utilisons $\twocoord{0}{\bullet}$ pour simplifier la table. Dans cette écriture, {\scriptsize$\bullet$} indique un caractère quelconque. Dans une transition, il indique le caractère initial qui est donc inchangé.


\begin{center}
	\emph{\small Phase 1 : a-t-on un pair via la bande du bas ?}
	
	\smallskip
	\begin{tabular}{|c||c|c|c|}
% MUTIPLE OF 3
		\hline
		$\delta$ 
			& $\twocoord{0}{B}$ 
			& $\twocoord{1}{B}$
			& $\twocoord{B}{B}$ \\
		\hline
		\hline
		$q_0$ 
			& \transition{\ell_0}{\twocoord{0}{B}}{\twocoord{D}{I}} 
			& \transition{\ell_1}{\twocoord{1}{B}}{\twocoord{D}{I}}
			&                                                       \\
		\hline
		$\ell_0$
			& \transition{\ell_0 }{\twocoord{0}{B}}{\twocoord{D}{I}} 
			& \transition{\ell_1 }{\twocoord{1}{B}}{\twocoord{D}{I}}
			& \transition{sp_0   }{\twocoord{B}{1}}{\twocoord{G}{I}} \\
		\hline
		$\ell_1$
			& \transition{\ell_0}{\twocoord{0}{B}}{\twocoord{D}{I}} 
			& \transition{\ell_1}{\twocoord{1}{B}}{\twocoord{D}{I}}
			& \transition{sp_0  }{\twocoord{B}{B}}{\twocoord{G}{I}} \\
		\hline
	\end{tabular}
\end{center}



\begin{center}
	\emph{\small Phase 2 : a-t-on un multiple de $3$ via la bande du bas ?}
	
	\smallskip
	\begin{tabular}{|c||c|c|c|c|}
% MUTIPLE OF 3
		\hline
		$\delta$ 
			& $\twocoord{0}{\bullet}$ 
			& $\twocoord{1}{\bullet}$
			& $\twocoord{B}{B}$       
			& $\twocoord{B}{1}$ \\
		\hline
		\hline
		$sp_0$ 
			& \transition{si_0}{\twocoord{0}{\bullet}}{\twocoord{G}{I}} 
			& \transition{si_1}{\twocoord{1}{\bullet}}{\twocoord{G}{I}}
			& \transition{f   }{\twocoord{B}{B}      }{\twocoord{I}{I}}
			&                                                           \\
		\hline
		$sp_1$ 
			& \transition{si_1}{\twocoord{0}{\bullet}}{\twocoord{G}{I}} 
			& \transition{si_2}{\twocoord{1}{\bullet}}{\twocoord{G}{I}}
			&
			& \transition{f   }{\twocoord{B}{B}      }{\twocoord{I}{I}} \\
		\hline
		$sp_2$ 
			& \transition{si_2}{\twocoord{0}{\bullet}}{\twocoord{G}{I}} 
			& \transition{si_0}{\twocoord{1}{\bullet}}{\twocoord{G}{I}}
			&
			& \transition{f   }{\twocoord{B}{B}}{\twocoord{I}{I}}       \\
		\hline
		\hline
		$si_0$ 
			& \transition{sp_0}{\twocoord{0}{\bullet}}{\twocoord{G}{I}} 
			& \transition{sp_2}{\twocoord{1}{\bullet}}{\twocoord{G}{I}}
			& \transition{f   }{\twocoord{B}{B}      }{\twocoord{I}{I}}
			&                                                           \\
		\hline
		$si_1$ 
			& \transition{sp_1}{\twocoord{0}{\bullet}}{\twocoord{G}{I}} 
			& \transition{sp_0}{\twocoord{1}{\bullet}}{\twocoord{G}{I}}
			&
			& \transition{f   }{\twocoord{B}{B}      }{\twocoord{I}{I}} \\
		\hline
		$si_2$ 
			& \transition{sp_2}{\twocoord{0}{\bullet}}{\twocoord{G}{I}} 
			& \transition{sp_1}{\twocoord{1}{\bullet}}{\twocoord{G}{I}}
			&
			& \transition{f   }{\twocoord{B}{B}      }{\twocoord{I}{I}} \\
		\hline
	\end{tabular}
\end{center}




\subsubsection{Coder pour tester}

	Sur le site de téléchargement de ce document, dans le sous-dossier \verb+turing/palindrome+, se trouve le fichier \verb+palindrome.txt+ contenant un code utilisable
pour des tests manuels sur le site \url{https://turingmachinesimulator.com}.




\newpage
\subsection{Écriture binaire d'un multiple soit de 2, soit de 3 mais pas des deux en même temps}

	\subsubsection{Une table des transitions}

	On cherche ici à faire une table pour un \emph{\og ou exclusif \fg}.
En utilisant l'identité booléenne $A \boolope{ OUEX } B \eq[id] (\boolope{NON } A \boolope{ ET } B) \boolope{ OU } (A \boolope{ ET NON } B)$, nous pouvons appliquer ce que nous avons utilisé précédemment pour traduire les opérateurs logiques $\boolope{ET}$, $\boolope{OU}$ et $\boolope{NON}$.
Bien que théoriquement correct, ce raisonnement automatique va nous conduire à une trop \myquote{grosse} table des transitions.


\medskip


Pour obtenir une table de taille raisonnable, nous allons ajouter une nouvelle bande.
Expliquons comment faire automatiquement via la table \myquote{optimisée} de la section \ref{2-or-3} \emph{(la méthode présentée est généralisable aux tables à plusieurs bandes)}.
\begin{enumerate}
	\item La bande supplémentaire va juste nous servir à \myquote{compter} les cas gagnants.

	\item Nous allons court-circuiter le 1\ier{} cas final pour aller à l'état $sp_0$  tout en gardant la trace du succès du 1\ier{} test.

	\item Il reste à gérer les états bloquants de la 2\ieme{} machine lorsque la 1\iere{} a fonctionné.
	      Dans l'optique d'utilisation séquentielle de machines, nous remettons à zéro la bande supplémentaire.
\end{enumerate}


\medskip

Nous utilisons $\twocoord{0}{\bullet}$ pour simplifier la table. Dans cette écriture, {\scriptsize$\bullet$} indique un caractère quelconque. Dans une transition, il indique le caractère initial qui est donc inchangé.


\begin{center}
	\emph{\small Phase 1 : a-t-on un pair via la bande du bas ?}
	
	\smallskip
	\begin{tabular}{|c||c|c|c|}
% MUTIPLE OF 3
		\hline
		$\delta$ 
			& $\twocoord{0}{B}$ 
			& $\twocoord{1}{B}$
			& $\twocoord{B}{B}$ \\
		\hline
		\hline
		$q_0$ 
			& \transition{\ell_0}{\twocoord{0}{B}}{\twocoord{D}{I}} 
			& \transition{\ell_1}{\twocoord{1}{B}}{\twocoord{D}{I}}
			&                                                       \\
		\hline
		$\ell_0$
			& \transition{\ell_0 }{\twocoord{0}{B}}{\twocoord{D}{I}} 
			& \transition{\ell_1 }{\twocoord{1}{B}}{\twocoord{D}{I}}
			& \transition{sp_0   }{\twocoord{B}{1}}{\twocoord{G}{I}} \\
		\hline
		$\ell_1$
			& \transition{\ell_0}{\twocoord{0}{B}}{\twocoord{D}{I}} 
			& \transition{\ell_1}{\twocoord{1}{B}}{\twocoord{D}{I}}
			& \transition{sp_0  }{\twocoord{B}{B}}{\twocoord{G}{I}} \\
		\hline
	\end{tabular}
\end{center}



\begin{center}
	\emph{\small Phase 2 : a-t-on un multiple de $3$ via la bande du bas ?}
	
	\smallskip
	\begin{tabular}{|c||c|c|c|c|}
% MUTIPLE OF 3
		\hline
		$\delta$ 
			& $\twocoord{0}{\bullet}$ 
			& $\twocoord{1}{\bullet}$
			& $\twocoord{B}{B}$       
			& $\twocoord{B}{1}$ \\
		\hline
		\hline
		$sp_0$ 
			& \transition{si_0}{\twocoord{0}{\bullet}}{\twocoord{G}{I}} 
			& \transition{si_1}{\twocoord{1}{\bullet}}{\twocoord{G}{I}}
			& \transition{f   }{\twocoord{B}{B}      }{\twocoord{I}{I}}
			&                                                           \\
		\hline
		$sp_1$ 
			& \transition{si_1}{\twocoord{0}{\bullet}}{\twocoord{G}{I}} 
			& \transition{si_2}{\twocoord{1}{\bullet}}{\twocoord{G}{I}}
			&
			& \transition{f   }{\twocoord{B}{B}      }{\twocoord{I}{I}} \\
		\hline
		$sp_2$ 
			& \transition{si_2}{\twocoord{0}{\bullet}}{\twocoord{G}{I}} 
			& \transition{si_0}{\twocoord{1}{\bullet}}{\twocoord{G}{I}}
			&
			& \transition{f   }{\twocoord{B}{B}}{\twocoord{I}{I}}       \\
		\hline
		\hline
		$si_0$ 
			& \transition{sp_0}{\twocoord{0}{\bullet}}{\twocoord{G}{I}} 
			& \transition{sp_2}{\twocoord{1}{\bullet}}{\twocoord{G}{I}}
			& \transition{f   }{\twocoord{B}{B}      }{\twocoord{I}{I}}
			&                                                           \\
		\hline
		$si_1$ 
			& \transition{sp_1}{\twocoord{0}{\bullet}}{\twocoord{G}{I}} 
			& \transition{sp_0}{\twocoord{1}{\bullet}}{\twocoord{G}{I}}
			&
			& \transition{f   }{\twocoord{B}{B}      }{\twocoord{I}{I}} \\
		\hline
		$si_2$ 
			& \transition{sp_2}{\twocoord{0}{\bullet}}{\twocoord{G}{I}} 
			& \transition{sp_1}{\twocoord{1}{\bullet}}{\twocoord{G}{I}}
			&
			& \transition{f   }{\twocoord{B}{B}      }{\twocoord{I}{I}} \\
		\hline
	\end{tabular}
\end{center}




\subsubsection{Coder pour tester}

	Sur le site de téléchargement de ce document, dans le sous-dossier \verb+turing/palindrome+, se trouve le fichier \verb+palindrome.txt+ contenant un code utilisable
pour des tests manuels sur le site \url{https://turingmachinesimulator.com}.




\newpage
\section{Détecter les palindromes avec une seule bande}

	\subsection{Un exemple avec le mot abbca}

	Voici les grandes étapes présentées sur deux colonnes.


\begin{multicols}{2}

% GESTION DE LA LETTRE LA PLUS À DROITE

\emptybox\emptybox%
	\boxit{a}\boxit{b}\boxit{b}\boxit{c}\boxit{a}%
\emptybox\emptybox

\phantom{\emptybox\emptybox}%
	\head


\medskip % RECHERCHE DE LA DERNIÈRE LETTRE
\emptybox\emptybox%
	\fboxit{a}\boxit{b}\boxit{b}\boxit{c}\boxit{a}%
\emptybox\emptybox

\phantom{\emptybox\emptybox%
	\emptybox\emptybox\emptybox\emptybox}%
	\head


\medskip % BONNE LETTRE
\emptybox\emptybox%
	\fboxit{a}\boxit{b}\boxit{b}\boxit{c}\fboxit{a}%
\emptybox\emptybox

\phantom{\emptybox\emptybox%
	\emptybox\emptybox\emptybox\emptybox}%
	\head


\medskip % EFFACEMENT DES LETTRES
\emptybox\emptybox%
	\emptybox\boxit{b}\boxit{b}\boxit{c}\emptybox%
\emptybox\emptybox

\phantom{\emptybox\emptybox%
	\emptybox}%
	\head

\vfill\null
\columnbreak

\medskip % GESTION DE LA NOUVELLE LETTRE LA PLUS À DROITE
\emptybox\emptybox%
	\emptybox\fboxit{b}\boxit{b}\boxit{c}\emptybox%
\emptybox\emptybox

\phantom{\emptybox\emptybox%
	\emptybox}%
	\head


\medskip % RECHERCHE DE LA DERNIÈRE LETTRE
\emptybox\emptybox%
	\emptybox\fboxit{b}\boxit{b}\boxit{c}\emptybox%
\emptybox\emptybox

\phantom{\emptybox\emptybox%
	\emptybox\emptybox\emptybox}%
	\head


\medskip % MAUVAISE LETTRE
\emptybox\emptybox%
	\emptybox\fboxit{b}\boxit{b}\wboxit{c}\emptybox%
\emptybox\emptybox

\phantom{\emptybox\emptybox%
	\emptybox\emptybox\emptybox}%
	\head

\end{multicols}


\vspace{-1em}

Qu'a-t-on fait ?
\begin{enumerate}
	\item On note la lettre pointée par la tête de lecture.
	      Commence alors une phase de recherche d'une lettre connue.

	\item On avance tant que l'on ne rencontre par une case vide. Une fois celle-ci repérée on revient d'une case en arrière.
	
	\item On compare alors la lettre pointée par la tête de lecture avec celle de la phase de recherche en cours. Deux cas sont possibles.
	\begin{enumerate}
		\item Si les lettres sont différentes alors on ne fait plus rien. 
		      On a un état bloquant et le mot n'est pas validé.

		\item Si les lettres sont identiques alors on efface la lettre en cours puis on va vers la gauche jusqu'à la prochaine case vide. 
		      Une fois celle-ci trouvée, on avance d'une case vers la droite pour effacer son contenu, puis on avance d'une autre case vers la droite pour recommencer les actions à partir du point 1.
	\end{enumerate}
\end{enumerate}


La méthode ci-dessus est en fait incomplète comme nous allons le voir dans la section suivante avec un exemple de palindrome à repérer.




\subsection{Un exemple avec le mot abbcbba}

	On fait comme précédemment en devant tout parcourir !
Voici les grandes étapes.


\begin{multicols}{2}

\emptybox\emptybox%
	\boxit{a}\boxit{b}\boxit{b}\boxit{c}\boxit{b}\boxit{b}\boxit{a}%
\emptybox\emptybox

\phantom{%
	\emptybox\emptybox}%
	\head
	

\medskip % 1ER EFFACEEMENT

\emptybox\emptybox%
	\fboxit{a}\boxit{b}\boxit{b}\boxit{c}\boxit{b}\boxit{b}\fboxit{a}%
\emptybox\emptybox

\phantom{%
	\emptybox\emptybox%
	\emptybox\emptybox\emptybox\emptybox\emptybox\emptybox}%
	\head
	

\vfill\null
\columnbreak


\emptybox\emptybox%
	\emptybox\boxit{b}\boxit{b}\boxit{c}\boxit{b}\boxit{b}\emptybox%
\emptybox\emptybox

\phantom{%
	\emptybox\emptybox\emptybox}%
	\head
	

\medskip % 2IÈME EFFACEEMENT

\emptybox\emptybox%
	\emptybox\fboxit{b}\boxit{b}\boxit{c}\boxit{b}\fboxit{b}\emptybox%
\emptybox\emptybox

\phantom{%
	\emptybox\emptybox%
	\emptybox\emptybox\emptybox\emptybox\emptybox}%
	\head

\vfill\null
\end{multicols}


\begin{multicols}{2}

\emptybox\emptybox%
	\emptybox\emptybox\boxit{b}\boxit{c}\boxit{b}\emptybox\emptybox%
\emptybox\emptybox

\phantom{%
	\emptybox\emptybox\emptybox\emptybox}%
	\head


\medskip % 3IÈME EFFACEEMENT

\emptybox\emptybox%
	\emptybox\emptybox\fboxit{b}\boxit{c}\fboxit{b}\emptybox\emptybox%
\emptybox\emptybox

\phantom{%
	\emptybox\emptybox%
	\emptybox\emptybox\emptybox\emptybox}%
	\head
	

\medskip

\emptybox\emptybox%
	\emptybox\emptybox\emptybox\boxit{c}\emptybox\emptybox\emptybox%
\emptybox\emptybox

\phantom{%
	\emptybox\emptybox\emptybox\emptybox\emptybox}%
	\head
	

\vfill\null
\columnbreak


% DERNIER EFFACEEMENT

\emptybox\emptybox%
	\emptybox\emptybox\emptybox\fboxit{c}\emptybox\emptybox\emptybox%
\emptybox\emptybox

\phantom{%
	\emptybox\emptybox%
	\emptybox\emptybox\emptybox}%
	\head
	

\medskip

\emptybox\emptybox%
	\emptybox\emptybox\emptybox\emptybox\emptybox\emptybox\emptybox%
\emptybox\emptybox

\phantom{%
	\emptybox\emptybox\emptybox\emptybox\emptybox}%
	\head

\vfill\null
\end{multicols}




\subsection{La table des transitions} \label{duplicate-table}

	On cherche ici à faire une table pour un \emph{\og ou exclusif \fg}.
En utilisant l'identité booléenne $A \boolope{ OUEX } B \eq[id] (\boolope{NON } A \boolope{ ET } B) \boolope{ OU } (A \boolope{ ET NON } B)$, nous pouvons appliquer ce que nous avons utilisé précédemment pour traduire les opérateurs logiques $\boolope{ET}$, $\boolope{OU}$ et $\boolope{NON}$.
Bien que théoriquement correct, ce raisonnement automatique va nous conduire à une trop \myquote{grosse} table des transitions.


\medskip


Pour obtenir une table de taille raisonnable, nous allons ajouter une nouvelle bande.
Expliquons comment faire automatiquement via la table \myquote{optimisée} de la section \ref{2-or-3} \emph{(la méthode présentée est généralisable aux tables à plusieurs bandes)}.
\begin{enumerate}
	\item La bande supplémentaire va juste nous servir à \myquote{compter} les cas gagnants.

	\item Nous allons court-circuiter le 1\ier{} cas final pour aller à l'état $sp_0$  tout en gardant la trace du succès du 1\ier{} test.

	\item Il reste à gérer les états bloquants de la 2\ieme{} machine lorsque la 1\iere{} a fonctionné.
	      Dans l'optique d'utilisation séquentielle de machines, nous remettons à zéro la bande supplémentaire.
\end{enumerate}


\medskip

Nous utilisons $\twocoord{0}{\bullet}$ pour simplifier la table. Dans cette écriture, {\scriptsize$\bullet$} indique un caractère quelconque. Dans une transition, il indique le caractère initial qui est donc inchangé.


\begin{center}
	\emph{\small Phase 1 : a-t-on un pair via la bande du bas ?}
	
	\smallskip
	\begin{tabular}{|c||c|c|c|}
% MUTIPLE OF 3
		\hline
		$\delta$ 
			& $\twocoord{0}{B}$ 
			& $\twocoord{1}{B}$
			& $\twocoord{B}{B}$ \\
		\hline
		\hline
		$q_0$ 
			& \transition{\ell_0}{\twocoord{0}{B}}{\twocoord{D}{I}} 
			& \transition{\ell_1}{\twocoord{1}{B}}{\twocoord{D}{I}}
			&                                                       \\
		\hline
		$\ell_0$
			& \transition{\ell_0 }{\twocoord{0}{B}}{\twocoord{D}{I}} 
			& \transition{\ell_1 }{\twocoord{1}{B}}{\twocoord{D}{I}}
			& \transition{sp_0   }{\twocoord{B}{1}}{\twocoord{G}{I}} \\
		\hline
		$\ell_1$
			& \transition{\ell_0}{\twocoord{0}{B}}{\twocoord{D}{I}} 
			& \transition{\ell_1}{\twocoord{1}{B}}{\twocoord{D}{I}}
			& \transition{sp_0  }{\twocoord{B}{B}}{\twocoord{G}{I}} \\
		\hline
	\end{tabular}
\end{center}



\begin{center}
	\emph{\small Phase 2 : a-t-on un multiple de $3$ via la bande du bas ?}
	
	\smallskip
	\begin{tabular}{|c||c|c|c|c|}
% MUTIPLE OF 3
		\hline
		$\delta$ 
			& $\twocoord{0}{\bullet}$ 
			& $\twocoord{1}{\bullet}$
			& $\twocoord{B}{B}$       
			& $\twocoord{B}{1}$ \\
		\hline
		\hline
		$sp_0$ 
			& \transition{si_0}{\twocoord{0}{\bullet}}{\twocoord{G}{I}} 
			& \transition{si_1}{\twocoord{1}{\bullet}}{\twocoord{G}{I}}
			& \transition{f   }{\twocoord{B}{B}      }{\twocoord{I}{I}}
			&                                                           \\
		\hline
		$sp_1$ 
			& \transition{si_1}{\twocoord{0}{\bullet}}{\twocoord{G}{I}} 
			& \transition{si_2}{\twocoord{1}{\bullet}}{\twocoord{G}{I}}
			&
			& \transition{f   }{\twocoord{B}{B}      }{\twocoord{I}{I}} \\
		\hline
		$sp_2$ 
			& \transition{si_2}{\twocoord{0}{\bullet}}{\twocoord{G}{I}} 
			& \transition{si_0}{\twocoord{1}{\bullet}}{\twocoord{G}{I}}
			&
			& \transition{f   }{\twocoord{B}{B}}{\twocoord{I}{I}}       \\
		\hline
		\hline
		$si_0$ 
			& \transition{sp_0}{\twocoord{0}{\bullet}}{\twocoord{G}{I}} 
			& \transition{sp_2}{\twocoord{1}{\bullet}}{\twocoord{G}{I}}
			& \transition{f   }{\twocoord{B}{B}      }{\twocoord{I}{I}}
			&                                                           \\
		\hline
		$si_1$ 
			& \transition{sp_1}{\twocoord{0}{\bullet}}{\twocoord{G}{I}} 
			& \transition{sp_0}{\twocoord{1}{\bullet}}{\twocoord{G}{I}}
			&
			& \transition{f   }{\twocoord{B}{B}      }{\twocoord{I}{I}} \\
		\hline
		$si_2$ 
			& \transition{sp_2}{\twocoord{0}{\bullet}}{\twocoord{G}{I}} 
			& \transition{sp_1}{\twocoord{1}{\bullet}}{\twocoord{G}{I}}
			&
			& \transition{f   }{\twocoord{B}{B}      }{\twocoord{I}{I}} \\
		\hline
	\end{tabular}
\end{center}




\subsection{Coder pour tester}

	Sur le site de téléchargement de ce document, dans le sous-dossier \verb+turing/palindrome+, se trouve le fichier \verb+palindrome.txt+ contenant un code utilisable
pour des tests manuels sur le site \url{https://turingmachinesimulator.com}.




\newpage
\section{Détecter les palindromes avec deux bandes}

	\subsection{Un exemple avec le mot abbca}

	Voici les grandes étapes présentées sur deux colonnes.


\begin{multicols}{2}

% GESTION DE LA LETTRE LA PLUS À DROITE

\emptybox\emptybox%
	\boxit{a}\boxit{b}\boxit{b}\boxit{c}\boxit{a}%
\emptybox\emptybox

\phantom{\emptybox\emptybox}%
	\head


\medskip % RECHERCHE DE LA DERNIÈRE LETTRE
\emptybox\emptybox%
	\fboxit{a}\boxit{b}\boxit{b}\boxit{c}\boxit{a}%
\emptybox\emptybox

\phantom{\emptybox\emptybox%
	\emptybox\emptybox\emptybox\emptybox}%
	\head


\medskip % BONNE LETTRE
\emptybox\emptybox%
	\fboxit{a}\boxit{b}\boxit{b}\boxit{c}\fboxit{a}%
\emptybox\emptybox

\phantom{\emptybox\emptybox%
	\emptybox\emptybox\emptybox\emptybox}%
	\head


\medskip % EFFACEMENT DES LETTRES
\emptybox\emptybox%
	\emptybox\boxit{b}\boxit{b}\boxit{c}\emptybox%
\emptybox\emptybox

\phantom{\emptybox\emptybox%
	\emptybox}%
	\head

\vfill\null
\columnbreak

\medskip % GESTION DE LA NOUVELLE LETTRE LA PLUS À DROITE
\emptybox\emptybox%
	\emptybox\fboxit{b}\boxit{b}\boxit{c}\emptybox%
\emptybox\emptybox

\phantom{\emptybox\emptybox%
	\emptybox}%
	\head


\medskip % RECHERCHE DE LA DERNIÈRE LETTRE
\emptybox\emptybox%
	\emptybox\fboxit{b}\boxit{b}\boxit{c}\emptybox%
\emptybox\emptybox

\phantom{\emptybox\emptybox%
	\emptybox\emptybox\emptybox}%
	\head


\medskip % MAUVAISE LETTRE
\emptybox\emptybox%
	\emptybox\fboxit{b}\boxit{b}\wboxit{c}\emptybox%
\emptybox\emptybox

\phantom{\emptybox\emptybox%
	\emptybox\emptybox\emptybox}%
	\head

\end{multicols}


\vspace{-1em}

Qu'a-t-on fait ?
\begin{enumerate}
	\item On note la lettre pointée par la tête de lecture.
	      Commence alors une phase de recherche d'une lettre connue.

	\item On avance tant que l'on ne rencontre par une case vide. Une fois celle-ci repérée on revient d'une case en arrière.
	
	\item On compare alors la lettre pointée par la tête de lecture avec celle de la phase de recherche en cours. Deux cas sont possibles.
	\begin{enumerate}
		\item Si les lettres sont différentes alors on ne fait plus rien. 
		      On a un état bloquant et le mot n'est pas validé.

		\item Si les lettres sont identiques alors on efface la lettre en cours puis on va vers la gauche jusqu'à la prochaine case vide. 
		      Une fois celle-ci trouvée, on avance d'une case vers la droite pour effacer son contenu, puis on avance d'une autre case vers la droite pour recommencer les actions à partir du point 1.
	\end{enumerate}
\end{enumerate}


La méthode ci-dessus est en fait incomplète comme nous allons le voir dans la section suivante avec un exemple de palindrome à repérer.




\subsection{Un exemple avec le mot abbcbba}

	On fait comme précédemment en devant tout parcourir !
Voici les grandes étapes.


\begin{multicols}{2}

\emptybox\emptybox%
	\boxit{a}\boxit{b}\boxit{b}\boxit{c}\boxit{b}\boxit{b}\boxit{a}%
\emptybox\emptybox

\phantom{%
	\emptybox\emptybox}%
	\head
	

\medskip % 1ER EFFACEEMENT

\emptybox\emptybox%
	\fboxit{a}\boxit{b}\boxit{b}\boxit{c}\boxit{b}\boxit{b}\fboxit{a}%
\emptybox\emptybox

\phantom{%
	\emptybox\emptybox%
	\emptybox\emptybox\emptybox\emptybox\emptybox\emptybox}%
	\head
	

\vfill\null
\columnbreak


\emptybox\emptybox%
	\emptybox\boxit{b}\boxit{b}\boxit{c}\boxit{b}\boxit{b}\emptybox%
\emptybox\emptybox

\phantom{%
	\emptybox\emptybox\emptybox}%
	\head
	

\medskip % 2IÈME EFFACEEMENT

\emptybox\emptybox%
	\emptybox\fboxit{b}\boxit{b}\boxit{c}\boxit{b}\fboxit{b}\emptybox%
\emptybox\emptybox

\phantom{%
	\emptybox\emptybox%
	\emptybox\emptybox\emptybox\emptybox\emptybox}%
	\head

\vfill\null
\end{multicols}


\begin{multicols}{2}

\emptybox\emptybox%
	\emptybox\emptybox\boxit{b}\boxit{c}\boxit{b}\emptybox\emptybox%
\emptybox\emptybox

\phantom{%
	\emptybox\emptybox\emptybox\emptybox}%
	\head


\medskip % 3IÈME EFFACEEMENT

\emptybox\emptybox%
	\emptybox\emptybox\fboxit{b}\boxit{c}\fboxit{b}\emptybox\emptybox%
\emptybox\emptybox

\phantom{%
	\emptybox\emptybox%
	\emptybox\emptybox\emptybox\emptybox}%
	\head
	

\medskip

\emptybox\emptybox%
	\emptybox\emptybox\emptybox\boxit{c}\emptybox\emptybox\emptybox%
\emptybox\emptybox

\phantom{%
	\emptybox\emptybox\emptybox\emptybox\emptybox}%
	\head
	

\vfill\null
\columnbreak


% DERNIER EFFACEEMENT

\emptybox\emptybox%
	\emptybox\emptybox\emptybox\fboxit{c}\emptybox\emptybox\emptybox%
\emptybox\emptybox

\phantom{%
	\emptybox\emptybox%
	\emptybox\emptybox\emptybox}%
	\head
	

\medskip

\emptybox\emptybox%
	\emptybox\emptybox\emptybox\emptybox\emptybox\emptybox\emptybox%
\emptybox\emptybox

\phantom{%
	\emptybox\emptybox\emptybox\emptybox\emptybox}%
	\head

\vfill\null
\end{multicols}




\subsection{Une table des transitions} \label{duplicate-table}

	On cherche ici à faire une table pour un \emph{\og ou exclusif \fg}.
En utilisant l'identité booléenne $A \boolope{ OUEX } B \eq[id] (\boolope{NON } A \boolope{ ET } B) \boolope{ OU } (A \boolope{ ET NON } B)$, nous pouvons appliquer ce que nous avons utilisé précédemment pour traduire les opérateurs logiques $\boolope{ET}$, $\boolope{OU}$ et $\boolope{NON}$.
Bien que théoriquement correct, ce raisonnement automatique va nous conduire à une trop \myquote{grosse} table des transitions.


\medskip


Pour obtenir une table de taille raisonnable, nous allons ajouter une nouvelle bande.
Expliquons comment faire automatiquement via la table \myquote{optimisée} de la section \ref{2-or-3} \emph{(la méthode présentée est généralisable aux tables à plusieurs bandes)}.
\begin{enumerate}
	\item La bande supplémentaire va juste nous servir à \myquote{compter} les cas gagnants.

	\item Nous allons court-circuiter le 1\ier{} cas final pour aller à l'état $sp_0$  tout en gardant la trace du succès du 1\ier{} test.

	\item Il reste à gérer les états bloquants de la 2\ieme{} machine lorsque la 1\iere{} a fonctionné.
	      Dans l'optique d'utilisation séquentielle de machines, nous remettons à zéro la bande supplémentaire.
\end{enumerate}


\medskip

Nous utilisons $\twocoord{0}{\bullet}$ pour simplifier la table. Dans cette écriture, {\scriptsize$\bullet$} indique un caractère quelconque. Dans une transition, il indique le caractère initial qui est donc inchangé.


\begin{center}
	\emph{\small Phase 1 : a-t-on un pair via la bande du bas ?}
	
	\smallskip
	\begin{tabular}{|c||c|c|c|}
% MUTIPLE OF 3
		\hline
		$\delta$ 
			& $\twocoord{0}{B}$ 
			& $\twocoord{1}{B}$
			& $\twocoord{B}{B}$ \\
		\hline
		\hline
		$q_0$ 
			& \transition{\ell_0}{\twocoord{0}{B}}{\twocoord{D}{I}} 
			& \transition{\ell_1}{\twocoord{1}{B}}{\twocoord{D}{I}}
			&                                                       \\
		\hline
		$\ell_0$
			& \transition{\ell_0 }{\twocoord{0}{B}}{\twocoord{D}{I}} 
			& \transition{\ell_1 }{\twocoord{1}{B}}{\twocoord{D}{I}}
			& \transition{sp_0   }{\twocoord{B}{1}}{\twocoord{G}{I}} \\
		\hline
		$\ell_1$
			& \transition{\ell_0}{\twocoord{0}{B}}{\twocoord{D}{I}} 
			& \transition{\ell_1}{\twocoord{1}{B}}{\twocoord{D}{I}}
			& \transition{sp_0  }{\twocoord{B}{B}}{\twocoord{G}{I}} \\
		\hline
	\end{tabular}
\end{center}



\begin{center}
	\emph{\small Phase 2 : a-t-on un multiple de $3$ via la bande du bas ?}
	
	\smallskip
	\begin{tabular}{|c||c|c|c|c|}
% MUTIPLE OF 3
		\hline
		$\delta$ 
			& $\twocoord{0}{\bullet}$ 
			& $\twocoord{1}{\bullet}$
			& $\twocoord{B}{B}$       
			& $\twocoord{B}{1}$ \\
		\hline
		\hline
		$sp_0$ 
			& \transition{si_0}{\twocoord{0}{\bullet}}{\twocoord{G}{I}} 
			& \transition{si_1}{\twocoord{1}{\bullet}}{\twocoord{G}{I}}
			& \transition{f   }{\twocoord{B}{B}      }{\twocoord{I}{I}}
			&                                                           \\
		\hline
		$sp_1$ 
			& \transition{si_1}{\twocoord{0}{\bullet}}{\twocoord{G}{I}} 
			& \transition{si_2}{\twocoord{1}{\bullet}}{\twocoord{G}{I}}
			&
			& \transition{f   }{\twocoord{B}{B}      }{\twocoord{I}{I}} \\
		\hline
		$sp_2$ 
			& \transition{si_2}{\twocoord{0}{\bullet}}{\twocoord{G}{I}} 
			& \transition{si_0}{\twocoord{1}{\bullet}}{\twocoord{G}{I}}
			&
			& \transition{f   }{\twocoord{B}{B}}{\twocoord{I}{I}}       \\
		\hline
		\hline
		$si_0$ 
			& \transition{sp_0}{\twocoord{0}{\bullet}}{\twocoord{G}{I}} 
			& \transition{sp_2}{\twocoord{1}{\bullet}}{\twocoord{G}{I}}
			& \transition{f   }{\twocoord{B}{B}      }{\twocoord{I}{I}}
			&                                                           \\
		\hline
		$si_1$ 
			& \transition{sp_1}{\twocoord{0}{\bullet}}{\twocoord{G}{I}} 
			& \transition{sp_0}{\twocoord{1}{\bullet}}{\twocoord{G}{I}}
			&
			& \transition{f   }{\twocoord{B}{B}      }{\twocoord{I}{I}} \\
		\hline
		$si_2$ 
			& \transition{sp_2}{\twocoord{0}{\bullet}}{\twocoord{G}{I}} 
			& \transition{sp_1}{\twocoord{1}{\bullet}}{\twocoord{G}{I}}
			&
			& \transition{f   }{\twocoord{B}{B}      }{\twocoord{I}{I}} \\
		\hline
	\end{tabular}
\end{center}




\subsection{Coder pour tester}

	Sur le site de téléchargement de ce document, dans le sous-dossier \verb+turing/palindrome+, se trouve le fichier \verb+palindrome.txt+ contenant un code utilisable
pour des tests manuels sur le site \url{https://turingmachinesimulator.com}.




\newpage
\section{Deux fois plus de b que de a}

	\subsection{Un exemple avec le mot babbbabab}

	Commençons par un exemple simple qui va nous permettre de fixer les notations que nous allons utiliser.
Voyons comment repérer un entier naturel pair à partir de son écriture binaire.
La réponse est évidente : un naturel est pair si et seulement si son écriture binaire se finit par un zéro.
Il suffit donc de parcourir cette écriture binaire et d'analyser le chiffre le plus à droite. Ceci se schématise comme suit où \head{} indique la tête de lecture.

\begin{multicols}{2}

%

\emptybox\emptybox%
	\boxit{1}\boxit{0}\boxit{0}\boxit{1}\boxit{1}%
\emptybox\emptybox

\phantom{%
	\emptybox\emptybox}%
	\head


\medskip %

\emptybox\emptybox%
	\boxit{1}\boxit{0}\boxit{0}\boxit{1}\boxit{1}%
\emptybox\emptybox

\phantom{%
	\emptybox\emptybox
	\emptybox}%
	\head


\medskip %

\emptybox\emptybox%
	\boxit{1}\boxit{0}\boxit{0}\boxit{1}\boxit{1}%
\emptybox\emptybox

\phantom{%
	\emptybox\emptybox
	\emptybox\emptybox}%
	\head


\vfill\null
\columnbreak

\medskip %

\emptybox\emptybox%
	\boxit{1}\boxit{0}\boxit{0}\boxit{1}\boxit{1}%
\emptybox\emptybox

\phantom{%
	\emptybox\emptybox
	\emptybox\emptybox\emptybox}%
	\head


\medskip %

\emptybox\emptybox%
	\boxit{1}\boxit{0}\boxit{0}\boxit{1}\boxit{1}%
\emptybox\emptybox

\phantom{%
	\emptybox\emptybox
	\emptybox\emptybox\emptybox\emptybox}%
	\head


\medskip %

\emptybox\emptybox%
	\boxit{1}\boxit{0}\boxit{0}\boxit{1}\boxit{1}%
\emptybox\emptybox

\phantom{%
	\emptybox\emptybox
	\emptybox\emptybox\emptybox\emptybox\emptybox}%
	\head

\vfill\null
\end{multicols}

\vspace{-1em}

Pourquoi s'arrêter à la première case vide ? L'idée va être de garder en mémoire la valeur de la dernière case visitée.
Dans notre exemple, lors du dernier mouvement, on sait que la case précédente était un $1$ et donc que l'écriture binaire n'est pas celle d'un entier naturel pair.



\subsection{Une table des transitions}

	On cherche ici à faire une table pour un \emph{\og ou exclusif \fg}.
En utilisant l'identité booléenne $A \boolope{ OUEX } B \eq[id] (\boolope{NON } A \boolope{ ET } B) \boolope{ OU } (A \boolope{ ET NON } B)$, nous pouvons appliquer ce que nous avons utilisé précédemment pour traduire les opérateurs logiques $\boolope{ET}$, $\boolope{OU}$ et $\boolope{NON}$.
Bien que théoriquement correct, ce raisonnement automatique va nous conduire à une trop \myquote{grosse} table des transitions.


\medskip


Pour obtenir une table de taille raisonnable, nous allons ajouter une nouvelle bande.
Expliquons comment faire automatiquement via la table \myquote{optimisée} de la section \ref{2-or-3} \emph{(la méthode présentée est généralisable aux tables à plusieurs bandes)}.
\begin{enumerate}
	\item La bande supplémentaire va juste nous servir à \myquote{compter} les cas gagnants.

	\item Nous allons court-circuiter le 1\ier{} cas final pour aller à l'état $sp_0$  tout en gardant la trace du succès du 1\ier{} test.

	\item Il reste à gérer les états bloquants de la 2\ieme{} machine lorsque la 1\iere{} a fonctionné.
	      Dans l'optique d'utilisation séquentielle de machines, nous remettons à zéro la bande supplémentaire.
\end{enumerate}


\medskip

Nous utilisons $\twocoord{0}{\bullet}$ pour simplifier la table. Dans cette écriture, {\scriptsize$\bullet$} indique un caractère quelconque. Dans une transition, il indique le caractère initial qui est donc inchangé.


\begin{center}
	\emph{\small Phase 1 : a-t-on un pair via la bande du bas ?}
	
	\smallskip
	\begin{tabular}{|c||c|c|c|}
% MUTIPLE OF 3
		\hline
		$\delta$ 
			& $\twocoord{0}{B}$ 
			& $\twocoord{1}{B}$
			& $\twocoord{B}{B}$ \\
		\hline
		\hline
		$q_0$ 
			& \transition{\ell_0}{\twocoord{0}{B}}{\twocoord{D}{I}} 
			& \transition{\ell_1}{\twocoord{1}{B}}{\twocoord{D}{I}}
			&                                                       \\
		\hline
		$\ell_0$
			& \transition{\ell_0 }{\twocoord{0}{B}}{\twocoord{D}{I}} 
			& \transition{\ell_1 }{\twocoord{1}{B}}{\twocoord{D}{I}}
			& \transition{sp_0   }{\twocoord{B}{1}}{\twocoord{G}{I}} \\
		\hline
		$\ell_1$
			& \transition{\ell_0}{\twocoord{0}{B}}{\twocoord{D}{I}} 
			& \transition{\ell_1}{\twocoord{1}{B}}{\twocoord{D}{I}}
			& \transition{sp_0  }{\twocoord{B}{B}}{\twocoord{G}{I}} \\
		\hline
	\end{tabular}
\end{center}



\begin{center}
	\emph{\small Phase 2 : a-t-on un multiple de $3$ via la bande du bas ?}
	
	\smallskip
	\begin{tabular}{|c||c|c|c|c|}
% MUTIPLE OF 3
		\hline
		$\delta$ 
			& $\twocoord{0}{\bullet}$ 
			& $\twocoord{1}{\bullet}$
			& $\twocoord{B}{B}$       
			& $\twocoord{B}{1}$ \\
		\hline
		\hline
		$sp_0$ 
			& \transition{si_0}{\twocoord{0}{\bullet}}{\twocoord{G}{I}} 
			& \transition{si_1}{\twocoord{1}{\bullet}}{\twocoord{G}{I}}
			& \transition{f   }{\twocoord{B}{B}      }{\twocoord{I}{I}}
			&                                                           \\
		\hline
		$sp_1$ 
			& \transition{si_1}{\twocoord{0}{\bullet}}{\twocoord{G}{I}} 
			& \transition{si_2}{\twocoord{1}{\bullet}}{\twocoord{G}{I}}
			&
			& \transition{f   }{\twocoord{B}{B}      }{\twocoord{I}{I}} \\
		\hline
		$sp_2$ 
			& \transition{si_2}{\twocoord{0}{\bullet}}{\twocoord{G}{I}} 
			& \transition{si_0}{\twocoord{1}{\bullet}}{\twocoord{G}{I}}
			&
			& \transition{f   }{\twocoord{B}{B}}{\twocoord{I}{I}}       \\
		\hline
		\hline
		$si_0$ 
			& \transition{sp_0}{\twocoord{0}{\bullet}}{\twocoord{G}{I}} 
			& \transition{sp_2}{\twocoord{1}{\bullet}}{\twocoord{G}{I}}
			& \transition{f   }{\twocoord{B}{B}      }{\twocoord{I}{I}}
			&                                                           \\
		\hline
		$si_1$ 
			& \transition{sp_1}{\twocoord{0}{\bullet}}{\twocoord{G}{I}} 
			& \transition{sp_0}{\twocoord{1}{\bullet}}{\twocoord{G}{I}}
			&
			& \transition{f   }{\twocoord{B}{B}      }{\twocoord{I}{I}} \\
		\hline
		$si_2$ 
			& \transition{sp_2}{\twocoord{0}{\bullet}}{\twocoord{G}{I}} 
			& \transition{sp_1}{\twocoord{1}{\bullet}}{\twocoord{G}{I}}
			&
			& \transition{f   }{\twocoord{B}{B}      }{\twocoord{I}{I}} \\
		\hline
	\end{tabular}
\end{center}




\subsection{Avec une seule bande ?}

	Il a été facile de résoudre la problème avec trois bandes. Essayons de voir si l'on peut se limiter à une seule bande
\footnote{
	Théoriquement on sait que toute machine à $n$ bandes peut être traduite en une machine à une seule bande.
	Malheureusement le procédé ne produit pas forcément des machines qu'un humain aurait conçu tout seul donc nous allons laisser de côté ce procédé.
}
en n'utilisant pas la lettre supplémentaire X
\footnote{
	Là est le mini-défi.
}.
Voici les grandes lignes de la méthode.

\begin{multicols}{2}
%
\emptybox\emptybox%
	\wboxit{\text{b}}\fboxit{\text{a}}\wboxit{\text{b}}\wboxit{\text{b}}\wboxit{\text{b}}\fboxit{\text{a}}\wboxit{\text{b}}\fboxit{\text{a}}\wboxit{\text{b}}%
\emptybox\emptybox

\phantom{\emptybox\emptybox}%
	\head


\medskip %


\emptybox\emptybox%
	\fboxit{\text{a}}\fboxit{\text{a}}\fboxit{\text{a}}\wboxit{\text{b}}\wboxit{\text{b}}\wboxit{\text{b}}\wboxit{\text{b}}\wboxit{\text{b}}\wboxit{\text{b}}%
\emptybox\emptybox

\phantom{\emptybox\emptybox\emptybox\emptybox\emptybox\emptybox\emptybox\emptybox\emptybox\emptybox\emptybox}%
	\head


\medskip %


\emptybox\emptybox%
	\fboxit{\text{a}}\fboxit{\text{a}}\fboxit{\text{a}}\wboxit{\text{b}}\wboxit{\text{b}}\wboxit{\text{b}}\wboxit{\text{b}}\emptybox\emptybox%
\emptybox\emptybox

\phantom{\emptybox}%
	\head


\medskip %


\emptybox\emptybox%
	\emptybox\fboxit{\text{a}}\fboxit{\text{a}}\wboxit{\text{b}}\wboxit{\text{b}}\wboxit{\text{b}}\wboxit{\text{b}}\emptybox\emptybox%
\emptybox\emptybox

\phantom{\emptybox\emptybox\emptybox\emptybox\emptybox\emptybox\emptybox\emptybox\emptybox}%
	\head


\medskip %


\emptybox\emptybox%
	\emptybox\fboxit{\text{a}}\fboxit{\text{a}}\wboxit{\text{b}}\wboxit{\text{b}}\emptybox\emptybox\emptybox\emptybox%
\emptybox\emptybox

\phantom{\emptybox\emptybox}%
	\head


\medskip %


\emptybox\emptybox%
	\emptybox\emptybox\fboxit{\text{a}}\wboxit{\text{b}}\wboxit{\text{b}}\emptybox\emptybox\emptybox\emptybox%
\emptybox\emptybox

\phantom{\emptybox\emptybox\emptybox\emptybox\emptybox\emptybox\emptybox}%
	\head


\medskip %


\emptybox\emptybox%
	\emptybox\emptybox\fboxit{\text{a}}\emptybox\emptybox\emptybox\emptybox\emptybox%
\emptybox\emptybox

\phantom{\emptybox\emptybox\emptybox}%
	\head


\medskip %


\emptybox\emptybox%
	\emptybox\emptybox\emptybox\emptybox\emptybox\emptybox\emptybox\emptybox%
\emptybox\emptybox

\phantom{\emptybox\emptybox\emptybox\emptybox\emptybox}%
	\head

\end{multicols}


On aboutit alors à la table des transitions suivante où le plus délicat est la réorganisation des a et des b
\emph{(on effectue un tri à bulles)}.
L'état $g$ et ceux de type $s$ sont chargés de cette réorganisation, tandis que les états de type $e$ s'occupent de la procédure d'effacement.


\begin{center}
	\emph{\small Phase 1 : réorganisation \emph{(tri à bulles)}.}
	
	\smallskip
	\begin{tabular}{|c||c|c|c|}
% MUTIPLE OF 3
		\hline
		$\delta$ 
			& a 
			& b
			& $B$ \\
		\hline
		\hline
		$q_0$ 
			& \transition{s_a}{\text{a}}{D} 
			& \transition{s_b}{\text{b}}{D}
			&                        \\
		\hline
		$s_a$ 
			& \transition{s_a}{\text{a}}{D} 
			& \transition{s_b}{\text{b}}{D}
			& \transition{e_b}{B       }{G} \\
		\hline
		$s_b$ 
			& \transition{s^{\,\prime}_b}{\text{b}}{G}
			& \transition{s_b           }{\text{b}}{D}
			& \transition{e_b           }{B       }{G} \\
		\hline
		$s^{\,\prime}_b$ 
			&
			& \transition{s^{\,\prime\prime}_b}{\text{a}}{D}
			&                                         \\
		\hline
		$s^{\,\prime\prime}_b$ 
			& \transition{s^{\,\prime}_b       }{\text{b}}{G}
			& \transition{s^{\,\prime\prime}_b }{\text{b}}{D}
			& \transition{g                    }{\text{b}}{G} \\
		\hline
		$g$ 
			& \transition{g  }{\text{a}}{G}
			& \transition{g  }{\text{b}}{G}
			& \transition{q_0}{B       }{D} \\
		\hline
	\end{tabular}
\end{center}


\begin{center}

	\emph{\small Phase 2 : effacement.}
	
	\smallskip
	\begin{tabular}{|c||c|c|c|}
% MUTIPLE OF 3
		\hline
		$\delta$ 
			& a 
			& b
			& $B$ \\
		\hline
		\hline
		$e_b$ 
			&
			& \transition{e^{\,\prime}_b}{B}{G} 
			& \transition{f             }{B}{I} \\
		\hline
		$e^{\,\prime}_b$
			&
			& \transition{g_a}{B}{G} 
			&                        \\
		\hline
		$g_a$
			& \transition{g_a}{\text{a}}{G}
			& \transition{g_a}{\text{b}}{G}
			& \transition{e_a}{B       }{D} \\
		\hline
		$e_a$
			& \transition{d_b}{B}{D}
			& 
			&                        \\
		\hline
		$d_b$
			& \transition{d_b}{\text{a}}{D}
			& \transition{d_b}{\text{b}}{D}
			& \transition{e_b}{B       }{G} \\
		\hline
	\end{tabular}
\end{center}


\subsection{Coder pour tester}

	Sur le site de téléchargement de ce document, dans le sous-dossier \verb+turing/palindrome+, se trouve le fichier \verb+palindrome.txt+ contenant un code utilisable
pour des tests manuels sur le site \url{https://turingmachinesimulator.com}.



\end{document}
