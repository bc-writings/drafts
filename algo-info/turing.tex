  % !TEX encoding = UTF-8 Unicode
\documentclass[a4paper, 12pt]{scrartcl}

\usepackage[utf8]{inputenc}
\usepackage[T1]{fontenc}

\usepackage[top=2.5cm, bottom=2.5cm, left=1.95cm, right=1.95cm]{geometry}

\usepackage[french]{babel}
\usepackage{listings}

\usepackage[hidelinks]{hyperref}

\usepackage[fr, apmep]{lyxam}
\usepackage{tnsmath}
\usepackage{fancyvrb}
\usepackage{bera}
\usepackage{enumitem}
\usepackage{multicol}

\usepackage{lastpage}
\usepackage{xcolor}
\usepackage{graphicx}
\usepackage{setspace}

\usepackage{tikz}
\usetikzlibrary{arrows,positioning, calc} 

\usepackage{fancyvrb}


\usepackage[
    type={CC},
    modifier={by-nc-sa},
	version={4.0},
]{doclicense}


\newcommand\boxit[1]{\fbox{\makebox[.85em]{#1}\vphantom{$pX^M$}}}
\newcommand\fboxit[1]{\fcolorbox{black}{yellow}{\makebox[.85em]{#1}\vphantom{$pX^M$}}}
\newcommand\nboxit[1]{\fcolorbox{black}{lightgray}{\makebox[.85em]{#1}\vphantom{$pX^M$}}}
\newcommand\wboxit[1]{\fcolorbox{black}{red!50}{\makebox[.85em]{#1}\vphantom{$pX^M$}}}
\newcommand\noboxit[1]{\fcolorbox{white}{white}{\makebox[.85em]{#1}\vphantom{$pX^M$}}}

\newcommand\emptybox{\boxit{\phantom{A}}}
\newcommand\nemptybox{\nboxit{\phantom{A}}}
\newcommand\wemptybox{\wboxit{\phantom{A}}}

\newcommand\head{\noboxit{$\uparrow$}}
\newcommand\deah{\noboxit{$\downarrow$}}


\newcommand\boxedU{\boxit{\bfseries ?}}
\newcommand\boxedB{\boxit{\bfseries K}}
\newcommand\boxedW{\boxit{\bfseries W}}

\newcommand\fboxedB{\fboxit{\bfseries K}}
\newcommand\fboxedW{\fboxit{\bfseries W}}

\newcommand\nboxedB{\nboxit{\bfseries K}}
\newcommand\nboxedW{\nboxit{\bfseries W}}

\newcommand\wboxedB{\wboxit{\bfseries K}}
\newcommand\wboxedW{\wboxit{\bfseries W}}



\newcommand\twocoord[2]{{\scriptsize\begin{matrix}#1\\#2\end{matrix}}}
\newcommand\threecoord[3]{{\scriptsize\begin{matrix}#1\\#2\\#3\end{matrix}}}


\newcommand\transition[3]{%
	$\left(\, #1 \,, #2 \,, #3 \,\right)^{\vphantom{4}}$
}

\newcommand\myquote[1]{\emph{\og #1 \fg}}


\newcommand\boolope[1]{\text{\bfseries#1}}


\begin{document}

\title{BROUILLON - Quelques machines de Turing déterministes}
\author{Christophe BAL}
\date{7 Février 2020 -- 20 Mars 2020}

\maketitle

\begin{center}
	\itshape
	Document, avec son source \LaTeX, disponible sur la page
	
	\url{https://github.com/bc-writing/drafts}.
\end{center}


\bigskip


\begin{center}
	\hrule\vspace{.3em}
	{
		\fontsize{1.35em}{1em}\selectfont
		\textbf{Mentions \og légales \fg}
	}
			
	\vspace{0.45em}
	\doclicenseThis
	\hrule
\end{center}


\bigskip
\setcounter{tocdepth}{2}
\tableofcontents



\newpage
\section{Écriture binaire des naturels pairs}

	\input{turing/divisible-by-2.tex}


\newpage
\section{Écriture binaire des multiples de 3} \label{divisibility-by-3}

	\input{turing/divisible-by-3.tex}


\newpage
\section{Écriture binaire des naturels impairs ou de ceux non multiples de 3}

	\input{turing/not-divisible-by-2-or-3.tex}


\newpage
\section{Avec deux bandes}

	Dans cette section, \textbf{nous cherchons à proposer des méthodes généralistes automatiques mais non nécessairement optimales}
	\emph{(voir à ce sujet la section \ref{better-2-and-3})}.

	\input{turing/two-tapes.tex}


\newpage
\section{Détecter les palindromes avec une seule bande}

	\input{turing/palindrome.tex}



\newpage
\section{Détecter les palindromes avec deux bandes}

	\input{turing/palindrome-2-tapes.tex}



\newpage
\section{Deux fois plus de b que de a}

	\input{turing/twice-more-b-than-a.tex}


\end{document}
