\documentclass[12pt]{amsart}
\usepackage[T1]{fontenc}
\usepackage[utf8]{inputenc}

\usepackage[top=1.95cm, bottom=1.95cm, left=2.35cm, right=2.35cm]{geometry}

\usepackage{hyperref}
\usepackage{enumitem}
\usepackage{tcolorbox}
\usepackage{float}
\usepackage{cleveref}
\usepackage{multicol}
\usepackage{fancyvrb}
\usepackage{amsmath}
\usepackage[french]{babel}
\usepackage[
    type={CC},
    modifier={by-nc-sa},
	version={4.0},
]{doclicense}

\newcommand\floor[1]{\left\lfloor #1 \right\rfloor}

\usepackage{tnsmath}


\newtheorem{fact}{Fait}[section]
\newtheorem{example}{Exemple}[section]
\newtheorem{remark}{Remarque}[section]
\newtheorem*{proof*}{Preuve}

\setlength\parindent{0pt}

\floatstyle{boxed} 
\restylefloat{figure}


\DeclareMathOperator{\taille}{\text{\normalfont\texttt{taille}}}

\newcommand\sqseq[2]{\fbox{$#1$}_{\,\,#2}}


\DefineVerbatimEnvironment{rawcode}%
	{Verbatim}%
	{tabsize=4,%
	 frame=lines, framerule=0.3mm, framesep=2.5mm}
	 
	 
	 
\begin{document}

\title{BROUILLON - Sommer les carrés des chiffres d'un naturel}
\author{Christophe BAL}
\date{6 Juin 2018 -- 28 Mars 2019}

\maketitle

\begin{center}
	\itshape
	Document, avec son source \LaTeX, disponible sur la page
	
	\url{https://github.com/bc-writing/drafts}.
\end{center}


\bigskip


\begin{center}
	\hrule\vspace{.3em}
	{
		\fontsize{1.35em}{1em}\selectfont
		\textbf{Mentions \og légales \fg}
	}
			
	\vspace{0.45em}
	\doclicenseThis
	\hrule
\end{center}


\bigskip
\setcounter{tocdepth}{1}
\tableofcontents



\section{Faire une tête au carré à tous les entiers naturels}

Voici un procédé facile à faire à l'aide d'une calculatrice.
Considérons un entier naturel $n$, puis calculons la somme de ses chiffres élevés au carré. Ceci nous donne un nouveau naturel auquel on peut appliquer le même procédé. Voici deux exemples.


\begin{example}
	Pour $n = 19$, nous obtenons :
	\begin{itemize}[label=\textbullet]
		\item $1^2 + 9^2 = 82$
		\item $8^2 + 2^2 = 68$
		\item $6^2 + 8^2 = 100$
		\item $1^2 + 0^2 + 0^2 = 1$ $\rightarrow$ Rien de nouveau à attendre.
	\end{itemize}
\end{example}


\begin{example}
	Pour $n = 1\,234\,567\,890$, après $1^2 + 2^2 + 3^2 + 4^2 + 5^2 + 6^2 + 7^2 + 8^2 + 9^2 + 0^2 = 285$ nous obtenons :
	\vspace{-.7em}
	\begin{multicols}{2}
		\begin{itemize}[label=\textbullet]
			\item $2^2 + 8^2 + 5^2 = 93$
			\item $9^2 + 3^2 = 90$
			\item $9^2 + 0^2 = 81$
			\item $8^2 + 1^2 = 65$
			\item $6^2 + 5^2 = 61$
			\item $6^2 + 1^2 = 37$
			\item $3^2 + 7^2 = 58$
		\end{itemize}
		\columnbreak
		\begin{itemize}[label=\textbullet]
			\item $5^2 + 8^2 = 89$
			\item $8^2 + 9^2 = 145$
			\item $1^2 + 4^2 + 5^2 = 42$
			\item $4^2 + 2^2 = 20$
			\item $2^2 + 0^2 = 4$
			\item $4^2 = 16$ 
			\item $1^2 + 6^2 = 37$ $\rightarrow$ Dèjà rencontré.
		\end{itemize}
	\end{multicols}
\end{example}

Dans le 1\ier{} cas, au bout d'un moment le procédé ne produit que des $1$. Ce sera le cas dès que l'on commence avec une puissance de $10$. Le 2\ieme{} exemple montre que le mieux que l'on puisse espérer c'est que le procédé devienne périodique à partir d'un moment \emph{(on parle de phénomène ultimement périodique)}.


\medskip

On peut explorer le comportement de ce procédé sur plusieurs valeurs grâce à un programme. Voici un code possible écrit en Python qui prend un peu de temps pour vérifier que pour tous les naturels $n \in \ZintervalC{1}{10^6}$, le procédé devient ultimement périodique.

\begin{rawcode}
NMAX    = 10**6
MAXLOOP = 10**20

for n in range(1, NMAX + 1):
    nbloops = 0
    results = []

    while nbloops < MAXLOOP and n not in results:
        nbloops += 1
        results.append(n)
        n = sum(int(d)**2 for d in str(n))

    if n not in results:
        print(f"Test raté pour n = {n}.")

print(f"Tests finis.")
\end{rawcode}

\medskip

Il reste à voir ce qu'il se passe dans le cas général. La section qui suit démontre que pour tout naturel $n$, le procédé sera toujours ultimement périodique.

\label{conjecture}


\section{Une preuve}\label{proof}

Pour un naturel 
$\displaystyle      n = \left[ c_{d-1} c_{d-2} \cdots c_1 c_0 \right]_{10} 
\stackrel{\text{def}}{=} \sum_{k=0}^{d-1} c_k 10^k$
avec $c_{d-1} \neq 0$,
on pose
$\displaystyle sq(n) = \sum_{k=0}^{d-1} (c_k)^2$
et
$\taille(n) = d$ qui sera appelé \emph{\og taille de $n$ \fg}.

\medskip

Pour $(n \,; k) \in \NN^2$, on définit 
$  \sqseq{n}{0} = n$
et
$  \sqseq{n}{k} = sq^k(n)
\stackrel{\text{def}}{=} sq \,\circ sq \,\circ \cdots \,\circ sq(n)$ avec $(k-1)$ compositions si $k > 0$.
Autrement dit,
$\sqseq{n}{k+1} = sq \left( \, \sqseq{n}{k} \right)$.

\medskip

On note enfin
$V_n = \geneset{ \, \sqseq{n}{k} \, | \, k \in \NN }$
l'ensemble des valeurs prises par la suite $\left( \, \sqseq{n}{k} \right)_k$.



\medskip

\begin{fact}
	$\forall n \in \NN$, $sq(n) \leqslant 81 d$ où $d = \taille(n)$.
\end{fact}

\begin{proof*}
	Si $n = [c_{d-1} c_{d-2} \cdots c_1 c_0]_{10}$
	alors 
	$\displaystyle sq(n) = \sum_{k=0}^{d-1} (c_k)^2 \leqslant \sum_{k=0}^{d-1} 9^2 = 81 d $.
\end{proof*}



\medskip

\begin{fact}
	$\forall n \in \NN$, notant $d = \taille(n)$, nous avons les résultats suivants :
	
	\begin{enumerate}
		\item Si $d \geqslant 4$ alors $\taille(sq(n)) < \taille(n)$.
		
		\item Si $d \leqslant 3$ alors $\taille(sq(n)) \leqslant 3$.
	\end{enumerate}
\end{fact}

\begin{proof*}
	Notons que $n \geqslant 10^{d-1}$.
	Le comportement des fonctions $10^{x-1}$ et $81x$ sur $\RRsp$ assure l'existence d'un naturel $D$ tel que $\forall d \in \NN$, $d \geqslant D$ implique $10^{d-1} > 81d \geqslant sq(n)$. On a même beaucoup mieux : si $10^{D-1} > 81D \geqslant sq(n)$ alors $d \geqslant D$ implique $10^{d-1} > 81d \geqslant sq(n)$.
	
	\medskip
	
	Comme $10^3 > 81 \times 4$, nous avons sans effort le 1er point \emph{(rappelons que $10^k > n$ implique que $n$ admet au plus $(k-1)$ chiffres)}.
	
	\medskip
	
	Pour $d \leqslant 3$, le 2nd point découle de $sq(999) = 243$, $sq(99) = 162$ et $sq(9) = 81$.
\end{proof*}



\medskip

\begin{fact}
	$\forall n \in \NN$, l'ensemble $V_n$ est fini et donc la suite $\left( \, \sqseq{n}{k} \right)_{k \in \NN}$ est ultimement périodique, i.e. périodique à partir d'un certain rang.
\end{fact}

\begin{proof*}
	Le 2nd point dépend directement du 1er point via le principe des tiroirs et la définition récursive de la suite $\left( \, \sqseq{n}{k} \right)_k$.
	
	\medskip
	
	Pour le 1er point, il suffit de montrer que $V_n \subset \intervalC{0}{10^{\taille(n)}}$ pour $n \geqslant 4$ via une petite récurrence descendante finie, et pour $n \leqslant 3$ on a directement $V_n \subset \intervalC{0}{10^3}$.
\end{proof*}


\section{Coder - Étudier la \og période \fg{} d'un naturel}

Quand il ne se fige pas, le code suivant donne la \textit{\og période \fg} d'un naturel auquel on applique le procédé présenté dans la section \ref{conjecture}.

\begin{rawcode}
n     = 20181209
nmemo = n

results = []

while n not in results:
    results.append(n)
    n = sum(int(d)**2 for d in str(n))

print(f"{nmemo} a la période suivante :")
print(results[results.index(n):])

print()

before = results[:results.index(n)]

if before:
    print("Avant la 1ère période nous avons :")
    print(before)
else:
    print("On commence directement par la période.")
\end{rawcode}

\medskip

Le code précédent, où \verb+n = 20181209+, nous affiche :

\begin{rawcode}
20181209 a la période suivante :
[16, 37, 58, 89, 145, 42, 20, 4]

Avant la 1ère période nous avons :
[20181209, 155, 51, 26, 40]
\end{rawcode}


\medskip

\begin{figure}[t]
	\centering
	\includegraphics[scale=.9]{squares-digits/periods.png}
  	\caption{Histogramme des tailles des périodes}
	\label{histogram}
\end{figure}



\medskip

Amusons-nous maintenant à représenter un histogramme des tailles des \og périodes \fg{}
À l'adresse \url{https://github.com/bc-writing/drafts}, dans le dossier \texttt{squares-digits}, vous trouverez le fichier \texttt{squareint-sizeplots.py} qui été utilisé pour obtenir le graphique
\footnote{
	À la même adresse dans le dossier \texttt{squares-digits} se trouve l'image \texttt{befores.png} qui est un histogramme des nombres de termes calculés avant l'apparition de la 1\iere{} \emph{\og période \fg{}}.
}.
Le traitement des données a été amélioré pour éviter de refaire des calculs déjà rencontrés \emph{(pour plus de précisions, se reporter aux commentaires du code)}.
Le résultat est donné dans la figure \ref{histogram} \cpageref{histogram}.



\medskip

Le graphique est frappant ! En effet, il semblerait que l'on ait soit des périodes de taille $1$, penser à $0$ et $1$, soit des périodes de taille $8$ comme pour $37 - 58 - 89 - 145 - 42 - 20 - 4 - 16$.
Magie ou coïncidence ? Les résultats de la section \ref{proof}, dont nous allons reprendre les notations, vont nous permettre de le savoir.
Tout d'abord,  d'après le fait \ref{magicmajo}, nous avons $\taille(sq(n)) < \taille(n)$ dès que $\taille(n) \geqslant 4$, donc la périodicité n'arrivera que lorsque $\taille\left( \, \sqseq{n}{k} \right) \leqslant 3$.
De plus, nous savons aussi que $\taille(sq(n)) \leqslant 3$ dès que $\taille(n) \leqslant 3$.
Tout ceci nous permet d'analyser brutalement via un programme ce qu'il se passe pour les périodes des naturels appartenant à $\ZintervalC{0}{999}$. Nous pouvons pour cela utiliser le code suivant, qui n'est absolument pas optimisé mais fait le travail immédiatement.


\newpage

\begin{rawcode}
nmax = 999

periodsfound = []

for n in range(nmax + 1):
    results = []

    while n not in results:
        results.append(n)
        n = sum(int(d)**2 for d in str(n))

    period = results[results.index(n):]

    if period not in periodsfound:
        periodsfound.append(period)

for oneperiod in periodsfound:
    print(oneperiod)
\end{rawcode}



\medskip

Le code précédent nous fournit toutes les périodes possibles.


\newpage

\begin{rawcode}
[0]
[1]
[4, 16, 37, 58, 89, 145, 42, 20]
[37, 58, 89, 145, 42, 20, 4, 16]
[89, 145, 42, 20, 4, 16, 37, 58]
[16, 37, 58, 89, 145, 42, 20, 4]
[20, 4, 16, 37, 58, 89, 145, 42]
[58, 89, 145, 42, 20, 4, 16, 37]
[42, 20, 4, 16, 37, 58, 89, 145]
[145, 42, 20, 4, 16, 37, 58, 89]
\end{rawcode}


\medskip

Et là cela devient joli car nous notons au passage que trois types de périodes : 
\verb+[0]+, \verb+[1]+ et
\verb+[4, 16, 37, 58, 89, 145, 42, 20]+ avec toutes ses \emph{\og permutées circulaires \fg}.


\section{\texorpdfstring{Peut-on généraliser à un exposant $k \geqslant 3$ ?}%
		                {Peut-on généraliser à un exposant k >= 3 ?}}

Pour finir, nous allons analyser ce qu'il se passe s'il on somme à la puissance $p \geqslant 3$ au lieu d'élever au carré.
Nous reprenons des notations similaires à celles de la section \ref{proof}.
\begin{itemize}[label = \textbullet]
	\item Pour un naturel $n =  \left[ \, c_{d-1} c_{d-2} \cdots c_1 c_0 \, \right]_{10}$ avec $c_{d-1} \neq 0$,
	on pose
	$\displaystyle pw(n) = \sum_{k=0}^{d-1} (c_k)^p$
	et
	$\taille(n) = d$.
	
	\item Pour $(n \,; k) \in \NN^2$, on définit 
	$\sqseq{n}{0} = n$
	et
	$\sqseq{n}{k+1} = pw \left( \, \sqseq{n}{k} \right)$.
\end{itemize}

 

\bigskip

\begin{fact}
	$\forall n \in \NN$, $pw(n) \leqslant 9^p \, d$ où $d = \taille(n)$.
\end{fact}

\begin{proof*}
	Si $n = \left[ \, c_{d-1} c_{d-2} \cdots c_1 c_0 \, \right]_{10}$
	alors 
	$\displaystyle pw(n) = \sum_{k=0}^{d-1} (c_k)^p \leqslant \sum_{k=0}^{d-1} 9^p = 9^p \, d $.
\end{proof*}




\medskip

\begin{fact}\label{magicmajo}
	Il existe $d_0 \in \NN$ tel qu'on ait : $\forall n \in \NN$,	
	$[ \, \taille(n) \geqslant d_0 \Rightarrow n > pw(n) \, ]$ .
	
	\smallskip
	On peut en fait choisir $d_0 = 3 + \floor{\dfrac{1}{\, \ln 10 \,} \ln \left( \dfrac{ \, 9^p \, d \, }{\, \ln 10 \,} \right)}$
	où $\floor{x}$ désigne la partie entière de $x$.
\end{fact}

\begin{proof*}
	Notons $d = \taille(n)$ , de sorte que $n \geqslant 10^{d-1}$.
	Compte tenu du fait précédent, nous cherchons à comparer $10^{d-1}$ et $9^p \, d$.
	Comme dans le cas $p = 2$ démontré dans la section \ref{proof}, on considère sur $\RRp$ la fonction $\Delta(x) = 10^{x-1} - 9^p \, d \, x$
	qui vérifie $\Delta^\prime(x) > 0$ 
	si et seulement si
	$x > 1 + \dfrac{1}{\, \ln 10 \,} \ln \left( \dfrac{ \, 9^p \, d \, }{\, \ln 10 \,} \right)$.
	
	
	\medskip
	
	Notant $\alpha$ le réel précédent, nous savons donc que $n \geqslant 10^{d-1} > 9^p \, d \geqslant pw(n)$ ,
	puis $n > pw(n)$ dès que $d > \alpha$.
	Cette dernière condition est vérifiée dès que $d \geqslant d_0$ avec le $d_0$ proposé un peu plus haut.
\end{proof*}



\medskip

\begin{fact}\label{beautifulproof}
	$\forall n \in \NN$, la suite $\left( \, \sqseq{n}{k} \right)_{k \in \NN}$ est ultimement périodique.
\end{fact}

\begin{proof*}
	Tout est en fait contenu dans le fait \ref{magicmajo}, dont on reprend la définition de $d_0$. Expliquons pourquoi.
	\begin{itemize}[label = \textbullet]
		\item Le fait \ref{magicmajo} donne l'existence d'un indice $k_0 \in \NN$ tel que $\taille\left( \, \sqseq{n}{k_0} \right) < d_0$ \emph{(dans le cas contraire, on pourrait construire une suite strictement décroissante de naturels)}.

		\item Si $\forall k \in \ZintervalCO{k_0}{+\infty}$, $\taille\left( \, \sqseq{n}{k} \right) < d_0$ , nous avons l'ultime périodicité via le principe des tiroirs \emph{(si besoin revoir la fin de la section \ref{proof})}.

		\item Sinon il existe $k^\prime_0 \in \ZintervalO{k_0}{+\infty}$ tel que $\taille\left( \, \sqseq{n}{k^\prime_0} \right) \geqslant d_0$. Comme dans le premier point, nous pouvons alors trouver $k_1 \in \ZintervalO{k^\prime_0}{+\infty}$ tel que $\taille\left( \, \sqseq{n}{k_1} \right) < d_0$.
		
		\item En répétant notre raisonnement,
		on peut aboutir à une situation similaire au 2\ieme{} point, et c'est gagné. 
		
		\noindent
		Sinon on arrive à construire une suitre strictement croissante $\left( k_i \right)_i$ d'indices tels que $\forall i \in \NN$, $\taille\left( \, \sqseq{n}{k_i} \right) < d_0$. Le principe des tiroirs s'applique ici aussi !
	\end{itemize}
\end{proof*}



\medskip

\begin{remark}
	La preuve précédente montre que pour rechercher toutes les périodes il \emph{\og suffit \fg} d'étudier les naturels appartenant à $\ZintervalC{0}{10^{d_0} - 1}$.
\end{remark}





\bigskip

\hrule

\section{AFFAIRE À SUIVRE...}

\bigskip

\hrule


%\section{Quelles périodes déterminées informatiquement}
%
%
{\Huge ????} La preuve du fait \ref{beautifulproof} permet de justifier que le programme suivant, pour peu qu'il ne se fige pas, fournit toutes les périodes possibles pour un exposant $p \geqslant 3$ connu.


n = 3
\begin{rawcode}
[0]
[1]
[55, 250, 133]
[136, 244]
[153]
[160, 217, 352]
[370]
[371]
[407]
[919, 1459]
\end{rawcode}


n = 4
\begin{rawcode}
[0]
[1]
[1138, 4179, 9219, 13139, 6725, 4338, 4514]
[1634]
[2178, 6514]
[8208]
[9474]
\end{rawcode}


n = 5
\begin{rawcode}

\end{rawcode}



\end{document}
