% Exemple code LUA pour prolonger : top utilisation des opérateurs !!!!




\documentclass[12pt]{amsart}
\usepackage[T1]{fontenc}
\usepackage[utf8]{inputenc}

\usepackage[top=1.95cm, bottom=1.95cm, left=2.35cm, right=2.35cm]{geometry}

\usepackage{hyperref}
\usepackage{enumitem}
\usepackage{tcolorbox}
\usepackage{multicol}
\usepackage{fancyvrb}
\usepackage{xstring}
\usepackage{amsmath}
\usepackage[french]{babel}
\usepackage[
    type={CC},
    modifier={by-nc-sa},
	version={4.0},
]{doclicense}
\usepackage{textcomp}
\usepackage{xcolor}
\usepackage{tcolorbox}
\usepackage{pgffor}
      
\usepackage{lymath}
\usepackage{circledsteps}
    
\newtheorem{fact}{Fait}%[section]


\newtheorem{definition}{Définition}%[section]
\newtheorem{theorem}{Théorème}%[section]
\newtheorem{example}{Exemple}%[section]
\newtheorem{exercise}{Exercice}%[section]
\newtheorem{remark}{Remarque}%[section]
\newtheorem*{notations}{Notations}

\setlength\parindent{0pt}




\DeclareMathMacro{\ediv}{\operatorname{\div}}


\newcommand\centerit[1]{%%
	\smallskip
	
	\begin{center}
		#1
	\end{center}
}


\newcommand\sign[1]{\fbox{\scriptsize $#1$}}
\newcommand\phantomsign{\phantom{\sign{-}}}


% Source pour le \@tfor :
% 	* https://tex.stackexchange.com/a/253205/6880
\makeatletter
	\newcommand\equality{\@ifstar{\@equality@star@}{\@equality@no@star@}}

	\newcommand\@equality@star@[1]{
		\smallskip
	
		$=$ {} \qquad  $\leftarrow$ #1
	
		\smallskip
	}

	\newcommand\@equality@no@star@{
		\smallskip
	
		$=$

		\smallskip
	}

% Nothing special for the cell...	
	\newcommand\elasticbox[1]{%
		\fcolorbox{black}{white}{\normalfont #1}%
	}

	\newcommand\fixedboxcolored[3]{%
		\fcolorbox{#1}{#2}{\makebox[1em]{\normalfont\vphantom{?}#3}}%
	}

	\newcommand\fixedbox[1]{%
		\fixedboxcolored{black}{white}{\makebox[1em]{\normalfont\vphantom{?}#1}}%
	}

% ... will move
	\newcommand\elasticboxwill[1]{%
		\fcolorbox{black}{red!15}{\normalfont #1}%
	}

	\newcommand\fixedboxwill[1]{%
		\fixedboxcolored{black}{red!15}{\normalfont #1}%
	}

% ... has moved
	\newcommand\elasticboxhas[1]{%
		\fcolorbox{black}{black!10}{\normalfont #1}%
	}
	
	\newcommand\fixedboxhas[1]{%
		\fixedboxcolored{black}{black!10}{\normalfont #1}%
	}

% Let's go !
	\newcommand\onefortwocomplemen{\fixedboxcolored{black}{green!15}{\bfseries 1}}


\newcommand\zerofortwocomplemen{\fixedboxcolored{black}{green!15}{\bfseries 0}}


	\newcommand\twocomplement{\fixedboxwill{$\bullet$}}

	\newcommand\@empty@content{\phantom{?}}
	\newcommand\emptycell{\fixedbox{\@empty@content}}
	\newcommand\one{\fixedbox{\bfseries 1}}
	\newcommand\zero{\fixedbox{\bfseries 0}}
	
	\newcommand\@ellipsis@{\,\,\vphantom{?}$\cdots$\,\,}
	\newcommand\myellipsis{\elasticbox{\@ellipsis@}}

% ... will move
	\newcommand\emptycellwill{\fixedboxwill{\@empty@content}}
	\newcommand\onewill{\fixedboxwill{\bfseries 1}}
	\newcommand\zerowill{\fixedboxwill{\bfseries 0}}
	\newcommand\myellipsiswill{\elasticboxwill{\@ellipsis@}}
	
% ... has moved
	\newcommand\emptycellhas{\fixedboxhas{\@empty@content}}
	\newcommand\onehas{\fixedboxhas{\bfseries 1}}
	\newcommand\zerohas{\fixedboxhas{\bfseries 0}}
	\newcommand\myellipsishas{\elasticboxhas{\@ellipsis@}}


% 0 : 0 normal
% 1 : 1 normal
% . : vide normal
% - : ellipsis
%
% Z : 0 qui va bouger
% U : 1 qui va bouger
% v : vide qui va bouger
% = : ellipsis qui va bouger
%
% z : 0 qui vient de bouger
% u : 1 qui vient de bouger
% a : vide qui vient de bouger
% + : ellipsis
%
%
% * : 1 pour complément à 2 + 1
% o : complément à 2 + 1
%
% < Ouvre un cadre de surlignement de plusieurs cellules
% < Ferme un cadre de surlignement de plusieurs cellules
	\newcommand\binary[1]{%
		\@tfor\next:=#1\do{%
			\if\next 0\zero{}\else%
		    \if\next 1\one{}\else%
			\if\next .\emptycell{}\else%
			\if\next -\myellipsis{}\else%
			%
			\if\next Z\zerowill{}\else%
		    \if\next U\onewill{}\else%
		    \if\next v\emptycellwill{}\else%
			\if\next =\myellipsiswill{}\else%
		    %
			\if\next z\zerohas{}\else%
		    \if\next u\onehas{}\else%
		    \if\next a\emptycellhas{}\else%
			\if\next +\myellipsishas{}\else%
			%
			\if\next *\onefortwocomplemen{}\else%
			\if\next c\zerofortwocomplemen{}\else%
			\if\next o\twocomplement{}\else%
		    %
\quad {\bfseries [[ illegal character : \, {\normalfont \next{}} \, ]]} \quad % BUG
			\fi%  :
			\fi%  a
			\fi%  u
			\fi%  p
			%
			\fi%  =
			\fi%  v
			\fi%  n
			\fi%  b
			%
			\fi%  -
			\fi%  .
			\fi%  N
			\fi%  B
			%
			\fi%  *
			\fi%  c
			\fi%  o
		}%
	}
	

	\newcommand\autobox[1]{%
      \foreach \i in {1, ..., #1} {%
        \fixedbox{\i}%
      }
	}
	

	\newcommand\binaryplus{\@ifstar{\@binaryplus@star@}{\@binaryplus@no@star@}}

	\newcommand\@binaryplus@no@star@[2]{
		\binary{#1}
		\setbox0=\hbox{#2\unskip}\ifdim\wd0=0pt \else {\small \ $\leftarrow$ #2}\fi
	}

	\newcommand\@binaryplus@star@[2]{
		\@binaryplus@no@star@{#1}{#2}
		
		\StrLen{#1}[\@length@]
		\noindent  
		\autobox{\@length@}
	}

	\newcommand\listbox[1]{%
		\@tfor\next:=#1\do{\fixedbox{\next}}%
	}
\makeatother


 
\begin{document}

\title{BROUILLON - CANDIDAT - Quels décalages entre les entiers signés et les non signés ?}
\author{Christophe BAL}
\date{17 Mars 2019 - 11 Janvier 2020}

\maketitle

\begin{center}
	\itshape
	Document, avec son source \LaTeX, disponible sur la page
	
	\url{https://github.com/bc-writing/drafts}.
\end{center}


\bigskip


\begin{center}
	\hrule\vspace{.3em}
	{
		\fontsize{1.35em}{1em}\selectfont
		\textbf{Mentions \og légales \fg}
	}
			
	\vspace{0.45em}
	\doclicenseThis
	\hrule
\end{center}

	
\bigskip
\setcounter{tocdepth}{2}
\tableofcontents



% ----------------- %


\section{Entiers non signés}

\input{shift-this-int-right/unsigned-integers}


% ----------------- %


\section{Entiers signés}

\subsection{Comment ça marche ?}

\input{shift-this-int-right/signed-integers/recipe}


\subsection{Qu'est-ce qui motive ces choix ?}

\input{shift-this-int-right/signed-integers/intuition}


\subsection{Pourquoi ça marche ?}

Rappelons que $\setgeo{H} : y = \frac{1}{x}$ , $E(1 ; 1)$ , $A \left( a ; \frac{1}{a} \right)$ , $B \left( b ; \frac{1}{b} \right)$ et $P \left( p ; \frac{1}{p} \right)$ où $p = a b$ avec $(a ; b) \in \left( \RRs \right)^2$ .



\bigskip

\textbf{Cas 1.} \emph{Supposons que $x_A x_B = 1$ .}

\medskip

Il est clair que $P = E$ dans ce cas.



\bigskip

\textbf{Cas 2.} \emph{Supposons que $x_A x_B \neq 1$ et $A \neq B$ .}

\medskip

La droite $(AB)$ a pour pente
$\frac{y_A - y_B}{x_A - x_B} = \frac{1}{a - b} \left( \frac{1}{a} - \frac{1}{b} \right) = - \frac{1}{ab}$ .
De plus, la droite $(EP)$ , qui existe car $p \neq 1$ , a pour pente
$\frac{y_P - y_E}{x_P - x_E} = \frac{1}{p - 1} \left( \frac{1}{p} - 1 \right) = - \frac{1}{p} = - \frac{1}{ab}$ .
Les droites $(AB)$ et $(EP)$ sont bien parallèles comme nous l'avons affirmé.



\bigskip

\textbf{Cas 3.} \emph{Supposons que $x_A x_B \neq 1$ et $A = B$ .}

\medskip

Comme ici $p = a^2 \neq 1$ . la droite $(EP)$ a pour pente $\left( - \frac{1}{a^2} \right)$ qui est aussi la pente de la tangente en $A \left( a ; \frac{1}{a} \right)$ à l'hyperbole $\setgeo{H}$ comme annoncé. 


% ----------------- %


\section{Division par $2$ et décalage à droite}

\subsection{Entiers non signés}

\input{shift-this-int-right/shift-right/signed}


\subsection{Entiers signés}

\input{shift-this-int-right/shift-right/unsigned}


\subsection{Synthèse}

\input{shift-this-int-right/shift-right/synthesis}


% ----------------- %


\section{Multiplication par $2$ et décalage à gauche}

\subsection{Entiers non signés}

\input{shift-this-int-right/shift-left/signed}


\subsection{Entiers signés}
\input{shift-this-int-right/shift-left/unsigned}


\subsection{Synthèse}
\input{shift-this-int-right/shift-left/synthesis}


\end{document}
