\documentclass[12pt]{amsart}
\usepackage[T1]{fontenc}
\usepackage[utf8]{inputenc}

\usepackage[top=1.95cm, bottom=1.95cm, left=2.35cm, right=2.35cm]{geometry}

\usepackage{amsmath}
\usepackage[french]{babel}
\usepackage{lymath}

\DeclareMathOperator{\taille}{\tau}

\newtheorem{fact}{Fait}
\newtheorem*{proof*}{Preuve}

\setlength\parindent{0pt}


\begin{document}

\title{BROUILLON - Racines rationnelles d'un polynôme symétrique de degré 4}
\author{Christophe BAL}
\date{6 Décembre 2018}
\maketitle

$P(X) = a X^4 + b x^3 + c X^2 + b X + a$, un polynôme symétrique de degré 4, peut-il n'avoir que des racines entières ? Que des racines rationnelles ?



\section{Constatations générales}

On peut supposer que $a = 1$, i.e. $P(X) = X^4 + b X^3 + c X^2 + b X + 1$.

Dès lors si $P(r) = 0$ alors $r \neq 0$ et $P\left( \dfrac1r \right) = 0$.

Ensuite, nous avons :

\medskip

$P(X) = X^4 P\left( \dfrac1X \right)$

\medskip

$P\,^{\prime}(X) = 4 X^3 P\left( \dfrac1X \right) 
            - X^2 P\,^{\prime}\left( \dfrac1X \right)$

%\medskip
%
%$P\,^{\prime\prime}(X) = 12 X^2 P\left( \dfrac1X \right)  - 4 X P\,^{\prime}\left( \dfrac1X \right)
%		    - 2 X P\,^{\prime}\left( \dfrac1X \right) + P\,^{\prime\prime}\left( \dfrac1X \right)$
%
%$P\,^{\prime\prime}(X) = 12 X^2 P\left( \dfrac1X \right)  
%            - 6 X P\,^{\prime}\left( \dfrac1X \right) 
%            + P\,^{\prime\prime}\left( \dfrac1X \right)$

On en déduit que si $r$ est une racine d'ordre au moins $2$, il en est de même pour $\dfrac1r$.


\section{Uniquement des racines entières ?}

Si $P$ n'a que des racines entières, alors ces racines ne peuvent être que $\pm 1$ qui sont les seuls entiers ayant un inverse entier. Ceci donne les uniques  possibilités suivantes :

\begin{enumerate}
	\item $P(X) = (X + 1)^4 = X^4 + 4 X^3 + 6 X^2 + 4 X + 1$ : ce polynôme est ok.

	\item $P(X) = (X - 1)^4 = X^4 - 4 X^3 + 6 X^2 - 4 X + 1$ : ce polynôme est ok.
	
	\item $P(X) = (X + 1)^3 (X - 1) = X^4 + 2 X^3 - 2 X - 1$ : on rejette. Notons que l'on a un polynôme anti-symétrique.
	
	\item $P(X) = (X - 1)^3 (X + 1) = X^4 - 2 X^3 + 2 X - 1$ : on rejette. Notons que l'on a un polynôme anti-symétrique.

	\item $P(X) = (X + 1)^2 (X - 1)^2 = X^4 - 2 X^2 + 1$ : ce polynôme est ok.
\end{enumerate}


\section{Uniquement des racines rationnelles ?}

Supposons que $r \in \QQ - \NN$ soit une racine de $P$.

\medskip

Le résultat sur la multiplicité supérieure ou égale à $2$ nous donne que si $r$ est de multiplicité au moins $2$ alors $\dfrac1r \neq r$ est aussi de multiplicité au moins $2$.
Ceci implique que $r$ est de multiplicité $1$ ou $2$.



\subsection*{$r$ est de multiplicité $1$}

Si $P$ admet une autre racine $s \in \QQ - \NN$ avec $s \neq r$ et $s \neq \dfrac1r$ alors nécessairement $P(X) = (X - r) \left( X - \dfrac1r \right) (X - s) \left( X - \dfrac1s \right)$.


D'où
$P(X) = X^4
      - \left( r + \dfrac1r + s + \dfrac1s \right) X^3
      + \left( 2 + \dfrac{s}{r} + \dfrac{r}{s} + r s + \dfrac{1}{r s} \right) X^2
      - \left( r + \dfrac1r + s + \dfrac1s \right) X
      + 1$

Ce polynôme est ok (utilisation d'un logiciel de calcul formel par flemme !).


\bigskip

Il reste à étudier les cas suivants.

\begin{enumerate}
	\item $P(X) = (X - r) \left( X - \dfrac1r \right) (X + 1)^2$ et $P(X) = (X - r) \left( X - \dfrac1r \right) (X - 1)^2$ sont ok car il suffit de reprendre le calcul formel précédent avec $s = \pm 1$.
	

	\item $P(X) = (X - r) \left( X - \dfrac1r \right) (X + 1) (X - 1)
	            = X^4
	            - \left( r + \dfrac1r \right) X^3 
	            + \left( r + \dfrac1r \right) X
	            - 1$ : on rejette. Notons que l'on a un polynôme anti-symétrique.
\end{enumerate}


\subsection*{$r$ est de multiplicité $2$}

Dans ce cas, $P(X) = (X - r)^2 \left( X - \dfrac1r \right)^2$ nécessairement !


$P(X) = (X - r)^2 \left( X - \dfrac1r \right)^2 
      = X^4
      - \left( 2 r + \dfrac2r \right) X^3 
      + \left( 4 + r^2 + \dfrac{1}{r^2} \right) X^2 
      - \left( 2 r + \dfrac2r \right) X
      + 1$
      
Ce polynôme est ok (utilisation d'un logiciel de calcul formel par flemme !).



\section{Problèmes similaires}

Que se passe-t-il pour un polynôme anti-symétrique $P(X) = a X^4 + b x^3 - b X - a$ ?

\medskip

Comment généraliser à d'autres degrés ?

\medskip

Et surtout, blanquette de veau ou moussaka ?

\end{document}
