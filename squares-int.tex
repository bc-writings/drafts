\documentclass[12pt]{amsart}
\usepackage[T1]{fontenc}
\usepackage[utf8]{inputenc}

\usepackage[top=1.95cm, bottom=1.95cm, left=2.35cm, right=2.35cm]{geometry}

\usepackage{hyperref}
\usepackage{enumitem}
\usepackage{multicol}
\usepackage{fancyvrb}
\usepackage{amsmath}
\usepackage[french]{babel}

\usepackage{lymath}


\newtheorem{fact}{Fait}
\newtheorem{example}{Exemple}
\newtheorem*{proof*}{Preuve}

\setlength\parindent{0pt}


\DeclareMathOperator{\taille}{\text{\normalfont\texttt{taille}}}

\newcommand\sqseq[2]{\fbox{$#1$}_{\,\,#2}}


\DefineVerbatimEnvironment{rawcode}%
	{Verbatim}%
	{tabsize=4,%
	 frame=lines, framerule=0.3mm, framesep=2.5mm}
	 
	 
	 
\begin{document}

\title{BROUILLON - Sommer les carrés des chiffres d'un naturel}
\author{Christophe BAL}
\date{6 Juin 2018 -- 8 Déc. 2018}
\maketitle

\setcounter{tocdepth}{1}
\tableofcontents


\section{Faire une tête au carré à tous les entiers naturels}

Voici un procédé facile à faire à l'aide d'une calculatrice.
Considérons un entier naturel $n$, puis calculons la somme de ses chiffres élevés au carré. Ceci nous donne un nouveau naturel auquel on peut appliquer le même procédé. Voici deux exemples.


\begin{example}
	Pour $n = 19$, nous obtenons :
	\begin{itemize}[label=\textbullet]
		\item $1^2 + 9^2 = 82$
		\item $8^2 + 2^2 = 68$
		\item $6^2 + 8^2 = 100$
		\item $1^2 + 0^2 + 0^2 = 1$ $\rightarrow$ Rien de nouveau à attendre.
	\end{itemize}
\end{example}


\begin{example}
	Pour $n = 1\,234\,567\,890$, après $1^2 + 2^2 + 3^2 + 4^2 + 5^2 + 6^2 + 7^2 + 8^2 + 9^2 + 0^2 = 285$ nous obtenons :
	\vspace{-.7em}
	\begin{multicols}{2}
		\begin{itemize}[label=\textbullet]
			\item $2^2 + 8^2 + 5^2 = 93$
			\item $9^2 + 3^2 = 90$
			\item $9^2 + 0^2 = 81$
			\item $8^2 + 1^2 = 65$
			\item $6^2 + 5^2 = 61$
			\item $6^2 + 1^2 = 37$
			\item $3^2 + 7^2 = 58$
		\end{itemize}
		\columnbreak
		\begin{itemize}[label=\textbullet]
			\item $5^2 + 8^2 = 89$
			\item $8^2 + 9^2 = 145$
			\item $1^2 + 4^2 + 5^2 = 42$
			\item $4^2 + 2^2 = 20$
			\item $2^2 + 0^2 = 4$
			\item $4^2 = 16$ 
			\item $1^2 + 6^2 = 37$ $\rightarrow$ Dèjà rencontré.
		\end{itemize}
	\end{multicols}
\end{example}

Dans le 1\ier{} cas, au bout d'un moment le procédé ne produit que des $1$. Ce sera le cas dès que l'on commence avec une puissance de $10$. Le 2\ieme{} exemple montre que le mieux que l'on puisse espérer c'est que le procédé devienne périodique à partir d'un moment \emph{(on parle de phénomène ultimement périodique)}.


\medskip

On peut explorer le comportement de ce procédé sur plusieurs valeurs grâce à un programme. Voici un code possible écrit en Python qui prend un peu de temps pour vérifier que pour tous les naturels $n \in \ZintervalC{1}{10^6}$, le procédé devient ultimement périodique.

\begin{rawcode}
NMAX    = 10**6
MAXLOOP = 10**20

for n in range(1, NMAX + 1):
    nbloops = 0
    results = []

    while nbloops < MAXLOOP and n not in results:
        nbloops += 1
        results.append(n)
        n = sum(int(d)**2 for d in str(n))

    if n not in results:
        print(f"Test raté pour n = {n}.")

print(f"Tests finis.")
\end{rawcode}

\medskip

Il reste à voir ce qu'il se passe dans le cas général. La section qui suit démontre que pour tout naturel $n$, le procédé sera toujours ultimement périodique.




\section{Une preuve}

Pour un naturel 
$\displaystyle      n = \left[ c_{d-1} c_{d-2} \cdots c_1 c_0 \right]_{10} 
\stackrel{\text{def}}{=} \sum_{k=0}^{d-1} c_k 10^k$
avec $c_{d-1} \neq 0$,
on pose
$\displaystyle sq(n) = \sum_{k=0}^{d-1} (c_k)^2$
et
$\taille(n) = d$ qui sera appelé \emph{\og taille de $n$ \fg}.

\medskip

Pour $(n \,; k) \in \NN^2$, on définit 
$  \sqseq{n}{0} = n$
et
$  \sqseq{n}{k} = sq^k(n)
\stackrel{\text{def}}{=} sq \,\circ sq \,\circ \cdots \,\circ sq(n)$ avec $(k-1)$ compositions si $k > 0$.
Autrement dit,
$\sqseq{n}{k+1} = sq \left( \, \sqseq{n}{k} \right)$.

\medskip

On note enfin
$V_n = \geneset{ \, \sqseq{n}{k} \, | \, k \in \NN }$
l'ensemble des valeurs prises par la suite $\left( \, \sqseq{n}{k} \right)_k$.



\medskip

\begin{fact}
	$\forall n \in \NN$, $sq(n) \leqslant 81 d$ où $d = \taille(n)$.
\end{fact}

\begin{proof*}
	Si $n = [c_{d-1} c_{d-2} \cdots c_1 c_0]_{10}$
	alors 
	$\displaystyle sq(n) = \sum_{k=0}^{d-1} (c_k)^2 \leqslant \sum_{k=0}^{d-1} 9^2 = 81 d $.
\end{proof*}



\medskip

\begin{fact}
	$\forall n \in \NN$, notant $d = \taille(n)$, nous avons les résultats suivants :
	
	\begin{enumerate}
		\item Si $d \geqslant 4$ alors $\taille(sq(n)) < \taille(n)$.
		
		\item Si $d \leqslant 3$ alors $\taille(sq(n)) \leqslant 3$.
	\end{enumerate}
\end{fact}

\begin{proof*}
	Notons que $n \geqslant 10^{d-1}$.
	Le comportement des fonctions $10^{x-1}$ et $81x$ sur $\RRsp$ assure l'existence d'un naturel $D$ tel que $\forall d \in \NN$, $d \geqslant D$ implique $10^{d-1} > 81d \geqslant sq(n)$. On a même beaucoup mieux : si $10^{D-1} > 81D \geqslant sq(n)$ alors $d \geqslant D$ implique $10^{d-1} > 81d \geqslant sq(n)$.
	
	\medskip
	
	Comme $10^3 > 81 \times 4$, nous avons sans effort le 1er point \emph{(rappelons que $10^k > n$ implique que $n$ admet au plus $(k-1)$ chiffres)}.
	
	\medskip
	
	Pour $d \leqslant 3$, le 2nd point découle de $sq(999) = 243$, $sq(99) = 162$ et $sq(9) = 81$.
\end{proof*}



\medskip

\begin{fact}
	$\forall n \in \NN$, l'ensemble $V_n$ est fini et donc la suite $\left( \, \sqseq{n}{k} \right)_{k \in \NN}$ est ultimement périodique, i.e. périodique à partir d'un certain rang.
\end{fact}

\begin{proof*}
	Le 2nd point dépend directement du 1er point via le principe des tiroirs et la définition récursive de la suite $\left( \, \sqseq{n}{k} \right)_k$.
	
	\medskip
	
	Pour le 1er point, il suffit de montrer que $V_n \subset \intervalC{0}{10^{\taille(n)}}$ pour $n \geqslant 4$ via une petite récurrence descendante finie, et pour $n \leqslant 3$ on a directement $V_n \subset \intervalC{0}{10^3}$.
\end{proof*}


\section{Code : déterminer la période d'un naturel}

Quand il ne se fige pas, le code suivant donne la \textit{\og période \fg} d'un naturel auquel on applique le procédé.

\begin{rawcode}
n = 19

results = []
print(n)

while n not in results:
    results.append(n)
    n = sum(int(d)**2 for d in str(n))

print("Période :")
print(results[results.index(n):])
\end{rawcode}


\end{document}
