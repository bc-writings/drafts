\documentclass[12pt]{amsart}
\usepackage[T1]{fontenc}
\usepackage[utf8]{inputenc}

\usepackage[top=1.95cm, bottom=1.95cm, left=2.35cm, right=2.35cm]{geometry}


\usepackage{wrapfig} 

\usepackage{hyperref}
\usepackage{enumitem}
\usepackage{tcolorbox}
\usepackage{float}
\usepackage{cleveref}
\usepackage{multicol}
\usepackage{fancyvrb}
\usepackage{enumitem}
\usepackage{amsmath}
\usepackage{textcomp}
\usepackage{numprint}
\usepackage[french]{babel}
\usepackage[
    type={CC},
    modifier={by-nc-sa},
	version={4.0},
]{doclicense}

\newcommand\floor[1]{\left\lfloor #1 \right\rfloor}

\usepackage{lymath}


\newtheorem{fact}{Fait}
\newtheorem*{fact*}{Fait}

\newtheorem{example}{Exemple}

\newtheorem{remark}{Remarque}
\newtheorem*{remark*}{Remarque}

\newtheorem*{proof*}{Preuve}

\setlength\parindent{0pt}

\floatstyle{boxed} 
\restylefloat{figure}


\DeclareMathOperator{\taille}{\text{\normalfont\texttt{taille}}}

\newcommand\sqseq[2]{\fbox{$#1$}_{\,\,#2}}


\DefineVerbatimEnvironment{rawcode}%
	{Verbatim}%
	{tabsize=4,%
	 frame=lines, framerule=0.3mm, framesep=2.5mm}
	 
	 
	 
\begin{document}

\title{BROUILLON - Dérivée et variations - Une preuve pour lycéen}
\author{Christophe BAL}
\date{16 Juillet 2019}

\maketitle

\begin{center}
	\itshape
	Document, avec son source \LaTeX, disponible sur la page
	
	\url{https://github.com/bc-writing/drafts}.
\end{center}


\bigskip


\begin{center}
	\hrule\vspace{.3em}
	{
		\fontsize{1.35em}{1em}\selectfont
		\textbf{Mentions \og légales \fg}
	}
			
	\vspace{0.45em}
	\doclicenseThis
	\hrule
\end{center}


\vspace{1em}


Voici un classique de l'analyse : \emph{\og pour toute fonction $f$ définie et dérivable sur un intervalle non trivial $I$ , si $f\,' > 0$ alors $f$ est strictement croissante sur l'intervalle $I$ \fg}. La preuve de ce résultat n'est pas simple car il nécessite l'usage du théorème des valeurs intermédiaires.


\medskip


Par contre il est facile de démontrer le résultat plus restrictif suivant
\footnote{
	On rencontre tout le temps cette situation au lycée
} :
\emph{\og pour toute fonction $f$ définie et dérivable sur un intervalle non trivial $I$ , si $f\,' > 0$ et si $f\,'$ est continue sur $I$ alors $f$ est strictement croissante sur l'intervalle $I$ \fg}. 
La démonstration utilise tout simplement la stricte positivité de l'intégrale sur $\RR$ et l'identité $f(b) - f(a) = \int_a^b f\,'(x) \dd{x}$ .

\end{document}
