Pour finir, nous allons analyser ce qu'il se passe s'il on somme à la puissance $p \geqslant 3$ au lieu d'élever au carré.
Nous reprenons des notations similaires à celles de la section \ref{proof}.
\begin{itemize}[label = \textbullet]
	\item Pour un naturel $n =  \left[ \, c_{d-1} c_{d-2} \cdots c_1 c_0 \, \right]_{10}$ avec $c_{d-1} \neq 0$,
	on pose
	$\displaystyle pw(n) = \sum_{k=0}^{d-1} (c_k)^p$
	et
	$\taille(n) = d$.
	
	\item Pour $(n \,; k) \in \NN^2$, on définit 
	$\sqseq{n}{0} = n$
	et
	$\sqseq{n}{k+1} = pw \left( \, \sqseq{n}{k} \right)$.
\end{itemize}

 

\bigskip

\begin{fact}
	$\forall n \in \NN$, $pw(n) \leqslant 9^p \, d$ où $d = \taille(n)$.
\end{fact}

\begin{proof*}
	Si $n = \left[ \, c_{d-1} c_{d-2} \cdots c_1 c_0 \, \right]_{10}$
	alors 
	$\displaystyle pw(n) = \sum_{k=0}^{d-1} (c_k)^p \leqslant \sum_{k=0}^{d-1} 9^p = 9^p \, d $.
\end{proof*}




\medskip

\begin{fact}\label{magicmajo}
	Il existe $d_0 \in \NN$ tel qu'on ait : $\forall n \in \NN$,	
	$[ \, \taille(n) \geqslant d_0 \Rightarrow n > pw(n) \, ]$ .
	
	\smallskip
	On peut en fait choisir $d_0 = 3 + \floor{\dfrac{1}{\, \ln 10 \,} \ln \left( \dfrac{ \, 9^p \, d \, }{\, \ln 10 \,} \right)}$
	où $\floor{x}$ désigne la partie entière de $x$.
\end{fact}

\begin{proof*}
	Notons $d = \taille(n)$ , de sorte que $n \geqslant 10^{d-1}$.
	Compte tenu du fait précédent, nous cherchons à comparer $10^{d-1}$ et $9^p \, d$.
	Comme dans le cas $p = 2$ démontré dans la section \ref{proof}, on considère sur $\RRp$ la fonction $\Delta(x) = 10^{x-1} - 9^p \, d \, x$
	qui vérifie $\Delta^\prime(x) > 0$ 
	si et seulement si
	$x > 1 + \dfrac{1}{\, \ln 10 \,} \ln \left( \dfrac{ \, 9^p \, d \, }{\, \ln 10 \,} \right)$.
	
	
	\medskip
	
	Notant $\alpha$ le réel précédent, nous savons donc que $n \geqslant 10^{d-1} > 9^p \, d \geqslant pw(n)$ ,
	puis $n > pw(n)$ dès que $d > \alpha$.
	Cette dernière condition est vérifiée dès que $d \geqslant d_0$ avec le $d_0$ proposé un peu plus haut.
\end{proof*}



\medskip

\begin{fact}\label{beautifulproof}
	$\forall n \in \NN$, la suite $\left( \, \sqseq{n}{k} \right)_{k \in \NN}$ est ultimement périodique.
\end{fact}

\begin{proof*}
	Tout est en fait contenu dans le fait \ref{magicmajo}, dont on reprend la définition de $d_0$. Expliquons pourquoi.
	\begin{itemize}[label = \textbullet]
		\item Le fait \ref{magicmajo} donne l'existence d'un indice $k_0 \in \NN$ tel que $\taille\left( \, \sqseq{n}{k_0} \right) < d_0$ \emph{(dans le cas contraire, on pourrait construire une suite strictement décroissante de naturels)}.

		\item Si $\forall k \in \ZintervalCO{k_0}{+\infty}$, $\taille\left( \, \sqseq{n}{k} \right) < d_0$ , nous avons l'ultime périodicité via le principe des tiroirs \emph{(si besoin revoir la fin de la section \ref{proof})}.

		\item Sinon il existe $k^\prime_0 \in \ZintervalO{k_0}{+\infty}$ tel que $\taille\left( \, \sqseq{n}{k^\prime_0} \right) \geqslant d_0$. Comme dans le premier point, nous pouvons alors trouver $k_1 \in \ZintervalO{k^\prime_0}{+\infty}$ tel que $\taille\left( \, \sqseq{n}{k_1} \right) < d_0$.
		
		\item En répétant notre raisonnement,
		on peut aboutir à une situation similaire au 2\ieme{} point, et c'est gagné. 
		
		\noindent
		Sinon on arrive à construire une suitre strictement croissante $\left( k_i \right)_i$ d'indices tels que $\forall i \in \NN$, $\taille\left( \, \sqseq{n}{k_i} \right) < d_0$. Le principe des tiroirs s'applique ici aussi !
	\end{itemize}
\end{proof*}



\medskip

\begin{remark}
	La preuve précédente montre que pour rechercher toutes les périodes il \emph{\og suffit \fg} d'étudier les naturels appartenant à $\ZintervalC{0}{10^{d_0} - 1}$.
\end{remark}
