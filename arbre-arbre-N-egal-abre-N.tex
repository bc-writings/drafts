\documentclass[12pt]{amsart}
\usepackage[T1]{fontenc}
\usepackage[utf8]{inputenc}

\usepackage[top=1.95cm, bottom=1.95cm, left=2.35cm, right=2.35cm]{geometry}

\usepackage{multicol}
\usepackage{fancyvrb}
\usepackage{enumitem}

\usepackage{amsmath}
\usepackage[french]{babel}
\usepackage[
    type={CC},
    modifier={by-nc-sa},
	version={4.0},
]{doclicense}

\usepackage{lymath}

\newcommand\infoarbre{info\,\text{-}\,arbre}

\newcommand\infoarbrearbre{info\,\text{-}\,arbre\,\text{-}\,arbre}


\newtheorem{theorem}{Théorème}
\newtheorem{definition}{Définition}
\newtheorem{example}{Exemple}
\newtheorem{remark}{Remarque}

\setlength\parindent{0pt}
\usepackage{ebproof}

\DefineVerbatimEnvironment{pseudocode}%
	{Verbatim}%
	{commandchars=\\\{\},%
	 codes={\catcode`$=3\catcode`_=8},%
	 tabsize=4,%
	 fontshape=nl,
	 frame=lines, framerule=0.3mm, framesep=2.5mm}


\begin{document}

\title{BROUILLON - Des arbres dont les feuilles sont... des arbres !}
\author{Christophe BAL}
\date{21 Mars 2019 - 10 Avril 2019}

\maketitle


\begin{center}
	\hrule\vspace{.3em}
	{
		\fontsize{1.35em}{1em}\selectfont
		\textbf{Mentions \og légales \fg}
	}
			
	\vspace{0.45em}
	\doclicenseThis
	\hrule
\end{center}

\setcounter{tocdepth}{2}
\tableofcontents



\section{Notre type de base}

\begin{definition} \label{basic-type}
	Le type $arbre(\alpha)$ est défini comme suit où $\alpha$ est un type.

	\medskip

	\begin{center}
	\begin{prooftree}
    	\hypo{a : \alpha}
	    \infer1[\footnotesize\itshape (feuille)]{%
	    	Leaf(a) : arbre(\alpha)%
		}
	\end{prooftree}
	\quad\quad\quad
	\begin{prooftree}
    	\hypo{t_1 : arbre(\alpha)}
	    \hypo{t_2 : arbre(\alpha)}
    	\infer2[\footnotesize\itshape (noeud)]{%
	    	Node(t_1, t_2) : arbre(\alpha)%
		}
	\end{prooftree}
	\end{center}
\end{definition}


\begin{example} \label{exa-tree}
	A l'aide des règles précédentes, on peut définir l'arbre suivant de type $arbre(\NN)$ où nous utilisons les abréviations suivantes $N = Node$, $L = Leaf$ et $AN = arbre(\NN)$.
	
	\medskip
	\scriptsize
	
	\begin{center}
	\begin{prooftree}
	    \hypo{1 : \NN}
	    \infer1[\footnotesize\itshape (feuille)]{%
	    	l_1 = L(1) : AN%
		}
	    \hypo{2 : \NN}
	    \infer1[\footnotesize\itshape (feuille)]{%
	    	l_2 = L(2) : AN%
		}
	    \hypo{3 : \NN}
	    \infer1[\footnotesize\itshape (feuille)]{%
	    	l_3 = L(3) : AN%
		}
    	\infer2[\footnotesize\itshape (noeud)]{%
	    	n_{23} = N(l_2, l_3) : AN%
		}
    	\infer2[\footnotesize\itshape (noeud)]{%
	    	n_{123} = N(l_1, n_{23}) : AN%
		}
		\hypo{4 : \NN}
	    \infer1[\footnotesize\itshape (feuille)]{%
	    	l_4 = L(4) : AN%
		}
    	\infer2[\footnotesize\itshape (noeud)]{%
	    	N(n_{123}, l_4) : AN%
		}
	\end{prooftree}
	\end{center}
	
	\medskip
	\normalsize
	
	Dans la suite, on utilisera aussi une écriture fonctionnelle pour représenter un arbre.
	Pour notre exemple, ceci donne :
	\begin{pseudocode}
Node(
	Node(
		Leaf(1),
		Node(
			Leaf(2),
			Leaf(3)
		)
	),
	Leaf(4)
)
	\end{pseudocode}
	
	\medskip

	Avec cette représentation, le plus \emph{\og simple \fg} des arbres de type $arbre(\NN)$ est le suivant avec $k \in \NN$ quelconque.
	\begin{pseudocode}
Leaf(k)
	\end{pseudocode}
\end{example}




\section{Un type dérivé}

\begin{definition}
	Le type $arbre(arbre(\alpha))$ est défini comme suit où $\alpha$ est un type \emph{(nous appliquons juste au type $arbre(\alpha)$ la définition \ref{basic-type})}.

	\medskip

	\begin{center}
		\begin{prooftree}
    		\hypo{a : arbre(\alpha)}
    		\infer1[\footnotesize\itshape (feuille)]{%
    			Leaf(a) : arbre(arbre(\alpha))%
			}
		\end{prooftree}
		\quad\quad\quad
		\begin{prooftree}
    		\hypo{t_1 : arbre(arbre(\alpha))}
    		\hypo{t_2 : arbre(arbre(\alpha))}
    		\infer2[\footnotesize\itshape (noeud)]{%
    			Node(t_1, t_2) : arbre(arbre(\alpha))%
			}
		\end{prooftree}
	\end{center}
\end{definition}


\begin{example}\label{exa-tree-tree}
	Dans cet exemple, nous noterons \verb+Leaf''+ et \verb+Node''+ les applications feuille et noeud pour le type $arbre(arbre(\NN))$.
	Le plus \emph{\og simple \fg} des arbres de type $arbre(arbre(\NN))$ est le suivant avec $k \in \NN$ quelconque.
	\begin{pseudocode}
Leaf''(Leaf(k))
	\end{pseudocode}
	
	\medskip
	
	On peut alors fabriquer un arbre similaire, mais pas identique, au 1\ier{} arbre de l'exemple \ref{exa-tree}.
	\begin{pseudocode}
Node''(
	Node''(
		Leaf''(Leaf(1)),
		Node''(
			Leaf''(Leaf(2)),
			Leaf''(Leaf(3))
		)
	),
	Leaf''(Leaf(4))
)
	\end{pseudocode}
\end{example}






\section{Une comparaison sans formalité de nos deux types}

Il est facile de passer des arbres de l'exemple \ref{exa-tree} à ceux de l'exemple \ref{exa-tree-tree} : chaque \verb+Leaf(k)+ doit être remplacé par \verb+Leaf(Leaf(k))+ pour $k \in \NN$.

\medskip

De façon similaire, on peut passer des arbres de l'exemple \ref{exa-tree-tree} à ceux de l'exemple \ref{exa-tree}.

\medskip

Mais n'allons pas trop vite... En effet, pour tout arbre \verb+t+ de type $arbre(\NN)$, nous avons \verb+Leaf(t)+ qui est de type $arbre(arbre(\NN))$. Ceci a des implications fortes. Pour le voir, considérons l'arbre \verb+TEX+ suivant qui est de type $arbre(\NN)$.

\begin{pseudocode}
Node(
	Leaf(1), 
	Node(
		Node(
			Leaf(2),
			Leaf(3)
		), 
		Leaf(4)
	)
)
\end{pseudocode}

	
Voici un premier arbre de type $arbre(arbre(\NN))$ pouvant être \emph{\og interprété naturellement \fg} comme étant l'arbre \verb+TEX+. De nouveau, nous noterons \verb+Leaf''+ et \verb+Node''+ les applications feuille et noeud pour le type $arbre(arbre(\NN))$.

\begin{pseudocode}
Leaf''(
	Node(
		Leaf(1), 
		Node(
			Node(
				Leaf(2),
				Leaf(3)
			), 
			Leaf(4)
		)
	)
)
\end{pseudocode}


Voici un autre exemple.

\begin{pseudocode}
Node''(
	Leaf''(Leaf(1)), 
	Node''(
		Node''(
			Leaf''(Leaf(2)),
			Leaf''(Leaf(3))
		), 
		Leaf''(Leaf(4))
	)
)
\end{pseudocode}


On peut aussi proposer l'arbre ci-dessous.

\begin{pseudocode}
Node''(
	Leaf''(Leaf(1)), 
	Node''(
		Leaf''(
			Node(
				Leaf(2),
				Leaf(3)
			)
		), 
		Leaf''(Leaf(4))
	)
)
\end{pseudocode}


Pour finir, voici une dernière proposition.

\begin{pseudocode}
Node''(
	Leaf''(Leaf(1)), 
	Leaf''(
		Node(
			Node(
				Leaf(2),
				Leaf(3)
			), 
			Leaf(4)
		)
	)
)
\end{pseudocode}






Si l'on regarde bien chacun des exemples proposés, nous notons que les \verb+Leaf''+ \emph{\og polluent \fg} les arbres mais pas les \verb+Node''+.
En fait, en transformant syntaxiquement \verb+Node''(...)+ en \verb+Node(...)+ , puis  \verb+Leaf''(Leaf(...))+ en \verb+Leaf(...)+ et enfin \verb+Leaf''(Node(...))+ en \verb+Node(...)+ alors pour chaque exemple nous retombons sur notre arbre initial \verb+TEX+ de type $arbre(\NN)$.


\medskip

Intuitivement le type $arbre(arbre(\NN))$ ne contient pas plus d'information que le type $arbre(\NN)$. 
L'objectif de la section suivante va être de définir rigoureusement ce que nous entendons par \emph{\og information \fg} puis ensuite de voir qu'effectivement le type $arbre(arbre(\NN))$ ne donne pas plus d'information que le type $arbre(\NN)$.




\section{Comparer (\!?) des types}

Sans donner de signification aux arbres que nous présentons, les arbres de type $arbre(arbre(\NN))$ et ceux de type $arbre(\NN)$ sont de natures structurelles totalement différentes comme le montrent les exemples précédents.


\medskip

D'un autre côté si nous nous intéressons non aux structures de nos arbres mais à ce qu'ils stockent, alors la section précédente nous pousse à trouver une grande similitude entre les arbres de type $arbre(arbre(\NN))$ et ceux de type $arbre(\NN)$. 


\medskip

Autrement dit, la comparaison des types $arbre(arbre(\NN))$ et $arbre(\NN)$ ne peut faire sens que du point de vie sémantique, et non simplement via une analyse syntaxique.
Nous devons donc arriver à donner une signification à nos arbres. C'est ce que proposent les deux définitions \emph{\og naturelles \fg} suivantes.


\begin{definition}
	KKKKK
\end{definition}



\begin{definition}
	KKKKK
\end{definition}



\begin{example}
	????    
	type $arbre(\NN)$
	\begin{pseudocode}
???
	\end{pseudocode}
	
	\medskip
	
	Ici nous avons ????
\end{example}



\begin{example}
	????    
	type $arbre(arbre(\NN))$
	\begin{pseudocode}
???
	\end{pseudocode}
	
	\medskip
	
	Ici nous avons ????
\end{example}



 de n souhaite définir l'ingformation contenue

on définit une fonciton $\infoarbre$ stockant profondeur usivi de la valeur d'un noued

l'info définit l'arbre de façon unique !

on montre que les type ont le smême info si on définit une fino récursivement sue arbre(arbre(N)) 


on crée une ou deux fnction qui fabrique une liste qui stocke -2 ou 2 si on va à gaiche ou à droite dans un arbre, et  si c'est une feuille on stocke valeur $10^n$ 

de la sorte on repère valeur et déplacement


pb pour arbre(arbre(N)), comment gfircer vélautaion de la feuille ???



\Huge

PLUS SIMPLE !

$\infoarbre$ et $\infoarbrearbre$ qui décrive le mêm enselble à carcatériser proprment !!!!

\end{document}
