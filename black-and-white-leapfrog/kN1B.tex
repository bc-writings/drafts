Avant de passer au cas général, nous allons démontrer le fait suivant où le nombre de moutons noirs est quelconque.


\factwin{k}{1}


\begin{proof}
	Il suffit d'utiliser un raisonnement par récurrence sur $k \in \NNs$.
	
	\begin{itemize}[label=\small\textbullet]
		\item \textbf{Cas de base :} pour $k = 1$, la configuration \config{kN.1B} est résoluble d'après le fait \ref{config-1-1}.


		\item \textbf{Hérédité :} supposons que la configuration \config{kN.1B} soit résoluble. Nous devons en déduire que la configuration \config{(k+1)N.1B} l'est aussi. Ceci vient directement de la dernière étape ci-dessous.
		\begin{mvts}
			\medskip
			\item  \gamelineplus{-NN.b}{Il y a $(k+1)$ noirs à gauche. On décide de bouger le blanc.} 

			\medskip
			\item  \gamelineplus{-Nnp.}{Un noir va aller prendre sa place définitive à droite.}

			\medskip
			\item  \gamelineplus{=nvbu}{Nous reconnaissons à gauche la sous-configuration \config{kN.1B}.}  
		\end{mvts}
	
		\noindent
		A partir de la dernière étape, il suffit de résoudre la sous-configuration \config{kN.1B} à gauche grâce à l'hypothèse de récurrence, et ceci sans toucher le mouton noir bien placé tout à droite qui de toute façon ne peut plus bouger.
	\end{itemize}
\end{proof}


Les preuves des faits \ref{greedy-2N2B} et \ref{greedy-3N2B} sont faciles à comprendre mais leur généralisation va nous demander un peu de travail et de prudence. Le chemin entre l'intuition est la démonstration n'est pas toujours direct.
