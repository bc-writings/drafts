\begin{definition}
	Une configuration désigne juste un ensemble de moutons noirs et blancs sur une ligne de cases dont une seule est vide \emph{(une configuration ne vient pas forcément de mouvements faits lors d'une partie)}.
\end{definition}


\begin{definition}
	Une configuration \config{kN.pB} désignera un début de parties avec $k$ moutons noirs et $p$ blancs.
	Par exemple, \config{5N.3B} correspond à la configuration suivante.
	\centerit{\gameline{NNNNN.BBB}}
\end{definition}


\begin{definition}
	La configuration \config{kN.pB} sera dite résoluble si l'on peut résoudre la devinette des moutons qui lui est associée.
\end{definition}


\begin{remark}
	Plus généralement, nous noterons certains configurations de façon naturelle. Par exemple, \config{3NB.4B} correspond à la configuration suivante.

	\centerit{\gameline{upupup.BBBB}}

	Un autre exemple : \config{2N3NB.4B} désigne la configuration suivante.
	
	\centerit{\gameline{NNupupup.BBBB}}
\end{remark}
