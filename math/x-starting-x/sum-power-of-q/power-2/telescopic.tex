\subsection{La preuve}

Pour trouver une formule explicite de $S_n \eq[def] \dsum_{k = 0}^{n} 2^k$, on peut noter que
$2^k = 2 \cdot 2^{k-1} = 2^{k-1} + 2^{k-1}$
donne
$2^{k-1} = 2^k - 2^{k-1}$
d'où l'on déduit que
$\forall k \in \NN$, $2^k = 2^{k+1} - 2^k$.
Ceci nous conduit aux calculs suivants.

\begin{focusproof}
\begin{stepcalc}[style = sar]
	S_n 
		\explnext[{\eq[def]}]{}
	\dsum_{k = 0}^{n} 2^k
		\explnext{}
	\dsum_{k = 0}^{n} (2^{k+1} - 2^k)
		\explnext{}
	\dsum_{k = 0}^{n} 2^{k+1} - \dsum_{k = 0}^{n} 2^k
		\explnext{\footnotesize $i = k+1 \iff k = i-1$}
	\dsum_{i = 1}^{n+1} 2^{i} - \dsum_{k = 0}^{n} 2^k
		\explnext*{\footnotesize On fait apparaître des sommes identiques.}{}
	\dsum_{i = 1}^{n} 2^{i} + 2^{n+1} - 2^0 - \dsum_{k = 1}^{n} 2^k
		\explnext{}
	2^{n+1} - 1
\end{stepcalc}
\end{focusproof}


% ------------- %


\subsection{Commentaires}

Les simplifications du type
$\dsum_{k = 0}^{n} (2^{k+1} - 2^k) = 2^{n+1} - 2^0$ ,
ou plus généralement du type
$\dsum_{k = 0}^{n} (u_{k+1} - u_{k}) = u_{n+1} - u_0$
ou
$\dsum_{k = 0}^{n} (u_{k} - u_{k+1}) = u_0 - u_{n+1}$ ,
sont un grand classique : on parle de \og sommes télescopiques \fg{}
\footnote{
	Cette technique permet par exemple de ramener l'étude d'une suite à celle d'une série.
	Or il se trouve que l'on dispose d'outils très pratiques pour étudier les séries.
}.
L'usage de cette astuce fonctionne sans souci avec les puissances de $q$ comme nous le verrons dans la section \ref{power-q:telescopic} page \pageref{power-q:telescopic}.
Bien qu'élégants du point de vue algébrique, les calculs ci-dessus ne donnent aucune information sur la signification de la formule trouvée.


% ------------- %


\subsection{D'autres applications}

Une application rigolote est l'obtention d'une formule explicite de la somme $I_n \eq[def] \dsum_{k = 1}^{n} \dfrac{1}{k(k+1)}$. 
Une fois noté que $\dfrac{1}{k(k+1)} = \dfrac{1}{k} -  \dfrac{1}{k+1}$
le calcul est très aisé
\footnote{
	Cet exemple est conçu comme un cas typique d'usage de sommes télescopiques.
}.

\medskip

\begin{stepcalc}[style = sar]
	I_n \explnext[{\eq[def]}]{}
	\dsum_{k = 1}^{n} \dfrac{1}{k(k+1)}
		\explnext{}
	\dsum_{k = 1}^{n} \left( \dfrac{1}{k} - \dfrac{1}{k+1} \right)
		\explnext{\footnotesize Usage de sommes télescopiques.}
	\dfrac{1}{1} - \dfrac{1}{n+1}
		\explnext{}
	\dfrac{n}{n+1}
\end{stepcalc}


\bigskip

On peut aussi utiliser des sommes télescopiques pour expliciter $\dsum_{k = 0}^{n} k^p$ .
Montrons par exemple comment trouver une formule explicite de la somme
\footnote{
	La lettre $G$ faire référence à Gauss à qui l'on attribue une méthode très astucieuse pour calculer cette somme en la réordonnant.
}
$G_n \eq[def] \dsum_{k = 0}^{n} k$.
L'idée astucieuse consiste à noter que $(k+1)^2 - k^2 = 2 k + 1$ puis à procéder comme suit.

\medskip

\begin{stepcalc}[style = sar]
	2 G_n
		\explnext{}
	\dsum_{k = 0}^{n} 2 k
		\explnext{}
	\dsum_{k = 0}^{n} \big[ \, (k+1)^2 - k^2 - 1 \, \big]
		\explnext{}
	\dsum_{k = 0}^{n} \big[ \, (k+1)^2 - k^2 \, \big] - \dsum_{k = 0}^{n} 1
		\explnext{\footnotesize Usage de sommes télescopiques.}
	(n + 1)^2 - 0^2 - (n+1)
		\explnext{}
	(n + 1) \cdot \big[ \, (n + 1) - 1 \, \big]
		\explnext{}
	n (n + 1)
\end{stepcalc}

\bigskip

Donc
$\forall n \in \NN$ , $\dsum_{k = 0}^{n} k = \dfrac{n(n + 1)}{2}$ .
La preuve précédente, bien que calculatoire, est relativement élégante
\footnote{
	Vous pourrez comparer avec celle proposée dans la section \ref{power-2:rewriting} page \pageref{power-2:rewriting}.
}.
Continuons avec $C_n \eq[def] \dsum_{k = 0}^{n} k^2$
via $(k+1)^3 - k^3 = 3 k^2 + 3 k + 1$
et la formule précédente.

\medskip

\begin{stepcalc}[style = sar]
	3 C_n
		\explnext{}
	\dsum_{k = 0}^{n} 3 k^2
		\explnext{}
	\dsum_{k = 0}^{n} \big[ \, (k+1)^3 - k^3 - 3 k - 1 \, \big]
		\explnext{}
	\dsum_{k = 0}^{n} \big[ \, (k+1)^3 - k^3 \, \big] - 3 \dsum_{k = 0}^{n} k - \dsum_{k = 0}^{n} 1
		\explnext{\footnotesize Usage de sommes télescopiques.}
	(n + 1)^3 - 0^3 - 3 \cdot \dfrac{n(n + 1)}{2} - (n+1)
%		\explnext{}
%	\dfrac{1}{2} \big[ \, 2 (n + 1)^3 - 3n (n + 1) - 2(n+1) \, \big]
		\explnext{}
	\dfrac{(n+1)}{2} \cdot \big[ \, 2 (n + 1)^2 - 3n - 2 \, \big]
		\explnext{}
	\dfrac{(n+1)}{2} \cdot \big[ \, 2 (n^2 + 2 n + 1) - 3n - 2 \, \big]
		\explnext{}
	\dfrac{(n+1)(2 n^2 + n)}{2}
		\explnext{}
	\dfrac{n(n+1)(2 n + 1)}{2}
\end{stepcalc}


\medskip

Donc
$\forall n \in \NN$ , $\dsum_{k = 0}^{n} k^2 = \dfrac{n(n + 1)(2n + 1)}{6} = \dfrac{n^3}{3} + \dfrac{n^2}{2} + \dfrac{n}{6}$ .


\medskip

Finissons en montrant que
$\forall n \in \NN$ , $\dsum_{k = 0}^{n} k^3 = \dfrac{n^2 (n + 1)^2}{4} = \left( \, \dsum_{k = 0}^{n} k \, \right)^2$ . Un très joli résultat !
Nous allons utiliser de $(k+1)^4 - k^4 = 4 k^3 + 6 k^2 + 4 k + 1$.
Notez que les calculs se compliquent vite et rendent la preuve de moins en moins élégante
\footnote{
	Un bon cadre d'étude pour des puissances plus élevées est celui utilisant les nombres de Bernoulli qui ont de jolies propriétés.
}.

\medskip

\begin{stepcalc}[style = sar]
	4 \dsum_{k = 0}^{n} k^3
		\explnext{}
	\dsum_{k = 0}^{n} \big[ \, (k+1)^4 - k^4 - 6 k^2 - 4 k - 1 \, \big]
		\explnext{}
	\dsum_{k = 0}^{n} \big[ \, (k+1)^4 - k^4 \, \big] - 6 \dsum_{k = 0}^{n} k^2 - 4 \dsum_{k = 0}^{n} k - \dsum_{k = 0}^{n} 1
		\explnext{}
	(n+1)^4 - 0^4 - 6 \cdot \dfrac{n(n + 1)(2n + 1)}{6} - 4 \cdot \dfrac{n(n + 1)}{2} - (n+1)
		\explnext{}
	(n+1) \cdot \big[ \, (n+1)^3 - n(2n + 1) - 2n - 1 \, \big]
		\explnext{}
	(n+1) \cdot \big[ \, (n+1)^3 - 2n^2 - 3n - 1 \, \big]
		\explnext*{\footnotesize $(-1)$ est une racine évidente de $2 X^2 + 3X + 1$.}{}
	(n+1) \cdot \big[ \, (n+1)^3 - (n+1)(2n + 1) \, \big]
		\explnext{}
	(n+1)^2 \cdot \big[ \, (n+1)^2 - 2n - 1 \, \big]
		\explnext{}
	n^2 (n+1)^2
\end{stepcalc}



