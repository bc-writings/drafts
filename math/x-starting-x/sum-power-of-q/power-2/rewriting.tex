\subsection{La preuve}

Les calculs suivants sont très simples à suivre mais malheureusement peu éclairants d'un point de vue conceptuel. Le cas où $n = 0$ étant évident, on suppose par la suite que $n \in \NNs$.

\begin{focusproof}
\begin{stepcalc}[style = sar, ope = \iff]
	S_n \eq[def] \dsum_{k = 0}^{n} 2^k
		\explnext{}
	S_n = 1 + \dsum_{k = 1}^{n} 2^k
		\explnext{\footnotesize $k = i+1 \iff i = k-1$}
	S_n = 1 + \dsum_{i = 0}^{n-1} 2^{i + 1}
		\explnext{}
	S_n = 1 + 2 \dsum_{i = 0}^{n-1} 2^i
		\explnext*{\footnotesize Faisons apparaître la somme \og réduite\fg{} à gauche. }{}
	\dsum_{k = 0}^{n-1} 2^k + 2^n = 1 + 2 \dsum_{i = 0}^{n-1} 2^i
		\explnext{}
	S_{n-1} + 2^n = 1 + 2 S_{n-1}
		\explnext{}
	2^n - 1 = S_{n-1}
		\explnext{}
	S_{n-1} = 2^n - 1
\end{stepcalc}

\medskip

Donc $S_0 = 1$ et $\forall n \in \NNs$ , $S_{n-1} = 2^n - 1$ .
Ceci se réécrit :
$\forall n \in \NN$ , $S_n = 2^{n+1} - 1$ .
\end{focusproof}


% ------------- %


\subsection{Commentaires}

La méthode précédente se généralise sans souci aux puissances de $q$ comme nous le verrons dans la section \ref{power-q:rewriting} page \pageref{power-q:rewriting}.
Par contre elle n'éclaire en rien sur la signification de la formule trouvée mais a l'avantage d'être prouvable via un ordinateur.


% ------------- %


\subsection{D'autres applications}

La réécriture de sommes peut permettre de trouver des sommes du type $\dsum_{k = 0}^{n} k^p$ .
Montrons par exemple comment trouver une formule explicite de la somme
\footnote{
	La lettre $G$ faire référence à Gauss à qui l'on attribue une méthode très astucieuse pour calculer cette somme en la réordonnant.
}
$G_n \eq[def] \dsum_{k = 0}^{n} k$ où nous laissons de nouveau de côté le cas trivial où $n = 0$.
L'idée est de réécrire $C_n \eq[def] \dsum_{k = 0}^{n} k^2$ et non $G_n$ car nous allons voir que la réécriture va éliminer les carrés
\footnote{
	Ce principe d'élimination se repère vite si l'on raisonne directement en réécrivant la somme cherchée $G_n$.
}.

\medskip

\begin{stepcalc}[style = sar, ope = \iff]
	C_n \eq[def]  \dsum_{k = 0}^{n} k^2
		\explnext{}
	C_n = 0 + \dsum_{k = 1}^{n} k^2
		\explnext{\footnotesize $k = i+1 \iff i = k-1$}
	C_n = \dsum_{i = 0}^{n-1} (i + 1)^2
		\explnext{}
	C_n = \dsum_{i = 0}^{n-1} (i^2 + 2 i + 1)
		\explnext{}
	C_n = \dsum_{i = 0}^{n-1} i^2 + 2 \dsum_{i = 0}^{n-1} i + \dsum_{i = 0}^{n-1} 1
		\explnext{}
	C_n = C_{n-1} + 2 G_{n-1} + n
		\explnext*{\footnotesize Apparition de la somme \og réduite\fg{} à gauche. }{}
	C_{n-1} + n^2 = C_{n-1} + 2 G_{n-1} + n
		\explnext{}
	n^2 = 2 G_{n-1} + n
\end{stepcalc}


\begin{stepcalc}[style = sar, ope = \iff]
	C_n \eq[def]  \dsum_{k = 0}^{n} k^2
		\explnext{}
	n^2 - n = 2 G_{n-1}
		\explnext{}
	G_{n-1} = \dfrac{n(n - 1)}{2}
\end{stepcalc}

\bigskip

Donc $G_0 = 0$ et $\forall n \in \NNs$ , $G_{n-1} = \dfrac{n(n - 1)}{2}$ .
Ceci se réécrit :
$\forall n \in \NN$ , $\dsum_{k = 0}^{n} k = \dfrac{n(n + 1)}{2}$ .

\medskip

On peut continuer de façon analogue pour obtenir une formule explicite de $C_n$ via la somme des cubes d'entiers successifs, puis ensuite on en aura une pour la somme des cubes elle-même mais les calculs deviennent vite pénibles
\footnote{
	En fait il existe une formulation générale faisant intervenir les nombres de Bernoulli qui ont de jolies propriétés.
}...
Voici la partie importante pour découvrir que 
$\forall n \in \NN$ , $\dsum_{k = 0}^{n} k^2 = \dfrac{n (n + 1) (2n + 1)}{6} = \dfrac{n^3}{3} + \dfrac{n^2}{2} + \dfrac{n}{6}$ .

\medskip

\begin{stepcalc}[style = sar, ope = \iff]
	D_n \eq[def]  \dsum_{k = 0}^{n} k^3
		\explnext{}
	D_n = \dsum_{i = 0}^{n-1} (i + 1)^3
		\explnext{}
	D_n = \dsum_{i = 0}^{n-1} (i^3 + 3 i^2 + 3 i + 1)
		\explnext{}
	D_n = D_{n-1} + 3 C_{n-1} + 3 G_{n-1} + n
		\explnext{}
	n^3 = 3 C_{n-1} + 3 G_{n-1} + n
		\explnext{}
	3 C_{n-1} = n^3 - 3 \cdot \dfrac{n(n - 1)}{2} - n
		\explnext{}
	6 C_{n-1} = 2 n^3 - 3n(n - 1) - 2n
		\explnext{}
	6 C_{n-1} = n (2 n^2 - 3n + 1)
		\explnext*{\footnotesize $1$ est une racine évidente de $2 X^2 - 3X + 1$.}{}
	6 C_{n-1} = n (n - 1) (2n - 1)
%		\explnext{}
%	C_{n-1} = \dfrac{n (n - 1) (2n - 1)}{6}
\end{stepcalc}


\medskip


Finissons en montrant que
$\forall n \in \NN$ , $\dsum_{k = 0}^{n} k^3 = \dfrac{n^2 (n + 1)^2}{4} = \left( \, \dsum_{k = 0}^{n} k \, \right)^2$ . Que c'est joli !
Notez que les calculs se compliquent vite et rendent la preuve très inélégante.

\medskip

\begin{stepcalc}[style = sar, ope = \iff]
	E_n \eq[def]  \dsum_{k = 0}^{n} k^4
		\explnext{}
	E_n = \dsum_{i = 0}^{n-1} (i^4 + 4 i^3 + 6 i^2 + 4 i + 1)
		\explnext{}
	n^4 = 4 D_{n-1} + 6 C_{n-1} + 4 G_{n-1} + n
		\explnext{}
	4 D_{n-1} = n^4 - n (n - 1) (2n - 1) - 2n(n - 1) - n
		\explnext{}
	4 D_{n-1} = n \cdot \big[ \, \textcolor{blue}{n^3} - (n - 1) (2n - 1) - 2(n - 1) - \textcolor{blue}{1} \, \big]
		\explnext*{%
			\tiny%
			\begin{stepcalc}[style = ar]
				n^3 - 1
					\explnext{}	
				(n - 1)(n^2 + n + 1)
			\end{stepcalc}%
		}{}
	4 D_{n-1} = n (n - 1) \cdot \big[ \, \textcolor{blue}{n^2 + n + 1} - (2n - 1) - 2 \, \big]
		\explnext{}
	4 D_{n-1} = n (n - 1) (n^2 - n)
		\explnext{}
	4 D_{n-1} = n^2 (n - 1)^2
\end{stepcalc}


