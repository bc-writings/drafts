La Physique nous indique que la lumière va parcourir le chemin le plus court, donc on cherche le point $M \in \setgeo{C}$  tel que la distance $MS + MI$ soit minimale avec $x_M > 0$. Calculons donc cette distance en fonction de $y_M$.

\medskip

\begin{stepcalc}[style=ar*]
	MS + MI
\explnext{}   
	\sqrt{ \big( x_M - s \big)^2 + y_M^2 }
	+
	\sqrt{ \big( x_M - i \big)^2 + \big( y_M - j \big)^2 }
\explnext{}   
	\sqrt{ x_M^2 + s^2 - 2 s x_M + y_M^2 }
	+
	\sqrt{ x_M^2 + i^2 - 2 i x_M + y_M^2 + j^2 - 2 j y_M }
\explnext*{$x_M^2 + y_M^2 = 1$}{}   
	\sqrt{ 1 + s^2 - 2 s x_M  }
	+
	\sqrt{ 1 + i^2 + j^2 - 2 i x_M - 2 j y_M }
\explnext*{$t \eq[nota] OI$}{}   
	\sqrt{ 1 + s^2 - 2 s x_M  }
	+
	\sqrt{ 1 + t^2 - 2 i x_M - 2 j y_M }
\explnext*{$x_M^2 + y_M^2 = 1$ et $x_M > 0$}{} 
	\sqrt{ 1 + s^2 - 2 s \sqrt{1 - y_M^2} }
	+
	\sqrt{ 1 + t^2 - 2 i \sqrt{1 - y_M^2} - 2 j y_M }  
\explnext*{$2 \alpha \eq[nota] 1 + s^2$ \\ $2 \beta \eq[nota] 1 + t^2$}{} 
	\sqrt{ 2 \alpha - 2 s \sqrt{1 - y_M^2} }
	+
	\sqrt{ 2 \beta - 2 i \sqrt{1 - y_M^2} - 2 j y_M }  
\end{stepcalc}

\bigskip

On doit donc commencer à chercher toutes les racines $y \in \intervalO{-1}{1}$ de l'équation $\der{f}{x}{1}(y) = 0$ où 
$f(y) \eq[def] \sqrt{ 2 \alpha - 2 s \sqrt{1 - y^2} } + \sqrt{ 2 \beta - 2 i \sqrt{1 - y^2} - 2 j y } $ . 
Or nous avons :

\bigskip

\begin{stepcalc}[style=sar]
	\der{f}{x}{1}(y)
\explnext{}
    \dfrac{2 s y}{ \sqrt{1 - y^2} }
	\cdot
    \dfrac{1}{ 2 \sqrt{ 2 \alpha - 2 s \sqrt{1 - y^2} } }
	%
	\\ &
	\kern-1em + \,
	%
    \bigg( \dfrac{2 i y}{ \sqrt{1 - y^2} } - 2 j \bigg)
	\cdot
    \dfrac{1}{ 2 \sqrt{ 2 \beta - 2 i \sqrt{1 - y^2} - 2 j y } }
\explnext{}
	\dfrac{1}{\sqrt{1 - y^2}}
	\bigg(
        \dfrac{s y}{ \sqrt{ 2 \alpha - 2 s \sqrt{1 - y^2} } }	%
    	+
        \dfrac{i y - j \sqrt{1 - y^2} }{ \sqrt{ 2 \beta - 2 i \sqrt{1 - y^2} - 2 j y } }
    \bigg)
\end{stepcalc}

\bigskip

Nous avons alors :

\medskip

\begin{stepcalc}[style=ar*, ope = {\implies[donc]}]
	\der{f}{x}{1}(y) = 0
\explnext{}
	s y \, \sqrt{ 2 \beta - 2 i \sqrt{1 - y^2} - 2 j y }
	+
	\big( i y - j \sqrt{1 - y^2} \big) \, \sqrt{ 2 \alpha - 2 s \sqrt{1 - y^2} }
	=
	0
\explnext{}
	s y \, \sqrt{ 2 \beta - 2 i \sqrt{1 - y^2} - 2 j y }
	=
	\big( j \sqrt{1 - y^2} - i y \big) \, \sqrt{ 2 \alpha - 2 s \sqrt{1 - y^2} }
\explnext{}
	s^2 y^2 \big( \beta - i \sqrt{1 - y^2} - j y \big)
	=
	\big( j \sqrt{1 - y^2} - i y \big)^2 \, \big( \alpha - s \sqrt{1 - y^2} \big)
\end{stepcalc}

\bigskip

Concentrons-nous \qty{2}{\min} sur la partie droite de la dernière équation.

\medskip

\begin{stepcalc}[style=ar*]
	\big( j \sqrt{1 - y^2} - i y \big)^2 \, \big( \alpha - s \sqrt{1 - y^2} \big)
\explnext{}
	\big( j^2 (1 - y^2) + i^2 y^2 - 2 i j y \sqrt{1 - y^2} \big) \, \big( \alpha - s \sqrt{1 - y^2} \big)
\explnext{}
	\big( j^2 + (i^2 - j^2) y^2 - 2 i j y \sqrt{1 - y^2} \big) \, \big( \alpha - s \sqrt{1 - y^2} \big)
\explnext{}
	j^2 \alpha - j^2 s \sqrt{1 - y^2}
	%
	\\ &
	\kern-1em + \,
	%
	(i^2 - j^2) y^2 \alpha - (i^2 - j^2) y^2 s \sqrt{1 - y^2}
	%
	\\ &
	\kern-1em -
	%
	2 \alpha i j y \sqrt{1 - y^2} + 2 i j s y (1 - y^2)
\end{stepcalc}


\bigskip

Nous devons de nouveau introduire des notations pour les constantes. Nous posons donc
$XXX$ ,
et
$YYY$ ,
afin d'obtenir :

\medskip

\begin{stepcalc}[style=ar*, ope = {\implies[donc]}]
	\der{f}{x}{1}(y) = 0
\explnext{}
	A y^3 + B y^2 + C y
	=
	( D y^2 + E y + F ) \sqrt{1 - y^2}
\explnext{}
	A^2 y^6 + B^2 y^4 + C^2 y^2
	+
	2 A B y^5 + 2 A C y^4 + 2 B C y^3
	%
	\\ &
	\kern-1em = \,
	%
	\big( 
	    D^2 y^4 + E^2 y^2 + F^2 
	    +
	    2 D E y^3 + 2 D F y^2 + 2 E F y
	\big)
	(1 - y^2)
\explnext{}
	XXX y^6 
	+ 
	XXX y^5
	+ 
	XXX y^4
	+ 
	XXX y^3
	+ 
	XXX y^2
	+ 
	XXX y
	+ 
	XXX
	=
	0
\end{stepcalc}


\bigskip

Il est de temps de passer à quelques applications numériques puisqu'il est bien connu que seules les équations polynomiales de degré au plus $4$ peuvent être résolues par radicaux de façon générique
\footnote{
	Par exemple, il est très simple de résoudre $x^{2048} = 2$ via des racines carrées successives.
}.



