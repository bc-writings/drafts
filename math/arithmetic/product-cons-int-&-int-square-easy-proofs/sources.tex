% ------------------ %


\bigskip
\textbf{Faits \ref{case-5}, \ref{case-7}, \ref{case-9}, \ref{case-10}, \ref{case-11}, \ref{case-12} et \ref{case-13}.}
	
\smallskip
\noindent
Un échange consulté le 13 février 2024, et titré
\emph{\enquote{\href{https://math.stackexchange.com/q/2361670/52365}{Product of 10 consecutive integers can never be a perfect square}}} 
sur le site \url{https://math.stackexchange.com}\,.

\smallskip
\noindent
\emph{La démonstration indiquée est celle du fait \ref{case-10}, une preuve venant d'une source Wordpress donnée dans une réponse de cet échange, mais cette source est très expéditive...}


% ------------------ %


\bigskip
\textbf{Fait \ref{case-6}.}
	
\smallskip
\noindent
Un échange consulté le 28 janvier 2024, et titré
\emph{\enquote{\href{https://math.stackexchange.com/q/90894/52365}{product of six consecutive integers being a perfect numbers}}} 
sur le site \url{https://math.stackexchange.com}\,.


% ------------------ %


\bigskip
\textbf{Fait \ref{case-7}.}
	
\smallskip
\noindent
Un échange consulté le 3 février 2024, et titré
\emph{\enquote{\href{https://math.stackexchange.com/q/2334887/52365}{Proof that the product of 7 successive positive integers is not a square}}} 
sur le site \url{https://math.stackexchange.com}\,.
	
\smallskip
\noindent
\emph{Il manque certaines justifications dans la démonstration donnée dans cet échange.}


% ------------------ %


\bigskip
\textbf{Fait \ref{case-8}.}
	
\smallskip
\noindent
Un échange consulté le 4 février 2024, et titré
\emph{\enquote{\href{https://math.stackexchange.com/a/2271715/52365}{How to prove that the product of eight consecutive numbers can't be a number raised to exponent 4?}}} 
sur le site \url{https://math.stackexchange.com}\,.

\smallskip
\noindent
\emph{La démonstration vient de l'une des réponses de cet échange, mais la justification des deux inégalités n'est pas donnée.}

