\begin{fact} \label{case-8}
	 $\forall n \in \NNs$\,, $\consprod<8> \notin \NNsquare$\,.
\end{fact}


% ------------------ %


Cette démonstration algébrique vient d'un document d'entraînement aux Olympiades Mathématiques (se reporter à la section \ref{sources}).


\begin{proof}[Preuve]
	\leavevmode

    \begin{itemize}
    	\item Nous avons les manipulations algébriques naturelles suivantes.
    
        \medskip
        \noindent\kern-6pt%
        \begin{stepcalc}[style = sar]
        	\consprod<8>
        \explnext{}
        	n (n + 7) \cdot (n + 1) (n + 6) \cdot (n + 2) (n + 5) \cdot (n + 4) (n + 3) 
        \explnext{}
        	(n^2 + 7n) (n^2 + 7n + 6) (n^2 + 7n + 10) (n^2 + 7n + 12) 
        \explnext*{$a = n^2 + 7n \in \ZintervalCO{8}{+\infty}$}{}
        	a (a + 6) (a + 10) (a + 12)
        \end{stepcalc}
    
    
    	\item Clairement $a \in 2 \NN$\,, donc $a = 2b$ avec $b \in \ZintervalCO{4}{+\infty}$\,, d'où les implications suivantes.
    
        \medskip
        \noindent\kern-10pt%
        \begin{stepcalc}[style = ar*, ope=\implies]
        	\consprod<8> \in \NNssquare
        \explnext{}
        	2b (2b + 6) (2b + 10) (2b + 12) \in \NNssquare
        \explnext{}
        	16 b (b + 3) (b + 5) (b + 6) \in \NNssquare
        \explnext*{Voir le fait \ref{facto-square}.}{}
        	b (b + 3) (b + 5) (b + 6) \in \NNssquare
        \explnext*{Via $b (b + 6) \cdot (b + 3) (b + 5)$\,.}{}
        	(b^2 + 6b) (b^2 + 8b + 15) \in \NNssquare
        \end{stepcalc}
    
    
    	\item Que faire de $(b^2 + 6b) (b^2 + 8b + 15)$ ?
    
        \smallskip
        \noindent%
        Tentons de passer via $c = b^2 + 7b$ la moyenne de $b^2 + 6b$ et $b^2 + 8b$\,. Voici une première constatation.
    
        \medskip
        \noindent\kern-10pt%
        \begin{stepcalc}[style = ar*]
        	(b^2 + 6b) (b^2 + 8b + 15)
        \explnext{}
        	(c - b) (c + b + 15)
        \explnext{}
        	(c - b) (c + b) + 15 (c - b)
        \explnext{}
        	c^2 - b^2 + 15 (b^2 + 6b)
        \explnext{}
        	c^2 + 14 b^2 + 90b
        \explnext{}
        	c^2 + 14 (c - 7b) + 90b
        \explnext{}
        	c^2 + 14 c - 8b
        \explnext[<]{}
        	c^2 + 14 c + 49
        \explnext[<]{}
        	(c + 7)^2
        \end{stepcalc}
        
        \smallskip
        \noindent
        Pas mal ! Avons-nous $(b^2 + 6b) (b^2 + 8b + 15) > (c + 6)^2$ ? Si oui, nous pourrons conclure. Voici ce que cela donne.
    
        \medskip
        \noindent\kern-10pt%
        \begin{stepcalc}[style = ar*]
        	(b^2 + 6b) (b^2 + 8b + 15) - (c + 6)^2
        \explnext{}
        	c^2 + 14 c - 8 b - (c + 6)^2
        \explnext{}
        	2 c - 8 b - 36
        \explnext{}
        	2 b^2 + 6 b - 36
        \explnext*[\geq]{$b \in \ZintervalCO{4}{+\infty}$}{}
        	32 + 24 - 36
        \explnext[>]{}
        	0
        \end{stepcalc}
    \end{itemize}
    
    \vspace{-2ex}	
    \leavevmode
\end{proof}
