\begin{fact} \label{case-5}
	 $\forall n \in \NNs$\,, $n(n+1)(n+2)(n+3)(n+4) \notin \NNssquare$\,.
\end{fact}


% ------------------ %


La preuve suivante s'inspire directement d'une démonstration citée via une source dans un échange sur \url{https://math.stackexchange.com} (voir la section \ref{sources}).


\begin{proof}[Preuve]%
    Supposons que $\consprod<5> \in \NNssquare$\,.
    
    \smallskip
    
    Clairement, 
    $\forall p \in \PP_{\geq 5}$\,, 
    $\forall i \in \ZintervalC{0}{4}$\,, 
    $\padicval{n + i} \in 2 \NN$\,.
    D'après le fait \ref{facto-square}, on doit s'intéresser à $p \in \setgene{2, 3}$\,, mais on peut observer très grossièrement qu'au maximum deux facteurs $(n + i)$ de $\consprod<5>$ sont divisibles par $3$\,, donc au moins $3$ facteurs sont de valuation $p$-adique paire dès que $p \in \PP_{\geq 3}$\,.
    Ces facteurs vérifient alors l'une des deux alternatives suivantes,
    chacune d'elles levant une contradiction.
    %
    \begin{itemize}
    	\medskip
		\item Deux facteurs différents $(n+i)$ et $(n+i^\prime)$ sont de valuations $2$-adiques impairs.
		
		\smallskip
		\noindent
		Dans ce cas, $(n+i, n+i^\prime) = (2 M^2, 2 N^2)$ avec $\abs{2(N^2 - M^2)} \in \ZintervalC{1}{4}$\,, c'est-à-dire $\abs{N^2 - M^2} \in \setgene{1, 2}$\,, mais c'est impossible d'après le fait \ref{diff-square-ko}.


    	\medskip
		\item Deux facteurs différents $(n+i)$ et $(n+i^\prime)$ sont de valuations $2$-adiques pairs.
		
		\smallskip
		\noindent
		Dans ce cas, $(n+i, n+i^\prime) = (M^2, N^2)$ avec $\abs{N^2 - M^2} \in \ZintervalC{1}{4}$\,, mais ceci n'est possible que si $\abs{N^2 - M^2} = 3$ d'après le fait \ref{diff-square-ko} qui donne aussi que soit $(M, N) = (1, 2)$\,, soit $(M, N) = (2, 1)$\,.
		Ceci impose d'avoir $n = 1$\,, mais $\consprod[1]<5> = 5! \notin \NNsquare$ car $\padicval[5]{5!} = 1$\,.
		%
		\qedhere
    \end{itemize}
\end{proof}

