Dans l'article \emph{\enquote{Note on Products of Consecutive Integers}}
\footnote{
	J. London Math. Soc. 14 (1939).
},
Paul Erdős démontre que pour tout couple $(n, k) \in \NNs \times \NNs$\,, le produit de $(k+1)$ entiers consécutifs $n (n + 1) \cdots (n + k)$ n'est jamais le carré d'un entier. 
Plus précisément, l'argument général de Paul Erdős est valable pour $k + 1 \geq 100$\,, soit à partir de $100$ facteurs.

\medskip

Dans ce document, nous donnes des preuves très simples de quelques cas particuliers. Quitte à nous répéter, nous avons rédigé au complet chaque preuve jusqu'au cas de $10$ facteurs, ceci permettant au lecteur de piocher des preuves au gré de ses envies.


\begin{remark}
	Vous trouverez dans mon document \emph{\enquote{Carrés parfaits et produits d'entiers consécutifs -- Des solutions à la main}} d'autres preuves, plus ou moins efficaces, mais toutes intéressantes dans leur approche.
\end{remark}

