\leavevmode
\smallskip

Supposons que $\consprod<2> = n(n+1)(n+2) \in \NNssquare$\,. Nous avons alors les $p$-tableaux de Vogler suivants pour $p \in \PP_{>2}$ divisant $\consprod<2>$\,, de nouveau car les valeurs $p$ doivent apparaître un nombre pair de fois, et être espacées par $(p-1)$ valeurs $1$ (voir les faits \ref{vogler-multiple} et \ref{vogler-parity-square}).

\begin{center}
	\begin{tblr}{
		colspec    = {Q[r,$]*{3}{Q[c,$]}},
		vline{2}   = {.95pt},
		vline{3-4} = {dashed},
		hline{2}   = {.95pt}
	}
		n + \bullet
			&  0  
			&  1 
			&  2
	\\
		p
			&  1
			&  1
			&  1
	\end{tblr}
\end{center}


Pour $p = 2$\,, via les faits \ref{vogler-multiple} et \ref{vogler-parity-square}, seulement deux $2$-tableaux de Vogler sont possibles. Nous utilisons un abus de notation évident pour indiquer ces deux possibilités.

\begin{center}
	\begin{tblr}{
		colspec    = {Q[r,$]*{3}{Q[c,$]}},
		vline{2}   = {.95pt},
		vline{3-4} = {dashed},
		hline{2}   = {.95pt}
	}
		n + \bullet
			&  0  
			&  1 
			&  2
	\\
		2
			&  1
			&  1
			&  1
	\\
			&  2
			&  1
			&  2
	\end{tblr}
\end{center}


La multiplication de tous les $d$-tableaux de Vogler précédents donne juste les deux tableaux de Vogler suivants, mais ceci est impossible d'après le fait \ref{illegal-vogler}.

\begin{center}
	\begin{tblr}{
		colspec    = {Q[r,$]*{3}{Q[c,$]}},
		vline{2}   = {.95pt},
		vline{3-4} = {dashed},
		hline{2}   = {.95pt},
		% Rejected
		cell{2}{2,3} = {red!15},
		cell{3}{2,4} = {red!15},
	}
		n + \bullet
			&  0  
			&  1 
			&  2
	\\
			&  1
			&  1
			&  1
	\\
		
			&  2
			&  1
			&  2
	\end{tblr}
\end{center}