\leavevmode
\smallskip

Supposons que $\consprod<3> = n(n+1)(n+2)(n+3) \in \NNssquare$\,. Nous avons alors les $p$-tableaux de Vogler suivants pour $p \in \PP_{>3}$ divisant $\consprod<3>$\,.

\begin{center}
	\begin{tblr}{
		colspec    = {Q[r,$]*{4}{Q[c,$]}},
		vline{2}   = {.95pt},
		vline{3-5} = {dashed},
		hline{2}   = {.95pt}
	}
		n + \bullet
			&  0  
			&  1 
			&  2 
			&  3
	\\
		p
			&  1
			&  1
			&  1
			&  1
	\end{tblr}
\end{center}


Pour $p = 2$\,, nous avons les trois $2$-tableaux de Vogler suivants.

\begin{center}
	\begin{tblr}{
		colspec    = {Q[r,$]*{4}{Q[c,$]}},
		vline{2}   = {.95pt},
		vline{3-5} = {dashed},
		hline{2}   = {.95pt}
	}
		n + \bullet
			&  0  
			&  1 
			&  2 
			&  3
	\\
		2
			&  1
			&  1
			&  1
			&  1
	\\
			&  2
			&  1
			&  2
			&  1
	\\
			&  1
			&  2
			&  1
			&  2
	\end{tblr}
\end{center}


Pour $p = 3$\,, nous obtenons les deux $3$-tableaux de Vogler suivants.

\begin{center}
	\begin{tblr}{
		colspec    = {Q[r,$]*{4}{Q[c,$]}},
		vline{2}   = {.95pt},
		vline{3-5} = {dashed},
		hline{2}   = {.95pt}
	}
		n + \bullet
			&  0  
			&  1 
			&  2 
			&  3
	\\
		3
			&  1
			&  1
			&  1
			&  1
	\\
			&  3
			&  1
			&  1
			&  3
	\end{tblr}
\end{center}


La multiplication des $d$-tableaux de Vogler précédents donne les tableaux de Vogler suivants.

\begin{center}
	\begin{tblr}{
		colspec    = {Q[r,$]*{4}{Q[c,$]}},
		vline{2}   = {.95pt},
		vline{3-5} = {dashed},
		hline{2}   = {.95pt},
		% Focus
		cell{5-7}{2-5} = {gray!15},
	}
		n + \bullet
			&  0  
			&  1 
			&  2 
			&  3
	\\
			&  1
			&  1
			&  1
			&  1
	\\
			&  2
			&  1
			&  2
			&  1
	\\
			&  1
			&  2
			&  1
			&  2
	\\
			&  3
			&  1
			&  1
			&  3
	\\
			&  6
			&  1
			&  2
			&  3
	\\
			&  3
			&  2
			&  1
			&  6
	\end{tblr}
\end{center}


%\newpage

Le fait \ref{illegal-vogler} rejette les quatre premiers tableaux de Vogler : voir les cellules surlignées ci-dessous.

\begin{center}
	\begin{tblr}{
		colspec    = {Q[r,$]*{4}{Q[c,$]}},
		vline{2}   = {.95pt},
		vline{3-5} = {dashed},
		hline{2}   = {.95pt},
		% Rejected
		cell{2}{2,3} = {red!15},
		cell{3}{2,4} = {red!15},
		cell{4}{3,5} = {red!15},
		cell{5}{3,4} = {red!15},
	}
		n + \bullet
			&  0  
			&  1 
			&  2 
			&  3
	\\
			&  1
			&  1
			&  1
			&  1
	\\
			&  2
			&  1
			&  2
			&  1
	\\
			&  1
			&  2
			&  1
			&  2
	\\
			&  3
			&  1
			&  1
			&  3
	\\
			&  6
			&  1
			&  2
			&  3
	\\
			&  3
			&  2
			&  1
			&  6
	\end{tblr}
\end{center}


Il nous reste à étudier les deux derniers tableaux reproduits ci-après.

\begin{center}
	\begin{tblr}{
		colspec    = {Q[r,$]*{4}{Q[c,$]}},
		vline{2}   = {.95pt},
		vline{3-5} = {dashed},
		hline{2}   = {.95pt}
	}
		n + \bullet
			&  0  
			&  1 
			&  2 
			&  3
	\\
			&  6
			&  1
			&  2
			&  3
	\\
			&  3
			&  2
			&  1
			&  6
	\end{tblr}
\end{center}


Commençons par démontrer que l'on ne peut pas avoir 
$n = 6 A^2$\,, $n + 1 = B^2$\,, $n + 2 = 2 C^2$ et $n + 3 = 3 D^2$ 
où $(A, B, C, D) \in \big( \NNs \big)^4$\,.

\begin{itemize}
	\item Posons $x = n + \frac32$\, de sorte que 
	$x - \frac32 = 6 A^2$\,, $x - \frac12 = B^2$\,, $x + \frac12 = 2 C^2$ et $x + \frac32 = 3 D^2$\,.

	\item Nous avons
	$\big( x - \frac32 \big) \big( x + \frac32 \big) = 2 E^2$\,, c'est-à-dire $x^2 - \frac94 = 2 E^2$\,, avec $E \in \NNs$.

	\item De même,
	$x^2 - \frac14 = 2 F^2$ avec $F \in \NNs$.	

	\item Par simple soustraction, nous obtenons $2 F^2 - 2 E^2 = 2$\,, puis $F^2 - E^2 = 1$\,, mais ceci contredit le fait \ref{dist-square}.
\end{itemize}


Le même type de raisonnement
\footnote{
	Noter la \enquote{symmétrie} respectée par les valeurs des deux tableaux.
}
démontre l'impossibilité d'avoir 
$n = 3 A^2$\,, $n + 1 = 2 B^2$\,, $n + 2 = C^2$ et $n + 3 = 6 D^2$ 
où $(A, B, C, D) \in \big( \NNs \big)^4$\,.

