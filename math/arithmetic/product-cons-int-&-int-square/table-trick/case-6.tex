\leavevmode
\smallskip

Supposons que $\consprod<4> = n(n+1)(n+2)(n+3)(n+4) \in \NNssquare$\,. Nous avons alors les $p$-tableaux de Vogler suivants pour $p \in \PP_{>5}$ divisant $\consprod<4>$\,.

\begin{center}
	\begin{tblr}{
		colspec    = {Q[r,$]*{5}{Q[c,$]}},
		vline{2}   = {.95pt},
		vline{3-6} = {dashed},
		hline{2}   = {.95pt}
	}
		n + \bullet
			&  0  
			&  1 
			&  2 
			&  3 
			&  4
	\\
		p
			&  1
			&  1
			&  1
			&  1
			&  1
	\end{tblr}
\end{center}


Pour $p = 2$\,, nous avons les $2$-tableaux de Vogler suivants.

\begin{center}
	\begin{tblr}{
		colspec    = {Q[r,$]*{5}{Q[c,$]}},
		vline{2}   = {.95pt},
		vline{3-6} = {dashed},
		hline{2}   = {.95pt}
	}
		n + \bullet
			&  0  
			&  1 
			&  2 
			&  3 
			&  4
	\\
		2
			&  1
			&  1
			&  1
			&  1
			&  1
	\\
			&  2
			&  1
			&  2
			&  1
			&  1
	\\
			&  2
			&  1
			&  1
			&  1
			&  2
	\\
			&  1
			&  2
			&  1
			&  2
			&  1
	\\
			&  1
			&  1
			&  2
			&  1
			&  2
	\end{tblr}
\end{center}


Pour $p = 3$\,, nous obtenons les $3$-tableaux de Vogler suivants.

\begin{center}
	\begin{tblr}{
		colspec    = {Q[r,$]*{5}{Q[c,$]}},
		vline{2}   = {.95pt},
		vline{3-6} = {dashed},
		hline{2}   = {.95pt}
	}
		n + \bullet
			&  0  
			&  1 
			&  2 
			&  3 
			&  4
	\\
		3
			&  1
			&  1
			&  1
			&  1
			&  1
	\\
			&  3
			&  1
			&  1
			&  3
			&  1
	\\
			&  1
			&  3
			&  1
			&  1
			&  3
	\end{tblr}
\end{center}


Pour $p = 5$\,, nous obtenons les $5$-tableaux de Vogler suivants.

\begin{center}
	\begin{tblr}{
		colspec    = {Q[r,$]*{5}{Q[c,$]}},
		vline{2}   = {.95pt},
		vline{3-6} = {dashed},
		hline{2}   = {.95pt}
	}
		n + \bullet
			&  0  
			&  1 
			&  2 
			&  3 
			&  4
	\\
		5
			&  1
			&  1
			&  1
			&  1
			&  1
	\\
			&  5
			&  1
			&  1
			&  1
			&  5
	\end{tblr}
\end{center}


La multiplication de tous les $d$-tableaux de Vogler précédents donnerait $30$ cas, mais comme nous savons que $\consprod<3> = n(n+1)(n+2)(n+3) \notin \NNssquare$\,, nous pouvons ignorer tous les tableaux commençant, ou finissant, par une valeur $1$ en plus de ceux rejetés par le le fait \ref{illegal-vogler}.
Ceci nous amène à ne considérer que les six tableaux de Vogler suivants.

\begin{center}
	\begin{tblr}{
		colspec  = {Q[r,$]*{5}{Q[c,$]}},
		vline{2} = {.95pt},
		vline{3-6} = {dashed},
		hline{2} = {.95pt}
	}
		n + \bullet
			&  0  
			&  1 
			&  2 
			&  3 
			&  4

	\\
			&  30
			&  1
			&  2
			&  3
			&  5
	\\
			&  15
			&  2
			&  1
			&  6
			&  5
	\\
			&  15
			&  1
			&  2
			&  3
			&  10
	\\
			&  10
			&  3
			&  2
			&  1
			&  15
	\\
			&  5
			&  6
			&  1
			&  2
			&  15
	\\
			&  5
			&  3
			&  2
			&  1
			&  30
	\end{tblr}
\end{center}


TODO