	
%    + Discussion du forum lesmathematiques.net
%    + math/arithmetic/prod-cons-carre-entier.pdf



Ce document n'aurait pas vu le jour sans les sources suivantes.


% ------------------ %


\subsection{Résolutions à fortes empreintes cognitives}

\leavevmode
\smallskip

\begin{enumerate}
	\item L'article \emph{\enquote{Le produit de 5 entiers consécutifs n'est pas le carré d'un entier.}} de T. Hayashi, Nouvelles Annales de Mathématiques, est consultable via \href{https://numdam.org}{Numdam}\,, la bibliothèque numérique française de mathématiques.
	
	\smallskip
	\noindent
	\emph{Cet article a fortement inspiré la preuve alternative du fait \ref{case-5}.}
\end{enumerate}


% ------------------ %


\subsection{Une méthode efficace}

\leavevmode
\smallskip

\begin{enumerate}
	\item Une discussion archivée consultée le 28 janvier 2024 : 
	
	\noindent
	\url{https://web.archive.org/web/20171110144534/http://mathforum.org/library/drmath/view/65589.html}\,.
	
	\smallskip
	\noindent
	\emph{Cette discussion utilise ce que nous avons nommé les tableaux de Vogler, mais le côté semi mécanisable de leur utilisation n'est pas souligné.}
\end{enumerate}