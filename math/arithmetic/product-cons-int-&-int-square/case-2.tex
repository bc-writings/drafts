\begin{fact} \label{case-2}
	 $\forall n \in \NNs$\,, $n(n+1)(n+2) \notin \NNsquare$\,.
\end{fact}


% ------------------ %


\begin{proof}
    Supposons que $\consprod<2> = n(n+1)(n+2) \in \NNssquare$\,.
    %
    Posant $m = n+1$\,, nous avons $\consprod<2> = (m-1)m(m+1) = m(m^2 - 1)$ où $m \in \NN_{\geq 2}$\,.
    %
    Comme $\forall p \in \PP$\,, $\padicval{\consprod<2>} \in 2 \NN$\,, et comme de plus $p \in \PP$ ne peut diviser à la fois $m$ et $m^2 - 1$\,, nous savons que 
    $\forall p \in \PP$\,, 
    $\padicval{m} \in 2 \NN$ et $\padicval{m^2 - 1} \in 2 \NN$\,,
    d'où 
    $(m, m^2 - 1) \in \NNsquare \times \NNsquare$\,.
    Or, d'après le fait \ref{dist-square}, $m^2 - 1 \in \NNsquare$ est impossible.
\end{proof}


% ------------------ %


\begin{proof}[Une preuve alternative]
    Supposons que $\consprod<2> = n(n+1)(n+2) \in \NNssquare$\,.
    %
    Comme $p \in \PP_{>2}$ ne peut diviser au maximum qu'un seul des trois facteurs $n$\,, $(n+1)$ et $(n+2)$\,, nous savons que 
    $\forall p \in \PP_{>2}$\,, 
    $\big( \padicval{n} , \padicval{n + 1} , \padicval{n + 2} \big) \in \big( 2 \NN \big)^3$\,.
    Mais que se passe-t-il pour $p = 2$ ?
    
    \medskip
    
    Supposons d'abord $n \in 2 \NN$\,.
	%
	\begin{itemize}
		\item Posant $n = 2 m$\,, nous avons $\consprod<2> = 4 m(2m+1)(m+1)$\,, d'où $m(2m+1)(m+1) \in \NNssquare$\,.
		
		\item Comme $\padicval[2]{2m+1} = 0$\,, nous savons que $2m+1 \in \NNssquare$\,.
		
		\item Donc $m(m+1)\in \NNssquare$\,, mais le fait \ref{case-1} interdit cela.
	\end{itemize}
    
    \medskip
    
    Supposons maintenant $n \in 2 \NN + 1$\,.
	%
	\begin{itemize}
		\item Nous savons que $n \in \NNssquare$ via $\padicval[2]{n} = 0$\,.

		\item Dès lors, on obtient $(n+1)(n+2) \in \NNssquare$\,, mais de nouveau ceci contredit le fait \ref{case-1}. \qedhere
	\end{itemize}
\end{proof}