\begin{fact} \label{case-2}
	 $\forall n \in \NNs$, $n(n+1) \notin \NNsquare$\,.
\end{fact}


% ------------------ %


\begin{proof}
	Il suffit de noter que $n^2 < n(n+1) < (n+1)^2$.
\end{proof}


% ------------------ %


\begin{proof}[Preuve alternative no.1]
    Supposons que $\consprod<1> = n(n+1) \in \NNssquare$\,.
    %
    Clairement $\forall p \in \PP$\,, $\padicval{\consprod<1>} \in 2 \NN$\,.
    %
    Or $p \in \PP$ ne peut diviser à la fois $n$ et $n+1$\,.
    %
    Nous savons donc que $\forall p \in \PP$\,, 
    $\padicval{n} \in 2 \NN$ et $\padicval{n+1} \in 2 \NN$\,,
    autrement dit 
    $(n, n + 1) \in \NNsquare \times \NNsquare$\,.
    D'après le fait \ref{dist-square}, nous savons que ceci est impossible.
\end{proof}


% ------------------ %


\begin{proof}[Preuve alternative no.2]
     Supposons que $\consprod<1> = n(n+1) = N^2$ où $N \in \NNs$.
     Les équivalences suivantes donnent alors une contradiction.
	
	\medskip
	
	\begin{stepcalc}[style = ar*, ope = \iff]
		n(n+1) = N^2
	\explnext*{$n(n+1) = 2 \dsum_{k=1}^{n} k$ et $N^2 = \dsum_{k=1}^{N} (2 k - 1)$\,.}{}
		2 \dsum_{k=1}^{n} k = \dsum_{k=1}^{N} (2 k - 1)
	\explnext{}
		\dsum_{k=1}^{n} 2k = \dsum_{k=1}^{N} 2 k - N
	\explnext*{$N > n$}{}
		\dsum_{k=n+1}^{N} 2k = N
	\explnext*{$N > 0$ rend impossible la dernière égalité.}{}
		\dsum_{k=n+1}^{N-1} 2k + N = 0
	\end{stepcalc}

	\vspace{-2ex}	
	\leavevmode
\end{proof}
