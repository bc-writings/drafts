$\forall N \in \NN$\,, $N^2 = \dsum_{k=1}^{N} (2 k - 1)$ donne le fait très utile suivant.


% -------------------- %


\begin{fact} \label{dist-square}
	$\forall (N, M) \in \NNs \times \NNs$, 
	si $N > M$\,, alors $N^2 - M^2 = \dsum_{k=M+1}^{N} (2 k - 1)$\,.
	
	En particulier, $N^2 - M^2 \geq 3 (N - M) \geq 3$ dès que $N > M$\,.
\end{fact}


% ------------------ %


%\begin{proof}
%	Quitte à échanger les rôles, on peut supposer $n > m$\,.
%	Par hypothèse, nous avons $(N, M) \in \NNs \times \NNs$ tel que $n = N^2$ et $m = M^2$\,.
%	Comme $n > m$\,, nous avons aussi $N > M$\,. 
%	Pour conclure, il suffit de s'appuyer sur les équivalences suivantes.
%	
%	\medskip
%	
%	\begin{stepcalc}[style = sar, ope = \iff]
%		N > M
%	\explnext{}
%		N \geq M + 1
%	\explnext{}
%		N^2 \geq (M + 1)^2
%	\explnext{}
%		n \geq m + 2 M + 1
%	\explnext{}
%		n - m \geq 2 M + 1 \hfill
%	\end{stepcalc}
%
%	\vspace{-2ex}	
%	\leavevmode
%\end{proof}
