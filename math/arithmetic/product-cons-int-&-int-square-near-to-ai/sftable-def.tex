% ------------------ %


\subsection{Les \sftab[x]}

\leavevmode
\smallskip

L'idée de départ est simple : d'après le fait \ref{facto-square}, il semble opportun de se concentrer sur les diviseurs sans facteur carré des $k$ facteurs $(n + i)$ de $\consprod = n (n + 1) \cdots (n + k - 1)$\,.


% ------------------ %


%\newpage
\begin{defi}
	Considérons $(n, k) \in ( \NNs )^2$\,,
	$( a_i )_{0 \leq i \leq k} \subset \NNsf$
	et
	$( s_i )_{0 \leq i \leq k} \subset \NNssquare$
	tels que
	$\forall i \in \ZintervalC{0}{k}$\,, $n + i = a_i s_i$\,.
	%
	Cette situation est résumée par le tableau suivant que nous nommerons \enquote{\sftab}
	\footnote{
		\enquote{sf} est pour \enquote{square free}\,.
	}.

	\begin{center}
		\begin{tblr}{
			colspec    = {Q[r,$]*{5}{Q[c,$]}},
			vline{2}   = {.95pt},
			vline{3-6} = {dashed},
			hline{2}   = {.95pt}
		}
			n + \bullet
				&  0    &  1    &  2    &  \dots  &  k
		\\
				&  a_0  &  a_1  &  a_2  &  \dots  &  a_k
		\end{tblr}
	\end{center}
%	Si de plus $( a_i )_{0 \leq i \leq k} \subseteq \NNsf$\,, le \sftab\ sera dit réduit.
\end{defi}


% ------------------ %


%\newpage
\begin{example}
	Supposons avoir le \sftab\ suivant où $n \in \NNs$.

	\begin{center}
		\begin{tblr}{
			colspec    = {Q[r,$]*{4}{Q[c,$]}},
			vline{2}   = {.95pt},
			vline{3-5} = {dashed},
			hline{2}   = {.95pt}
		}
			n + \bullet
				&  0  &  1  &  2  &  3
		\\
				&  2  &  5  &  6  &  1
		\end{tblr}
	\end{center}

	Ceci résume la situation suivante.

	\vspace{-1ex}
	\begin{multicols}{2}
	\begin{itemize}
		\item $\exists A \in \NNs$ tel que $n     = 2 A^2$\,.

		\item $\exists B \in \NNs$ tel que $n + 1 = 5 B^2$\,.

		\item $\exists C \in \NNs$ tel que $n + 2 = 6 C^2$\,.

		\item $\exists D \in \NNs$ tel que $n + 3 =   D^2$\,.
	\end{itemize}
	\end{multicols}
\end{example}


% ------------------ %


\begin{defi}
	Soient $r \in \NNs$,
	$( n_i )_{1 \leq i \leq r} \in \NNseq[r]$\,,
	$( a_i )_{1 \leq i \leq r} \subset \NNsf$
	et
	$( s_i )_{1 \leq i \leq r} \subset \NNssquare$
	tels que
	$\forall i \in \ZintervalC{1}{r}$\,, $n_i = a_i s_i$\,.
	%
	Cette situation est résumée par le tableau suivant que nous nommerons \enquote{\sftab\ généralisé}\,.

	\begin{center}
		\begin{tblr}{
			colspec    = {Q[r,$]*{5}{Q[c,$]}},
			vline{2}   = {.95pt},
			vline{3-6} = {dashed},
			hline{2}   = {.95pt}
		}
			\bullet
				&  n_1  &  n_2  &  n_3  &  \dots  &  n_r
		\\
				&  a_1  &  a_2  &  a_3  &  \dots  &  a_r
		\end{tblr}
	\end{center}
%	Si de plus $( a_i )_{0 \leq i \leq k} \subseteq \NNsf$\,, le \sftab\ sera dit réduit.
\end{defi}


% ------------------ %


\subsection{Construire des \sftab[x]}

\leavevmode

\smallskip
Pour fabriquer des \sftab[x], nous allons \enquote{multiplier} des \sftab[x] dits partiels.


\begin{defi}
	Soient $(n, k, r) \in ( \NNs )^3$\,,
	$( p_j )_{1 \leq j \leq r} \in \PPseq[r]$\,,
	$( \epsilon_{i,j} )_{0 \leq i \leq k \,, 1 \leq j \leq r} \subseteq \setgene{0, 1}$
	et aussi
	$( f_i )_{0 \leq i \leq k} \subset \NNs$
	vérifiant les conditions suivantes.
	%
	\begin{itemize}
		\item $\forall i \in \ZintervalC{0}{k}$\,,
		$n + i = f_i \cdot \dprod_{j = 1}^{r} p_j^{\,\padicval[p_j]{n+i}}$\,.
		Noter que
		$\forall i \in \ZintervalC{0}{k}$\,,
		$\forall j \in \ZintervalC{1}{r}$\,,
		$\GCD{f_i}{p_j} = 1$\,.

		\item $\forall i \in \ZintervalC{0}{k}$\,,
		$\forall j \in \ZintervalC{1}{r}$\,,
		$\padicval[p_j]{n+i} \equiv \epsilon_{i,j}$ modulo $2$\,.
	\end{itemize}

	\smallskip

	Cette situation est résumée par le tableau suivant qui sera nommé \enquote{\sftab\ partiel}\,, voire \enquote{\sftab\ partiel d'ordre $( p_j )_{1 \leq j \leq r}$\!}\,
	\footnote{
		Noter que $\forall i \in \ZintervalC{0}{k}$\,, $\forall j \in \ZintervalC{1}{r}$\,, $p_j^{\,\epsilon_{i,j}} \in \setgene{1, p_j}$\,.
	}.

	\begin{center}
		\begin{tblr}{
			colspec     = {Q[r,$]*{5}{Q[c,$]}},
			vline{2}    = {.95pt},
			vline{3-6}  = {dashed},
			hline{2}    = {.95pt},
			column{2-Z} = {2.75em},
		}
			n + \bullet
				&  0
				&  1
				&  2
				&  \dots
				&  k
		\\
			( p_j )_{1 \leq j \leq r}
				&  \dprod_{j = 1}^{r} p_j^{\,\epsilon_{0,j}}
				&  \dprod_{j = 1}^{r} p_j^{\,\epsilon_{1,j}}
				&  \dprod_{j = 1}^{r} p_j^{\,\epsilon_{2,j}}
				&  \dots
				&  \dprod_{j = 1}^{r} p_j^{\,\epsilon_{k,j}}
		\end{tblr}
	\end{center}
\end{defi}


% ------------------ %


\begin{example}
	Supposons avoir le \sftab\ partiel suivant où $n \in \NNs$.

	\begin{center}
		\begin{tblr}{
			colspec    = {Q[r,$]*{4}{Q[c,$]}},
			vline{2}   = {.95pt},
			vline{3-5} = {dashed},
			hline{2}   = {.95pt},
%			column{2-Z} = {1.75em},
		}
			n + \bullet
				&  0  &  1  &  2  &  3
		\\
			(2, 3)
				&  2  &  6  &  1  &  3
		\end{tblr}
	\end{center}

%	\newpage
	Ceci résume la situation suivante.
	%
	\begin{itemize}
		\item $\exists (a, \alpha, A) \in \NN^2 \times \NNs$
		      tel que $A \wedge 6 = 1$
		      et
		      $n     = 2^{2a+1} 3^{2\alpha} A$\,.

		\item $\exists (b, \beta, B) \in \NN^2 \times \NNs$
		      tel que $B \wedge 6 = 1$
		      et
		      $n + 1 = 2^{2b+1} 3^{2\beta+1} B$\,.

		\item $\exists (c, \gamma, C) \in \NN^2 \times \NNs$
		      tel que $C \wedge 6 = 1$
		      et
		      $n + 2 = 2^{2c} 3^{2\gamma} C$\,.

		\item $\exists (d, \delta, D) \in \NN^2 \times \NNs$
		      tel que $D \wedge 6 = 1$
		      et
		      $n + 3 = 2^{2d} 3^{2\delta+1} D$\,.
	\end{itemize}
\end{example}


% ------------------ %


\begin{example}
	La multiplication de deux \sftab[x] partiels de deux suites
	$( p_j )_{1 \leq j \leq r} \in \PPseq[r]$
	et
	$( q_j )_{1 \leq j \leq s} \in \PPseq[s]$
	d'intersection vide, c'est-à-dire sans nombre premier commun, est \enquote{naturelle}\,.
	Considérons les deux \sftab[x] partiels suivants où l'on note $2$ et $3$ au lieu de $(2)$ et $(3)$\,.

	\vspace{-1.5ex}
	\begin{multicols}{2}
	\begin{center}
		\begin{tblr}{
			colspec    = {Q[r,$]*{4}{Q[c,$]}},
			vline{2}   = {.95pt},
			vline{3-5} = {dashed},
			hline{2}   = {.95pt}
		}
			n + \bullet
				&  0  &  1  &  2  &  3
		\\
			2
				&  1  &  2  &  1  &  2
		\end{tblr}
	\end{center}

	\begin{center}
		\begin{tblr}{
			colspec    = {Q[r,$]*{4}{Q[c,$]}},
			vline{2}   = {.95pt},
			vline{3-5} = {dashed},
			hline{2}   = {.95pt}
		}
			n + \bullet
				&  0  &  1  &  2  &  3
		\\
			3
				&  3  &  1  &  1  &  3
		\end{tblr}
	\end{center}
	\end{multicols}


	\vspace{-1ex}
	La multiplication de ces \sftab[x] partiels est le \sftab\  suivant, partiel a priori, mais si l'on sait que $2$ et $3$ sont les seuls diviseurs premiers de $\consprod<4>$\,, alors le \sftab\ est non partiel.

	\begin{center}
		\begin{tblr}{
			colspec    = {Q[r,$]*{4}{Q[c,$]}},
			vline{2}   = {.95pt},
			vline{3-5} = {dashed},
			hline{2}   = {.95pt}
		}
			n + \bullet
				&  0  &  1  &  2  &  3
		\\
			(2, 3)
				&  3  &  2  &  1  &  6
		\end{tblr}
	\end{center}

	Ceci résume la situation suivante qui est équivalente à ce que donne la conjonction des deux premiers \sftab[x] partiels (les abus de notations sont évidents).


	\vspace{-1ex}
	\begin{multicols}{2}
	\begin{itemize}
		\item $A \wedge 6 = 1$
		      et
		      $n     = 2^{2a}   3^{2\alpha+1} A$\,.

		\item $B \wedge 6 = 1$
		      et
		      $n + 1 = 2^{2b+1} 3^{2\beta}    B$\,.

		\item $C \wedge 6 = 1$
		      et
		      $n + 2 = 2^{2c}   3^{2\gamma}   C$\,.

		\item $D \wedge 6 = 1$
		      et
		      $n + 3 = 2^{2d+1} 3^{2\delta+1} D$\,.
	\end{itemize}
	\end{multicols}
\end{example}

