% ------------------ %


\subsection{A propos des \sftab[x] partiels} \label{sftable-constraint}


\begin{fact} \label{sftable-multiple}

	Dans la deuxième ligne d'un \sftab\ partiel d'ordre $p$\,, les positions des valeurs $p$ sont congrues modulo $p$\,.
\end{fact}


\begin{proof}
	Penser aux multiples de $p$\,.
\end{proof}


% ------------------ %


\begin{fact} \label{sftable-parity-square}
	$\forall (n, k, p) \in ( \NNs )^2 \times \PP$\,,
	si $\consprod \in \NNsquare$\,,
	alors dans le \sftab\ partiel d'ordre $p$ associé à $\consprod$\,, le nombre de valeurs $p$ est forcément pair.
\end{fact}


\begin{proof}
	Évident, mais très pratique, comme nous le verrons dans la suite.
\end{proof}


% ------------------ %


\subsection{A propos des \sftab[x] non partiels} \label{sftab-illegal}


\begin{fact} \label{sftab-recu}
	Dans les tableaux ci-dessous, où $k \geq 2$\,, les puces $\bullet$ indiquent des valeurs quelconques.

	\begin{enumerate}
		\item Si nous avons un \sftab\ du type suivant, alors $\consprod<k-1> \in \NNssquare$\,.
	\end{enumerate}

	\begin{center}
		\begin{tblr}{
			colspec     = {Q[r,$]*{5}{Q[c,$]}},
			vline{2}    = {.95pt},
			vline{3-6}  = {dashed},
			hline{2}    = {.95pt},
			column{2-Z} = {2em},
			% Subsquare
			cell{2}{6} = {blue!15},
		}
			n + \bullet
				&  0        &  1        &  \dots  &  k - 1    &  k
		\\
				&  \bullet  &  \bullet  &  \dots  &  \bullet  &  1
		\end{tblr}
	\end{center}


	\begin{enumerate}[start=2]
		\item Si nous avons un \sftab\ du type suivant, alors $\consprod[n+1]<k-1> \in \NNssquare$\,.
	\end{enumerate}

	\begin{center}
		\begin{tblr}{
			colspec     = {Q[r,$]*{5}{Q[c,$]}},
			vline{2}    = {.95pt},
			vline{3-6}  = {dashed},
			hline{2}    = {.95pt},
			column{2-Z} = {2em},
			% Subsquare
			cell{2}{2} = {blue!15},
		}
			n + \bullet
				&  0  &  1        &  \dots  &  k - 1    &  k
		\\
				&  1  &  \bullet  &  \dots  &  \bullet  &  \bullet
		\end{tblr}
	\end{center}
\end{fact}


\begin{proof}
	Immédiat via le fait \ref{facto-square}, car nous avons soit $n + k \in \NNssquare$\,, soit $n \in \NNssquare$\,.
\end{proof}


% ------------------ %


\begin{fact} \label{sftable-illegal-0-sol}
	Soit le \sftab\ généralisé ci-après où
	$r \in \NN_{\geq 2}$\,,
	$( n_i )_{1 \leq i \leq r} \in \NNseq[r]$
	et
	$d \in \NNsf$\,.

    \begin{center}
    	\begin{tblr}{
    		colspec     = {Q[r,$]*{3}{Q[c,$]}},
    		vline{2}    = {.95pt},
    		vline{3-Y}  = {dashed},
   			hline{2}    = {.95pt},
			column{2-Z} = {2em},
			% Rejected
			cell{2}{2,Z} = {red!15},
    	}
    		\bullet
   				&  n_1  &  \dots  &  n_r
    	\\
    			&  d    &  \dots  &  d
    	\end{tblr}
    \end{center}

	Ce \sftab\ est impossible si l'une des deux conditions suivantes est validée.
	
	\begin{enumerate}
		\item $\dfrac{n_r - n_1}{d} \notin \NN$\,.

		\item $\dfrac{n_r - n_1}{d} \in \{ 1, 2, 4, 6, 10, 14, 18, 22, 26, 30, 34, 38, 42, 46, 50, 54, 58, 62, 66, 70, 74, 78, 82,$ \\ $86, 90, 94, 98 \}$\,.
	\end{enumerate}
\end{fact}


\begin{proof}
	$n_1 = d A^2$ et $n_r = d B^2$ nous donnent $d (B^2 - A^2) = n_r - n_1$\,. On conclut directement pour le premier cas, et via le fait \ref{diff-square-ko} dans le second.
\end{proof}

