% ------------------ %


\subsection{A propos des \sftab[x] partiels} \label{sftable-constraint}


\begin{fact} \label{sftable-multiple}

	Dans la deuxième ligne d'un \sftab\ partiel d'ordre $p$\,, les positions des valeurs $p$ sont congrues modulo $p$\,.
\end{fact}


\begin{proof}
	Penser aux multiples de $p$\,.
\end{proof}


% ------------------ %


\begin{fact} \label{sftable-parity-square}
	$\forall (n, k, p) \in \NNs \times \NN \times \PP$\,,
	si $\consprod \in \NNsquare$\,,
	alors dans le \sftab\ partiel d'ordre $p$ associé à $\consprod$\,, le nombre de valeurs $p$ est forcément pair.
\end{fact}


\begin{proof}
	Évident, mais très pratique, comme nous le verrons dans la suite.
\end{proof}


% ------------------ %


\subsection{A propos des \sftab[x] non partiels} \label{sftab-illegal}


\begin{fact} \label{sftab-recu}
	Dans les tableaux ci-dessous, les puces $\bullet$ indiquent des valeurs quelconques.

	\begin{enumerate}
		\item Si nous avons un \sftab\ du type suivant, alors $\consprod<k-1> \in \NNssquare$\,.
	\end{enumerate}

	\begin{center}
		\begin{tblr}{
			colspec     = {Q[r,$]*{5}{Q[c,$]}},
			vline{2}    = {.95pt},
			vline{3-6}  = {dashed},
			hline{2}    = {.95pt},
			column{2-Z} = {2em},
			% Subsquare
			cell{2}{6} = {blue!15},
		}
			n + \bullet
				&  0
				&  1
				&  \dots
				&  k - 1
				&  k
		\\
				&  \bullet
				&  \bullet
				&  \dots
				&  \bullet
				&  1
		\end{tblr}
	\end{center}


	\begin{enumerate}[start=2]
		\item Si nous avons un \sftab\ du type suivant, alors $\consprod[n+1]<k-1> \in \NNssquare$\,.
	\end{enumerate}

	\begin{center}
		\begin{tblr}{
			colspec     = {Q[r,$]*{5}{Q[c,$]}},
			vline{2}    = {.95pt},
			vline{3-6}  = {dashed},
			hline{2}    = {.95pt},
			column{2-Z} = {2em},
			% Subsquare
			cell{2}{2} = {blue!15},
		}
			n + \bullet
				&  0
				&  1
				&  \dots
				&  k - 1
				&  k
		\\
				&  1
				&  \bullet
				&  \dots
				&  \bullet
				&  \bullet
		\end{tblr}
	\end{center}
\end{fact}


\begin{proof}
	Immédiat via le fait \ref{facto-square}, car nous avons soit $n + k \in \NNssquare$\,, soit $n \in \NNssquare$\,.
\end{proof}


% ------------------ %


\begin{fact} \label{sftable-illegal-0-sol}
	Soit $(n, d, a, k) \in ( \NNs )^4$ et $i \in \NN$\,.
	Considérons le \sftab\ ci-après où les puces $\bullet$ indiquent des valeurs quelconques.

    \begin{center}
    	\begin{tblr}{
    		colspec     = {Q[r,$]*{5}{Q[c,$]}},
    		vline{2}    = {.95pt},
    		vline{3-6}  = {dashed},
   			hline{2}    = {.95pt},
			column{2-Z} = {3.75em},
			% Rejected
			cell{2}{2,6} = {red!15},
    	}
    		n + \bullet
   				&  i
    			&  i + 1
    			&  \dots
    			&  i + k d - 1
    			&  i + k d
    	\\
    			&  ad
				&  \bullet
    			&  \dots
				&  \bullet
				&  ad
    	\end{tblr}
    \end{center}

	Ce \sftab\ est impossible si l'une des deux conditions suivantes est validée.

	\begin{enumerate}[start=2]
		\item $a \ndivides k$\,.

		\item $a \divides k$ et $\frac{k}{a} \in \setgene{1, 2, 4, 6, 10, 14, 18}$\,.
	\end{enumerate}
\end{fact}


\begin{proof}
	$n + i = ad A^2$ et $n + i + k d = ad B^2$ nous donnent $ad (B^2 - A^2) = k d$\,, puis $a (B^2 - A^2) = k$\,. On conclut directement si $a \ndivides k$\,, et via le fait \ref{diff-square-ko} dans le second cas.
\end{proof}

