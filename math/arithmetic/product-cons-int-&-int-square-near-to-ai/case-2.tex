Supposons que $\consprod<1> = n(n+1) \in \NNssquare$\,. Nous avons alors les tableaux de Vogler partiels suivants pour $p \in \PP$ divisant $\consprod<1>$\,, car les valeurs $p$ de la deuxième ligne doivent apparaître un nombre pair de fois tout en étant espacées par $(p-1)$ valeurs $1$ (voir les faits \ref{vogler-multiple} et \ref{vogler-parity-square}). 

\begin{center}
	\begin{tblr}{
		colspec  = {Q[r,$]*{2}{Q[c,$]}},
		vline{2} = {.95pt},
		vline{3} = {dashed},
		hline{2} = {.95pt}
	}
		n + \bullet
			&  0  
			&  1
	\\
		p
			&  1
			&  1
	\end{tblr}
\end{center}


%\newpage
La multiplication de tous les tableaux de Vogler partiels précédents donne le tableau de Vogler, non partiel, ci-après, mais ceci contredit le fait \ref{illegal-vogler}.

\begin{center}
	\begin{tblr}{
		colspec  = {Q[r,$]*{2}{Q[c,$]}},
		vline{2} = {.95pt},
		vline{3} = {dashed},
		hline{2} = {.95pt},
		% Rejected
		cell{2}{2,3} = {red!15},
	}
		n + \bullet
			&  0  
			&  1
	\\
			&  1
			&  1
	\end{tblr}
\end{center}