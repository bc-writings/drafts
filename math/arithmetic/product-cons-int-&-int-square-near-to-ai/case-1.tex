\subsection{Structure}


\begin{fact} \label{facto-square}
	$\forall n \in \NNssquare$\,, s'il existe $m \in \NNssquare$ tel que $n =  f m$ alors $f  \in \NNssquare$\,.
\end{fact}


\begin{proof}
	Clairement, $\forall p \in \PP$\,, nous avons 
	$\padicval{f m} \in 2 \NN$
	et
	$\padicval{m} \in 2 \NN$
	qui donnent
	$\padicval{f} \in 2 \NN$
	car $\padicval{f m} = \padicval{f} + \padicval{m}$\,.
\end{proof}


% ------------------ %


\begin{fact} \label{prime-square}
	$\forall (a, b) \in \NNs \times \NNs$, 
	si $\GCD{a}{b} = 1$ et $a b \in \NNssquare$\,,
	alors $a \in \NNssquare$ et $b \in \NNssquare$\,.
\end{fact}


\begin{proof}
	Clairement, $\forall p \in \PP$\,, nous avons $\padicval{ab} \in 2 \NN$\,.
    %
    Or $p \in \PP$ ne peut diviser à la fois $a$ et $b$\,, donc
    $\forall p \in \PP$\,, 
    $\padicval{a} \in 2 \NN$ et $\padicval{b} \in 2 \NN$\,,
    autrement dit 
    $(a, b) \in \NNssquare \times \NNssquare$\,.
\end{proof}


% ------------------ %


\begin{fact} \label{same-square-free}
	Soit $(a, b) \in \NNs \times \NNs$ tel que $a b \in \NNssquare$\,,
	ainsi que $(\alpha, \beta, A, B) \in ( \NNsf )^2 \times \NN^2$ tel que $a = \alpha A^2$ et $b = \beta B^2$.
	Nous avons alors forcément $\alpha = \beta$\,.
\end{fact}


\begin{proof}
	Le fait \ref{facto-square} donne $\alpha \beta \in \NNssquare$\,.
	De plus, $\forall p \in \PP$\,, nous avons 
	$\padicval{\alpha} \in \setgene{0, 1}$
	et
	$\padicval{\beta} \in \setgene{0, 1}$\,.
	Finalement, $\forall p \in \PP$\,, $\padicval{\alpha} = \padicval{\beta}$\,, autrement dit $\alpha = \beta$\,.
\end{proof}


% ------------------ %


\subsection{Distance entre deux carrés parfaits}

	
\begin{fact} \label{dist-square}
	$\forall (N, M) \in \NNs \times \NNs$, 
	si $N > M$\,, alors $N^2 - M^2 = \dsum_{k=M+1}^{N} (2 k - 1)$\,.
\end{fact}


\begin{proof}
	$N^2 = \dsum_{k=1}^{N} (2 k - 1)$ donne l'identité indiquée
	\footnote{
		La formule utilisée est facile à démontrer algébriquement, et évidente à découvrir géométriquement.
	}.
\end{proof}


% ------------------ %


L'identité précédente permet d'éliminer beaucoup de situations en s'aidant, si besoin, d'un petit programme informatique (voir un peu plus bas).

\begin{fact} \label{diff-square-ko}
	Soit $(N, M) \in \NNs \times \NNs$ tel que $N > M$\,.
	%
	\begin{enumerate}
		\item $N^2 - M^2 \geq 2M + 1$\,.
		
		\item Notons $nb_{sol}$ le nombre de solutions $(N, M) \in \NNs \times \NNs$ de $N^2 - M^2 = \delta$\,.
		Pour $\delta \in \ZintervalC{1}{20}$, nous avons :
		\begin{enumerate}
			\item $nb_{sol}= 0$ si $\delta \in \setgene{1, 2, 4, 6, 10, 14, 18}$\,.

			\item $nb_{sol}= 1$ si $\delta \in \setgene{3, 5, 7, 8, 9, 11, 12, 13, 16}$\,.

			\item $nb_{sol}= 2$ si $\delta = 15$\,.
		\end{enumerate}
	\end{enumerate}
\end{fact}


\begin{proof}
	\leavevmode
	
	\vspace{-1ex}
	\begin{enumerate}
		\item $N^2 - M^2 = \dsum_{k=M+1}^{N} (2 k - 1) \geq 2(M + 1) - 1 = 2M + 1$
		\footnote{
			On peut aussi noter que 
			$N^2 - M^2 = (N - M)(N + M) \geq 1 \cdot (M + 1 + M) = 2M + 1$\,.
		}


		\item Notant $\delta = N^2 - M^2$ \,, nous avons $2 N - 1 \leq \dsum_{k=M+1}^{N} (2 k - 1) = \delta$\,, soit $N  \leq \dfrac{\delta + 1}{2}$\,. 
		Ceci permet de comprendre le programme \verb#Python# donné ci-dessous qui nous donne les nombres de solutions.
		%
		\qedhere
	\end{enumerate}
\end{proof}


\bgroup
\small
\begin{Python}
from math import sqrt, floor

def sol(diff):
    solfound = []

    for i in range(1, (diff + 1) // 2 + 1):
        tested = i**2 - diff

        if tested < 0:
            continue

        tested = floor(sqrt(i**2 - diff))

        if tested == 0:
            continue

        if tested**2 == i**2 - diff:
            solfound.append((i, tested))

    return solfound
\end{Python}
\egroup
