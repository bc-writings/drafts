Il est facile de trouver des démonstrations à la main du fait que $n(n+1) \cdots (n + k) \notin \NNssquare$ pour les premières valeurs de $k$.
Certains preuves sont très sympathiques, mais, malheureusement, elles ne reposent pas sur un schéma commun de raisonnement.
Dans la suite, nous allons tenter de limiter au maximum l'emploi de fourberies déductives ; pour cela, commençons par noter le fait suivant
\footnote{
	$N^2 - M^2 = (N - M)(N + M)$ donne directement $N^2 - M^2 \geq 3$ dès que $(N, M) \in \big( \NNs \big)^2$ vérifie $N > M$\,.
}.


\begin{fact} \label{dist-square}
	$\forall (N, M) \in \NNs \times \NNs$, 
	si $N > M$\,, alors $N^2 - M^2 = \dsum_{k=M+1}^{N} (2 k - 1)$\,.
\end{fact}


\begin{proof}
	Il suffit d'utiliser $N^2 = \dsum_{k=1}^{N} (2 k - 1)$\,.
\end{proof}


% ------------------ %


\begin{fact} \label{facto-square}
	$\forall n \in \NNssquare$\,, s'il existe $m \in \NNssquare$ tel que $n =  f m$ alors $f  \in \NNssquare$\,.
\end{fact}


\begin{proof}
	Il suffit de passer via les décompositions en facteurs premiers de $n$\,, $m$ et $f$\,.
\end{proof}


% ------------------ %


Passons aux diviseurs sans facteur carré des facteurs $(n + i)$ de $\consprod = \dprod_{i = 0}^{k} (n + i)$\,. 


\begin{defi}
	Soient $(n, k) \in \big( \NNs \big)^2$\,,
	$( a_i )_{0 \leq i \leq k} \subseteq \NNs$
	et
	$( s_i )_{0 \leq i \leq k} \subseteq \NNssquare$
	tels que 
	$\forall i \in \ZintervalC{0}{k}$\,, $n + i = a_i s_i$\,.
	%
	Ce type de situation sera résumé par le tableau suivant que nous nommerons tableau de Vogler en référence à la discussion où l'auteur a découvert ce concept.

	\begin{center}
		\begin{tblr}{
			colspec    = {Q[r,$]*{5}{Q[c,$]}},
			vline{2}   = {.95pt},
			vline{3-6} = {dashed},
			hline{2}   = {.95pt}
		}
			n + \bullet
				&  0  
				&  1  
				&  2  
				&  \dots
				&  k
		\\
				&  a_0
				&  a_1
				&  a_2
				&  \dots  
				&  a_k
		\end{tblr}
	\end{center}
\end{defi}


% ------------------ %


%\newpage
\begin{example}
	Supposons avoir le tableau de Vogler suivant où $n \in \NNs$.

	\begin{center}
		\begin{tblr}{
			colspec    = {Q[r,$]*{4}{Q[c,$]}},
			vline{2}   = {.95pt},
			vline{3-5} = {dashed},
			hline{2}   = {.95pt}
		}
			n + \bullet
				&  0  
				&  1  
				&  2  
				&  3
		\\
				&  2
				&  5
				&  6
				&  1
		\end{tblr}
	\end{center}
	
	Ceci résume la situation suivante. 
	
	\begin{itemize}
		\item $\exists A \in \NNs$ tel que $n     = 2 A^2$\,.

		\item $\exists B \in \NNs$ tel que $n + 1 = 5 B^2$\,.

		\item $\exists C \in \NNs$ tel que $n + 2 = 6 C^2$\,.

		\item $\exists D \in \NNs$ tel que $n + 3 = D^2  $\,.
	\end{itemize}
\end{example}


% ------------------ %


\begin{fact} \label{vogler-sub-square}
	Dans les tableaux ci-dessous, les puces $\bullet$ indiquent des valeurs quelconques.
	
	\begin{enumerate}
		\item Si nous avons un tableau de Vogler du type suivant, alors $\consprod<k-1> \in \NNssquare$\,.
	\end{enumerate}

	\begin{center}
		\begin{tblr}{
			colspec     = {Q[r,$]*{5}{Q[c,$]}},
			vline{2}    = {.95pt},
			vline{3-6}  = {dashed},
			hline{2}    = {.95pt},
			column{2-Z} = {2em},
			% Subsquare
			cell{2}{6} = {blue!15},
		}
			n + \bullet
				&  0  
				&  1  
				&  \dots
				&  k - 1
				&  k
		\\
				&  \bullet
				&  \bullet
				&  \dots  
				&  \bullet
				&  1
		\end{tblr}
	\end{center}


	\begin{enumerate}[start=2]
		\item Si nous avons un tableau de Vogler du type suivant, alors $\consprod[n+1]<k-1> \in \NNssquare$\,.
	\end{enumerate}

	\begin{center}
		\begin{tblr}{
			colspec     = {Q[r,$]*{5}{Q[c,$]}},
			vline{2}    = {.95pt},
			vline{3-6}  = {dashed},
			hline{2}    = {.95pt},
			column{2-Z} = {2em},
			% Subsquare
			cell{2}{2} = {blue!15},
		}
			n + \bullet
				&  0  
				&  1  
				&  \dots
				&  k - 1
				&  k
		\\
				&  1
				&  \bullet
				&  \dots  
				&  \bullet
				&  \bullet
		\end{tblr}
	\end{center}
\end{fact}


\begin{proof}
	C'est immédiat via le fait \ref{facto-square}.
\end{proof}


% ------------------ %


\begin{fact} \label{illegal-vogler}
	Soit $(n, d, i, a) \in \big( \NNs \big)^4$.
	Les tableaux de Vogler ci-après sont impossibles. 


%	\begin{enumerate}
%		\item Pas de facteurs carrés consécutifs.
%	\end{enumerate}
%		
%    \begin{center}
%    	\begin{tblr}{
%    		colspec     = {Q[r,$]*{2}{Q[c,$]}},
%    		vline{2}    = {.95pt},
%    		vline{3}    = {dashed},
%   			hline{2}    = {.95pt},
%			column{2-Z} = {2em},
%			% Rejected
%			cell{2}{2,3} = {red!15},
%    	}
%    		n + \bullet
%   				&  i 
%    			&  i + 1
%    	\\
%    			&  a
%				&  a
%    	\end{tblr}
%    \end{center}


	\begin{enumerate}
		\item Pas de facteurs carrés trop près (les puces $\bullet$ indiquent des valeurs inconnues).
	\end{enumerate}
		
    \begin{center}
    	\begin{tblr}{
    		colspec     = {Q[r,$]*{5}{Q[c,$]}},
    		vline{2}    = {.95pt},
    		vline{3-6}  = {dashed},
   			hline{2}    = {.95pt},
			column{2-Z} = {3em},
			% Rejected
			cell{2}{2,6} = {red!15},
    	}
    		n + \bullet
   				&  i 
    			&  i + 1
    			&  \dots
    			&  i + d - 1
    			&  i + d
    	\\
    			&  ad
				&  \bullet
    			&  \dots
				&  \bullet
				&  ad
    	\end{tblr}
    \end{center}


	\begin{enumerate}[start=2]
		\item Pas de facteurs carrés pas trop loin.
	\end{enumerate}
		
    \begin{center}
    	\begin{tblr}{
    		colspec     = {Q[r,$]*{5}{Q[c,$]}},
    		vline{2}    = {.95pt},
    		vline{3-6}  = {dashed},
   			hline{2}    = {.95pt},
			column{2-Z} = {3.75em},
			% Rejected
			cell{2}{2,6} = {red!15},
    	}
    		n + \bullet
   				&  i 
    			&  i + 1
    			&  \dots
    			&  i + 2d - 1
    			&  i + 2d
    	\\
    			&  ad
				&  \bullet
    			&  \dots
				&  \bullet
				&  ad
    	\end{tblr}
    \end{center}
\end{fact}


\begin{proof}
	Tout est contenu dans le fait \ref{dist-square}.
	
	\begin{enumerate}
%		\item \label{vogler-no-witness}
%		Ici, $n + i = a A^2$ et $n + i + 1 = a B^2$ donnent $a (B^2 - A^2) = 1$\,, d'où $B^2 - A^2 = 1$ qui ne se peut pas car nous sommes dans $\NNs$, d'où $A^2 > B^2 \geq 1$\,.

		\item Ici, $n + i = ad A^2$ et $n + i + d = ad B^2$ donnent $ad (B^2 - A^2) = d$\,, puis $a (B^2 - A^2) = 1$, d'où $B^2 - A^2 = 1$ qui ne se peut pas car $B^2 > A^2 \geq 1$\,.

		\item Ici, $n + i = ad A^2$ et $n + i + 2 d = ad B^2$ donnent $ad (B^2 - A^2) = 2d$\,, \emph{i.e.} $a (B^2 - A^2) = 2$\,, d'où $B^2 - A^2 \in \setgene{1, 2}$ qui est impossible.
	\end{enumerate}
\end{proof}


% ------------------ %


Pour fabriquer des tableaux de Vogler, nous allons \enquote{multiplier} des $d$-tableaux de Vogler qui sont moins restrictifs ; ils sont définis comme suit.
	

\begin{defi}
	Soient $(n, k, d) \in \big( \NNs \big)^3$\,,
	$( q_i )_{0 \leq i \leq k} \subseteq \NN$\,,
	$( \epsilon_i )_{0 \leq i \leq k} \subseteq \setgene{0, 1}$
	et
	$( f_i )_{0 \leq i \leq k} \subseteq \NNs$
	tels que 
	$\forall i \in \ZintervalC{0}{k}$\,, $n + i = d^{\,2q_i + \epsilon_i} f_i$ avec $f_i \wedge d = 1$\,.
	%
	Ce type de situation sera résumé par le tableau suivant que nous nommerons $d$-tableau de Vogler où $d^{\,\epsilon_i} \in \setgene{1, d}$\,.

	\begin{center}
		\begin{tblr}{
			colspec    = {Q[r,$]*{5}{Q[c,$]}},
			vline{2}   = {.95pt},
			vline{3-6} = {dashed},
			hline{2}   = {.95pt}
		}
			n + \bullet
				&  0  
				&  1  
				&  2  
				&  \dots
				&  k
		\\
			d
				&  d^{\,\epsilon_0}
				&  d^{\,\epsilon_1}
				&  d^{\,\epsilon_2}
				&  \dots  
				&  d^{\,\epsilon_k}
		\end{tblr}
	\end{center}
\end{defi}


% ------------------ %


\begin{example}
	Supposons avoir le $5$-tableau de Vogler suivant où $n \in \NNs$.

	\begin{center}
		\begin{tblr}{
			colspec    = {Q[r,$]*{4}{Q[c,$]}},
			vline{2}   = {.95pt},
			vline{3-5} = {dashed},
			hline{2}   = {.95pt}
		}
			n + \bullet
				&  0  
				&  1  
				&  2  
				&  3
		\\
			5
				&  1
				&  5
				&  1
				&  1
		\end{tblr}
	\end{center}
	
	Ceci résume la situation suivante. 
	
	\begin{itemize}
		\item $\exists (a, A) \in \NN \times \NNs$
		      tel que $A \wedge 5 = 1$ et $n     = 5^{\,2a} A$\,.
		
		\item $\exists (b, B) \in \NN \times \NNs$
		      tel que $B \wedge 5 = 1$ et $n + 1 = 5^{\,2b + 1} B$\,.
		
		\item $\exists (c, C) \in \NN \times \NNs$
		      tel que $C \wedge 5 = 1$ et $n + 2 = 5^{\,2c} C$\,.
		
		\item $\exists (d, D) \in \NN \times \NNs$
		      tel que $D \wedge 5 = 1$ et $n + 3 = 5^{\,2d} D$\,.
	\end{itemize}
\end{example}


% ------------------ %


\begin{example}
	La multiplication de deux $d$-tableaux de Vogler est \enquote{naturelle} lorsqu'elle porte sur des nombres $d$ premiers entre eux.
	Considérons le $2$-tableau de Vogler et le $3$-tableau de Vogler suivants.
	
	\vspace{-1.5ex}
	\begin{multicols}{2}
	\begin{center}
		\begin{tblr}{
			colspec    = {Q[r,$]*{4}{Q[c,$]}},
			vline{2}   = {.95pt},
			vline{3-5} = {dashed},
			hline{2}   = {.95pt}
		}
			n + \bullet
				&  0  
				&  1  
				&  2  
				&  3
		\\
			2
				&  1
				&  2
				&  1
				&  2
		\end{tblr}
	\end{center}

	\begin{center}
		\begin{tblr}{
			colspec    = {Q[r,$]*{4}{Q[c,$]}},
			vline{2}   = {.95pt},
			vline{3-5} = {dashed},
			hline{2}   = {.95pt}
		}
			n + \bullet
				&  0  
				&  1  
				&  2  
				&  3
		\\
			3
				&  3
				&  1
				&  1
				&  3
		\end{tblr}
	\end{center}
	\end{multicols}


	\vspace{-1ex}
	La multiplication de ces $d$-tableaux de Vogler est le $6$-tableau de Vogler suivant.

	\begin{center}
		\begin{tblr}{
			colspec    = {Q[r,$]*{4}{Q[c,$]}},
			vline{2}   = {.95pt},
			vline{3-5} = {dashed},
			hline{2}   = {.95pt}
		}
			n + \bullet
				&  0  
				&  1  
				&  2  
				&  3
		\\
			6
				&  3
				&  2
				&  1
				&  6
		\end{tblr}
	\end{center}
	
	Ceci résume la situation suivante avec des notations \enquote{évidentes}\,. 
	
	\begin{itemize}
		\item $A \wedge 6 = 1$
		      et
		      $n     = 2^{\,2a}   3^{\,2\alpha+1} A$\,.

		\item $B \wedge 6 = 1$
		      et
		      $n + 1 = 2^{\,2b+1} 3^{\,2\beta}    B$\,.

		\item $C \wedge 6 = 1$
		      et
		      $n + 2 = 2^{\,2c}   3^{\,2\gamma}   C$\,.

		\item $D \wedge 6 = 1$
		      et
		      $n + 3 = 2^{\,2d+1} 3^{\,2\delta+1} D$\,.
	\end{itemize}
\end{example}


% ------------------ %


\begin{fact} \label{vogler-multiple}
	Dans la deuxième ligne d'un $d$-tableau de Vogler, les valeurs $d$ sont séparées par exactement $(d-1)$ valeurs $1$\,.
\end{fact}


\begin{proof}
	Penser aux multiples de $d$\,.
\end{proof}


% ------------------ %


\begin{fact} \label{vogler-parity-square}
	$\forall p \in \PP$\,, si $\consprod \in \NNsquare$\,, alors dans le $p$-tableau de Vogler associé à $\consprod$\,, le nombre de valeurs $p$ est forcément pair.
\end{fact}


\begin{proof}
	Évident, mais très pratique, comme nous le verrons dans la suite.
\end{proof}

