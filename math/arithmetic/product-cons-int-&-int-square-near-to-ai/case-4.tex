Supposons que $\consprod<4> = n(n+1)(n+2)(n+3) \in \NNssquare$\,. Nous avons alors les \sftab[x] partiels suivants pour $p \in \PP_{\geq 4}$ divisant $\consprod<4>$\,.

\begin{center}
	\begin{tblr}{
		colspec    = {Q[r,$]*{4}{Q[c,$]}},
		vline{2}   = {.95pt},
		vline{3-5} = {dashed},
		hline{2}   = {.95pt}
	}
		n + \bullet
			&  0
			&  1
			&  2
			&  3
	\\
		p
			&  1
			&  1
			&  1
			&  1
	\end{tblr}
\end{center}


Pour $p = 2$\,, nous avons les trois \sftab[x] partiels d'ordre $2$ donnés ci-après.

\begin{center}
	\begin{tblr}{
		colspec    = {Q[r,$]*{4}{Q[c,$]}},
		vline{2}   = {.95pt},
		vline{3-5} = {dashed},
		hline{2}   = {.95pt}
	}
		n + \bullet
			&  0
			&  1
			&  2
			&  3
	\\
		2
			&  1
			&  1
			&  1
			&  1
	\\
			&  2
			&  1
			&  2
			&  1
	\\
			&  1
			&  2
			&  1
			&  2
	\end{tblr}
\end{center}


\newpage
Pour $p = 3$\,, nous obtenons les deux \sftab[x] partiels d'ordre $3$ donnés ci-après.

\begin{center}
	\begin{tblr}{
		colspec    = {Q[r,$]*{4}{Q[c,$]}},
		vline{2}   = {.95pt},
		vline{3-5} = {dashed},
		hline{2}   = {.95pt}
	}
		n + \bullet
			&  0
			&  1
			&  2
			&  3
	\\
		3
			&  1
			&  1
			&  1
			&  1
	\\
			&  3
			&  1
			&  1
			&  3
	\end{tblr}
\end{center}


%\newpage
La multiplication des \sftab[x] partiels précédents donne les \sftab[x]
\footnote{
	Tableaux non partiels forcément.
},
suivants.

\vspace{-1.75ex}
\begin{multicols}{2}
\begin{center}
	\begin{tblr}{
		colspec    = {Q[r,$]*{4}{Q[c,$]}},
		vline{2}   = {.95pt},
		vline{3-5} = {dashed},
		hline{2}   = {.95pt},
	}
		n + \bullet
			&  0
			&  1
			&  2
			&  3
	\\
			&  1
			&  1
			&  1
			&  1
	\\
			&  2
			&  1
			&  2
			&  1
	\\
			&  1
			&  2
			&  1
			&  2
	\end{tblr}
\end{center}


\begin{center}
	\begin{tblr}{
		colspec    = {Q[r,$]*{4}{Q[c,$]}},
		vline{2}   = {.95pt},
		vline{3-5} = {dashed},
		hline{2}   = {.95pt},
	}
		n + \bullet
			&  0
			&  1
			&  2
			&  3
	\\
			&  3
			&  1
			&  1
			&  3
	\\
			&  6
			&  1
			&  2
			&  3
	\\
			&  3
			&  2
			&  1
			&  6
	\end{tblr}
\end{center}
\end{multicols}


\vspace{-1.5ex}
Le fait \ref{sftable-illegal-0-sol} rejette quatre \sftab[x] : voir les cellules surlignées ci-dessous.

\vspace{-1.75ex}
\begin{multicols}{2}
\begin{center}
	\begin{tblr}{
		colspec    = {Q[r,$]*{4}{Q[c,$]}},
		vline{2}   = {.95pt},
		vline{3-5} = {dashed},
		hline{2}   = {.95pt},
		% Rejected
		cell{2}{2,3} = {red!15},
		cell{3}{2,4} = {red!15},
		cell{4}{3,5} = {red!15},
	}
		n + \bullet
			&  0
			&  1
			&  2
			&  3
	\\
			&  1
			&  1
			&  1
			&  1
	\\
			&  2
			&  1
			&  2
			&  1
	\\
			&  1
			&  2
			&  1
			&  2
	\end{tblr}
\end{center}


\begin{center}
	\begin{tblr}{
		colspec    = {Q[r,$]*{4}{Q[c,$]}},
		vline{2}   = {.95pt},
		vline{3-5} = {dashed},
		hline{2}   = {.95pt},
		% Rejected
		cell{2}{3,4} = {red!15},
	}
		n + \bullet
			&  0
			&  1
			&  2
			&  3
	\\
			&  3
			&  1
			&  1
			&  3
	\\
			&  6
			&  1
			&  2
			&  3
	\\
			&  3
			&  2
			&  1
			&  6
	\end{tblr}
\end{center}
\end{multicols}


\vspace{-1.5ex}
Ceci nous amène à étudier les deux \sftab[x] généralisés suivants.

\begin{center}
	\begin{tblr}{
		colspec     = {Q[r,$]*{4}{Q[c,$]}},
		vline{2}    = {.95pt},
		vline{3-5}  = {dashed},
		hline{2}    = {.95pt},
		column{2-Z} = {2em},
	}
		\bullet
			&  n
			&  n+1
			&  n+2
			&  n+3
	\\
			&  6
			&  1
			&  2
			&  3
	\\
			&  3
			&  2
			&  1
			&  6
	\end{tblr}
\end{center}


En posant $x = n + \frac32 = \frac{n + (n + 3)}{2}$\,, nous obtenons les \sftab[x] généralisés suivants.

\begin{center}
	\begin{tblr}{
		colspec     = {Q[r,$]*{4}{Q[c,$]}},
		vline{2}    = {.95pt},
		vline{3-5}  = {dashed},
		hline{2}    = {.95pt},
		column{2-Z} = {2.75em},
	}
		\bullet
			&  x - \frac32
			&  x - \frac12
			&  x + \frac12
			&  x + \frac32
	\\
			&  6
			&  1
			&  2
			&  3
	\\
			&  3
			&  2
			&  1
			&  6
	\end{tblr}
\end{center}


En multipliant les colonnes extrêmes ensemble, et celles centrales aussi, et en notant que $6 \times 3 = 2 \times 3^2$, nous arrivons au même \sftab\ généralisé ci-dessous.

\begin{center}
	\begin{tblr}{
		colspec     = {Q[r,$]*{2}{Q[c,$]}},
		vline{2}    = {.95pt},
		vline{3-Y}  = {dashed},
		hline{2}    = {.95pt},
		column{2-Z} = {3em},
	}
		\bullet
			&  x^2 - \frac94
			&  x^2 - \frac14
	\\
			&  2
			&  2
	\end{tblr}
\end{center}


Comme $x^2 - \frac14 - \big( x^2 - \frac94 \big) = 2$\,, le fait \ref{sftable-illegal-0-sol} nous montre que le \sftab\ généralisé précédent est impossible. Joli ! Non ?


\begin{remark}
	Noter que la fin du raisonnement n'a fait appel à aucune hypothèse sur $\consprod<4>$\,. 
\end{remark}

