La méthode présentée ci-dessus permet de faire appel à un programme pour limiter les traitements à la main, et à la sueur des neurones, de \sftab[x] problématiques comme nous avons dû le faire dans la section \ref{apply-4}.
Expliquons cette tactique semi-automatique en traitant le cas de $6$ facteurs.

%\newpage
\begin{enumerate}
	\item On raisonne par l'absurde en supposant que $\consprod<6> \in \NNssquare$\,.


	\item Comme $\forall p \in \PP_{\geq 6}$\,, $p$ divise au maximum un seul des facteurs $(n + i)$ de $\consprod<6>$\,,
	nous avons juste besoin de considérer l'ensemble $\setproba{P} = \setgene{2, 3, 5}$ des diviseurs premiers stricts de $6$\,.


	\item Pour chaque élément $p$ de $\setproba{P}$\,, on construit la liste $\setproba*{V}{p}$ des \sftab[x] partiels relatifs à $p$ et $\consprod<6> \in \NNssquare$ en s'appuyant sur la section \ref{sftable-constraint}.


	\item Via les listes $\setproba*{V}{p}$\,, on calcule toutes les multiplications de tous les \sftab[x] partiels relatifs à des nombres $p$ différents, et pour chacune d'elles, on ne la garde que si elle ne vérifie aucune des conditions suivantes, celles du dernier cas devant être indiquées à la main au programme qui va donc évoluer au gré des démonstrations faites par un humain (démonstrations que l'on espère le plus rare possible).
	%
	\begin{enumerate}
		\item Le tableau commence, ou se termine, par la valeur $1$\,. Dans ce cas, on sait par récurrence que le tableau produit n'est pas possible (voir le fait \ref{sftab-recu}).

		\item Le tableau est rejeté par le fait \ref{sftable-illegal-0-sol}.

		\item Le tableau \enquote{produit} contient un sous-tableau que nous savons impossible suite à un raisonnement humain fait \emph{localement}\,, c'est-à-dire que seul les facteurs indiqués dans le sous-tableau, et le sous-tableau lui-même sont utilisés pour raisonner.
		C'est le cas des \sftab[x] du fait \ref{no-sftab-6.1.2.3}.
	\end{enumerate}
\end{enumerate}


{\Huge YAPLUKA !}

Dans le dépôt en ligne associé à ce document sont placés les fichiers \verb#find-pb-sftab.py# et \verb#common.py# qui nous fournissent les informations suivantes pour $\consprod<6>$\,.
