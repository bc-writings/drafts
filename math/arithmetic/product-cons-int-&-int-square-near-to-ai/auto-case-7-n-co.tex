\subsection{Au-delà de 6 facteurs ?}

\leavevmode
\smallskip

Voici ce que donne nos programmes \verb#Pyhon# sans trop d'efforts, mais avec du temps de calcul 
\footnote{
	Nous commençons à entrer dans un monde à la combinatoire élevée.
}.
Rappelons que chaque nouveau cas problématique est indiqué au programme qui évolue donc au gré de l'intervention humaine.


% ------------------ %


\nopb{k \in \setgene{7, 8}}


% ------------------ %


\byhand{9}
%
\begin{center}
	\begin{tblr}{
		colspec    = {Q[r,$]*{9}{Q[c,$]}},
		vline{2}   = {.95pt},
		vline{3-Y} = {dashed},
		hline{2}   = {.95pt},
	}
		n + \bullet
			&  0   &  1  &  2  &  3  &  4  &  5  &  6  &  7  &  8
	\\
			&  14  &  1  &  6  &  5  &  2  &  3  &  1  &  7  &  10
	\\
			&  10  &  7  &  1  &  3  &  2  &  5  &  6  &  1  &  14
	\end{tblr}
\end{center}

Extrayons du premier \sftab[], le \sftab\ généralisé suivant.
%
\begin{center}
	\begin{tblr}{
		colspec    = {Q[r,$]*{9}{Q[c,$]}},
		vline{2}   = {.95pt},
		vline{3-Y} = {dashed},
		hline{2}   = {.95pt},
	}
		\bullet
			&  n+1  &  n+2  &  n+4  &  n+5
	\\
			&  1    &  6    &  2    &  3
	\end{tblr}
\end{center}

\newpage
En posant $m = n + 3$\,, nous obtenons le tableau ci-après.
%
\begin{center}
	\begin{tblr}{
		colspec    = {Q[r,$]*{9}{Q[c,$]}},
		vline{2}   = {.95pt},
		vline{3-Y} = {dashed},
		hline{2}   = {.95pt},
		% Focus
		cell{1-2}{2,5} = {green!15},
		cell{1-2}{3,4} = {orange!15},
	}
		\bullet
			&  m-2  &  m-1  &  m+1  &  m+2
	\\
			&  1    &  6    &  2    &  3
	\end{tblr}
\end{center}


En multipliant les colonnes $1$ et $4$\,, et aussi la $2$ et la $3$\,, nous obtenons le \sftab\ généralisé ci-dessous après avoir noté que $6 \times 2 = 3 \times 2^2$.

\begin{center}
	\begin{tblr}{
		colspec     = {Q[r,$]*{2}{Q[c,$]}},
		vline{2}    = {.95pt},
		vline{3-Y}  = {dashed},
		hline{2}    = {.95pt},
		column{2-Z} = {3em},
		% Focus
		cell{1-3}{2} = {green!15},
		cell{1-3}{3} = {orange!15},
	}
		\bullet
			&  m^2 - 4  &  m^2 - 1
	\\
			&  3        &  3
	\end{tblr}
\end{center}


Comme $m^2 - 4 - ( m^2 - 1 ) = 3$\,, le fait \ref{sftable-illegal-0-sol} nous montre que le premier \sftab[], celui commençant par $14$, est impossible.
Le cas du deuxième se traite de façon analogue, d'où finalement $\consprod<9> \notin \NNsquare$\,. Notons au passage un nouveau fait.


\begin{fact} \label{no-sftab-1.6.*.2.3}
	Aucun \sftab\ ne peut contenir l'un des deux \sftab[x] généralisés suivants.
	\begin{center}
	\begin{tblr}{
		colspec    = {Q[r,$]*{5}{Q[c,$]}},
		vline{2}   = {.95pt},
		vline{3-5} = {dashed},
		hline{2}   = {.95pt}
	}
		m + \bullet
			&  0  &  1  &  3  &  4
	\\
			&  1  &  6  &  2  &  3
	\\
			&  3  &  2  &  6  &  1
	\end{tblr}
	\end{center} 
\end{fact}


% ------------------ %


\nopb{k \in \ZintervalC{10}{17}}
