Dans la suite, nous emploierons les notations suivantes.

\begin{itemize}
	\item $\forall (n, k) \in ( \NNs )^2$\,, $\consprod = \dprod_{i = 0}^{k - 1} (n + i)$\,. 
	
	\noindent
	Par exemple,
	$\consprod<1> = n$\,,
	$\consprod<2> = n (n+1)$
	et
	$\consprod[n+2]<4> = (n+2) (n+3) (n+4) (n+5)$\,.


	\medskip
	\item $\NNsquare = \setgene{n^2, n \in \NN}$ est l'ensemble des carrés parfaits.
	On note aussi $\NNssquare = \NNsquare \cap \NNs$.

	\noindent
	$\NNsf$ est l'ensemble des naturels non nuls sans facteur carré
	\footnote{
		En anglais, on dit \enquote{square free}\,.
	}.


	\medskip
	\item $\PP$ désigne l'ensemble des nombres premiers.
	
	\noindent
	$\forall (p ; n) \in \PP \times \NNs$\,, $\padicval{n} \in \NN$ est la valuation $p$-adique de $n$\,,
	c'est-à-dire 
	$p^{\padicval{n}} \divides n$ et $p^{\padicval{n} + 1} \ndivides n$\,,
	autrement dit
	$p^{\padicval{n}}$ divise $n$\,, contrairement à $p^{\padicval{n} + 1}$\,.


	\medskip
	\item $\NNseq[r]$ désigne l'ensemble des suites finies strictement croissantes de $r$ entiers naturels.
	
	
	\noindent
	$\PPseq[r]$ désigne l'ensemble des suites finies strictement croissantes de $r$ nombres premiers.


	\medskip
	\item $\forall (n , m) \in \NN^2$, $n \wedge m$ désigne le PGCD de $n$ et $m$.

	
	\medskip
	\item $2\,\NN$ désigne l'ensemble des nombres naturels pairs.
	
	\noindent
	$2\,\NN + 1$ est l'ensemble des nombres naturels impairs.
\end{itemize}