La méthode présentée ci-dessus permet de faire appel à un programme pour n'avoir à traiter à la main, et aux neurones, que certains tableaux de Vogler problématiques comme nous avons dû le faire dans la section \ref{apply-4}.
Expliquons cette tactique semi-automatique en traitant le cas de $6$ facteurs.


\begin{enumerate}
	\item On raisonne par l'absurde en supposant que $\consprod<6> \in \NNssquare$\,.
	
 
	\item On fabrique la liste $\setproba{P}$ des diviseurs premiers stricts de $6$ : nous avons juste $2$\,, $3$ et $5$\,.
	Notons qu'avec $7$ facteurs, nous n'aurions pas garder $7$ car il est forcément de valuation paire dans chaque facteur $(n + i)$ de $\consprod<7>$ si $\consprod<7> \in \NNssquare$\,.
	
 
	\item Pour chaque élément $p$ de $\setproba{P}$\,, on construit la liste $\setproba*{V}{p}$ des $p$-tableaux de Vogler possibles relativement à $\consprod<6>$\,.
	
 
	\item Via les listes $\setproba*{V}{p}$\,, on calcule toutes les multiplications de $p$-tableaux de Vogler, et pour chacune d'elles on ne la garde que si elle ne vérifie aucune des conditions suivantes, celles du dernier cas devant être indiquées à la main au programme.
	%
	\begin{enumerate}
		\item Le tableau \enquote{produit} commence, ou se termine, par la valeur $1$\,. Dans ce cas, on sait par récurrence que le tableau produit n'est pas possible (voir le fait \ref{vogler-sub-square}). 

		\item L'une des interdictions du fait \ref{illegal-vogler} est validée par le tableau \enquote{produit}\,.
		
		\item Le tableau \enquote{produit} contient un sous-tableau que nous savons impossible suite à un raisonnement humain fait \emph{localement}\,, c'est-à-dire que seul les facteurs indiqués dans le sous-tableau, et le sous-tableau lui-même sont utilisés pour raisonner.
		Comme c'est ce qui a été fait en fin de section \ref{apply-4}, nous pouvons indiquer les deux sous-tableaux impossibles suivants.
	\end{enumerate}
\end{enumerate}

\begin{center}
	\begin{tblr}{
		colspec    = {Q[r,$]*{4}{Q[c,$]}},
		vline{2}   = {.95pt},
		vline{3-5} = {dashed},
		hline{2}   = {.95pt}
	}
		m + \bullet
			&  0  
			&  1 
			&  2 
			&  3
	\\
			&  6
			&  1
			&  2
			&  3
	\\
			&  3
			&  2
			&  1
			&  6
	\end{tblr}
	
	\smallskip
	
	\emph{\small Deux sous-tableaux de Vogler impossibles.}
\end{center}


{\Huge YAPLUKA !}
