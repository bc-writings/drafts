Supposons que $\consprod<4> = n(n+1)(n+2)(n+3)(n+4) \in \NNssquare$\,. Nous avons alors les $p$-tableaux de Vogler suivants pour $p \in \PP_{>4}$ divisant $\consprod<4>$\,.

\begin{center}
	\begin{tblr}{
		colspec    = {Q[r,$]*{5}{Q[c,$]}},
		vline{2}   = {.95pt},
		vline{3-6} = {dashed},
		hline{2}   = {.95pt}
	}
		n + \bullet
			&  0  
			&  1 
			&  2 
			&  3 
			&  4
	\\
		p
			&  1
			&  1
			&  1
			&  1
			&  1
	\end{tblr}
\end{center}


%\newpage
Pour $p = 2$\,, nous avons les $2$-tableaux de Vogler suivants.

\begin{center}
	\begin{tblr}{
		colspec    = {Q[r,$]*{5}{Q[c,$]}},
		vline{2}   = {.95pt},
		vline{3-6} = {dashed},
		hline{2}   = {.95pt}
	}
		n + \bullet
			&  0  
			&  1 
			&  2 
			&  3 
			&  4
	\\
		2
			&  1
			&  1
			&  1
			&  1
			&  1
	\\
			&  2
			&  1
			&  2
			&  1
			&  1
	\\
			&  2
			&  1
			&  1
			&  1
			&  2
	\\
			&  1
			&  2
			&  1
			&  2
			&  1
	\\
			&  1
			&  1
			&  2
			&  1
			&  2
	\end{tblr}
\end{center}


%\newpage

Pour $p = 3$\,, nous obtenons les $3$-tableaux de Vogler suivants.

\begin{center}
	\begin{tblr}{
		colspec    = {Q[r,$]*{5}{Q[c,$]}},
		vline{2}   = {.95pt},
		vline{3-6} = {dashed},
		hline{2}   = {.95pt}
	}
		n + \bullet
			&  0  
			&  1 
			&  2 
			&  3 
			&  4
	\\
		3
			&  1
			&  1
			&  1
			&  1
			&  1
	\\
			&  3
			&  1
			&  1
			&  3
			&  1
	\\
			&  1
			&  3
			&  1
			&  1
			&  3
	\end{tblr}
\end{center}


\newpage
La multiplication de tous les $d$-tableaux de Vogler précédents donne les $15$ cas suivants.

\vspace{-1ex}
\begin{multicols}{3}
\begin{center}
	\begin{tblr}{
		colspec  = {Q[r,$]*{5}{Q[c,$]}},
		vline{2} = {.95pt},
		vline{3-6} = {dashed},
		hline{2} = {.95pt},
	}
		n + \bullet
			&  0  
			&  1 
			&  2 
			&  3 
			&  4

	\\
			&  1
			&  1
			&  1
			&  1
			&  1
	\\
			&  2
			&  1
			&  2
			&  1
			&  1
	\\
			&  2
			&  1
			&  1
			&  1
			&  2
	\\
			&  1
			&  2
			&  1
			&  2
			&  1
	\\
			&  1
			&  1
			&  2
			&  1
			&  2
	\end{tblr}
\end{center}


\begin{center}
	\begin{tblr}{
		colspec  = {Q[r,$]*{5}{Q[c,$]}},
		vline{2} = {.95pt},
		vline{3-6} = {dashed},
		hline{2} = {.95pt},
	}
		n + \bullet
			&  0  
			&  1 
			&  2 
			&  3 
			&  4

	\\
			&  3
			&  1
			&  1
			&  3
			&  1
	\\
			&  6
			&  1
			&  2
			&  3
			&  1
	\\
			&  6
			&  1
			&  1
			&  3
			&  2
	\\
			&  3
			&  2
			&  1
			&  6
			&  1
	\\
			&  3
			&  1
			&  2
			&  3
			&  2
	\end{tblr}
\end{center}


\begin{center}
	\begin{tblr}{
		colspec  = {Q[r,$]*{5}{Q[c,$]}},
		vline{2} = {.95pt},
		vline{3-6} = {dashed},
		hline{2} = {.95pt},
	}
		n + \bullet
			&  0  
			&  1 
			&  2 
			&  3 
			&  4

	\\
			&  1
			&  3
			&  1
			&  1
			&  3
	\\
			&  2
			&  3
			&  2
			&  1
			&  3
	\\
			&  2
			&  3
			&  1
			&  1
			&  6
	\\
			&  1
			&  6
			&  1
			&  2
			&  3
	\\
			&  1
			&  3
			&  2
			&  1
			&  6
	\end{tblr}
\end{center}
\end{multicols}


\vspace{-1ex}
Comme $\consprod<3> = n(n+1)(n+2)(n+3) \notin \NNssquare$ et $\consprod[n+1]<3> = (n+1)(n+2)(n+3)(n+4) \notin \NNssquare$\,, nous pouvons ignorer tous les tableaux commençant, ou finissant, par une valeur $1$ d'après le fait \ref{vogler-sub-square}. Cela laisse les tableaux de Vogler ci-après, mais ces derniers sont rejetés par le fait \ref{illegal-vogler}.

\begin{center}
	\begin{tblr}{
		colspec  = {Q[r,$]*{5}{Q[c,$]}},
		vline{2} = {.95pt},
		vline{3-6} = {dashed},
		hline{2} = {.95pt},
		% Rejected
		cell{2-3}{3,4} = {red!15},
		cell{4}{2,5}   = {red!15},
		cell{5}{2,4}   = {red!15},
		cell{6}{4,5}   = {red!15},
	}
		n + \bullet
			&  0  
			&  1 
			&  2 
			&  3 
			&  4

	\\
			&  2
			&  1
			&  1
			&  1
			&  2
	\\
			&  6
			&  1
			&  1
			&  3
			&  2
	\\
			&  3
			&  1
			&  2
			&  3
			&  2
	\\
			&  2
			&  3
			&  2
			&  1
			&  3
	\\
			&  2
			&  3
			&  1
			&  1
			&  6
	\end{tblr}
\end{center}


\begin{remark}
	Notons qu'un cas comme 
	$6 \cdot 1 \cdot 1 \cdot 3 \cdot 2$\,,
	c'est-à-dire
	$n = 6 A^2$\,, $n + 1 = B^2$\,, $n + 2 = C^2$\,, $n + 3 = 3 D^2$ et $n + 4 = 2 E^2$ 
	où $(A, B, C, D, E) \in \big( \NNs \big)^4$
	peut se traiter de façon analogue à ce qui a été fait dans la section \ref{apply-4} via
	$x - 2 = 6 A^2$\,, $x - 1 = B^2$\,, $x = C^2$\,, $x + 1 = 3 D^2$ et $x + 2 = 2 E^2$
	qui donnent
	$x^2 - 4 = 3 F^2$ et $x^2 - 1 = 3 G^2$
	où $(F, G) \in \big( \NNs \big)^4$\,.
\end{remark}
