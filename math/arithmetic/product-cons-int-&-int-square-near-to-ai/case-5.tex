Supposons que $\consprod<5> = n(n+1)(n+2)(n+3)(n+4) \in \NNssquare$\,. Nous avons alors les \sftab[x] partiels suivants pour $p \in \PP_{\geq 5}$ divisant $\consprod<5>$\,.

\begin{center}
	\begin{tblr}{
		colspec    = {Q[r,$]*{5}{Q[c,$]}},
		vline{2}   = {.95pt},
		vline{3-6} = {dashed},
		hline{2}   = {.95pt}
	}
		n + \bullet  
			&  0  &  1  &  2  &  3  &  4
	\\
		p  
			&  1  &  1  &  1  &  1  &  1
	\end{tblr}
\end{center}


%\newpage
Pour $p = 2$\,, nous avons les \sftab[x] partiels d'ordre $2$ donnés ci-après.

\begin{center}
	\begin{tblr}{
		colspec    = {Q[r,$]*{5}{Q[c,$]}},
		vline{2}   = {.95pt},
		vline{3-6} = {dashed},
		hline{2}   = {.95pt}
	}
		n + \bullet  
			&  0  &  1  &  2  &  3  &  4
	\\
		2  
			&  1  &  1  &  1  &  1  &  1
	\\  
			&  2  &  1  &  2  &  1  &  1
	\\  
			&  2  &  1  &  1  &  1  &  2
	\\  
			&  1  &  2  &  1  &  2  &  1
	\\  
			&  1  &  1  &  2  &  1  &  2
	\end{tblr}
\end{center}


%\newpage

Pour $p = 3$\,, nous obtenons les \sftab[x] partiels d'ordre $3$ donnés ci-après.

\begin{center}
	\begin{tblr}{
		colspec    = {Q[r,$]*{5}{Q[c,$]}},
		vline{2}   = {.95pt},
		vline{3-6} = {dashed},
		hline{2}   = {.95pt}
	}
		n + \bullet  
			&  0  &  1  &  2  &  3  &  4
	\\
		3  
			&  1  &  1  &  1  &  1  &  1
	\\  
			&  3  &  1  &  1  &  3  &  1
	\\  
			&  1  &  3  &  1  &  1  &  3
	\end{tblr}
\end{center}


%\newpage
La multiplication de tous les \sftab[x] partiels précédents donne les $15$ cas suivants.

\vspace{-1.75ex}
\begin{multicols}{3}
\begin{center}
	\begin{tblr}{
		colspec    = {Q[r,$]*{5}{Q[c,$]}},
		vline{2}   = {.95pt},
		vline{3-6} = {dashed},
		hline{2}   = {.95pt},
	}
		n + \bullet  
			&  0  &  1  &  2  &  3  &  4
	\\  
			&  1  &  1  &  1  &  1  &  1
	\\  
			&  2  &  1  &  2  &  1  &  1
	\\  
			&  2  &  1  &  1  &  1  &  2
	\\  
			&  1  &  2  &  1  &  2  &  1
	\\  
			&  1  &  1  &  2  &  1  &  2
	\end{tblr}
\end{center}


\begin{center}
	\begin{tblr}{
		colspec    = {Q[r,$]*{5}{Q[c,$]}},
		vline{2}   = {.95pt},
		vline{3-6} = {dashed},
		hline{2}   = {.95pt},
	}
		n + \bullet
			&  0  &  1  &  2  &  3  &  4
	\\  
			&  3  &  1  &  1  &  3  &  1
	\\  
			&  6  &  1  &  2  &  3  &  1
	\\  
			&  6  &  1  &  1  &  3  &  2
	\\  
			&  3  &  2  &  1  &  6  &  1
	\\  
			&  3  &  1  &  2  &  3  &  2
	\end{tblr}
\end{center}


\begin{center}
	\begin{tblr}{
		colspec    = {Q[r,$]*{5}{Q[c,$]}},
		vline{2}   = {.95pt},
		vline{3-6} = {dashed},
		hline{2}   = {.95pt},
	}
		n + \bullet
			&  0  &  1  &  2  &  3  &  4
	\\  
			&  1  &  3  &  1  &  1  &  3
	\\  
			&  2  &  3  &  2  &  1  &  3
	\\  
			&  2  &  3  &  1  &  1  &  6
	\\  
			&  1  &  6  &  1  &  2  &  3
	\\  
			&  1  &  3  &  2  &  1  &  6
	\end{tblr}
\end{center}
\end{multicols}


\vspace{-1.5ex}
Comme
$\consprod<4> = n(n+1)(n+2)(n+3) \notin \NNssquare$
et
$\consprod[n+1]<4> = (n+1)(n+2)(n+3)(n+4) \notin \NNssquare$
d'après la section \ref{apply-4},
les tableaux commençant, ou finissant, par une valeur $1$ sont à ignorer d'après le fait \ref{sftab-recu}. Cela laisse les \sftab[x] ci-après, mais ces derniers sont rejetés par le fait \ref{sftable-illegal-0-sol}.

\begin{center}
	\begin{tblr}{
		colspec    = {Q[r,$]*{5}{Q[c,$]}},
		vline{2}   = {.95pt},
		vline{3-6} = {dashed},
		hline{2}   = {.95pt},
		% Rejected
		cell{2-3}{3,4} = {red!15},
		cell{4}{2,5}   = {red!15},
		cell{5}{2,4}   = {red!15},
		cell{6}{4,5}   = {red!15},
	}
		n + \bullet
			&  0  &  1  &  2  &  3  &  4
	\\  
			&  2  &  1  &  1  &  1  &  2
	\\  
			&  6  &  1  &  1  &  3  &  2
	\\  
			&  3  &  1  &  2  &  3  &  2
	\\  
			&  2  &  3  &  2  &  1  &  3
	\\  
			&  2  &  3  &  1  &  1  &  6
	\end{tblr}
\end{center}

