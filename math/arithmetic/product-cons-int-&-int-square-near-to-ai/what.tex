Dans l'article \enquote{Note on Products of Consecutive Integers}
\footnote{
	J. London Math. Soc. 14 (1939).
},
Paul Erdos démontre que pour tout couple $(n, k) \in \NNs \times \NNs$\,, le produit d'entiers consécutifs $\dprod_{i = 0}^{k} (n + i)$ n'est jamais le carré d'un entier.

\smallskip

Dans ce document, nous proposons une méthode efficace
\footnote{
	Cette méthode s'appuie sur une représentation trouvée dans \href{https://web.archive.org/web/20171110144534/http://mathforum.org/library/drmath/view/65589.html}{un message archivé} que l'auteur a consulté le 28 janvier 2024.
	Voir \url{https://web.archive.org/web/20171110144534/http://mathforum.org/library/drmath/view/65589.html}\,.
},
élémentaire, et semi-automatisable, pour gérer plus facilement les premiers cas d'impossibilité.

