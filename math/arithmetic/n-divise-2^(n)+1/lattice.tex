Un ensemble $\mathcal{T}$ est appelé treillis s'il vérifie les conditions suivantes.
%
	\begin{itemize}
		\item $\big( \mathcal{T} ; \leq \big)$ est un ensemble ordonné.

		\item $\forall (a ; b) \in \mathcal{T}^2$\,, l'ensemble $\setgene{a ; b}$ possède une borne inférieure et une borne supérieure
		\footnote{
			Rappelons que ces bornes ne sont pas forcément dans $\setgene{a ; b}$\,.
		}\,. 
	\end{itemize}


\begin{fact}
	La relation de divisibilité ordonne l'ensemble $\ourset$ via $n  \leq m$ si, et seulement si, $n \divides m$\,.
	
	\medskip
	
	Muni de cet ordre, $\ourset$ est un treillis.
\end{fact}

\begin{proof}
	Voir les faits \ref{lcm} et \ref{gcd}.
\end{proof}



% -------------------- %


\begin{fact}
	$\forall n \in \ourset_{>1}$\,, $3 \divides n$\,, autrement dit $3$ est le minimum de $\ourset_{>1}$ dans le treillis $\big( \ourset ; \divides \big)$\,.
\end{fact}

\begin{proof}
	Soit $p \in \primeset$ tel que $p \divides n$\,.
	Modulo $p$\,, nous avons
	$2^{2n} \equiv (- 1)^2 \equiv 1$
	et
	$2^{p-1} \equiv 1$
	d'où
	$2^{(2n) \wedge (p-1)} \equiv 1$\,.
	%
	Or, on sait que $2$ est impair, donc $(2n) \wedge (p-1) = 2 \cdot \big( n \wedge \frac{p-1}{2} \big)$\,.
	%
	Dès lors, l'ordre $\sigma$ de $2$ divise $2 \cdot \big( n \wedge \frac{p-1}{2} \big)$\,.
	
	\medskip
	
	Considérons maintenant $p$ minimal, pour l'ordre usuel, parmi les diviseurs premiers de $n$\,.
	Clairement, $n \wedge \frac{p-1}{2} = 1$
	\footnote{
		Tout diviseur premier $q$ de $n \wedge \frac{p-1}{2}$ vérifierait $q \leq \frac{p-1}{2} < p$\,.
	},
	d'où $\sigma = 2$ puisque forcément $\sigma \neq 1$\,.
	%
	Finalement, $p = 3$\,.
\end{proof}


% -------------------- %


