\begin{fact}
	$\forall n \in \ourset$, $\forall p \in \primeset$\,,
	si $p \divides n$ alors $p n \in \ourset$\,.
\end{fact}

\begin{proof}
	Nous savons que $p > 2$\,.
	Ensuite, comme $2^n = -1 + k n$ où $k \in \ZZ$\,,
	nous avons :

    \medskip
    
    \begin{stepcalc}[style=sar]
    	2^{p n}
    \explnext{}
    	\big( 2^n \big)^p
    \explnext{}
    	\big( -1 + k n \big)^p
    \explnext{}
    	\dsum_{i=0}^p \binom{p}{i} \, (-1)^{p-i} \cdot (k n)^i
    \explnext*{$p \divides \binom{p}{i}$ si $0 < i < p$}{}
    	- 1 + \dsum_{i=1}^{p-1} p c_i \cdot (-1)^{p-i} \cdot (k n)^i + k^p \cdot n^p
    \explnext*{$n = p q$}{}
    	- 1 + pn \, \dsum_{i=1}^{p-1} c_i \cdot (-1)^{p-i} \cdot k^i n^{i-1} + pq \cdot n \cdot k^p \cdot n^{p-2}
    \end{stepcalc}

    \medskip

    On obtient finalement $2^{p n} = - 1 + pn \cdot r$ avec $r \in \ZZ$ comme souhaité.
\end{proof}


% -------------------- %


Notons au passage que le fait \ref{prime-sol} et le fait précédent donnent un exemple de preuve par récurrence où l'initialisation est essentielle car nous avons :
$\forall p \in \PP$\,, $p^k \divides 2^{( p^k )} + 1$ 
implique
$p^{k+1} \divides 2^{( p^{k+1} )} + 1$\,.


% -------------------- %


\begin{fact}
	$\forall (n , m) \in \ourset^2$, $n \vee m \in \ourset$\,.
\end{fact}

\begin{proof}
	TODO
\end{proof}


% -------------------- %


\begin{fact}
	$\forall (n , m) \in \ourset^2$, $n \wedge m \in \ourset$\,.
\end{fact}

\begin{proof}
	TODO
\end{proof}


% -------------------- %


\begin{fact}
	Ordonné via la relation de divisibilité, l'ensemble $\ourset$ est un treillis.
\end{fact}

\begin{proof}
	TODO
\end{proof}


% -------------------- %


\begin{fact}
	$\forall (n , m) \in \ourset^2$, $n m \in \ourset$\,.
\end{fact}

\begin{proof}
	TODO
\end{proof}


% -------------------- %


\begin{fact}
	$\forall n \in \ourset$, $2^n + 1 \in \ourset$\,.
\end{fact}

\begin{proof}
	TODO
\end{proof}

