La preuve du fait \ref{power-of-3} amène naturellement au fait suivant.

\begin{fact} \label{prime-divisor}
	$\forall n \in \ourset$, $\forall p \in \PP$\,,
	si $p \divides n$ alors $p n \in \ourset$\,.
\end{fact}

\begin{proof}
	$2^n = -1 + k n$, où $k \in \ZZ$\,, donne :

    \medskip
    
    \begin{stepcalc}[style=sar]
    	2^{p n}
    \explnext{}
    	\big( 2^n \big)^p
    \explnext{}
    	\big( -1 + k n \big)^p
    \explnext{}
    	\dsum_{i=0}^p \binom{p}{i} \, (-1)^{p-i} \cdot (k n)^i
    \explnext*{$p \divides \binom{p}{i}$ si $0 < i < p$}{}
    	- 1 + \dsum_{i=1}^{p-1} p c_i \cdot (-1)^{p-i} \cdot (k n)^i + k^p \cdot n^p
    \explnext*{$n = p q$}{}
    	- 1 + pn \, \dsum_{i=1}^{p-1} c_i \cdot (-1)^{p-i} \cdot k^i n^{i-1} + pq \cdot n \cdot k^p \cdot n^{p-2}
    \end{stepcalc}

    \medskip

    On obtient finalement $2^{p n} = - 1 + pn \cdot r$ avec $r \in \ZZ$ comme souhaité.
\end{proof}


% -------------------- %


Notons au passage que ce qui précède et le fait \ref{prime-sol} donnent un exemple non trivial pour insister sur la nécessité de l'initialisation dans une preuve par récurrence car nous avons :
$\forall p \in \PP$\,, $p^k \divides 2^{( p^k )} + 1$ 
implique
$p^{k+1} \divides 2^{( p^{k+1} )} + 1$\,.


% -------------------- %


\begin{fact} \label{lcm}
	$\forall (n , m) \in \ourset^2$, $n \vee m \in \ourset$\,.
\end{fact}

\begin{proof}
	Soit $r \in \NN$ tel que $n \vee m = n r$\,. Rappelons que d'après le fait \ref{no-even}, aucun des entiers considérés ne peut être pair.
	
	\medskip
	
	Posons $d = 2^n$\,. Comme $r \in 2 \NN + 1$\,, nous avons :
	
	\medskip
	
	\begin{stepcalc}[style = ar*]
		2^{nr} + 1
	\explnext*{$r \in 2 \NN + 1$}{}
		1 - (-d)^r
	\explnext{}
		\big( 1 + d \big) \, \big( 1 + (-d) + \cdots + (-d)^{r-1} \big)
	\end{stepcalc}
	
	\medskip
	
	Comme $n \divides 2^n + 1$\,, c'est-à-dire $n \divides d + 1$\,, nous obtenons que $n \divides 2^{nr} + 1$\,, i.e. $n \divides 2^{n \vee m} + 1$\,.
	Par symétrie des rôles, nous avons aussi $m \divides 2^{n \vee m} + 1$\,.
	Finalement, $n \vee m \in \ourset$\,.
\end{proof}


Notons que la preuve précédente donne une démonstration alternative du fait \ref{prime-divisor} mais pour tout diviseur $p$ non trivial, premier ou non, de $n \in \ourset$\,.
En effet,
posons $d = 2^n$ et partons de nouveau de $2^{np} + 1 = \big( 1 + d \big) \, \big( 1 + (-d) + \cdots + (-d)^{p-1}  \big)$\,.
Comme $p \divides n \divides 2^n + 1$\,, nous avons modulo $p$ :

\medskip

\begin{stepcalc}[style = ar*, ope = \equiv]
	1 + (-d) + \cdots + (-d)^{p-1} 
\explnext*{$d \equiv 2^n \equiv - 1 \mod p$}{}
	1 + 1 + \cdots + 1^{p-1} 
\explnext{}
	p
\explnext{}
	0
\end{stepcalc}

\medskip 

Finalement,
$n \divides d + 1$ et $p \divides \big( 1 + (-d) + \cdots + (-d)^{p-1}  \big)$\, de sorte que $n p \divides 2^{n p} + 1$\,.


% -------------------- %


\begin{fact} \label{product}
	$\forall (n , m) \in \ourset^2$, $n m \in \ourset$\,.
\end{fact}

\begin{proof}
	Nous avons
	$n = \dprod_{p \divides n} p^{\padicval{n}}$
	et
	$m = \dprod_{p \divides m} p^{\padicval{m}}$
	où les produits sont finis.
	Les faits suivants permettent de conclure.
%
	\begin{itemize}
		\item $n \vee m = \dprod_{p \divides m} p^{\max ( \padicval{n} ; \padicval{m} )}$

		\item Si $\max ( \padicval{n} ; \padicval{m} ) < \padicval{n} + \padicval{m}$\,, alors le fait \ref{prime-divisor} donne que $p^\delta \cdot ( n \vee m ) \in \ourset$ où $\delta = \padicval{n} + \padicval{m} - \max ( \padicval{n} ; \padicval{m} )$\,.

		\item En répétant l'opération précédente autant de fois que nécessaire, on arrive à obtenir que $n m \in \ourset$\,.
	\end{itemize}
\end{proof}


% -------------------- %


\begin{fact} \label{gcd}
	$\forall (n , m) \in \ourset^2$, $n \wedge m \in \ourset$\,.
\end{fact}

\begin{proof}
	TODO
\end{proof}


% -------------------- %


\begin{fact}
	$\forall n \in \ourset$, $2^n + 1 \in \ourset$\,.
\end{fact}

\begin{proof}
	Le principe est similaire à la preuve du fait \ref{lcm}.
	Notant $M = 2^n + 1 = n k$ et $d = 2^n$\,, nous avons :

	\medskip
	
	\begin{stepcalc}[style = sar]
		2^M + 1
	\explnext{}
		2^{nk} + 1
	\explnext{}
		\big( 1 + d \big) \, \big( 1 + (-d) + \cdots + (-d)^{k-1} \big)
	\explnext{}
		M \, \big( 1 + (-d) + \cdots + (-d)^{k-1} \big)
	\end{stepcalc}
\end{proof}

