La preuve du fait \ref{power-of-3} amène naturellement au fait suivant.

\begin{fact} \label{prime-divisor}
	$\forall n \in \ourset$, $\forall p \in \PP$\,,
	si $p \divides n$ alors $p n \in \ourset$\,.
\end{fact}

\begin{proof}
	$2^n = -1 + k n$, où $k \in \ZZ$\,, donne :

    \medskip
    
    \begin{stepcalc}[style=sar]
    	2^{p n}
    \explnext{}
    	\big( 2^n \big)^p
    \explnext{}
    	\big( -1 + k n \big)^p
    \explnext{}
    	\dsum_{i=0}^p \binom{p}{i} \, (-1)^{p-i} \cdot (k n)^i
    \explnext*{$p \divides \binom{p}{i}$ si $0 < i < p$}{}
    	(- 1)^p + \dsum_{i=1}^{p-1} p c_i \cdot (-1)^{p-i} \cdot (k n)^i + k^p \cdot n^p
    \explnext*{$n = p q$ où $q \in \NN$ \\ $p \in 2 \NN + 1$ car $\ourset \cap 2 \NN = \emptyset$\,.}{}
    	- 1 + pn \, \dsum_{i=1}^{p-1} c_i \cdot (-1)^{p-i} \cdot k^i n^{i-1} + pq \cdot n \cdot k^p \cdot n^{p-2}
    \end{stepcalc}

    \medskip

    On obtient finalement $2^{p n} = - 1 + pn \cdot r$ avec $r \in \ZZ$ comme souhaité.
\end{proof}


% -------------------- %


Notons au passage que ce qui précède et le fait \ref{prime-sol} donnent un exemple non trivial pour insister sur la nécessité de l'initialisation dans une preuve par récurrence car nous avons
$\forall p \in \PP$\,, $p^k \divides 2^{( p^k )} + 1$ 
implique
$p^{k+1} \divides 2^{( p^{k+1} )} + 1$\,, mais juste $\ourset \cap \PP = \setgene{3}$\,.


% -------------------- %


\begin{fact} \label{lcm}
	$\forall (n , m) \in \ourset^2$\,, $n \vee m \in \ourset$\,.
\end{fact}

\begin{proof}
	Soit $r \in \NN$ tel que $n \vee m = n r$\,. Rappelons que, d'après le fait \ref{no-even}, aucun des entiers considérés ne peut être pair.
	Posant $d = 2^n$\,, nous avons :
	
	\medskip
	
	\begin{stepcalc}[style = ar*]
		2^{nr} + 1
	\explnext*{$r \in 2 \NN + 1$}{}
		1 - (-d)^r
	\explnext{}
		(1 + d) \cdot \dsum_{i=0}^{r-1} (-d)^i
	\end{stepcalc}
	
	\medskip
	
	Comme $n \divides 2^n + 1$\,, c'est-à-dire $n \divides d + 1$\,, nous obtenons que $n \divides 2^{nr} + 1$\,, c'est-à-dire $n \divides 2^{n \vee m} + 1$\,.
	Par symétrie des rôles, nous avons aussi $m \divides 2^{n \vee m} + 1$\,.
	Finalement, $n \vee m \in \ourset$\,.
\end{proof}


Notons que la preuve précédente donne une démonstration alternative du fait \ref{prime-divisor}, et ceci pour tout diviseur $p$\,, non forcément premier, de $n \in \ourset$\,.
En effet,
posons $d = 2^n$ et partons de nouveau de $2^{np} + 1 = (1 + d) \cdot \dsum_{i=0}^{p-1} (-d)^i$\,.

Comme $p \divides n \divides 2^n + 1$\,, nous avons, modulo $p$\,, $d \equiv 2^n \equiv - 1 $\, d'où 
$\dsum_{i=0}^{p-1} (-d)^i \equiv \dsum_{i=0}^{p-1} 1 \equiv p \equiv 0$\,.
%
Finalement,
$n \divides d + 1$ et $p \divides \dsum_{i=0}^{p-1} (-d)^i$\, de sorte que $n p \divides 2^{n p} + 1$\,.


% -------------------- %


\begin{fact} \label{product}
	$\forall (n , m) \in \ourset^2$, $n m \in \ourset$\,.
\end{fact}

\begin{proof}
	Nous avons
	$n = \dprod_{p \divides n} p^{\padicval{n}}$
	et
	$m = \dprod_{p \divides m} p^{\padicval{m}}$
	où les produits sont finis.
	Les faits suivants permettent de conclure.

	\begin{itemize}
		\item $n \vee m = \dprod_{p \divides m} p^{\max ( \padicval{n} ; \padicval{m} )}$

		\item Le fait \ref{prime-divisor} donne $p^{\delta_p} \cdot ( n \vee m ) \in \ourset$ où $\delta_p = \padicval{n} + \padicval{m} - \max ( \padicval{n} ; \padicval{m} )$\,.

		\item En répétant l'opération précédente chaque fois que $\delta_p > 0$\,, on obtient $n m \in \ourset$\,.
	\end{itemize}
\end{proof}


% -------------------- %


\begin{fact} \label{gcd}
	$\forall (n , m) \in \ourset^2$, $n \wedge m \in \ourset$\,.
\end{fact}

\begin{proof}
	Comme $(n , m) \in \big( 2 \NN + 1 \big)^2$, la preuve vient directement du joli résultat suivant.
\end{proof}


% -------------------- %


\begin{fact}
	$\forall (n , m) \in \big( 2 \NN + 1 \big)^2$, on a :
	$(2^n + 1) \wedge (2^m + 1) = 2^{n \wedge m} + 1$
	\footnote{
		Les conditions de parité sont essentielles puisque
		$(2^3 + 1) \wedge (2^6 + 1) \neq 2^{3 \wedge 6} + 1$
		et
		$(2^2 + 1) \wedge (2^4 + 1) \neq 2^{2 \wedge 4} + 1$\,.
	}.
\end{fact}

\begin{proof}
	Notons $\delta = (2^n + 1) \wedge (2^m + 1)$\,, et supposons avoir $n \leq m$ quitte à échanger les rôles de $n$ et $m$\,.
	Essayons de localiser $\delta$\,.
		
	\medskip
	\begin{stepcalc}[style = ar*, ope = \implies]
		\delta \divides 2^n + 1 \text{ et } \delta \divides 2^m + 1
	\explnext*{Via $2^m + 1 - \big( 2^n + 1 \big)$\,.}{}
		\delta \divides 2^n \big( 2^{m-n} - 1 \big)
	\explnext*{$\delta \in 2 \NN + 1$ car $(n, m) \in \NNs \times \NNs$.}{}
		\delta \divides 2^{m-n} - 1
	\end{stepcalc}
	
	\medskip
	
	Ensuite $m - n \in 2 \NN$ donne $\delta \divides d^{m-n} - 1$ où $d = -2$\,.
	Comme $m \in 2 \NN + 1$\,, nous avons aussi $2^m + 1 = 1 - d^m$, et donc $\delta \divides d^m - 1$\,. 
	L'algorithme des différences du calcul d'un PGCD nous donne $\delta \divides d^{n \wedge m} - 1$\,, soit $\delta \divides 2^{n \wedge m} + 1$ puisque $n \wedge m \in 2 \NN + 1$\,.
	Notons que ceci suffit à la justification du fait \ref{gcd}.
	
	\medskip
	
	En fait, $\delta = 2^{n \wedge m} + 1$ car nous avons les implications suivantes.
		
	\medskip
	\begin{stepcalc}[style = ar*, ope = \implies]
		\delta \divides d^m - 1 \text{ et } \delta \divides d^{m-n} - 1
	\explnext{}
		\delta \divides d^m - 1 \text{ et } \delta \divides d^m - d^n
	\explnext*{$(n , m) \in \big( 2 \NN + 1 \big)^2$ et $d = -2$\,.}{}
		\delta \divides 2^m + 1 \text{ et } \delta \divides 2^n - 2^m
	\explnext*{Via $2^m + 1 + 2^n - 2^m$\,.}{}
		\delta \divides 2^m + 1 \text{ et } \delta \divides 2^n + 1
	\end{stepcalc}
\end{proof}


% -------------------- %


\begin{fact}
	$\forall n \in \ourset$, $2^n + 1 \in \ourset$\,.
\end{fact}

\begin{proof}
	Le principe est similaire à la preuve du fait \ref{lcm}.
	Notant $M = 2^n + 1 = n k$ et $d = 2^n$\,, nous avons
	$2^M + 1 = 2^{nk} + 1 = (1 + d) \cdot \dsum_{i=0}^{k-1} (-d)^i = M \cdot \dsum_{i=0}^{k-1} (-d)^i$\,.
\end{proof}

