\subsubsection{Cibler la recherche pour des tests brutaux} \label{algos-used-kill}

\leavevmode
\smallskip

Reprenons les notations de la section \ref{algos-used-select} en supposant $\consprod \in \NNssquare$ avec $k \in \NN_{\geq 2}$\,, et plaçons-nous dans la situation où l'ensemble $\setgeo{C}$ renvoyé par l'algorithme \ref{algo-select} est non vide (dans le cas contraire, notre tactique est mise en échec).
Nous avons donc au moins deux facteurs $(n+i)$ et $(n+i^\prime)$ de $\consprod$ vérifiant les points suivants.
%
\begin{itemize}
	\item $\forall p \in \PP - \setgeo{C}$\,, $( \padicval{n+i}, \padicval{n+i^\prime} ) \in ( 2 \NN )^2$\,.

	\item $\forall p \in \setgeo{C}$\,, $\padicval{n+i}$ et $\padicval{n+i^\prime}$ ont la même parité.
\end{itemize}

Ceci permet de \enquote{localiser} comme suit les valeurs de $n$ pouvant vérifier $\consprod \in \NNssquare$\,.
%
\begin{enumerate}
	\item Notons $\setproba{C}$ l'ensemble des entiers $\prod_{p \in \setgeo{C}} p^{(\epsilon_p)}$ avec $(\epsilon_p)_{p \in \setgeo{C}} \subseteq \setgene{0, 1}$\,.


	\item Il existe $c \in \setproba{C}$ tel que $n+i = c M^2$ et $n+i^\prime = c N^2$ avec $(c, M, N) \in \NNsf \times ( \NNs )^2$\,.
	
	\explainthis{Comme nous n'avons aucune idée de la suite $(\epsilon_p)_{p \in \setgeo{C}}$ à choisir, il faudra les tester toutes.}


	\item $c ( N^2 - M^2 ) \in \ZintervalC{1}{k-1}$ donne $N^2 - M^2 \in \ZintervalC{1}{\quot{k-1}{c}}$\,.


	\item L'algorithme \ref{algo-square-ko} permet de construire l'ensemble fini $\setproba*{D}{c}$\,, éventuellement vide, des couples $(M, N)$ tels que $N^2 - M^2 \in \ZintervalC{1}{\quot{k-1}{c}}$\,,
	à partir duquel nous construisons 
	$\setproba*{F}{c} = \setgene{(c M^2, c N^2) \, | \, (M, N) \in \setproba*{D}{c}}$\,.
	
	\explainthis{Nous pouvons affirmer que $\consprod$ contient au moins un couple de facteurs appartenant à $\cup_{c \in \setproba{C}} \setproba*{F}{c}$\,.}


	\item XXX


	\item XXX


	\item XXX


	\item XXX
\end{enumerate}






???






% ------------------ %


\medskip
%\newpage

Tout ce qui précède nous amène à l'algorithme suivant.

{\small
\begin{algo}[frame] \label{algo-kill}
%	\caption{Classique et efficace} 
	%%%
    \Data{?}
    \Result{?}
	\BlankLine
    \Actions{
		?
	}
\end{algo}
}