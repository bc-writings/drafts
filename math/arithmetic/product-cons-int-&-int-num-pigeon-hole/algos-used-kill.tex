\subsubsection{Cibler la recherche pour des tests brutaux} \label{algos-used-kill}

\leavevmode
\smallskip

Reprenons les notations de la section \ref{algos-used-select} en supposant $\consprod \in \NNssquare$ avec $k \in \NN_{\geq 2}$\,.
De plus, nous nous plaçons dans la situation où l'ensemble $\setgeo{C}$ renvoyé par l'algorithme \ref{algo-select} est non vide (dans le cas contraire, notre tactique est mise en échec).
Il existe donc au moins deux facteurs $(n+i)$ et $(n+i^\prime)$ de $\consprod$ vérifiant les deux points suivants.
%
\begin{itemize}
	\item $\forall p \in \PP - \setgeo{C}$\,, $( \padicval{n+i}, \padicval{n+i^\prime} ) \in ( 2 \NN )^2$\,.

	\item $\forall p \in \setgeo{C}$\,, $\padicval{n+i}$ et $\padicval{n+i^\prime}$ ont la même parité.
\end{itemize}

Nous pouvons alors \enquote{localiser} les valeurs possible de $n$ pour tester brutalement si $\consprod \in \NNssquare$ est vrai.
%
\begin{enumerate}
	\item 


	\item $c ( N^2 - M^2 ) \in \ZintervalC{1}{k-1}$\,, d'où $N^2 - M^2 \in \ZintervalC{1}{\quot{k-1}{c}}$\,.


	\item Le fait \ref{diff-square-ko} donne un ensemble fini $\setproba*{M}{k,c}$ tel que nécessairement $M \in \setproba*{M}{k,c}$\,.
\end{enumerate}


Notant $\epsilon_{i,p} = \reste{\padicval{n+i}}{2} \in \setgene{0, 1}$ pour $p \in \setgeo{C}$\,, puis posant $c = \prod_{p \in \setgeo{C}} p^{(\epsilon_{i,p})}$\,, nous avons $n+i = c M^2$ et $n+i^\prime = c N^2$ avec $(c, N, M) \in \NNsf \times ( \NNs )^2$\,.
Sans perte de généralité, nous pouvons supposer que $n + i > n+i^\prime$\,.s
Nous avons alors les faits suivants. 
???

YYYY





% ------------------ %


%\medskip
\newpage

Tout ceci nous amène au premier algorithme suivant.

{\small
\begin{algo}[frame] \label{algo-kill}
%	\caption{Classique et efficace} 
	%%%
    \Data{?}
    \Result{?}
	\BlankLine
    \Actions{
		?
	}
\end{algo}
}