Nous allons présenter dans cette section une démonstration qui va nous donner sans effort un algorithme permettant de valider, ou rejeter, la proposition $\consprod \notin \NNsquare$ au cas par cas
\footnote{
	Ou au $k$ par $k$... Ce jeu de mots bien médiocre est totalement assumé par l'auteur de ces lignes.
}. 

\medskip

Dans un échange sur \url{https://math.stackexchange.com}\,, voir la section \ref{sources}, il est indiqué une preuve de $\consprod<10> \notin \NNsquare$ pour $n \in \NNs$ quelconque.
Voici cette preuve complétée avec certains arguments laissés sous silence dans la source utilisée.
Noter au passage que l'essentiel consiste en des actes algorithmiques basiques permettant d'appliquer le fait \ref{diff-square-ko} sans user de fourberies déductives. Ennuyeux, mais efficace !


% ------------------ %


\begin{proof}[Preuve]%
    Supposons que $\consprod<10> \in \NNssquare$ pour $n \in \NNs$\,.
    
    \smallskip
    
    Clairement, 
    $\forall p \in \PP_{\geq 10}$\,, 
    $\forall i \in \ZintervalC{0}{9}$\,, 
    $\padicval{n + i} \in 2 \NN$\,.
    Concentrons-nous sur les nombres premiers dans $\PP_{< 10} = \setgene{2, 3, 5, 7}$\,. Voici ce que l'on peut observer très grossièrement.
    %
    \begin{itemize}
		\item Au maximum deux facteurs $(n + i)$ de $\consprod<10>$ sont divisibles par $7$\,.

		\item Au maximum deux facteurs $(n + i)$ de $\consprod<10>$ sont divisibles par $5$\,.

		\item Les points précédents donnent au moins $6$ facteurs $(n + i)$ de $\consprod<10>$ de valuation $p$-adique paire dès que $p \in \PP_{\geq 5}$\,.
    \end{itemize}
    
    Nous avons alors l'une des alternatives suivantes pour chacun des $6$ facteurs $(n+i)$ vérifiant $\padicval{n + i} \in 2 \NN$ dès que $p \in \PP_{\geq 5}$\,.
    %
    \begin{itemize}
    	\smallskip
		\item \alt{1}\,
		$\big( \padicval[2]{n + i} , \padicval[3]{n + i} \big) \in 2 \NN \times 2 \NN$

    	\smallskip
		\item \alt{2}\,
		$\big( \padicval[2]{n + i} , \padicval[3]{n + i} \big) \in 2 \NN \times \big( 2 \NN + 1)$

    	\smallskip
		\item \alt{3}\,
		$\big( \padicval[2]{n + i} , \padicval[3]{n + i} \big) \in \big( 2 \NN + 1 \big) \times 2 \NN$

    	\smallskip
		\item \alt{4}\,
		$\big( \padicval[2]{n + i} , \padicval[3]{n + i} \big) \in \big( 2 \NN + 1 \big) \times \big( 2 \NN + 1)$
    \end{itemize}
    
    \medskip
    
    Comme nous avons six facteurs pour quatre alternatives, ce bon vieux principe des tiroirs va nous permettre de lever des contradictions.
    %
    \begin{itemize}
    	\medskip
		\item Deux facteurs différents $(n+i)$ et $(n+i^\prime)$ vérifient \alt{1}\,.
		
		\smallskip
		\noindent
		Dans ce cas, $(n+i, n+i^\prime) = (N^2, M^2)$ avec $(M, N) \in \NNs$.
		Par symétrie des rôles, on peut supposer $N > M$\,, de sorte que $N^2 - M^2 \in \ZintervalC{1}{9}$\,. 
		Selon le fait \ref{diff-square-ko}, seuls les cas suivants sont possibles mais ils lèvent tous une contradiction.
		%
		\begin{enumerate}
			\item $N^2 - M^2 = 3$ avec $(M, N) = (1, 2)$ est possible, mais ceci donne $n = 1^2 = 1$\,, puis $\consprod[1]<10> = 10 ! \in \NNsquare$\,, or ceci est faux car $\padicval[7]{10!} = 1$\,.


			\item $N^2 - M^2 = 5$ avec $(M, N) = (2, 3)$ est possible
			d'où $n \in \ZintervalC{1}{4}$\,.
			Nous venons de voir que $n = 1$ est impossible.
			De plus, pour $n \in \ZintervalC{2}{4}$\,, $\padicval[7]{\consprod[n]<10>} = 1$ montre que $\consprod[n]<10> \in \NNsquare$ est faux.
			

			\item $N^2 - M^2 = 7$ avec $(M, N) = (3, 4)$ est possible
			d'où $n \in \ZintervalC{1}{9}$\,, puis $n \in \ZintervalC{5}{9}$ d'après ce qui précède.
			Mais ici, $\forall n \in \ZintervalC{5}{9}$\,, $\padicval[11]{\consprod[n]<10>} = 1$ montre que $\consprod[n]<10> \in \NNsquare$ est faux.


			\item $N^2 - M^2 = 8$ avec $(M, N) = (1, 3)$ est possible
			d'où $n = 1$\,, mais ceci est impossible comme nous l'avons vu ci-dessus.


			\item $N^2 - M^2 = 9$ avec $(M, N) = (4, 5)$ est possible
			d'où $n \in \ZintervalC{10}{16}$ d'après ce qui précède.
			Or $\forall n \in \ZintervalC{10}{16}$\,, $\padicval[17]{\consprod[n]<10>} = 1$\,, donc $\consprod[n]<10> \in \NNsquare$ est faux.
		\end{enumerate}


    	\medskip
		\item Deux facteurs différents $(n+i)$ et $(n+i^\prime)$ vérifient \alt{2}\,.
		
		\smallskip
		\noindent
		Dans ce cas, $(n+i, n+i^\prime) = (3 N^2, 3 M^2)$ avec $(M, N) \in \NNs$.
		Par symétrie des rôles, on peut supposer $N > M$\,, de sorte que $3(N^2 - M^2) \in \ZintervalC{1}{9}$\,, puis $N^2 - M^2 \in \ZintervalC{1}{3}$\,. 
		Selon le fait \ref{diff-square-ko}, nécessairement $N^2 - M^2 = 3$ avec $(M, N) = (1, 2)$\,, d'où $n \in \ZintervalC{1}{3}$\,, mais on sait que cela est impossible.


    	\medskip
		\item Deux facteurs différents $(n+i)$ et $(n+i^\prime)$ vérifient \alt{3}\,.
		
		\smallskip
		\noindent
		Dans ce cas, $(n+i, n+i^\prime) = (2 N^2, 2 M^2)$ avec $(M, N) \in \NNs$.
		Par symétrie des rôles, on peut supposer $N > M$\,, de sorte que $2(N^2 - M^2) \in \ZintervalC{1}{9}$\,, puis $N^2 - M^2 \in \ZintervalC{1}{4}$\,. 
		Selon le fait \ref{diff-square-ko}, nécessairement $N^2 - M^2 = 3$ avec $(M, N) = (1, 2)$\,, d'où $n \in \ZintervalC{1}{2}$\,, mais on sait que cela est impossible.


    	\medskip
		\item Deux facteurs différents $(n+i)$ et $(n+i^\prime)$ vérifient \alt{4}\,.
		
		\smallskip
		\noindent
		Dans ce cas, $(n+i, n+i^\prime) = (6 N^2, 6 M^2)$ avec $(M, N) \in \NNs$.
		Par symétrie des rôles, on peut supposer $N > M$\,, de sorte que $6(N^2 - M^2) \in \ZintervalC{1}{9}$\,, puis $N^2 - M^2 = 1$\,, mais c'est impossible d'après le fait \ref{diff-square-ko}.
		%
		\qedhere
    \end{itemize}
\end{proof}


% ------------------ %


Ce qui est intéressant avec la preuve précédente est qu'avec quelques adaptations \enquote{mécaniques}\,, on démontre que $\forall n \in \NNs$\,, $\consprod<k> \notin \NNsquare$ dès que $k \in \setgene{3, 5, 7, 9, 11, 12, 13}$\,.
\footnote{
	Voir mon document \emph{\enquote{Carrés parfaits et produits d'entiers consécutifs -- Des solutions à la main}}\,.
}
Ces preuves semblant peu gourmandes informatiquement, il semble opportun de tenter un traitement numérique des cas absents dans l'article de Paul Erdős. 

