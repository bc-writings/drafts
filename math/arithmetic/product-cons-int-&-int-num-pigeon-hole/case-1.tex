\subsection{Structure}


% ------------------ %


\leavevmode
\smallskip

\begin{fact} \label{prime-square}
	$n \in \NNssquare$ si, et seulement si,
	$\forall p \in \PP$\,,
	$\padicval{n} \in 2 \NN$\,.
\end{fact}


\begin{proof}
	Immédiat à valider.
\end{proof}


%% ------------------ %
%
%
%\begin{fact} \label{facto-square}
%	$\forall n \in \NNssquare$\,, s'il existe $m \in \NNssquare$ tel que $n =  f m$ alors $f  \in \NNssquare$\,.
%\end{fact}
%
%
%\begin{proof}
%	$\forall p \in \PP$\,, 
%	$\padicval{f m} \in 2 \NN$\,,
%	$\padicval{m} \in 2 \NN$
%	et
%	$\padicval{f m} = \padicval{f} + \padicval{m}$
%	donnent
%	$\padicval{f} \in 2 \NN$\,.
%\end{proof}


%% ------------------ %
%
%
%\begin{fact} \label{prime-square}
%	$\forall (a, b) \in \NNs \times \NNs$, 
%	si $\GCD{a}{b} = 1$ et $a b \in \NNssquare$\,,
%	alors $a \in \NNssquare$ et $b \in \NNssquare$\,.
%\end{fact}
%
%
%\begin{proof}
%	$\forall p \in \PP$\,, $\padicval{ab} \in 2 \NN$\,,
%	et $p$ ne peut diviser à la fois $a$ et $b$\,,
%	donc
%    $\forall p \in \PP$\,, 
%    $\padicval{a} \in 2 \NN$ et $\padicval{b} \in 2 \NN$\,,
%    autrement dit 
%    $(a, b) \in \NNssquare \times \NNssquare$\,.
%\end{proof}


%% ------------------ %
%
%
%\begin{fact} \label{same-square-free}
%	Soit $(a, b) \in \NNs \times \NNs$ tel que $a b \in \NNssquare$\,,
%	ainsi que $(\alpha, \beta, A, B) \in ( \NNsf )^2 \times \NN^2$ tel que $a = \alpha A^2$ et $b = \beta B^2$.
%	Nous avons alors forcément $\alpha = \beta$\,.
%\end{fact}
%
%%\newpage
%\begin{proof}
%	Le fait \ref{facto-square} donne $\alpha \beta \in \NNssquare$\,.
%	De plus, $\forall p \in \PP$\,, nous avons 
%	$\padicval{\alpha} \in \setgene{0, 1}$
%	et
%	$\padicval{\beta} \in \setgene{0, 1}$\,.
%	Finalement, $\forall p \in \PP$\,, $\padicval{\alpha} = \padicval{\beta}$\,, autrement dit $\alpha = \beta$\,.
%\end{proof}


% ------------------ %


\subsection{Distance entre deux carrés parfaits}

\begin{fact} \label{diff-square-ko}
	Soit $(N, M) \in \NNs \times \NNs$ tel que $N > M$\,.
	%
	\begin{enumerate}
		\item $N^2 - M^2 \geq 2N - 1$\,.
		
%		\newpage
		\item Notons $nb_{sol}$ le nombre de solutions $(N, M) \in \NNs \times \NNs$ de $N^2 - M^2 = \delta$\,.
		
		\smallskip
		\noindent
		Par exemple, pour $\delta \in \ZintervalC{1}{20}$, nous avons :
		\begin{enumerate}
			\item $nb_{sol}= 0$ si $\delta \in \{ 1, 2, 4, 6, 10, 14, 18\}$\,.

			\item $nb_{sol}= 1$ si $\delta \in \{ 3, 5, 7, 8, 9, 11, 12, 13, 16, 17, 19, 20 \}$\,.

			\item $nb_{sol}= 2$ si $\delta = 15$\,.
		\end{enumerate}
	\end{enumerate}
\end{fact}


\begin{proof}
	\leavevmode
	
	\vspace{-1ex}
	\begin{enumerate}
		\item Comme $N - 1 \geq M$\,, nous obtenons :
		$N^2 - M^2 \geq N^2 - (N - 1)^2 = 2N - 1$\,.

		\item Le point précédent permet d'utiliser le programme \python\ suivant afin d'obtenir rapidement les listes de nombres indiquées.
	\end{enumerate}

\bgroup
\small
\begin{Python}
from collections import defaultdict
from math        import sqrt, floor

# N**2 - M**2 = diff ?
def sol(diff):
    solfound = []

    for N in range(1, (diff + 1) // 2 + 1):
        M_square = N**2 - diff

        if M_square > 0:
            M = floor(sqrt(M_square))

            if M != 0 and M**2 == M_square:
                solfound.append((N, M))

    return solfound

all_nbsol = defaultdict(list)

for d in range(1, 101):
    all_nbsol[len(sol(d))].append(d)

print(all_nbsol)
\end{Python}
\egroup	


		%
		\qedhere
\end{proof}

	
% ------------------ %


\begin{remark}
	La fonction \emph{\piton{sol}} du programme précédent traduit l'algorithme suivant que nous utiliserons dans la suite de ce document.
\end{remark}

{\small
\begin{algo}[frame] \label{algo-square-ko}
%	\caption{Classique et efficace} 
	%%%
    \Data{$\delta \in \NNs$}
    \Result{$\setproba{S}$ l'ensemble des couples $(N, M) \in \NNs \times \NNs$ vérifiant $N^2 - M^2 = \delta$\,.}
	\BlankLine
    \Actions{
		$\setproba{S} \Store \emptyset$
		\\
		\BlankLine
		\ForRange{N}{1}{1 + \quot{\delta}{2}}{	
			$M_{square} \Store N^2 - \delta$
			\\
			\BlankLine
			\If{$M_{square} > 0$}{
				$M \Store \floor{\sqrt{M_{square}}}$
				\\
				\BlankLine
				\If{$M \neq 0$ \And $M^2 = M_{square}$}{
					$\setproba{S} \Store \setproba{S} \cup \setgene{(N, M)}$
				}
			}
		}
		\BlankLine
		\Return{$\setproba{S}$}
	}
\end{algo}
}

