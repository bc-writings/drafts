\leavevmode
\smallskip

????


La démonstration donnée dans la section \ref{case-10} commence par repérer le moins possible de nombres premiers $p$ de valuation -p--adiques non nécessaireemnt paires 



suit les grandes lignes suivantes qui mênent directement à un algorithme.
%
\begin{enumerate}
	\item XXXX

	\item XXXX

	\item XXXX
\end{enumerate}





\newpage
Idée : on compte grossièrement des nombres de valuation p adiques paires avec le plus de p différents, dison en avoir k, il reste alors d priemirs et on veut juste que d > 2**d, on fait en sorte d'aller vers d le plus petit possible.

une fois ceci faitn on se ramène à des eq du type $N^2 - M^2 = d$ ou d w= nbre de facteurs


De 2 à 100


5 cas KO : 
2, 3, 4, 6, 8


1 cas avec 1 premier gênant :
5


27 cas avec 2 premier gênants :
7, 9, 10, 11, 12, 13, 14, 15, 16, 17, 18, 19, 20, 21, 22, 23, 25, 26, 27, 28, 29, 30, 31, 33, 34, 35, 37


66 cas avec 3 premier gênants (tous les autres)


par exemple


pour 5 on a 2 nbrs premires
	 

pour 37 on a 11 nbrs premires
	 
\vspace{-1ex}
\begin{center}
    \begin{tblr}{
        width = \linewidth,
        stretch = 1.75,
        colspec = {X[3,r] *{10}{X[1,c,$]}},
        vline{2-Y},
        hline{2-Y},
        rowsep=2pt,
        colsep=3pt,
		% GOOD!
		column{W-X} = {blue!15},
		column{Y}   = {green!15},
		% STOP!
		column{Z} = {red!15},
    }
      $p\,$
    	& 31 & 29 & 23 & 19 & 17 & 13 & 11 & 7  & 5  & 3
    \\
      Occu. max.
		& 2  & 2  & 2  & 2  & 3  & 3  & 4  & 6  & 8  & 13
    \\
      Occu. libres.
		& 35 & 33 & 31 & 29 & 26 & 23 & 19 & 13 & 5  & 0
    \\
      Alternatives.
		& 2^{10}
		& 2^9
		& 2^8
		& 2^7
		& 2^6
		& 2^5
		& 2^4
		& 2^3
		& 2^2
		& 2
    \end{tblr}
\end{center}



pour 40

\vspace{-1ex}
\begin{center}
    \begin{tblr}{
        width = \linewidth,
        stretch = 1.75,
        colspec = {X[3,r] *{11}{X[1,c,$]}},
        vline{2-Y},
        hline{2-Y},
        rowsep=2pt,
        colsep=3pt,
		% GOOD!
		column{W} = {blue!15},
		column{X} = {green!15},
		% STOP!
		column{Z} = {red!15},
    }
      $p\,$
    	& 37 & 31 & 29 & 23 & 19 & 17 & 13 & 11 & 7  & 5  & 3
    \\
      Occu. max.
		& 2  & 2  & 2  & 2  & 3  & 3  & 4  & 4  & 6  & 8  & 14
    \\
      Occu. libres.
		& 38 & 36 & 34 & 32 & 29 & 26 & 22 & 18 & 12 & 4  & 0
    \\
      Alternatives.
		& 2^{11}
		& 2^{10}
		& 2^9
		& 2^8
		& 2^7
		& 2^6
		& 2^5
		& 2^4
		& 2^3
		& 2^2
		& 2
    \end{tblr}
\end{center}


ne pas rêver : 
824
          FAILS!