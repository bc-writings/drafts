\subsubsection{Sélection de potentiels bons candidats} \label{algo-select}

\leavevmode
\smallskip

La première phase consiste à tenter de trouver le moins possible de nombres premiers $p$ tel que tous les facteurs $(n+i)$ de $\consprod$ soient de valuation $p$-adique non nécessairement paire. 
Comme $p \in \PP_{\geq k}$ divise au maximum un facteur $(n+i)$ de $\consprod$\,, nous avons $\forall i \in \ZintervalC{0}{k-1}$\,, $\padicval{n+i} \in 2\,\NN$ dès que $p \in \PP_{\geq k}$ puisque  $\consprod \in \NNssquare$ par hypothèse. Ceci permet de cibler notre analyse sur les nombres premiers dans $\PP_{< k}$\,. 


% ------------------ %


\medskip

Voici un premier exemple de sélection avec $\consprod<3>$ en notant que $\PP_{< 3} = \setgene{2}$\,. Nous expliquons juste après comment lire le tableau ci-dessous.


% \vspace{-1ex}
\begin{center}
    \begin{tblr}{
        width = \linewidth,
%        stretch = 1.75,
		colspec = {Q[r]*{2}{Q[c,$]}},
        vline{2-Y},
        hline{2-Y},
        rowsep      = 2pt,
        colsep      = 3pt,
		column{1}   = {6em},
		column{2-Z} = {1.5em},
		% GOOD!
		column{Y} = {green!15},
		% STOP!
		column{Z} = {red!15},
    }
      $p_m\,$
    	&   & 2
    \\
      Occu. max.
		&   & 2
    \\
      Occu. libres.
		& 3 & 1
    \\
      Alternatives.
		& 2^1
		& 0
    \end{tblr}
\end{center}

Le tableau se lit comme suit.
%
\begin{itemize}
	\item $p_m$ désigne le plus grand nombre premier disponible non encore éliminé.

	\item La deuxième ligne indique le nombre maximum de facteurs $(n+i)$ de $\consprod$ pouvant être divisibles par $p_m$\,.

	\item La troisième ligne donne le nombre minimum de facteurs de valuations $p$-adiques nécessairement paires dès que $p \in \PP_{\geq p_m}$\,.

	\item La dernière ligne donne le nombre d'alternatives possibles relativement aux parités des valuations $p$-adiques pour les nombres premiers $p$ dans $\PP_{< p_m}$\,, les autres valuations $p$-adiques restantes étant paires.

	\item La colonne sur fond vert indique le \enquote{meilleur bon} candidat, c'est-à-dire celui avec le moins d'alternatives.
	Nous utiliserons du bleu pour de bons candidats non gardés.

	\item La colonne sur fond rouge indique que l'on ne peut plus avancer (évident ici mais nous verrons que cela peut arriver plus tôt dans l'analyse).
\end{itemize}


Nous voyons ici que $2$ est un bon candidat pour rejeter $\consprod<3> \in \NNssquare$ puisqu'au moins deux facteurs différents $(n+i)$ et $(n+i^\prime)$ de $\consprod<3>$ vérifient la même alternative, d'où $n+i = c M^2$ et $n+i^\prime = c N^2$ avec $(c, N, M) \in \NNsf \times ( \NNs )^2$\,, une information qui sera utilisée par notre second algorithme pour \enquote{localiser}\,, via le fait \ref{diff-square-ko}, des entiers naturels $n$ afin de tester presque brutalement si $\consprod<3> \in \NNssquare$ est vrai, ou non.


% ------------------ %


\medskip

Voici un autre exemple montrant que la sélection peut échouer : il suffit de considérer par exemple $\consprod<4>$ en notant que $\PP_{< 4} = \setgene{2, 3}$\,.

% \vspace{-1ex}
\begin{center}
    \begin{tblr}{
        width = \linewidth,
%        stretch = 1.75,
		colspec = {Q[r]*{3}{Q[c,$]}},
        vline{2-Y},
        hline{2-Y},
        rowsep      = 2pt,
        colsep      = 3pt,
		column{1}   = {6em},
		column{2-Z} = {1.5em},
		% STOP!
		column{Z} = {red!15},
    }
      $p_m\,$
    	&   & 3 & 2
    \\
      Occu. max.
		&   & 2 & 2
    \\
      Occu. libres.
		& 4 & 2 & 0
    \\
      Alternatives.
		& 2^2
		& 2^1
		& 0
    \end{tblr}
\end{center}



% ------------------ %


\medskip
%\newpage

Afin de clarifier la démarche que nous allons suivre, donnons un dernier exemple via $\consprod<37>$ en notant que $\card ( \PP_{< 37} ) = 11$\,.
	 
% \vspace{-1ex}
\begin{center}
    \begin{tblr}{
        width = \linewidth,
%        stretch = 1.75,
        colspec = {X[3,r] *{11}{X[1,c,$]}},
        vline{2-Y},
        hline{2-Y},
        rowsep      = 2pt,
        colsep      = 3pt,
		column{1}   = {6em},
		column{2-Z} = {1.5em},
		% GOOD!
		column{W-X} = {blue!15},
		column{Y}   = {green!15},
		% STOP!
		column{Z} = {red!15},
    }
      $p_m\,$
    	&    & 31 & 29 & 23 & 19 & 17 & 13 & 11 & 7  & 5  & 3
    \\
      Occu. max.
		&    & 2  & 2  & 2  & 2  & 3  & 3  & 4  & 6  & 8  & 13
    \\
      Occu. libres.
		& 37 & 35 & 33 & 31 & 29 & 26 & 23 & 19 & 13 & 5  & 0
    \\
      Alternatives.
		& 2^{11}
		& 2^{10}
		& 2^9
		& 2^8
		& 2^7
		& 2^6
		& 2^5
		& 2^4
		& 2^3
		& 2^2
		& 2
    \end{tblr}
\end{center}


% ------------------ %


\medskip

Nous décidons donc de procéder grosso modo comme suit.

\begin{enumerate}
	\item Nous supposons par l'absurde que $\consprod \in \NNssquare$ avec $k \in \NNs$.


	\item Nous fabriquons $\setgeo{P} = \PP_{<k}$\,.


	\item Nous posons $\setgeo{C} = \emptyset$\,.
	
	\explainthis{Cet ensemble sera celui des nombres premiers \enquote{candidats} utilisés dans notre second algorithme de tests brutaux. Nous cherchons à obtenir l'ensemble $\setgeo{C}$ non vide le plus petit possible.}


	\item Nous posons aussi $occu_{libre} = k$\,.
	
	\explainthis{Cette variable va nous servir à compter les facteurs $(n+i)$ de $\consprod$ ayant un \enquote{maximum} de valuations $p$-adiques pairs.}


	\item \label{algo-select-restart}
	\textbf{Début des actions répétitives.}
	
	\noindent
	Si $\setgeo{P} \neq \emptyset$ et $occu_{libre} > 2^{\card( \setgeo{P} )}$\,, nous posons $\setgeo{C} = \setgeo{P}$\,.
	
	\explainthis{Nous avons $2^{\card ( \setgeo{P} )}$ alternatives \alt{${}_j$} possibles relativement aux parités possible des valuations $p$-adiques pour les nombres premiers $p$ dans $\setgeo{P} = \PP_{< p_m}$\,, les valuations $p$-adiques restantes étant paires. %
	De l'autre côté, nous avons au moins $occu_{libre}$ facteurs $(n+i)$ de $\consprod$ tels que $\padicval{n+i} \in 2\,\NN$ dès que $p \in \PP_{\geq p_m}$\,. %
	Finalement, si $occu_{libre} > 2^{\card ( \setgeo{P} )}$\,, nous avons au moins deux facteurs différents $(n+i)$ et $(n+i^\prime)$ vérifiant la même alternative \alt{${}_j$}\,, d'où $c M^2$ et $c N^2$ avec $(c, N, M) \in \NNsf \times ( \NNs )^2$\,, une information qui sera utilisée par notre second algorithme pour \enquote{localiser} des $\consprod$ à tester brutalement.}


	\item Si $\setgeo{P} = \emptyset$\,, ou $occu_{libre} = 0$\,, nous stoppons tout !
	
	\explainthis{Si $\setgeo{C} = \emptyset$\,, nous avons perdu. Dans le cas contraire, nous pourrons continuer avec l'algorithme qui sera présenté dans la section \ref{algo-kill} suivante.}


	\item Sinon, nous considérons $p_m = \max ( \setgeo{P})$\,, puis retirons $p_m$ de $\setgeo{P}$\,, d'où $\setgeo{P} = \PP_{< p_m}$\,.
	
	\explainthis{Le choix du maximum tente de limiter les rejets de facteurs dans les étapes suivantes.} 


	\item Nous calculons $occu_{max}$ le nombre maximum de facteurs $(n+i)$ de $\consprod$ pouvant être divisés par $p_m$\,.
	
	\explainthis{Le calcul de $occu_{max}$ est simple puisqu'il suffit de considérer le cas où $p$ divise $n$\,, nous obtenons alors $occu_{max} = 1 + \quot{k-1}{p}$ car $\consprod = n (n + 1) \cdots (n + k - 1)$\,.}


	\item $occu_{libre}$ devient $occu_{libre} - occu_{max}$\,.
	
	\explainthis{Maintenant, nous savons qu'au moins $occu_{libre}$ facteurs $(n+i)$ de $\consprod$ vérifient $\padicval{n+i} \in 2\,\NN$ dès que $p \in \PP_{\geq p_m}$\,.}


	\item Nous reprenons les étapes à partir du point \ref{algo-select-restart}.
\end{enumerate}


% ------------------ %


%\medskip
\newpage

Tout ceci nous amène au premier algorithme suivant.

{\small
\begin{algo}[frame] \label{algo-select}
%	\caption{Classique et efficace} 
	%%%
    \Data{$k \in \NN_{\geq 2}$\,, le nombre de facteurs considérés}
    \Result{$\setgeo{C}$ un ensemble, éventuellement vide, de nombres premiers \enquote{candidats} tels que si $\setgeo{C} \neq \emptyset$ alors il existe au moins deux facteurs $(n+i)$ et $(n+i^\prime)$ de $\consprod$ vérifiant $\forall p \in \PP - \setgeo{C}$\,, $( \padicval{n+i}, \padicval{n+i^\prime} ) \in ( 2 \NN )^2$\,, ainsi que $\padicval{n+i}$ et $\padicval{n+i^\prime}$ ont la même parité dès que $p \in \setgeo{C}$\,.}
	\BlankLine
    \Actions{
		$u^{\prime} \Store 1$
		\\
		$u^{\prime\prime} \Store 0$
		\\
    	$v^{\prime} \Store 0$
		\\
		$v^{\prime\prime} \Store 1$
		\\
		\BlankLine
        \While{$b \neq 0$}{
			$a = q b + r$ est la division euclidienne standard.
			\\
			$temp_u \Store u^{\prime} - q u^{\prime\prime}$
			\\
			$u^{\prime} \Store u^{\prime\prime}$
			\\
			$u^{\prime\prime} \Store temp_u$
			\\
			$temp_v \Store v^{\prime} - q v^{\prime\prime}$
			\\
			$v^{\prime} \Store v^{\prime\prime}$
			\\
			$v^{\prime\prime} \Store temp_v$
		}
		\Return{$(u^{\prime} ; v^{\prime})$}
	}
\end{algo}
}


% ------------------ %


\medskip

Une fois l'algorithme \ref{algo-select} traduit en \python, nous obtiennons instantanément les informations suivantes pour $k \in \ZintervalC{2}{100}$\,.
%
\begin{itemize}
	\item \textbf{Mauvais candidats.}
	
	\noindent
	Il y en a 4 correspondant aux entiers $2$\,, $4$\,, $6$ et $8$\,.
	
	\item \textbf{Bons candidats avec un seul nombre premier à gérer.}
	
	\noindent
	Il y en a 2 correspondant aux entiers $3$ et $5$\,.
	
	\item \textbf{Bons candidats avec deux nombres premiers à gérer.}
	
	\noindent
	Il y en a 27 correspondant aux entiers $7$\,, $9$\,, $10$\,, $11$\,, $12$\,, $13$\,, $14$\,, $15$\,, $16$\,, $17$\,, $18$\,, $19$\,, $20$\,, $21$\,, $22$\,, $23$\,, $25$\,, $26$\,, $27$\,, $28$\,, $29$\,, $30$\,, $31$\,, $33$\,, $34$\,, $35$ et $37$\,.

	\item\textbf{Bons candidats avec trois nombres premiers à gérer.}
	
	\noindent
	Il y en a 66 correspondant aux entiers restants.
\end{itemize}


Ce qui précède est encourageant, car peu de cas sont rejetés.
De plus, les mauvais candidats sont faciles à gérer par un humain, ou un programme : voir la section \ref{algo-KO}.
Quant aux candidats acceptés, le nombre maximum d'alternatives est $2^3 = 8$ qui n'est pas une valeur informatiquement bloquante (nous verrons dans la section \ref{algo-kill} que d'autres paramètres peuvent bloquer la recherche).

% ------------------ %


\begin{remark}
	Ne rêvons pas trop à un principe général, car le programme donne aussi que $824$ est le premier naturel, après $8$\,, non sélectionné par notre algorithme.
\end{remark}