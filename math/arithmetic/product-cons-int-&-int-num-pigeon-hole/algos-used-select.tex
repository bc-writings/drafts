\subsubsection{Sélection de potentiels bons candidats} \label{algo-select}

\leavevmode
\smallskip

La première phase consiste à tenter de trouver le moins possible de nombres premiers $p$ tel que tous les facteurs $(n+i)$ de $\consprod$ soient de valuation $p$-adique non nécessairement paire. Pour ce faire, on procède grosso modo comme suit.

\begin{enumerate}
	\item On suppose par l'absurde que $\consprod \in \NNssquare$ avec $k \in \NNs$.


	\item On fabrique $\setgeo{P} = \PP_{<k}$\,.
	Ce choix vient de ce que pour $p \in \PP_{\geq k}$\,, nous savons que $p$ divise au maximum un facteur $(n+i)$ de $\consprod$\,, et donc $\forall i \in \ZintervalC{0}{k-1}$\,, $\padicval{n+i} \in 2\,\NN$ puisque $\consprod \in \NNssquare$ par hypothèse.


	\item On pose $\setgeo{C} = \emptyset$\,. Cet ensemble sera celui des nombres premiers \enquote{candidats} qui seront utilisés dans notre second algorithme.


	\item On pose $f = k$\,. Cette variable va nous servir à compter les facteurs $(n+i)$ de $\consprod$ ayant un \enquote{maximum} de valuations $p$-adiques pairs.


	\item \label{algo-select-restart}
	\textbf{Début des actions répétitives.}
	
	\noindent
	Si $\setgeo{P} = \emptyset$\,, on renvoie $\setgeo{C}$\,.


	\item On considère $p_m = \max ( \setgeo{P})$\,, et on retire $p_m$ de $\setgeo{P}$\,, de sorte que $\setgeo{P} = \PP_{< p_m}$\,. Le choix du maximum va permettre d'éliminer le moins possible de facteurs gênants dans les étapes suivantes. 


	\item On compte alors $f_m$ le nombre maximum de facteurs $(n+i)$ de $\consprod$ pouvant être divisibles par $p_m$\,.
	Le calcul de $f_m$ est simple puisqu'il suffit de considérer le cas où $p$ divise $n$ : on obtient $f_m = 1 + \quot{k-1}{p}$ car $\consprod = n (n + 1) \cdots (n + k - 1)$\,.


	\item $f$ devient $f - f_m$ : on sait ici qu'il y a au moins $f$ facteurs $(n+i)$ de $\consprod$ tels que $\padicval{n+i} \in 2\,\NN$ dès que $p \in \PP_{\geq p_m}$\,.


	\item Nous avons $2^{\card ( \setgeo{P} )}$ alternatives \alt{${}_j$} possibles relativement aux parités des valuations $p$-adiques pour les nombres premiers $p$ dans $\setgeo{P} = \PP_{< p_m}$\,, les valuations $p$-adiques restantes étant paires.
	De l'autre côté, nous avons au moins $f$ facteurs $(n+i)$ de $\consprod$ tels que $\padicval{n+i} \in 2\,\NN$ dès que $p \in \PP_{\geq p_m}$\,.
	Finalement, si $f > 2^{\card ( \setgeo{P} )}$\,, nous aurons au moins deux facteurs différents $(n+i)$ et $(n+i^\prime)$ vérifiant la même alternative \alt{${}_j$}\,, d'où nous déduirons que $c M^2$ et $c N^2$ avec $(c, N, M) \in \NNsf \times ( \NNs )^2$\,, une information qui sera utilisée par notre second algorithme pour \enquote{localiser} des $\consprod$ à tester brutalement.


	\item Si $f > 2^{\card ( \setgeo{P} )}$\,, on pose $\setgeo{C} = \setgeo{P}$\,.


	\item XXX 


	\item XXX 


	\item XXX 


	\item XXX 


	\item XXX ??? sinon on reprend les étapes à partir du point \ref{algo-select-restart}.


	\item Si $\setgeo{C} = \emptyset$\,, alors on a perdu, sinon on pourra continuer avec l'algorithme présentée dans la section \ref{algo-kill} suivante.
\end{enumerate}


Tout ceci nous amène au premier algorithme suivant.



\newpage
Idée : on compte grossièrement des nombres de valuation p adiques paires avec le plus de p différents, dison en avoir k, il reste alors d priemirs et on veut juste que d > 2**d, on fait en sorte d'aller vers d le plus petit possible.

une fois ceci faitn on se ramène à des eq du type $N^2 - M^2 = d$ ou d w= nbre de facteurs


De 2 à 100


5 cas KO : 
2, 4, 6, 8


1 cas avec 1 premier gênant :
5


27 cas avec 2 premier gênants :
7, 9, 10, 11, 12, 13, 14, 15, 16, 17, 18, 19, 20, 21, 22, 23, 25, 26, 27, 28, 29, 30, 31, 33, 34, 35, 37


66 cas avec 3 premier gênants (tous les autres)


par exemple


pour 5 on a 2 nbrs premires
	 

pour 37 on a 11 nbrs premires
	 
\vspace{-1ex}
\begin{center}
    \begin{tblr}{
        width = \linewidth,
        stretch = 1.75,
        colspec = {X[3,r] *{10}{X[1,c,$]}},
        vline{2-Y},
        hline{2-Y},
        rowsep=2pt,
        colsep=3pt,
		% GOOD!
		column{W-X} = {blue!15},
		column{Y}   = {green!15},
		% STOP!
		column{Z} = {red!15},
    }
      $p\,$
    	& 31 & 29 & 23 & 19 & 17 & 13 & 11 & 7  & 5  & 3
    \\
      Occu. max.
		& 2  & 2  & 2  & 2  & 3  & 3  & 4  & 6  & 8  & 13
    \\
      Occu. libres.
		& 35 & 33 & 31 & 29 & 26 & 23 & 19 & 13 & 5  & 0
    \\
      Alternatives.
		& 2^{10}
		& 2^9
		& 2^8
		& 2^7
		& 2^6
		& 2^5
		& 2^4
		& 2^3
		& 2^2
		& 2
    \end{tblr}
\end{center}



pour 40

\vspace{-1ex}
\begin{center}
    \begin{tblr}{
        width = \linewidth,
        stretch = 1.75,
        colspec = {X[3,r] *{11}{X[1,c,$]}},
        vline{2-Y},
        hline{2-Y},
        rowsep=2pt,
        colsep=3pt,
		% GOOD!
		column{W} = {blue!15},
		column{X} = {green!15},
		% STOP!
		column{Z} = {red!15},
    }
      $p\,$
    	& 37 & 31 & 29 & 23 & 19 & 17 & 13 & 11 & 7  & 5  & 3
    \\
      Occu. max.
		& 2  & 2  & 2  & 2  & 3  & 3  & 4  & 4  & 6  & 8  & 14
    \\
      Occu. libres.
		& 38 & 36 & 34 & 32 & 29 & 26 & 22 & 18 & 12 & 4  & 0
    \\
      Alternatives.
		& 2^{11}
		& 2^{10}
		& 2^9
		& 2^8
		& 2^7
		& 2^6
		& 2^5
		& 2^4
		& 2^3
		& 2^2
		& 2
    \end{tblr}
\end{center}


ne pas rêver : 
824
          FAILS!