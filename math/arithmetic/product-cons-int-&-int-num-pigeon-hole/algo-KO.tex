\leavevmode
\smallskip

Aussi surprenant que cela puisse paraître, il est très facile, bien que fastidieux, de démontrer humainement que $\consprod \notin \NNssquare$ pour $k \in \setgene{2, 4, 6}$ :
voir mon document \emph{\enquote{Carrés parfaits et produits d’entiers consécutifs – Une méthode efficace}} pour savoir comment cela fonctionne
\footnote{
	Le cas $k = 6$ est rédigé dans mon document \emph{\enquote{Carrés parfaits et produits d’entiers consécutifs – Des solutions à la main}}\,.
}.
La méthode présentée étant facile à coder, un programme \python, fait sans astuce, démontre instantanément, ou presque, que $\consprod \notin \NNssquare$ pour $k \in \ZintervalC{2}{8}$\,, ce qui achève notre péril informatique.