\leavevmode
\smallskip

Aussi surprenant que cela puisse paraître, il est très facile de démontrer humainement que $\consprod \notin \NNssquare$ pour $k \in \ZintervalC{2}{6}$ :
se reporter à mon document \emph{\enquote{Carrés parfaits et produits d’entiers consécutifs – Une méthode efficace}} pour savoir comment cela fonctionne
\footnote{
	Dans mon document \emph{\enquote{Carrés parfaits et produits d’entiers consécutifs – Des solutions à la main}}\,, vous trouverez le cas $k = 6$ rédigé à la sueur des neurones.
}.
La méthode citée étant facile à coder, un programme \python, fait sans astuce, démontre instantanément, ou presque, que $\consprod \notin \NNssquare$ pour $k \in \ZintervalC{2}{8}$\,, ce qui achève notre périple informatique.