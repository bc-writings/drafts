 \documentclass[12pt]{amsart}
%\usepackage[T1]{fontenc}
%\usepackage[utf8]{inputenc}

\usepackage[top=1.95cm, bottom=1.95cm, left=2.35cm, right=2.35cm]{geometry}

\usepackage%[hidelinks]%
           {hyperref}  
\usepackage{pgfpages}
%\pgfpagesuselayout{2 on 1}[a4paper,landscape,border shrink=5mm]

\usepackage{hyperref}
\usepackage{enumitem}
\usepackage{tcolorbox}
\usepackage{float}
\usepackage{cleveref}
\usepackage{multicol}
\usepackage{fancyvrb}
\usepackage{enumitem}
\usepackage{amsmath}
\usepackage{textcomp}
\usepackage{numprint}
\usepackage{tabularray}
\usepackage[french]{babel}
\frenchsetup{StandardItemLabels=true}
\usepackage{csquotes}
\usepackage{piton}

\NewPitonEnvironment{Python}{}
  {\begin{tcolorbox}}
  {\end{tcolorbox}}
  
\SetPitonStyle{
 	Number = ,
    String = \itshape ,
    String.Doc = \color{gray} \slshape ,
    Operator = ,
    Operator.Word = \bfseries ,
    Name.Builtin = ,
    Name.Function = ,
    Comment = \color{gray} ,
    Comment.LaTeX = \normalfont \color{gray},
    Keyword = \bfseries ,
    Name.Namespace = ,
    Name.Class = ,
    Name.Type = ,
    InitialValues = \color{gray}
}

\usepackage[
    type={CC},
    modifier={by-nc-sa},
	version={4.0},
]{doclicense}

\newcommand\floor[1]{\left\lfloor #1 \right\rfloor}

\usepackage{tnsmath}
\usepackage{tnsalgo}


\newtheorem{fact}{Fait}[section]
\newtheorem{defi}{Définition}[section]
\newtheorem{example}{Exemple}[section]
\newtheorem{remark}{Remarque}[section]

\npthousandsep{.}
\setlength\parindent{0pt}

\floatstyle{boxed} 
\restylefloat{figure}


\DeclareMathOperator{\taille}{\text{\normalfont\texttt{taille}}}

\newcommand{\logicneg}{\text{\normalfont non \!}}

\newcommand\sqseq[2]{\fbox{$#1$}_{\,\,#2}}


\DefineVerbatimEnvironment{rawcode}%
	{Verbatim}%
	{tabsize=4,%
	 frame=lines, framerule=0.3mm, framesep=2.5mm}
	 
	 
\newcommand\contentdir{\jobname}

\newcommand\python{\texttt{Python}}

\newcommand\NNsf{\NN_{\kern-1pt s\kern-1pt f}}

\newcommand\NNsquare{\seqsuprageo{\NN}{}{}{}{2}}
\newcommand\NNssquare{\seqsuprageo{\NN}{*}{}{}{2}}

\NewDocumentCommand\GCD{ m  m }{#1 \wedge #2}
\NewDocumentCommand\reste{ m  m }{\mathrm{reste}( #1 , #2)}
\NewDocumentCommand\quot{ m  m }{\mathrm{quot}( #1 , #2)}

\NewDocumentCommand\padicval{ O{p} m }{v_{#1}(#2)}
\NewDocumentCommand\consprod{ O{n} D<>{k} }{\pi_{#1}^{#2}}

\NewDocumentCommand\alt{ m }{\textbf{[A\kern1pt#1]}}

\newcommand\mycheckmark{{\color{green!60!black} \checkmark}}
\newcommand\myboxtimes{{\color{red!80!black} \boxtimes}}

\newcommand\explainthis[1]{

	\noindent
	\bgroup
	\small
	\emph{#1}
	\egroup
}


\begin{document}

\title{BROUILLON - Carrés parfaits et produits d'entiers consécutifs -- Jusqu'à 100 facteurs ?}
\author{Christophe BAL}
\date{14 Fév. 2024 -- 21 Fév. 2024}

\maketitle

\begin{center}
	\itshape
	Document, avec son source \LaTeX, disponible sur la page
	
	\url{https://github.com/bc-writing/drafts}.
\end{center}


\bigskip


\begin{center}
	\hrule\vspace{.3em}
	{
		\fontsize{1.35em}{1em}\selectfont
		\textbf{Mentions \enquote{légales}}
	}
			
	\vspace{0.45em}
	\small
	\doclicenseThis
	\hrule
\end{center}


\setcounter{tocdepth}{3}
\tableofcontents


% ------------------ %


\newpage
\section{Ce qui nous intéresse}

Dans l'article \enquote{Note on Products of Consecutive Integers}
\footnote{
	J. London Math. Soc. 14 (1939).
},
Paul Erdos démontre que pour tout couple $(n, k) \in \NNs \times \NNs$\,, le produit de $(k+1)$ entiers consécutifs $n (n + 1) \cdots (n + k)$ n'est jamais le carré d'un entier. 

\smallskip

Il est facile de trouver sur le web des preuves à la main de $n(n+1) \cdots (n + k) \notin \NNssquare$ pour $k \in \ZintervalC{1}{7}$\,.
Bien que certaines de ces preuves soient très sympathiques, leur lecture ne fait pas ressortir de schéma commun de raisonnement.
%
Dans ce document, nous allons tenter de limiter au maximum l'emploi de fourberies déductives en présentant une méthode très élémentaire
\footnote{
	Cette méthode s'appuie sur une représentation trouvée dans \href{https://web.archive.org/web/20171110144534/http://mathforum.org/library/drmath/view/65589.html}{un message archivé} : voir la section \ref{sources}.
},
efficace, et semi-automatisable, pour démontrer, avec peu d'efforts cognitifs, les premiers cas d'impossibilité.




% ------------------ %


\bigskip
\section{Notations utilisées}

Dans la suite, nous utiliserons les notations suivantes.
\begin{itemize}
	\item $2\,\NN$ désigne l'ensemble des nombres naturels pairs.
	
	\item $2\,\NN + 1$ désigne l'ensemble des nombres naturels impairs.
	
	\item $\forall (n , m) \in \NN^2$, $n \vee m$ désigne le PPCM de $n$ et $m$.

	\item $\forall (n , m) \in \NN^2$, $n \wedge m$ désigne le PGCD de $n$ et $m$.

	\item $a \strictdivides b$ signifie que $a \divides b$ et $a \neq b$ (division stricte).

	\item $\PP$ désigne l'ensemble des nombres premiers.
	
	\item $\forall (p ; n) \in \PP \times \NNs$\,, $\padicval{n} \in \NN$ est la valuation $p$-adique de $n$\,, c'est-à-dire $p^{\padicval{n}} \divides n$\,, mais $p^{\padicval{n} + 1} \ndivides n$\,.
\end{itemize}


% ------------------ %


%\newpage
%\medskip
\section{Les carrés parfaits} \label{case-1}

\leavevmode
\smallskip

Via $N^2 - M^2 = (N - M)(N + M)$\,, il est immédiat de noter que 
$\forall (N, M) \in \NNs \times \NNs$\,, si $N > M$\,, alors $N^2 - M^2 \geq 3$\,. Le fait suivant précise ceci.


\begin{fact} \label{dist-square}
	$\forall (N, M) \in \NNs \times \NNs$, 
	si $N > M$\,, alors $N^2 - M^2 = \dsum_{k=M+1}^{N} (2 k - 1)$\,.
\end{fact}


% ------------------ %


\begin{proof}
	Il suffit d'utiliser $N^2 = \dsum_{k=1}^{N} (2 k - 1)$\,.
\end{proof}



% ------------------ %


\newpage
%\medskip
\section{Une démonstration intéressante} \label{case-10}

Dans un échange sur \url{https://math.stackexchange.com} est indiquée une référence vers une preuve du fait $\forall n \in \NNs$\,, $\consprod<10> \notin \NNsquare$ (voir la section \ref{sources}).
Voici cette preuve complétée avec certains arguments laissés sous silence dans la source utilisée.


% ------------------ %


\begin{proof}[Preuve]%
    Supposons que $\consprod<10> \in \NNssquare$\,.
    
    \smallskip
    
    Clairement, 
    $\forall p \in \PP_{\geq 10}$\,, 
    $\forall i \in \ZintervalC{0}{9}$\,, 
    $\padicval{n + i} \in 2 \NN$\,.
    On doit donc s'intéresser à $p \in \setgene{2, 3, 5, 7}$\,. Voici ce que l'on peut observer très grossièrement.
    %
    \begin{itemize}
		\item Au maximum deux facteurs $(n + i)$ de $\consprod<10>$ sont divisibles par $5$\,.

		\item Au maximum deux facteurs $(n + i)$ de $\consprod<10>$ sont divisibles par $7$\,.

		\item Les points précédents donnent au moins $6$ facteurs $(n + i)$ de $\consprod<10>$ non divisibles par $5$ et $7$\,, c'est-à-dire du type $2^\alpha 3^\beta C^2$ avec $(\alpha, \beta, C) \in ( \NNs )^3$.
    \end{itemize}
    
    Nous avons alors l'une des alternatives suivantes pour chacun des $6$ facteurs $(n+i)$ vérifiant $\padicval{n + i} \in 2 \NN$ pour $p \in \PP_{\geq 5}$\,.
    %
    \begin{itemize}
    	\smallskip
		\item \alt{1}\,
		$\big( \padicval[2]{n + i} , \padicval[3]{n + i} \big) \in 2 \NN \times 2 \NN$

    	\smallskip
		\item \alt{2}\,
		$\big( \padicval[2]{n + i} , \padicval[3]{n + i} \big) \in 2 \NN \times \big( 2 \NN + 1)$

    	\smallskip
		\item \alt{3}\,
		$\big( \padicval[2]{n + i} , \padicval[3]{n + i} \big) \in \big( 2 \NN + 1 \big) \times 2 \NN$

    	\smallskip
		\item \alt{4}\,
		$\big( \padicval[2]{n + i} , \padicval[3]{n + i} \big) \in \big( 2 \NN + 1 \big) \times \big( 2 \NN + 1)$
    \end{itemize}
    
    \medskip
    
    Comme nous avons six facteurs pour quatre alternatives, ce bon vieux principe des tiroirs va nous permettre de lever des contradictions.
    %
    \begin{itemize}
    	\medskip
		\item Deux facteurs différents $(n+i)$ et $(n+i^\prime)$ vérifient \alt{1}\,.
		
		\smallskip
		\noindent
		Dans ce cas, $(n+i, n+i^\prime) = (N^2, M^2)$ avec $(N, M) \in \NNs$.
		Par symétrie des rôles, on peut supposer $N > M$\,, de sorte que $N^2 - M^2 \in \ZintervalC{1}{9}$\,. 
		Selon le fait \ref{diff-square-ko}, seuls les cas suivants sont possibles mais ils lèvent tous une contradiction.
		%
		\begin{enumerate}
			\item $N^2 - M^2 = 3$ avec $(N, M) = (2, 1)$ est possible, mais ceci donne $n = 1^2 = 1$\,, puis $\consprod[1]<10> = 10 ! \in \NNsquare$\,, or ceci est faux car $\padicval[7]{10!} = 1$\,.


			\item $N^2 - M^2 = 5$ avec $(N, M) = (3, 2)$ est possible
			d'où $n \in \ZintervalC{1}{4}$\,.
			Nous venons de voir que $n = 1$ est impossible.
			De plus, pour $n \in \ZintervalC{2}{4}$\,, $\padicval[7]{\consprod[n]<10>} = 1$ montre que $\consprod[n]<10> \in \NNsquare$ est faux.
			

			\item $N^2 - M^2 = 7$ avec $(N, M) = (4, 3)$ est possible
			d'où $n \in \ZintervalC{1}{9}$\,, puis $n \in \ZintervalC{5}{9}$ d'après ce qui précède.
			Mais ici, $\forall n \in \ZintervalC{5}{9}$\,, $\padicval[11]{\consprod[n]<10>} = 1$ montre que $\consprod[n]<10> \in \NNsquare$ est faux.


			\item $N^2 - M^2 = 8$ avec $(N, M) = (3, 1)$ est possible
			d'où $n = 1$\,, mais ceci est impossible comme nous l'avons vu ci-dessus.


			\item $N^2 - M^2 = 9$ avec $(N, M) = (5, 4)$ est possible
			d'où $n \in \ZintervalC{9}{16}$\,, puis $n \in \ZintervalC{10}{16}$ d'après ce qui précède.
			Or $\forall n \in \ZintervalC{10}{16}$\,, $\padicval[17]{\consprod[n]<10>} = 1$\,, donc $\consprod[n]<10> \in \NNsquare$ est faux.
		\end{enumerate}


    	\medskip
		\item Deux facteurs différents $(n+i)$ et $(n+i^\prime)$ vérifient \alt{2}\,.
		
		\smallskip
		\noindent
		Dans ce cas, $(n+i, n+i^\prime) = (3 N^2, 3 M^2)$ avec $(N, M) \in \NNs$.
		Par symétrie des rôles, on peut supposer $N > M$\,, de sorte que $3(N^2 - M^2) \in \ZintervalC{1}{9}$\,, puis $N^2 - M^2 \in \ZintervalC{1}{3}$\,. 
		Selon le fait \ref{diff-square-ko}, nécessairement $N^2 - M^2 = 3$ avec $(N, M) = (2, 1)$\,, d'où $n \in \ZintervalC{1}{3}$\,, mais on sait que cela est impossible.


    	\medskip
		\item Deux facteurs différents $(n+i)$ et $(n+i^\prime)$ vérifient \alt{3}\,.
		
		\smallskip
		\noindent
		Dans ce cas, $(n+i, n+i^\prime) = (2 N^2, 2 M^2)$ avec $(N, M) \in \NNs$.
		Par symétrie des rôles, on peut supposer $N > M$\,, de sorte que $2(N^2 - M^2) \in \ZintervalC{1}{9}$\,, puis $N^2 - M^2 \in \ZintervalC{1}{4}$\,. 
		Selon le fait \ref{diff-square-ko}, nécessairement $N^2 - M^2 = 3$ avec $(N, M) = (2, 1)$\,, d'où $n \in \ZintervalC{1}{2}$\,, mais on sait que cela est impossible.


    	\medskip
		\item Deux facteurs différents $(n+i)$ et $(n+i^\prime)$ vérifient \alt{4}\,.
		
		\smallskip
		\noindent
		Dans ce cas, $(n+i, n+i^\prime) = (6 N^2, 6 M^2)$ avec $(N, M) \in \NNs$.
		Par symétrie des rôles, on peut supposer $N > M$\,, de sorte que $6(N^2 - M^2) \in \ZintervalC{1}{9}$\,, puis $N^2 - M^2 = 1$\,, mais c'est impossible d'après le fait \ref{diff-square-ko}.
		%
		\qedhere
    \end{itemize}
\end{proof}


% ------------------ %


Dans le document \enquote{Carrés parfaits et produits d'entiers consécutifs -- Des solutions à la main}\,, nous avons adapté la preuve ci-dessus, avec patience, pour démontrer que $\forall n \in \NNs$\,, $\consprod<k> \notin \NNsquare$ si $k \in \setgene{9, 11, 12, 13}$\,.
Comme ces adaptations sont très mécaniques, et a priori peu gourmandes informatiquement, il semble opportun de tenter un traitement numérique des cas laissés de côté dans l'article de Paul Erdős. 




% ------------------ %


%\newpage
%\medskip
\section{Une tactique informatique}

\subsection{Deux algorithmes basiques} \label{algos-used}

\leavevmode
\smallskip

La démonstration donnée dans la section \ref{case-10} part de $\consprod \in \NNssquare$ par hypothèse, puis elle s'appuie sur deux idées simples que nous allons transformer en algorithme.


\subsubsection{Sélection de potentiels bons candidats} \label{algos-used-select}

\leavevmode
\smallskip

La première phase consiste à tenter de trouver le moins possible de nombres premiers $p$ tel que tous les facteurs $(n+i)$ de $\consprod$ soient de valuation $p$-adique non nécessairement paire. 
Comme $p \in \PP_{\geq k}$ divise au maximum un facteur $(n+i)$ de $\consprod$\,, nous avons $\forall i \in \ZintervalC{0}{k-1}$\,, $\padicval{n+i} \in 2\,\NN$ dès que $p \in \PP_{\geq k}$ puisque  $\consprod \in \NNssquare$ par hypothèse. Ceci permet de cibler notre analyse sur les nombres premiers dans $\PP_{< k}$\,. 


% ------------------ %


\medskip

Voici un premier exemple de sélection avec $\consprod<3>$ en notant que $\PP_{< 3} = \setgene{2}$\,. Nous expliquons juste après comment lire le tableau ci-dessous.


% \vspace{-1ex}
\begin{center}
    \begin{tblr}{
        width = \linewidth,
%        stretch = 1.75,
		colspec = {Q[r]*{2}{Q[c,$]}},
        vline{2-Y},
        hline{2-Y},
        rowsep      = 2pt,
        colsep      = 3pt,
		column{1}   = {6em},
		column{2-Z} = {1.5em},
		% GOOD!
		column{Y} = {green!15},
		% STOP!
		column{Z} = {red!15},
    }
      $p_m\,$
    	&   & 2
    \\
      Occu. max.
		&   & 2
    \\
      Occu. libres.
		& 3 & 1
    \\
      Alternatives.
		& 2^1
		& 0
    \end{tblr}
\end{center}

Le tableau se lit comme suit.
%
\begin{itemize}
	\item $p_m$ désigne le plus grand nombre premier disponible non encore éliminé.

	\item La deuxième ligne indique le nombre maximum de facteurs $(n+i)$ de $\consprod$ pouvant être divisibles par $p_m$\,.

	\item La troisième ligne donne le nombre minimum de facteurs de valuations $p$-adiques nécessairement paires dès que $p \in \PP_{\geq p_m}$\,.

	\item La dernière ligne donne le nombre d'alternatives possibles relativement aux parités des valuations $p$-adiques pour les nombres premiers $p$ dans $\PP_{< p_m}$\,, les autres valuations $p$-adiques restantes étant paires.

	\item La colonne sur fond vert indique le \enquote{meilleur bon} candidat, c'est-à-dire celui avec le moins d'alternatives.
	Nous utiliserons du bleu pour de bons candidats non gardés.

	\item La colonne sur fond rouge indique que l'on ne peut plus avancer (évident ici mais nous verrons que cela peut arriver plus tôt dans l'analyse).
\end{itemize}


Nous voyons ici que $2$ est un bon candidat pour rejeter $\consprod<3> \in \NNssquare$ puisqu'au moins deux facteurs différents $(n+i)$ et $(n+i^\prime)$ de $\consprod<3>$ vérifient la même alternative, d'où $n+i = c M^2$ et $n+i^\prime = c N^2$ avec $(c, N, M) \in \NNsf \times ( \NNs )^2$\,, une information qui sera utilisée par notre second algorithme pour \enquote{localiser}\,, via le fait \ref{diff-square-ko}, des entiers naturels $n$ afin de tester presque brutalement si $\consprod<3> \in \NNssquare$ est vrai, ou non.


% ------------------ %


\medskip

Voici un autre exemple montrant que la sélection peut échouer : il suffit de considérer par exemple $\consprod<4>$ en notant que $\PP_{< 4} = \setgene{2, 3}$\,.

% \vspace{-1ex}
\begin{center}
    \begin{tblr}{
        width = \linewidth,
%        stretch = 1.75,
		colspec = {Q[r]*{3}{Q[c,$]}},
        vline{2-Y},
        hline{2-Y},
        rowsep      = 2pt,
        colsep      = 3pt,
		column{1}   = {6em},
		column{2-Z} = {1.5em},
		% STOP!
		column{Z} = {red!15},
    }
      $p_m\,$
    	&   & 3 & 2
    \\
      Occu. max.
		&   & 2 & 2
    \\
      Occu. libres.
		& 4 & 2 & 0
    \\
      Alternatives.
		& 2^2
		& 2^1
		& 0
    \end{tblr}
\end{center}



% ------------------ %


\medskip
%\newpage

Afin de clarifier la démarche que nous allons suivre, donnons un dernier exemple via $\consprod<37>$ en notant que $\card ( \PP_{< 37} ) = 11$\,.
	 
% \vspace{-1ex}
\begin{center}
    \begin{tblr}{
        width = \linewidth,
%        stretch = 1.75,
        colspec = {X[3,r] *{11}{X[1,c,$]}},
        vline{2-Y},
        hline{2-Y},
        rowsep      = 2pt,
        colsep      = 3pt,
		column{1}   = {6em},
		column{2-Z} = {1.5em},
		% GOOD!
		column{W-X} = {blue!15},
		column{Y}   = {green!15},
		% STOP!
		column{Z} = {red!15},
    }
      $p_m\,$
    	&    & 31 & 29 & 23 & 19 & 17 & 13 & 11 & 7  & 5  & 3
    \\
      Occu. max.
		&    & 2  & 2  & 2  & 2  & 3  & 3  & 4  & 6  & 8  & 13
    \\
      Occu. libres.
		& 37 & 35 & 33 & 31 & 29 & 26 & 23 & 19 & 13 & 5  & 0
    \\
      Alternatives.
		& 2^{11}
		& 2^{10}
		& 2^9
		& 2^8
		& 2^7
		& 2^6
		& 2^5
		& 2^4
		& 2^3
		& 2^2
		& 2
    \end{tblr}
\end{center}


% ------------------ %


\medskip

Nous décidons donc de procéder grosso modo comme suit.

\begin{enumerate}
	\item Nous supposons par l'absurde que $\consprod \in \NNssquare$ avec $k \in \NNs$.


	\item Nous fabriquons $\setgeo{P} = \PP_{<k}$\,.


	\item Nous posons $\setgeo{C} = \emptyset$\,.
	
	\explainthis{Cet ensemble sera celui des nombres premiers \enquote{candidats} utilisés dans notre second algorithme de tests brutaux. Nous cherchons à obtenir l'ensemble $\setgeo{C}$ non vide le plus petit possible.}


	\item Nous posons aussi $occu_{libre} = k$\,.
	
	\explainthis{Cette variable va nous servir à compter les facteurs $(n+i)$ de $\consprod$ ayant un \enquote{maximum} de valuations $p$-adiques pairs.}


	\item \label{algo-select-restart}
	\textbf{Début des actions répétitives.}
	
	\noindent
	Si $\setgeo{P} \neq \emptyset$ et $occu_{libre} > 2^{\card( \setgeo{P} )}$\,, nous posons $\setgeo{C} = \setgeo{P}$\,.
	
	\explainthis{Nous avons $2^{\card ( \setgeo{P} )}$ alternatives \alt{${}_j$} possibles relativement aux parités possible des valuations $p$-adiques pour les nombres premiers $p$ dans $\setgeo{P} = \PP_{< p_m}$\,, les valuations $p$-adiques restantes étant paires. %
	De l'autre côté, nous avons au moins $occu_{libre}$ facteurs $(n+i)$ de $\consprod$ tels que $\padicval{n+i} \in 2\,\NN$ dès que $p \in \PP_{\geq p_m}$\,. %
	Finalement, si $occu_{libre} > 2^{\card ( \setgeo{P} )}$\,, nous avons au moins deux facteurs différents $(n+i)$ et $(n+i^\prime)$ vérifiant la même alternative \alt{${}_j$}\,, d'où $c M^2$ et $c N^2$ avec $(c, N, M) \in \NNsf \times ( \NNs )^2$\,, une information qui sera utilisée par notre second algorithme pour \enquote{localiser} des $\consprod$ à tester brutalement.}


	\item Si $\setgeo{P} = \emptyset$\,, ou $occu_{libre} = 0$\,, nous stoppons tout !
	
	\explainthis{Si $\setgeo{C} = \emptyset$\,, nous avons perdu. Dans le cas contraire, nous pourrons continuer avec l'algorithme qui sera présenté dans la section \ref{algo-kill} suivante.}


	\item Sinon, nous considérons $p_m = \max ( \setgeo{P})$\,, puis retirons $p_m$ de $\setgeo{P}$\,, d'où $\setgeo{P} = \PP_{< p_m}$\,.
	
	\explainthis{Le choix du maximum tente de limiter les rejets de facteurs dans les étapes suivantes.} 


	\item Nous calculons $occu_{max}$ le nombre maximum de facteurs $(n+i)$ de $\consprod$ pouvant être divisés par $p_m$\,.
	
	\explainthis{Le calcul de $occu_{max}$ est simple puisqu'il suffit de considérer le cas où $p$ divise $n$\,, nous obtenons alors $occu_{max} = 1 + \quot{k-1}{p}$ car $\consprod = n (n + 1) \cdots (n + k - 1)$\,.}


	\item $occu_{libre}$ devient $occu_{libre} - occu_{max}$\,.
	
	\explainthis{Maintenant, nous savons qu'au moins $occu_{libre}$ facteurs $(n+i)$ de $\consprod$ vérifient $\padicval{n+i} \in 2\,\NN$ dès que $p \in \PP_{\geq p_m}$\,.}


	\item Nous reprenons les étapes à partir du point \ref{algo-select-restart}.
\end{enumerate}


% ------------------ %


%\medskip
\newpage

Tout ceci nous amène au premier algorithme suivant.

{\small
\begin{algo}[frame] \label{algo-select}
%	\caption{Classique et efficace} 
	%%%
    \Data{$k \in \NN_{\geq 2}$\,, le nombre de facteurs considérés}
    \Result{$\setgeo{C}$ un ensemble, éventuellement vide, de nombres premiers \enquote{candidats} tel que si $\setgeo{C} \neq \emptyset$ alors il existe au moins deux facteurs $(n+i)$ et $(n+i^\prime)$ de $\consprod$ vérifiant $\forall p \in \PP - \setgeo{C}$\,, $( \padicval{n+i}, \padicval{n+i^\prime} ) \in ( 2 \NN )^2$\,, ainsi que $\padicval{n+i}$ et $\padicval{n+i^\prime}$ ont la même parité dès que $p \in \setgeo{C}$\,.}
	\BlankLine
    \Actions{
		$u^{\prime} \Store 1$
		\\
		$u^{\prime\prime} \Store 0$
		\\
    	$v^{\prime} \Store 0$
		\\
		$v^{\prime\prime} \Store 1$
		\\
		\BlankLine
        \While{$b \neq 0$}{
			$a = q b + r$ est la division euclidienne standard.
			\\
			$temp_u \Store u^{\prime} - q u^{\prime\prime}$
			\\
			$u^{\prime} \Store u^{\prime\prime}$
			\\
			$u^{\prime\prime} \Store temp_u$
			\\
			$temp_v \Store v^{\prime} - q v^{\prime\prime}$
			\\
			$v^{\prime} \Store v^{\prime\prime}$
			\\
			$v^{\prime\prime} \Store temp_v$
		}
		\Return{$(u^{\prime} ; v^{\prime})$}
	}
\end{algo}
}


% ------------------ %


\medskip

Une fois l'algorithme \ref{algo-select} traduit en \python, nous obtiennons instantanément les informations suivantes pour $k \in \ZintervalC{2}{100}$\,.
%
\begin{itemize}
	\item \textbf{Mauvais candidats.}
	
	\noindent
	Il y en a 4 correspondant aux entiers $2$\,, $4$\,, $6$ et $8$\,.
	
	\item \textbf{Bons candidats avec un seul nombre premier à gérer.}
	
	\noindent
	Il y en a 2 correspondant aux entiers $3$ et $5$\,.
	
	\item \textbf{Bons candidats avec deux nombres premiers à gérer.}
	
	\noindent
	Il y en a 27 correspondant aux entiers $7$\,, $9$\,, $10$\,, $11$\,, $12$\,, $13$\,, $14$\,, $15$\,, $16$\,, $17$\,, $18$\,, $19$\,, $20$\,, $21$\,, $22$\,, $23$\,, $25$\,, $26$\,, $27$\,, $28$\,, $29$\,, $30$\,, $31$\,, $33$\,, $34$\,, $35$ et $37$\,.

	\item\textbf{Bons candidats avec trois nombres premiers à gérer.}
	
	\noindent
	Il y en a 66 correspondant aux entiers restants.
\end{itemize}


Ce qui précède est encourageant, car peu de cas sont rejetés.
De plus, les mauvais candidats sont faciles à gérer par un humain, ou un programme : voir la section \ref{algo-KO}.
Quant aux candidats acceptés, les nombres premiers à gérer sont forcément dans $\setgene{2, 3, 5}$\,, et le nombre maximum d'alternatives est $2^3 = 8$\,, tout ceci n'étant pas informatiquement bloquant (nous verrons dans la section \ref{algos-used-kill} qu'un autre paramètre peut bloquer la recherche).
 

% ------------------ %


\begin{remark}
	Ne rêvons pas trop à un principe général, car le programme donne aussi que $824$ est le premier naturel, après $8$\,, non sélectionné par notre algorithme.
\end{remark}
\subsubsection{XXX} \label{algo-kill}

\leavevmode
\smallskip

YYYY


\subsection{Les cas gagnants} \label{algo-OK}

\leavevmode
\smallskip

Une fois les algorithmes \ref{algo-square-ko}, \ref{algo-select} et \ref{algo-kill} traduits en \python
\footnote{
	Voir sur le dépôt associé à ce document.
},
nous validons sans effort que $\consprod \notin \NNssquare$ pour $k \in \ZintervalC{2}{100} - \setgene{4, 6, 8}$\,.


\begin{remark}
	Notre tactique informatique a été testée avec succès sur $\ZintervalC{2}{600} - \setgene{4, 6, 8}$\,.
\end{remark}



\subsection{Que faire des cas perdants ?} \label{algo-KO}

\leavevmode
\smallskip

Aussi surprenant que cela puisse paraître, il est très facile de démontrer humainement que $\consprod \notin \NNssquare$ pour $k \in \ZintervalC{2}{6}$ :
se reporter à mon document \emph{\enquote{Carrés parfaits et produits d’entiers consécutifs – Une méthode efficace}} pour savoir comment cela fonctionne
\footnote{
	Le cas $k = 6$ est rédigé à la main dans \emph{\enquote{Carrés parfaits et produits d’entiers consécutifs – Des solutions à la main}}\,, un autre de mes documents.
}.
La méthode citée étant facile à coder, un programme \python, fait sans astuce, démontre instantanément, ou presque, que $\consprod \notin \NNssquare$ pour $k \in \ZintervalC{2}{8}$\,, ce qui achève notre périple informatique.


% ------------------ %


\section{Conclusion}

Nous avons démontré informatiquement que $\consprod \notin \NNssquare$ pour $k \in \ZintervalC{2}{100}$\,. Il ne reste plus qu'à lire, et comprendre pleinement, l'article \emph{\enquote{Note on Products of Consecutive Integers}} de Paul Erdős. Bon courage !



% ------------------ %



% ------------------ %


%\newpage

\section{Sources utilisées} \label{sources}

% ------------------ %


\bigskip
\textbf{Fait \ref{case-4}.}
	
%\smallskip
%\noindent
%Voir la source du fait  \ref{case-7}.

\smallskip
\noindent
\emph{La démonstration non algébrique a été impulsée par la source du fait \ref{case-7} donnée plus bas.}


% ------------------ %


\bigskip
\textbf{Fait \ref{case-5}.}
	
\begin{itemize}
	\item Un échange consulté le 28 janvier 2024, et titré 
	\emph{\enquote{\href{https://les-mathematiques.net/vanilla/discussion/comment/351293}{n(n+1)...(n+k) est un carré ?}}} 
	sur le site \url{lesmathematiques.net}\,.

    \smallskip
    \noindent
    \emph{La démonstration via le principe des tiroirs trouve sa source dans cet échange.}


	\item Un échange consulté le 12 février 2024, et titré 
	\emph{\enquote{\href{https://artisticmathematics.quora.com/Is-there-an-easier-way-of-proving-the-product-of-any-5-consecutive-positive-integers-is-never-a-perfect-square}{Is there an easier way of proving the product of any 5 consecutive positive integers is never a perfect square?}}} 
	sur le site \url{www.quora.com/}\,.

    \smallskip
    \noindent
    \emph{La démonstration \enquote{élémentaire} sans le principe des tiroirs vient de cet échange.}


	\item L'article \emph{\enquote{Le produit de 5 entiers consécutifs n'est pas le carré d'un entier.}} de T. Hayashi, Nouvelles Annales de Mathématiques, est consultable via \href{https://numdam.org}{Numdam}\,, la bibliothèque numérique française de mathématiques.
	
	\smallskip
	\noindent
	\emph{Cet article a fortement inspiré la longue preuve.}
\end{itemize}
\vspace{-1ex}


% ------------------ %


\bigskip
\textbf{Fait \ref{case-6}.}
	
\begin{itemize}
	\item Un échange consulté le 28 janvier 2024, et titré
\emph{\enquote{\href{https://math.stackexchange.com/q/90894/52365}{product of six consecutive integers being a perfect numbers}}} 
sur le site \url{https://math.stackexchange.com}\,.
	
	\smallskip
	\noindent
	\emph{La courte démonstration est donnée dans cet échange. Vous y trouverez aussi un très joli argument basé sur les courbes elliptiques rationnelles.}


	\item Une discussion archivée consultée le 28 janvier 2024 : 
	
	\noindent
	\url{https://web.archive.org/web/20171110144534/http://mathforum.org/library/drmath/view/65589.html}\,.
	
	\smallskip
	\noindent
	\emph{Cette discussion a impulsé la preuve fastidieuse, mais facile d'accès, via des tableaux.}
\end{itemize}
\vspace{-1ex}


% ------------------ %


\bigskip
\textbf{Fait \ref{case-7}.}
	
\smallskip
\noindent
Un échange consulté le 3 février 2024, et titré
\emph{\enquote{\href{https://math.stackexchange.com/q/2334887/52365}{Proof that the product of 7 successive positive integers is not a square}}} 
sur le site \url{https://math.stackexchange.com}\,.
	
\smallskip
\noindent
\emph{La courte démonstration est donnée dans cet échange, mais certaines justifications manquent.}


% ------------------ %


\bigskip
\textbf{Fait \ref{case-8}.}
	
\begin{itemize}
	\item Le document \emph{\enquote{Products of consecutive Integers}} de Vadim Bugaenko, Konstantin Kokhas, Yaroslav Abramov et Maria Ilyukhina obtenu via un moteur de recherche le 28 février 2024.


	\item Un échange consulté le 4 février 2024, et titré \emph{\enquote{\href{https://math.stackexchange.com/a/2271715/52365}{How to prove that the product of eight consecutive numbers can't be a number raised to exponent 4?}}} sur le site \url{https://math.stackexchange.com}\,.

    \smallskip
    \noindent
    \emph{La démonstration astucieuse vient de l'une des réponses de cet échange, mais la justification des deux inégalités n'est pas donnée.}
\end{itemize}
\vspace{-1ex}






\smallskip
\noindent



% ------------------ %


\bigskip
\textbf{Fait \ref{case-10}.}
	
\smallskip
\noindent
Un échange consulté le 13 février 2024, et titré
\emph{\enquote{\href{https://math.stackexchange.com/q/2361670/52365}{Product of 10 consecutive integers can never be a perfect square}}} 
sur le site \url{https://math.stackexchange.com}\,.

\smallskip
\noindent
\emph{La démonstration vient d'une source Wordpress donnée dans une réponse de cet échange, mais cette source est très expéditive...}




% ------------------ %


%\bigskip
\newpage

\hrule

\section{AFFAIRE À SUIVRE...}

\bigskip

\hrule

\end{document}
