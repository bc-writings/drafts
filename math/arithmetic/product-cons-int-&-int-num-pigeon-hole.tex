 \documentclass[12pt]{amsart}
%\usepackage[T1]{fontenc}
%\usepackage[utf8]{inputenc}

\usepackage[top=1.95cm, bottom=1.95cm, left=2.35cm, right=2.35cm]{geometry}

\usepackage%[hidelinks]%
           {hyperref}  
\usepackage{pgfpages}
%\pgfpagesuselayout{2 on 1}[a4paper,landscape,border shrink=5mm]

\usepackage{hyperref}
\usepackage{enumitem}
\usepackage{tcolorbox}
\usepackage{float}
\usepackage{cleveref}
\usepackage{multicol}
\usepackage{fancyvrb}
\usepackage{enumitem}
\usepackage{amsmath}
\usepackage{textcomp}
\usepackage{numprint}
\usepackage{tabularray}
\usepackage[french]{babel}
\frenchsetup{StandardItemLabels=true}
\usepackage{csquotes}
\usepackage{piton}

\NewPitonEnvironment{Python}{}
  {\begin{tcolorbox}}
  {\end{tcolorbox}}
  
\SetPitonStyle{
 	Number = ,
    String = \itshape ,
    String.Doc = \color{gray} \slshape ,
    Operator = ,
    Operator.Word = \bfseries ,
    Name.Builtin = ,
    Name.Function = ,
    Comment = \color{gray} ,
    Comment.LaTeX = \normalfont \color{gray},
    Keyword = \bfseries ,
    Name.Namespace = ,
    Name.Class = ,
    Name.Type = ,
    InitialValues = \color{gray}
}

\usepackage[
    type={CC},
    modifier={by-nc-sa},
	version={4.0},
]{doclicense}

\newcommand\floor[1]{\left\lfloor #1 \right\rfloor}

\usepackage{tnsmath}


\newtheorem{fact}{Fait}[section]
\newtheorem{defi}{Définition}[section]
\newtheorem{example}{Exemple}[section]
\newtheorem{remark}{Remarque}[section]

\npthousandsep{.}
\setlength\parindent{0pt}

\floatstyle{boxed} 
\restylefloat{figure}


\DeclareMathOperator{\taille}{\text{\normalfont\texttt{taille}}}

\newcommand{\logicneg}{\text{\normalfont non \!}}

\newcommand\sqseq[2]{\fbox{$#1$}_{\,\,#2}}


\DefineVerbatimEnvironment{rawcode}%
	{Verbatim}%
	{tabsize=4,%
	 frame=lines, framerule=0.3mm, framesep=2.5mm}
	 
	 
\newcommand\contentdir{\jobname}

\newcommand\NNsf{\NN_{\kern-1pt s\kern-1pt f}}

\newcommand\NNsquare{\seqsuprageo{\NN}{}{}{}{2}}
\newcommand\NNssquare{\seqsuprageo{\NN}{*}{}{}{2}}

\NewDocumentCommand\GCD{ m  m }{#1 \wedge #2}

\NewDocumentCommand\padicval{ O{p} m }{v_{#1}(#2)}
\NewDocumentCommand\consprod{ O{n} D<>{k} }{\pi_{#1}^{#2}}

\NewDocumentCommand\alt{ m }{\textbf{[A\kern1pt#1]}}

\newcommand\mycheckmark{{\color{green!60!black} \checkmark}}
\newcommand\myboxtimes{{\color{red!80!black} \boxtimes}}



\begin{document}

\title{BROUILLON - Carrés parfaits et produits d'entiers consécutifs -- Jusqu'à 100 facteurs traités numériquement}
\author{Christophe BAL}
\date{14 Fév. 2024}

\maketitle

\begin{center}
	\itshape
	Document, avec son source \LaTeX, disponible sur la page
	
	\url{https://github.com/bc-writing/drafts}.
\end{center}


\bigskip


\begin{center}
	\hrule\vspace{.3em}
	{
		\fontsize{1.35em}{1em}\selectfont
		\textbf{Mentions \enquote{légales}}
	}
			
	\vspace{0.45em}
	\small
	\doclicenseThis
	\hrule
\end{center}


\setcounter{tocdepth}{2}
\tableofcontents


% ------------------ %


\newpage
\section{Ce qui nous intéresse}

Dans l'article \enquote{Note on Products of Consecutive Integers}
\footnote{
	J. London Math. Soc. 14 (1939).
},
Paul Erdos démontre que pour tout couple $(n, k) \in \NNs \times \NNs$\,, le produit de $(k+1)$ entiers consécutifs $n (n + 1) \cdots (n + k)$ n'est jamais le carré d'un entier. 

\smallskip

Il est facile de trouver sur le web des preuves à la main de $n(n+1) \cdots (n + k) \notin \NNssquare$ pour $k \in \ZintervalC{1}{7}$\,.
Bien que certaines de ces preuves soient très sympathiques, leur lecture ne fait pas ressortir de schéma commun de raisonnement.
%
Dans ce document, nous allons tenter de limiter au maximum l'emploi de fourberies déductives en présentant une méthode très élémentaire
\footnote{
	Cette méthode s'appuie sur une représentation trouvée dans \href{https://web.archive.org/web/20171110144534/http://mathforum.org/library/drmath/view/65589.html}{un message archivé} : voir la section \ref{sources}.
},
efficace, et semi-automatisable, pour démontrer, avec peu d'efforts cognitifs, les premiers cas d'impossibilité.




% ------------------ %


\bigskip
\section{Notations utilisées}

Dans la suite, nous utiliserons les notations suivantes.
\begin{itemize}
	\item $2\,\NN$ désigne l'ensemble des nombres naturels pairs.
	
	\item $2\,\NN + 1$ désigne l'ensemble des nombres naturels impairs.
	
	\item $\forall (n , m) \in \NN^2$, $n \vee m$ désigne le PPCM de $n$ et $m$.

	\item $\forall (n , m) \in \NN^2$, $n \wedge m$ désigne le PGCD de $n$ et $m$.

	\item $a \strictdivides b$ signifie que $a \divides b$ et $a \neq b$ (division stricte).

	\item $\PP$ désigne l'ensemble des nombres premiers.
	
	\item $\forall (p ; n) \in \PP \times \NNs$\,, $\padicval{n} \in \NN$ est la valuation $p$-adique de $n$\,, c'est-à-dire $p^{\padicval{n}} \divides n$\,, mais $p^{\padicval{n} + 1} \ndivides n$\,.
\end{itemize}


% ------------------ %


%\newpage
%\medskip
\section{Les carrés parfaits} \label{case-1}

\leavevmode
\smallskip

Via $N^2 - M^2 = (N - M)(N + M)$\,, il est immédiat de noter que 
$\forall (N, M) \in \NNs \times \NNs$\,, si $N > M$\,, alors $N^2 - M^2 \geq 3$\,. Le fait suivant précise ceci.


\begin{fact} \label{dist-square}
	$\forall (N, M) \in \NNs \times \NNs$, 
	si $N > M$\,, alors $N^2 - M^2 = \dsum_{k=M+1}^{N} (2 k - 1)$\,.
\end{fact}


% ------------------ %


\begin{proof}
	Il suffit d'utiliser $N^2 = \dsum_{k=1}^{N} (2 k - 1)$\,.
\end{proof}



% ------------------ %


%\newpage
%\medskip
\section{Une démonstration intéressante} \label{case-10}

Dans un échange sur \url{https://math.stackexchange.com} est indiquée une référence vers une preuve du fait $\forall n \in \NNs$\,, $\consprod<10> \notin \NNsquare$ (voir la section \ref{sources}).
Voici cette preuve complétée avec certains arguments laissés sous silence dans la source utilisée.


% ------------------ %


\begin{proof}[Preuve]%
    Supposons que $\consprod<10> \in \NNssquare$\,.
    
    \smallskip
    
    Clairement, 
    $\forall p \in \PP_{\geq 10}$\,, 
    $\forall i \in \ZintervalC{0}{9}$\,, 
    $\padicval{n + i} \in 2 \NN$\,.
    On doit donc s'intéresser à $p \in \setgene{2, 3, 5, 7}$\,. Voici ce que l'on peut observer très grossièrement.
    %
    \begin{itemize}
		\item Au maximum deux facteurs $(n + i)$ de $\consprod<10>$ sont divisibles par $5$\,.

		\item Au maximum deux facteurs $(n + i)$ de $\consprod<10>$ sont divisibles par $7$\,.

		\item Les points précédents donnent au moins $6$ facteurs $(n + i)$ de $\consprod<10>$ non divisibles par $5$ et $7$\,, c'est-à-dire du type $2^\alpha 3^\beta C^2$ avec $(\alpha, \beta, C) \in ( \NNs )^3$.
    \end{itemize}
    
    Nous avons alors l'une des alternatives suivantes pour chacun des $6$ facteurs $(n+i)$ vérifiant $\padicval{n + i} \in 2 \NN$ pour $p \in \PP_{\geq 5}$\,.
    %
    \begin{itemize}
    	\smallskip
		\item \alt{1}\,
		$\big( \padicval[2]{n + i} , \padicval[3]{n + i} \big) \in 2 \NN \times 2 \NN$

    	\smallskip
		\item \alt{2}\,
		$\big( \padicval[2]{n + i} , \padicval[3]{n + i} \big) \in 2 \NN \times \big( 2 \NN + 1)$

    	\smallskip
		\item \alt{3}\,
		$\big( \padicval[2]{n + i} , \padicval[3]{n + i} \big) \in \big( 2 \NN + 1 \big) \times 2 \NN$

    	\smallskip
		\item \alt{4}\,
		$\big( \padicval[2]{n + i} , \padicval[3]{n + i} \big) \in \big( 2 \NN + 1 \big) \times \big( 2 \NN + 1)$
    \end{itemize}
    
    \medskip
    
    Comme nous avons six facteurs pour quatre alternatives, ce bon vieux principe des tiroirs va nous permettre de lever des contradictions.
    %
    \begin{itemize}
    	\medskip
		\item Deux facteurs différents $(n+i)$ et $(n+i^\prime)$ vérifient \alt{1}\,.
		
		\smallskip
		\noindent
		Dans ce cas, $(n+i, n+i^\prime) = (N^2, M^2)$ avec $(N, M) \in \NNs$.
		Par symétrie des rôles, on peut supposer $N > M$\,, de sorte que $N^2 - M^2 \in \ZintervalC{1}{9}$\,. 
		Selon le fait \ref{diff-square-ko}, seuls les cas suivants sont possibles mais ils lèvent tous une contradiction.
		%
		\begin{enumerate}
			\item $N^2 - M^2 = 3$ avec $(N, M) = (2, 1)$ est possible, mais ceci donne $n = 1^2 = 1$\,, puis $\consprod[1]<10> = 10 ! \in \NNsquare$\,, or ceci est faux car $\padicval[7]{10!} = 1$\,.


			\item $N^2 - M^2 = 5$ avec $(N, M) = (3, 2)$ est possible
			d'où $n \in \ZintervalC{1}{4}$\,.
			Nous venons de voir que $n = 1$ est impossible.
			De plus, pour $n \in \ZintervalC{2}{4}$\,, $\padicval[7]{\consprod[n]<10>} = 1$ montre que $\consprod[n]<10> \in \NNsquare$ est faux.
			

			\item $N^2 - M^2 = 7$ avec $(N, M) = (4, 3)$ est possible
			d'où $n \in \ZintervalC{1}{9}$\,, puis $n \in \ZintervalC{5}{9}$ d'après ce qui précède.
			Mais ici, $\forall n \in \ZintervalC{5}{9}$\,, $\padicval[11]{\consprod[n]<10>} = 1$ montre que $\consprod[n]<10> \in \NNsquare$ est faux.


			\item $N^2 - M^2 = 8$ avec $(N, M) = (3, 1)$ est possible
			d'où $n = 1$\,, mais ceci est impossible comme nous l'avons vu ci-dessus.


			\item $N^2 - M^2 = 9$ avec $(N, M) = (5, 4)$ est possible
			d'où $n \in \ZintervalC{9}{16}$\,, puis $n \in \ZintervalC{10}{16}$ d'après ce qui précède.
			Or $\forall n \in \ZintervalC{10}{16}$\,, $\padicval[17]{\consprod[n]<10>} = 1$\,, donc $\consprod[n]<10> \in \NNsquare$ est faux.
		\end{enumerate}


    	\medskip
		\item Deux facteurs différents $(n+i)$ et $(n+i^\prime)$ vérifient \alt{2}\,.
		
		\smallskip
		\noindent
		Dans ce cas, $(n+i, n+i^\prime) = (3 N^2, 3 M^2)$ avec $(N, M) \in \NNs$.
		Par symétrie des rôles, on peut supposer $N > M$\,, de sorte que $3(N^2 - M^2) \in \ZintervalC{1}{9}$\,, puis $N^2 - M^2 \in \ZintervalC{1}{3}$\,. 
		Selon le fait \ref{diff-square-ko}, nécessairement $N^2 - M^2 = 3$ avec $(N, M) = (2, 1)$\,, d'où $n \in \ZintervalC{1}{3}$\,, mais on sait que cela est impossible.


    	\medskip
		\item Deux facteurs différents $(n+i)$ et $(n+i^\prime)$ vérifient \alt{3}\,.
		
		\smallskip
		\noindent
		Dans ce cas, $(n+i, n+i^\prime) = (2 N^2, 2 M^2)$ avec $(N, M) \in \NNs$.
		Par symétrie des rôles, on peut supposer $N > M$\,, de sorte que $2(N^2 - M^2) \in \ZintervalC{1}{9}$\,, puis $N^2 - M^2 \in \ZintervalC{1}{4}$\,. 
		Selon le fait \ref{diff-square-ko}, nécessairement $N^2 - M^2 = 3$ avec $(N, M) = (2, 1)$\,, d'où $n \in \ZintervalC{1}{2}$\,, mais on sait que cela est impossible.


    	\medskip
		\item Deux facteurs différents $(n+i)$ et $(n+i^\prime)$ vérifient \alt{4}\,.
		
		\smallskip
		\noindent
		Dans ce cas, $(n+i, n+i^\prime) = (6 N^2, 6 M^2)$ avec $(N, M) \in \NNs$.
		Par symétrie des rôles, on peut supposer $N > M$\,, de sorte que $6(N^2 - M^2) \in \ZintervalC{1}{9}$\,, puis $N^2 - M^2 = 1$\,, mais c'est impossible d'après le fait \ref{diff-square-ko}.
		%
		\qedhere
    \end{itemize}
\end{proof}


% ------------------ %


Dans le document \enquote{Carrés parfaits et produits d'entiers consécutifs -- Des solutions à la main}\,, nous avons adapté la preuve ci-dessus, avec patience, pour démontrer que $\forall n \in \NNs$\,, $\consprod<k> \notin \NNsquare$ si $k \in \setgene{9, 11, 12, 13}$\,.
Comme ces adaptations sont très mécaniques, et a priori peu gourmandes informatiquement, il semble opportun de tenter un traitement numérique des cas laissés de côté dans l'article de Paul Erdős. 




% ------------------ %


%\newpage
%\medskip
\section{Une tactique informatique}

\subsection{De potentiels bons candidats} \label{algo-select}

\leavevmode
\smallskip

????


La démonstration donnée dans la section \ref{case-10} commence par repérer le moins possible de nombres premiers $p$ de valuation -p--adiques non nécessaireemnt paires 



suit les grandes lignes suivantes qui mênent directement à un algorithme.
%
\begin{enumerate}
	\item XXXX

	\item XXXX

	\item XXXX
\end{enumerate}





\newpage
Idée : on compte grossièrement des nombres de valuation p adiques paires avec le plus de p différents, dison en avoir k, il reste alors d priemirs et on veut juste que d > 2**d, on fait en sorte d'aller vers d le plus petit possible.

une fois ceci faitn on se ramène à des eq du type $N^2 - M^2 = d$ ou d w= nbre de facteurs


De 2 à 100


5 cas KO : 
2, 3, 4, 6, 8


1 cas avec 1 premier gênant :
5


27 cas avec 2 premier gênants :
7, 9, 10, 11, 12, 13, 14, 15, 16, 17, 18, 19, 20, 21, 22, 23, 25, 26, 27, 28, 29, 30, 31, 33, 34, 35, 37


66 cas avec 3 premier gênants (tous les autres)


par exemple


pour 5 on a 2 nbrs premires
	 

pour 37 on a 11 nbrs premires
	 
\vspace{-1ex}
\begin{center}
    \begin{tblr}{
        width = \linewidth,
        stretch = 1.75,
        colspec = {X[3,r] *{10}{X[1,c,$]}},
        vline{2-Y},
        hline{2-Y},
        rowsep=2pt,
        colsep=3pt,
		% GOOD!
		column{W-X} = {blue!15},
		column{Y}   = {green!15},
		% STOP!
		column{Z} = {red!15},
    }
      $p\,$
    	& 31 & 29 & 23 & 19 & 17 & 13 & 11 & 7  & 5  & 3
    \\
      Occu. max.
		& 2  & 2  & 2  & 2  & 3  & 3  & 4  & 6  & 8  & 13
    \\
      Occu. libres.
		& 35 & 33 & 31 & 29 & 26 & 23 & 19 & 13 & 5  & 0
    \\
      Alternatives.
		& 2^{10}
		& 2^9
		& 2^8
		& 2^7
		& 2^6
		& 2^5
		& 2^4
		& 2^3
		& 2^2
		& 2
    \end{tblr}
\end{center}



pour 40

\vspace{-1ex}
\begin{center}
    \begin{tblr}{
        width = \linewidth,
        stretch = 1.75,
        colspec = {X[3,r] *{11}{X[1,c,$]}},
        vline{2-Y},
        hline{2-Y},
        rowsep=2pt,
        colsep=3pt,
		% GOOD!
		column{W} = {blue!15},
		column{X} = {green!15},
		% STOP!
		column{Z} = {red!15},
    }
      $p\,$
    	& 37 & 31 & 29 & 23 & 19 & 17 & 13 & 11 & 7  & 5  & 3
    \\
      Occu. max.
		& 2  & 2  & 2  & 2  & 3  & 3  & 4  & 4  & 6  & 8  & 14
    \\
      Occu. libres.
		& 38 & 36 & 34 & 32 & 29 & 26 & 22 & 18 & 12 & 4  & 0
    \\
      Alternatives.
		& 2^{11}
		& 2^{10}
		& 2^9
		& 2^8
		& 2^7
		& 2^6
		& 2^5
		& 2^4
		& 2^3
		& 2^2
		& 2
    \end{tblr}
\end{center}


ne pas rêver : 
824
          FAILS!


\subsection{Les cas gagnants} \label{algo-OK}

\leavevmode
\smallskip

Une fois les algorithmes \ref{algo-square-ko}, \ref{algo-select} et \ref{algo-kill} traduits en \python
\footnote{
	Voir sur le dépôt associé à ce document.
},
nous validons sans effort que $\consprod \notin \NNssquare$ pour $k \in \ZintervalC{2}{100} - \setgene{4, 6, 8}$\,.


\begin{remark}
	Notre tactique informatique a été testée avec succès sur $\ZintervalC{2}{600} - \setgene{4, 6, 8}$\,.
\end{remark}



\subsection{Que faire des cas perdants ?} \label{algo-KO}

\leavevmode
\smallskip

Aussi surprenant que cela puisse paraître, il est très facile de démontrer humainement que $\consprod \notin \NNssquare$ pour $k \in \ZintervalC{2}{6}$ :
se reporter à mon document \emph{\enquote{Carrés parfaits et produits d’entiers consécutifs – Une méthode efficace}} pour savoir comment cela fonctionne
\footnote{
	Le cas $k = 6$ est rédigé à la main dans \emph{\enquote{Carrés parfaits et produits d’entiers consécutifs – Des solutions à la main}}\,, un autre de mes documents.
}.
La méthode citée étant facile à coder, un programme \python, fait sans astuce, démontre instantanément, ou presque, que $\consprod \notin \NNssquare$ pour $k \in \ZintervalC{2}{8}$\,, ce qui achève notre périple informatique.


% ------------------ %



% ------------------ %


%\newpage

\section{Sources utilisées} \label{sources}

% ------------------ %


\bigskip
\textbf{Fait \ref{case-4}.}
	
%\smallskip
%\noindent
%Voir la source du fait  \ref{case-7}.

\smallskip
\noindent
\emph{La démonstration non algébrique a été impulsée par la source du fait \ref{case-7} donnée plus bas.}


% ------------------ %


\bigskip
\textbf{Fait \ref{case-5}.}
	
\begin{itemize}
	\item Un échange consulté le 28 janvier 2024, et titré 
	\emph{\enquote{\href{https://les-mathematiques.net/vanilla/discussion/comment/351293}{n(n+1)...(n+k) est un carré ?}}} 
	sur le site \url{lesmathematiques.net}\,.

    \smallskip
    \noindent
    \emph{La démonstration via le principe des tiroirs trouve sa source dans cet échange.}


	\item Un échange consulté le 12 février 2024, et titré 
	\emph{\enquote{\href{https://artisticmathematics.quora.com/Is-there-an-easier-way-of-proving-the-product-of-any-5-consecutive-positive-integers-is-never-a-perfect-square}{Is there an easier way of proving the product of any 5 consecutive positive integers is never a perfect square?}}} 
	sur le site \url{www.quora.com/}\,.

    \smallskip
    \noindent
    \emph{La démonstration \enquote{élémentaire} sans le principe des tiroirs vient de cet échange.}


	\item L'article \emph{\enquote{Le produit de 5 entiers consécutifs n'est pas le carré d'un entier.}} de T. Hayashi, Nouvelles Annales de Mathématiques, est consultable via \href{https://numdam.org}{Numdam}\,, la bibliothèque numérique française de mathématiques.
	
	\smallskip
	\noindent
	\emph{Cet article a fortement inspiré la longue preuve.}
\end{itemize}
\vspace{-1ex}


% ------------------ %


\bigskip
\textbf{Fait \ref{case-6}.}
	
\begin{itemize}
	\item Un échange consulté le 28 janvier 2024, et titré
\emph{\enquote{\href{https://math.stackexchange.com/q/90894/52365}{product of six consecutive integers being a perfect numbers}}} 
sur le site \url{https://math.stackexchange.com}\,.
	
	\smallskip
	\noindent
	\emph{La courte démonstration est donnée dans cet échange. Vous y trouverez aussi un très joli argument basé sur les courbes elliptiques rationnelles.}


	\item Une discussion archivée consultée le 28 janvier 2024 : 
	
	\noindent
	\url{https://web.archive.org/web/20171110144534/http://mathforum.org/library/drmath/view/65589.html}\,.
	
	\smallskip
	\noindent
	\emph{Cette discussion a impulsé la preuve fastidieuse, mais facile d'accès, via des tableaux.}
\end{itemize}
\vspace{-1ex}


% ------------------ %


\bigskip
\textbf{Fait \ref{case-7}.}
	
\smallskip
\noindent
Un échange consulté le 3 février 2024, et titré
\emph{\enquote{\href{https://math.stackexchange.com/q/2334887/52365}{Proof that the product of 7 successive positive integers is not a square}}} 
sur le site \url{https://math.stackexchange.com}\,.
	
\smallskip
\noindent
\emph{La courte démonstration est donnée dans cet échange, mais certaines justifications manquent.}


% ------------------ %


\bigskip
\textbf{Fait \ref{case-8}.}
	
\begin{itemize}
	\item Le document \emph{\enquote{Products of consecutive Integers}} de Vadim Bugaenko, Konstantin Kokhas, Yaroslav Abramov et Maria Ilyukhina obtenu via un moteur de recherche le 28 février 2024.


	\item Un échange consulté le 4 février 2024, et titré \emph{\enquote{\href{https://math.stackexchange.com/a/2271715/52365}{How to prove that the product of eight consecutive numbers can't be a number raised to exponent 4?}}} sur le site \url{https://math.stackexchange.com}\,.

    \smallskip
    \noindent
    \emph{La démonstration astucieuse vient de l'une des réponses de cet échange, mais la justification des deux inégalités n'est pas donnée.}
\end{itemize}
\vspace{-1ex}






\smallskip
\noindent



% ------------------ %


\bigskip
\textbf{Fait \ref{case-10}.}
	
\smallskip
\noindent
Un échange consulté le 13 février 2024, et titré
\emph{\enquote{\href{https://math.stackexchange.com/q/2361670/52365}{Product of 10 consecutive integers can never be a perfect square}}} 
sur le site \url{https://math.stackexchange.com}\,.

\smallskip
\noindent
\emph{La démonstration vient d'une source Wordpress donnée dans une réponse de cet échange, mais cette source est très expéditive...}




% ------------------ %


%\bigskip
\newpage

\hrule

\section{AFFAIRE À SUIVRE...}

\bigskip

\hrule

\end{document}
