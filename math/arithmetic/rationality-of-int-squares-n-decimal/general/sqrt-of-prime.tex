Le lecteur attentif aura noté que tous nos exemples ont porté sur des nombres premiers.
En fait, nous avons le fait suivant qui se démontre aisément avec la décomposition d'un entier en facteurs premiers, une preuve facile à trouver sur le web
\footnote{
	Cette preuve ne nécessite absolument pas l'unicité d'une telle décomposition, une unicité qui n'est pas immédiate à prouver proprement.
}.


\begin{fact}  \label{sqrt-p-not-in-Q}
	Pour tout nombre premier $p$, $\sqrt{p} \not\in \QQ$ .
\end{fact}


Nous allons essayer de voir si cette information n'est pas \emph{\og cachée \fg} juste dans les derniers chiffres de certaines écritures décimales.
Par exemple, pour $2$ , $3$ et $7$ , nous avons juste raisonné sur des chiffres des unités, tandis que pour $5$ et $11$, nous avons dû considérer les deux chiffres les moins significatifs, ceux à droite.
Une question vient naturellement : \emph{\og Peut-on généraliser les méthodes vues précédemment ? \fg}.