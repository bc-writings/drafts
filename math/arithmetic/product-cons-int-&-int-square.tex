% Source.
%    + "Le produit de 5 entiers consécutifs n'est pas le carré d'un entier." de T. Hayashi [Nouvelles Annales de Mathématiques]
%    + Discussion du forum lesmathematiques.net
%    + math/arithmetic/prod-cons-carre-entier.pdf

\documentclass[12pt]{amsart}
\usepackage[T1]{fontenc}
\usepackage[utf8]{inputenc}

\usepackage[top=1.95cm, bottom=1.95cm, left=2.35cm, right=2.35cm]{geometry}

\usepackage{hyperref}
\usepackage{enumitem}
\usepackage{tcolorbox}
\usepackage{float}
\usepackage{cleveref}
\usepackage{multicol}
\usepackage{fancyvrb}
\usepackage{enumitem}
\usepackage{amsmath}
\usepackage{textcomp}
\usepackage{numprint}
\usepackage{tabularray}
\usepackage[french]{babel}
\frenchsetup{StandardItemLabels=true}
\usepackage{csquotes}

\usepackage[
    type={CC},
    modifier={by-nc-sa},
	version={4.0},
]{doclicense}

\newcommand\floor[1]{\left\lfloor #1 \right\rfloor}

\usepackage{tnsmath}


\newtheorem{fact}{Fait}[section]
\newtheorem{example}{Exemple}[section]
\newtheorem{remark}{Remarque}[section]
\newtheorem*{proof*}{Une preuve alternative}

\npthousandsep{.}
\setlength\parindent{0pt}

\floatstyle{boxed} 
\restylefloat{figure}


\DeclareMathOperator{\taille}{\text{\normalfont\texttt{taille}}}

\newcommand{\logicneg}{\text{\normalfont non \!}}

\newcommand\sqseq[2]{\fbox{$#1$}_{\,\,#2}}


\DefineVerbatimEnvironment{rawcode}%
	{Verbatim}%
	{tabsize=4,%
	 frame=lines, framerule=0.3mm, framesep=2.5mm}
	 
	 
\newcommand\contentdir{\jobname}
\newcommand\NNsquare{\seqsuprageo{\NN}{}{}{}{2}}
\newcommand\NNssquare{\seqsuprageo{\NN}{}{*}{}{2}}
\NewDocumentCommand\padicval{ O{p} m }{v_{#1}(#2)}
\NewDocumentCommand\consprod{ O{n} D<>{k} }{\pi_{#1}^{#2}}

\NewDocumentCommand\alt{ m }{\textbf{[A\kern1pt#1]}}

\newcommand\mycheckmark{{\color{green!60!black} \checkmark}}
\newcommand\myboxtimes{{\color{red!80!black} \boxtimes}}



\begin{document}

\title{BROUILLON - Carrés parfaits et produits d'entiers consécutifs}
\author{Christophe BAL}
\date{25 Jan. 2024 -- 28 Jan. 2024}

\maketitle

\begin{center}
	\itshape
	Document, avec son source \LaTeX, disponible sur la page
	
	\url{https://github.com/bc-writing/drafts}.
\end{center}


\bigskip


\begin{center}
	\hrule\vspace{.3em}
	{
		\fontsize{1.35em}{1em}\selectfont
		\textbf{Mentions \og légales \fg}
	}
			
	\vspace{0.45em}
	\doclicenseThis
	\hrule
\end{center}


\setcounter{tocdepth}{2}
\tableofcontents


\newpage
\section{Ce qui nous intéresse}

Dans l'article \enquote{Note on Products of Consecutive Integers}
\footnote{
	J. London Math. Soc. 14 (1939).
},
Paul Erdos démontre que pour tout couple $(n, k) \in \NNs \times \NNs$\,, le produit de $(k+1)$ entiers consécutifs $n (n + 1) \cdots (n + k)$ n'est jamais le carré d'un entier. 

\smallskip

Il est facile de trouver sur le web des preuves à la main de $n(n+1) \cdots (n + k) \notin \NNssquare$ pour $k \in \ZintervalC{1}{7}$\,.
Bien que certaines de ces preuves soient très sympathiques, leur lecture ne fait pas ressortir de schéma commun de raisonnement.
%
Dans ce document, nous allons tenter de limiter au maximum l'emploi de fourberies déductives en présentant une méthode très élémentaire
\footnote{
	Cette méthode s'appuie sur une représentation trouvée dans \href{https://web.archive.org/web/20171110144534/http://mathforum.org/library/drmath/view/65589.html}{un message archivé} : voir la section \ref{sources}.
},
efficace, et semi-automatisable, pour démontrer, avec peu d'efforts cognitifs, les premiers cas d'impossibilité.




\section{Notations utilisées}

Dans la suite, nous utiliserons les notations suivantes.
\begin{itemize}
	\item $2\,\NN$ désigne l'ensemble des nombres naturels pairs.
	
	\item $2\,\NN + 1$ désigne l'ensemble des nombres naturels impairs.
	
	\item $\forall (n , m) \in \NN^2$, $n \vee m$ désigne le PPCM de $n$ et $m$.

	\item $\forall (n , m) \in \NN^2$, $n \wedge m$ désigne le PGCD de $n$ et $m$.

	\item $a \strictdivides b$ signifie que $a \divides b$ et $a \neq b$ (division stricte).

	\item $\PP$ désigne l'ensemble des nombres premiers.
	
	\item $\forall (p ; n) \in \PP \times \NNs$\,, $\padicval{n} \in \NN$ est la valuation $p$-adique de $n$\,, c'est-à-dire $p^{\padicval{n}} \divides n$\,, mais $p^{\padicval{n} + 1} \ndivides n$\,.
\end{itemize}


\foreach \k in {1,...,5} {
	\ifthenelse{\k = 1}{
		\section{Avec \k\ seul facteur}
	}{
		\section{Avec \k\ facteurs}
	}

	\input{\contentdir/case-\k}
}



%\section{Et après ?}
%
%\input{\contentdir/case-gene}



%\bigskip
\newpage

\hrule

\section{AFFAIRE À SUIVRE...}

\bigskip

\hrule

\end{document}
