 \documentclass[12pt]{amsart}
%\usepackage[T1]{fontenc}
%\usepackage[utf8]{inputenc}

\usepackage[top=1.95cm, bottom=1.95cm, left=2.35cm, right=2.35cm]{geometry}

\usepackage%[hidelinks]%
           {hyperref}  
\usepackage{pgfpages}
%\pgfpagesuselayout{2 on 1}[a4paper,landscape,border shrink=5mm]

\usepackage{hyperref}
\usepackage{enumitem}
\usepackage{tcolorbox}
\usepackage{float}
\usepackage{cleveref}
\usepackage{multicol}
\usepackage{fancyvrb}
\usepackage{enumitem}
\usepackage{amsmath}
\usepackage{textcomp}
\usepackage{numprint}
\usepackage{tabularray}
\usepackage[french]{babel}
\frenchsetup{StandardItemLabels=true}
\usepackage{csquotes}
\usepackage{piton}

\NewPitonEnvironment{Python}{}
  {\begin{tcolorbox}}
  {\end{tcolorbox}}
  
\SetPitonStyle{
 	Number = ,
    String = \itshape ,
    String.Doc = \color{gray} \slshape ,
    Operator = ,
    Operator.Word = \bfseries ,
    Name.Builtin = ,
    Name.Function = ,
    Comment = \color{gray} ,
    Comment.LaTeX = \normalfont \color{gray},
    Keyword = \bfseries ,
    Name.Namespace = ,
    Name.Class = ,
    Name.Type = ,
    InitialValues = \color{gray}
}

\usepackage[
    type={CC},
    modifier={by-nc-sa},
	version={4.0},
]{doclicense}

\newcommand\floor[1]{\left\lfloor #1 \right\rfloor}

\usepackage{tnsmath}


\newtheorem{fact}{Fait}[section]
\newtheorem{defi}{Définition}[section]
\newtheorem{example}{Exemple}[section]
\newtheorem{remark}{Remarque}[section]

\npthousandsep{.}
\setlength\parindent{0pt}

\floatstyle{boxed} 
\restylefloat{figure}


\DeclareMathOperator{\taille}{\text{\normalfont\texttt{taille}}}

\newcommand{\logicneg}{\text{\normalfont non \!}}

\newcommand\sqseq[2]{\fbox{$#1$}_{\,\,#2}}


\DefineVerbatimEnvironment{rawcode}%
	{Verbatim}%
	{tabsize=4,%
	 frame=lines, framerule=0.3mm, framesep=2.5mm}
	 
	 
\newcommand\contentdir{\jobname}

\newcommand\NNsf{\NN_{\kern-1pt s\kern-1pt f}}

\newcommand\NNsquare{\seqsuprageo{\NN}{}{}{}{2}}
\newcommand\NNssquare{\seqsuprageo{\NN}{*}{}{}{2}}

\NewDocumentCommand\GCD{ m  m }{#1 \wedge #2}

\NewDocumentCommand\padicval{ O{p} m }{v_{#1}(#2)}
\NewDocumentCommand\consprod{ O{n} D<>{k} }{\pi_{#1}^{#2}}

\NewDocumentCommand\alt{ m }{\textbf{[A\kern1pt#1]}}

\newcommand\mycheckmark{{\color{green!60!black} \checkmark}}
\newcommand\myboxtimes{{\color{red!80!black} \boxtimes}}



\begin{document}

\title{Carrés parfaits et produits d'entiers consécutifs -- Des solutions à la main}
\author{Christophe BAL}
\date{25 Jan. 2024 -- 18 Fév. 2024}

\maketitle

\begin{center}
	\itshape
	Document, avec son source \LaTeX, disponible sur la page
	
	\url{https://github.com/bc-writing/drafts}.
\end{center}


\bigskip


\begin{center}
	\hrule\vspace{.3em}
	{
		\fontsize{1.35em}{1em}\selectfont
		\textbf{Mentions \enquote{légales}}
	}
			
	\vspace{0.45em}
	\small
	\doclicenseThis
	\hrule
\end{center}


\setcounter{tocdepth}{2}
\tableofcontents


% ------------------ %


\newpage
\section{Ce qui nous intéresse}

Dans l'article \enquote{Note on Products of Consecutive Integers}
\footnote{
	J. London Math. Soc. 14 (1939).
},
Paul Erdos démontre que pour tout couple $(n, k) \in \NNs \times \NNs$\,, le produit de $(k+1)$ entiers consécutifs $n (n + 1) \cdots (n + k)$ n'est jamais le carré d'un entier. 

\smallskip

Il est facile de trouver sur le web des preuves à la main de $n(n+1) \cdots (n + k) \notin \NNssquare$ pour $k \in \ZintervalC{1}{7}$\,.
Bien que certaines de ces preuves soient très sympathiques, leur lecture ne fait pas ressortir de schéma commun de raisonnement.
%
Dans ce document, nous allons tenter de limiter au maximum l'emploi de fourberies déductives en présentant une méthode très élémentaire
\footnote{
	Cette méthode s'appuie sur une représentation trouvée dans \href{https://web.archive.org/web/20171110144534/http://mathforum.org/library/drmath/view/65589.html}{un message archivé} : voir la section \ref{sources}.
},
efficace, et semi-automatisable, pour démontrer, avec peu d'efforts cognitifs, les premiers cas d'impossibilité.




% ------------------ %


\bigskip
\section{Notations utilisées}

Dans la suite, nous utiliserons les notations suivantes.
\begin{itemize}
	\item $2\,\NN$ désigne l'ensemble des nombres naturels pairs.
	
	\item $2\,\NN + 1$ désigne l'ensemble des nombres naturels impairs.
	
	\item $\forall (n , m) \in \NN^2$, $n \vee m$ désigne le PPCM de $n$ et $m$.

	\item $\forall (n , m) \in \NN^2$, $n \wedge m$ désigne le PGCD de $n$ et $m$.

	\item $a \strictdivides b$ signifie que $a \divides b$ et $a \neq b$ (division stricte).

	\item $\PP$ désigne l'ensemble des nombres premiers.
	
	\item $\forall (p ; n) \in \PP \times \NNs$\,, $\padicval{n} \in \NN$ est la valuation $p$-adique de $n$\,, c'est-à-dire $p^{\padicval{n}} \divides n$\,, mais $p^{\padicval{n} + 1} \ndivides n$\,.
\end{itemize}


% ------------------ %


\foreach \k in {1,...,13} {
%\foreach \k in {5,7,9,...,13} {
	\newpage

	\ifthenelse{\k = 1}{
		\section{Les carrés parfaits}
	}{
		\section{Avec \k\ facteurs}
	}

	\input{\contentdir/case-\k}
}


%% Searching...
%\foreach \k in {8} {
%	\newpage
%	\section{\k\ facteurs ?}
%
%	\input{\contentdir/case-\k-OKKO}
%}


% ------------------ %


\newpage

\section{Sources utilisées} \label{sources}

% ------------------ %


\bigskip
\textbf{Fait \ref{case-4}.}
	
%\smallskip
%\noindent
%Voir la source du fait  \ref{case-7}.

\smallskip
\noindent
\emph{La démonstration non algébrique a été impulsée par la source du fait \ref{case-7} donnée plus bas.}


% ------------------ %


\bigskip
\textbf{Fait \ref{case-5}.}
	
\begin{itemize}
	\item Un échange consulté le 28 janvier 2024, et titré 
	\emph{\enquote{\href{https://les-mathematiques.net/vanilla/discussion/comment/351293}{n(n+1)...(n+k) est un carré ?}}} 
	sur le site \url{lesmathematiques.net}\,.

    \smallskip
    \noindent
    \emph{La démonstration via le principe des tiroirs trouve sa source dans cet échange.}


	\item Un échange consulté le 12 février 2024, et titré 
	\emph{\enquote{\href{https://artisticmathematics.quora.com/Is-there-an-easier-way-of-proving-the-product-of-any-5-consecutive-positive-integers-is-never-a-perfect-square}{Is there an easier way of proving the product of any 5 consecutive positive integers is never a perfect square?}}} 
	sur le site \url{www.quora.com/}\,.

    \smallskip
    \noindent
    \emph{La démonstration \enquote{élémentaire} sans le principe des tiroirs vient de cet échange.}


	\item L'article \emph{\enquote{Le produit de 5 entiers consécutifs n'est pas le carré d'un entier.}} de T. Hayashi, Nouvelles Annales de Mathématiques, est consultable via \href{https://numdam.org}{Numdam}\,, la bibliothèque numérique française de mathématiques.
	
	\smallskip
	\noindent
	\emph{Cet article a fortement inspiré la longue preuve.}
\end{itemize}
\vspace{-1ex}


% ------------------ %


\bigskip
\textbf{Fait \ref{case-6}.}
	
\begin{itemize}
	\item Un échange consulté le 28 janvier 2024, et titré
\emph{\enquote{\href{https://math.stackexchange.com/q/90894/52365}{product of six consecutive integers being a perfect numbers}}} 
sur le site \url{https://math.stackexchange.com}\,.
	
	\smallskip
	\noindent
	\emph{La courte démonstration est donnée dans cet échange. Vous y trouverez aussi un très joli argument basé sur les courbes elliptiques rationnelles.}


	\item Une discussion archivée consultée le 28 janvier 2024 : 
	
	\noindent
	\url{https://web.archive.org/web/20171110144534/http://mathforum.org/library/drmath/view/65589.html}\,.
	
	\smallskip
	\noindent
	\emph{Cette discussion a impulsé la preuve fastidieuse, mais facile d'accès, via des tableaux.}
\end{itemize}
\vspace{-1ex}


% ------------------ %


\bigskip
\textbf{Fait \ref{case-7}.}
	
\smallskip
\noindent
Un échange consulté le 3 février 2024, et titré
\emph{\enquote{\href{https://math.stackexchange.com/q/2334887/52365}{Proof that the product of 7 successive positive integers is not a square}}} 
sur le site \url{https://math.stackexchange.com}\,.
	
\smallskip
\noindent
\emph{La courte démonstration est donnée dans cet échange, mais certaines justifications manquent.}


% ------------------ %


\bigskip
\textbf{Fait \ref{case-8}.}
	
\begin{itemize}
	\item Le document \emph{\enquote{Products of consecutive Integers}} de Vadim Bugaenko, Konstantin Kokhas, Yaroslav Abramov et Maria Ilyukhina obtenu via un moteur de recherche le 28 février 2024.


	\item Un échange consulté le 4 février 2024, et titré \emph{\enquote{\href{https://math.stackexchange.com/a/2271715/52365}{How to prove that the product of eight consecutive numbers can't be a number raised to exponent 4?}}} sur le site \url{https://math.stackexchange.com}\,.

    \smallskip
    \noindent
    \emph{La démonstration astucieuse vient de l'une des réponses de cet échange, mais la justification des deux inégalités n'est pas donnée.}
\end{itemize}
\vspace{-1ex}






\smallskip
\noindent



% ------------------ %


\bigskip
\textbf{Fait \ref{case-10}.}
	
\smallskip
\noindent
Un échange consulté le 13 février 2024, et titré
\emph{\enquote{\href{https://math.stackexchange.com/q/2361670/52365}{Product of 10 consecutive integers can never be a perfect square}}} 
sur le site \url{https://math.stackexchange.com}\,.

\smallskip
\noindent
\emph{La démonstration vient d'une source Wordpress donnée dans une réponse de cet échange, mais cette source est très expéditive...}




% ------------------ %


%%\bigskip
%\newpage
%
%\hrule
%
%\section{AFFAIRE À SUIVRE...}
%
%\bigskip
%
%\hrule

\end{document}
