\documentclass[12pt]{amsart}
%\usepackage[T1]{fontenc}
%\usepackage[utf8]{inputenc}

\usepackage[top=1.95cm, bottom=1.95cm, left=2.35cm, right=2.35cm]{geometry}

\usepackage%[hidelinks]%
           {hyperref}  
\usepackage{pgfpages}
%\pgfpagesuselayout{2 on 1}[a4paper,landscape,border shrink=5mm]

\usepackage{enumitem}
\usepackage{tcolorbox}
\usepackage{float}
\usepackage{cleveref}
\usepackage{multicol}
\usepackage{fancyvrb}
\usepackage{enumitem}
\usepackage{amsmath}
\usepackage{textcomp}
\usepackage{numprint}
\usepackage{tabularray}
\usepackage[french]{babel}
\frenchsetup{StandardItemLabels=true}
\usepackage{csquotes}
\usepackage{scalerel}
\usepackage{piton}

\NewPitonEnvironment{Python}{}
  {\begin{tcolorbox}}
  {\end{tcolorbox}}
  
\SetPitonStyle{
 	Number = ,
    String = \itshape ,
    String.Doc = \color{gray} \slshape ,
    Operator = ,
    Operator.Word = \bfseries ,
    Name.Builtin = ,
    Name.Function = ,
    Comment = \color{gray} ,
    Comment.LaTeX = \normalfont \color{gray},
    Keyword = \bfseries ,
    Name.Namespace = ,
    Name.Class = ,
    Name.Type = ,
    InitialValues = \color{gray}
}

\usepackage[
    type={CC},
    modifier={by-nc-sa},
	version={4.0},
]{doclicense}

\newcommand\floor[1]{\left\lfloor #1 \right\rfloor}

\usepackage{tnsmath}


\newtheorem{fact}{Fait}[section]
\newtheorem{defi}{Définition}[section]
\newtheorem{example}{Exemple}[section]
\newtheorem{remark}{Remarque}[section]
\newtheorem*{proof*}{Une preuve alternative}

\npthousandsep{.}
\setlength\parindent{0pt}

\floatstyle{boxed} 
\restylefloat{figure}


\DeclareMathOperator{\taille}{\text{\normalfont\texttt{taille}}}

\newcommand{\logicneg}{\text{\normalfont non \!}}

\newcommand\sqseq[2]{\fbox{$#1$}_{\,\,#2}}

\DefineVerbatimEnvironment{rawcode}%
	{Verbatim}%
	{tabsize=4,%
	 frame=lines, framerule=0.3mm, framesep=2.5mm}
	 
	 
\newcommand\contentdir{\jobname}



\NewDocumentCommand\commonlayout{ m m }{%
\vspace{-5pt}
\null\hfill\textit{#1}\hfill\null

\smallskip
\smallskip

#2}

\NewDocumentCommand\nopb{ m }{%
\commonlayout{Sans intervention humaine.}%
             {Pour $#1$\,, nous avons $\consprod<k> \notin \NNsquare$ sans aucun effort cognitif.}}


\NewDocumentCommand\byhand{ m }{%
\commonlayout{De nouveaux cas problématiques}%
             {$\consprod<#1> \notin \NNsquare$ demande de gérer les \sftab[x] suivants.}}


\NewDocumentCommand\sftab{ O{} }{\textsf{s}\kern.5pt\textsf{f}\kern1pt-\kern1pt tableau#1}


\newcommand\NNsf{\NN_{\kern-1pt s\kern-1pt f}}

\newcommand\NNsquare{\seqsuprageo{\NN}{}{}{}{2}}
\newcommand\NNssquare{\seqsuprageo{\NN}{*}{}{}{2}}


\NewDocumentCommand\NNseq{ O{n} }{\seqsuprageo{\NN}{}{\!sc}{#1}{}}
\NewDocumentCommand\PPseq{ O{n} }{\seqsuprageo{\PP}{}{\!sc}{#1}{}}

\NewDocumentCommand\GCD{ m  m }{#1 \wedge #2}

\NewDocumentCommand\padicval{ O{p} m }{v_{#1}(#2)}
\NewDocumentCommand\consprod{ O{n} D<>{k} }{\pi_{#1}^{#2}}

\NewDocumentCommand\alt{ m }{\textbf{[A\kern1pt#1]}}

\newcommand\mycheckmark{{\color{green!60!black} \checkmark}}
\newcommand\myboxtimes{{\color{red!80!black} \boxtimes}}



\begin{document}

\title{BROUILLON - Carrés parfaits et produits d'entiers consécutifs -- Une méthode efficace}
\author{Christophe BAL}
\date{25 Jan. 2024 -- 21 Fév. 2024}

\maketitle

\begin{center}
	\itshape
	Document, avec son source \LaTeX, disponible sur la page
	
	\url{https://github.com/bc-writing/drafts}.
\end{center}


\bigskip


\begin{center}
	\hrule\vspace{.3em}
	{
		\fontsize{1.35em}{1em}\selectfont
		\textbf{Mentions \enquote{légales}}
	}
			
	\vspace{0.45em}
	\small
	\doclicenseThis
	\hrule
\end{center}


\setcounter{tocdepth}{2}
\tableofcontents


% ------------------ %


\newpage
\section{Ce qui nous intéresse}

Dans l'article \enquote{Note on Products of Consecutive Integers}
\footnote{
	J. London Math. Soc. 14 (1939).
},
Paul Erdos démontre que pour tout couple $(n, k) \in \NNs \times \NNs$\,, le produit de $(k+1)$ entiers consécutifs $n (n + 1) \cdots (n + k)$ n'est jamais le carré d'un entier. 

\smallskip

Il est facile de trouver sur le web des preuves à la main de $n(n+1) \cdots (n + k) \notin \NNssquare$ pour $k \in \ZintervalC{1}{7}$\,.
Bien que certaines de ces preuves soient très sympathiques, leur lecture ne fait pas ressortir de schéma commun de raisonnement.
%
Dans ce document, nous allons tenter de limiter au maximum l'emploi de fourberies déductives en présentant une méthode très élémentaire
\footnote{
	Cette méthode s'appuie sur une représentation trouvée dans \href{https://web.archive.org/web/20171110144534/http://mathforum.org/library/drmath/view/65589.html}{un message archivé} : voir la section \ref{sources}.
},
efficace, et semi-automatisable, pour démontrer, avec peu d'efforts cognitifs, les premiers cas d'impossibilité.




% ------------------ %


%\bigskip
\section{Notations utilisées}

Dans la suite, nous utiliserons les notations suivantes.
\begin{itemize}
	\item $2\,\NN$ désigne l'ensemble des nombres naturels pairs.
	
	\item $2\,\NN + 1$ désigne l'ensemble des nombres naturels impairs.
	
	\item $\forall (n , m) \in \NN^2$, $n \vee m$ désigne le PPCM de $n$ et $m$.

	\item $\forall (n , m) \in \NN^2$, $n \wedge m$ désigne le PGCD de $n$ et $m$.

	\item $a \strictdivides b$ signifie que $a \divides b$ et $a \neq b$ (division stricte).

	\item $\PP$ désigne l'ensemble des nombres premiers.
	
	\item $\forall (p ; n) \in \PP \times \NNs$\,, $\padicval{n} \in \NN$ est la valuation $p$-adique de $n$\,, c'est-à-dire $p^{\padicval{n}} \divides n$\,, mais $p^{\padicval{n} + 1} \ndivides n$\,.
\end{itemize}


% ------------------ %


%\newpage
%\medskip
\section{Les carrés parfaits}

\leavevmode
\smallskip

Via $N^2 - M^2 = (N - M)(N + M)$\,, il est immédiat de noter que 
$\forall (N, M) \in \NNs \times \NNs$\,, si $N > M$\,, alors $N^2 - M^2 \geq 3$\,. Le fait suivant précise ceci.


\begin{fact} \label{dist-square}
	$\forall (N, M) \in \NNs \times \NNs$, 
	si $N > M$\,, alors $N^2 - M^2 = \dsum_{k=M+1}^{N} (2 k - 1)$\,.
\end{fact}


% ------------------ %


\begin{proof}
	Il suffit d'utiliser $N^2 = \dsum_{k=1}^{N} (2 k - 1)$\,.
\end{proof}




% ------------------ %


%\newpage
%\medskip
\section{Prenons du recul}

% ------------------ %


\subsection{Les \sftab[x]}

\leavevmode
\smallskip

L'idée de départ est simple : d'après le fait \ref{facto-square}, il semble opportun de se concentrer sur les diviseurs sans facteur carré des $k$ facteurs $(n + i)$ de $\consprod = n (n + 1) \cdots (n + k - 1)$\,.


% ------------------ %


%\newpage
\begin{defi}
	Considérons $(n, k) \in ( \NNs )^2$\,,
	$( a_i )_{0 \leq i \leq k} \subset \NNsf$
	et
	$( s_i )_{0 \leq i \leq k} \subset \NNssquare$
	tels que
	$\forall i \in \ZintervalC{0}{k}$\,, $n + i = a_i s_i$\,.
	%
	Cette situation est résumée par le tableau suivant que nous nommerons \enquote{\sftab}
	\footnote{
		\enquote{sf} est pour \enquote{square free}\,.
	}.

	\begin{center}
		\begin{tblr}{
			colspec    = {Q[r,$]*{5}{Q[c,$]}},
			vline{2}   = {.95pt},
			vline{3-6} = {dashed},
			hline{2}   = {.95pt}
		}
			n + \bullet
				&  0    &  1    &  2    &  \dots  &  k
		\\
				&  a_0  &  a_1  &  a_2  &  \dots  &  a_k
		\end{tblr}
	\end{center}
%	Si de plus $( a_i )_{0 \leq i \leq k} \subseteq \NNsf$\,, le \sftab\ sera dit réduit.
\end{defi}


% ------------------ %


%\newpage
\begin{example}
	Supposons avoir le \sftab\ suivant où $n \in \NNs$.

	\begin{center}
		\begin{tblr}{
			colspec    = {Q[r,$]*{4}{Q[c,$]}},
			vline{2}   = {.95pt},
			vline{3-5} = {dashed},
			hline{2}   = {.95pt}
		}
			n + \bullet
				&  0  &  1  &  2  &  3
		\\
				&  2  &  5  &  6  &  1
		\end{tblr}
	\end{center}

	Ceci résume la situation suivante.

	\vspace{-1ex}
	\begin{multicols}{2}
	\begin{itemize}
		\item $\exists A \in \NNs$ tel que $n     = 2 A^2$\,.

		\item $\exists B \in \NNs$ tel que $n + 1 = 5 B^2$\,.

		\item $\exists C \in \NNs$ tel que $n + 2 = 6 C^2$\,.

		\item $\exists D \in \NNs$ tel que $n + 3 =   D^2$\,.
	\end{itemize}
	\end{multicols}
\end{example}


% ------------------ %


\begin{defi}
	Soient $r \in \NNs$,
	$( n_i )_{1 \leq i \leq r} \in \NNseq[r]$\,,
	$( a_i )_{1 \leq i \leq r} \subset \NNsf$
	et
	$( s_i )_{1 \leq i \leq r} \subset \NNssquare$
	tels que
	$\forall i \in \ZintervalC{1}{r}$\,, $n_i = a_i s_i$\,.
	%
	Cette situation est résumée par le tableau suivant que nous nommerons \enquote{\sftab\ généralisé}\,.

	\begin{center}
		\begin{tblr}{
			colspec    = {Q[r,$]*{5}{Q[c,$]}},
			vline{2}   = {.95pt},
			vline{3-6} = {dashed},
			hline{2}   = {.95pt}
		}
			\bullet
				&  n_1  &  n_2  &  n_3  &  \dots  &  n_r
		\\
				&  a_1  &  a_2  &  a_3  &  \dots  &  a_r
		\end{tblr}
	\end{center}
%	Si de plus $( a_i )_{0 \leq i \leq k} \subseteq \NNsf$\,, le \sftab\ sera dit réduit.
\end{defi}


% ------------------ %


\subsection{Construire des \sftab[x]}

\leavevmode

\smallskip
Pour fabriquer des \sftab[x], nous allons \enquote{multiplier} des \sftab[x] dits partiels.


\begin{defi}
	Soient $(n, k, r) \in ( \NNs )^3$\,,
	$( p_j )_{1 \leq j \leq r} \in \PPseq[r]$\,,
	$( \epsilon_{i,j} )_{0 \leq i \leq k \,, 1 \leq j \leq r} \subseteq \setgene{0, 1}$
	et aussi
	$( f_i )_{0 \leq i \leq k} \subset \NNs$
	vérifiant les conditions suivantes.
	%
	\begin{itemize}
		\item $\forall i \in \ZintervalC{0}{k}$\,,
		$n + i = f_i \cdot \dprod_{j = 1}^{r} p_j^{\,\padicval[p_j]{n+i}}$\,.
		Noter que
		$\forall i \in \ZintervalC{0}{k}$\,,
		$\forall j \in \ZintervalC{1}{r}$\,,
		$\GCD{f_i}{p_j} = 1$\,.

		\item $\forall i \in \ZintervalC{0}{k}$\,,
		$\forall j \in \ZintervalC{1}{r}$\,,
		$\padicval[p_j]{n+i} \equiv \epsilon_{i,j}$ modulo $2$\,.
	\end{itemize}

	\smallskip

	Cette situation est résumée par le tableau suivant qui sera nommé \enquote{\sftab\ partiel}\,, voire \enquote{\sftab\ partiel d'ordre $( p_j )_{1 \leq j \leq r}$\!}\,
	\footnote{
		Noter que $\forall i \in \ZintervalC{0}{k}$\,, $\forall j \in \ZintervalC{1}{r}$\,, $p_j^{\,\epsilon_{i,j}} \in \setgene{1, p_j}$\,.
	}.

	\begin{center}
		\begin{tblr}{
			colspec     = {Q[r,$]*{5}{Q[c,$]}},
			vline{2}    = {.95pt},
			vline{3-6}  = {dashed},
			hline{2}    = {.95pt},
			column{2-Z} = {2.75em},
		}
			n + \bullet
				&  0
				&  1
				&  2
				&  \dots
				&  k
		\\
			( p_j )_{1 \leq j \leq r}
				&  \dprod_{j = 1}^{r} p_j^{\,\epsilon_{0,j}}
				&  \dprod_{j = 1}^{r} p_j^{\,\epsilon_{1,j}}
				&  \dprod_{j = 1}^{r} p_j^{\,\epsilon_{2,j}}
				&  \dots
				&  \dprod_{j = 1}^{r} p_j^{\,\epsilon_{k,j}}
		\end{tblr}
	\end{center}
\end{defi}


% ------------------ %


\begin{example}
	Supposons avoir le \sftab\ partiel suivant où $n \in \NNs$.

	\begin{center}
		\begin{tblr}{
			colspec    = {Q[r,$]*{4}{Q[c,$]}},
			vline{2}   = {.95pt},
			vline{3-5} = {dashed},
			hline{2}   = {.95pt},
%			column{2-Z} = {1.75em},
		}
			n + \bullet
				&  0  &  1  &  2  &  3
		\\
			(2, 3)
				&  2  &  6  &  1  &  3
		\end{tblr}
	\end{center}

%	\newpage
	Ceci résume la situation suivante.
	%
	\begin{itemize}
		\item $\exists (a, \alpha, A) \in \NN^2 \times \NNs$
		      tel que $A \wedge 6 = 1$
		      et
		      $n     = 2^{2a+1} 3^{2\alpha} A$\,.

		\item $\exists (b, \beta, B) \in \NN^2 \times \NNs$
		      tel que $B \wedge 6 = 1$
		      et
		      $n + 1 = 2^{2b+1} 3^{2\beta+1} B$\,.

		\item $\exists (c, \gamma, C) \in \NN^2 \times \NNs$
		      tel que $C \wedge 6 = 1$
		      et
		      $n + 2 = 2^{2c} 3^{2\gamma} C$\,.

		\item $\exists (d, \delta, D) \in \NN^2 \times \NNs$
		      tel que $D \wedge 6 = 1$
		      et
		      $n + 3 = 2^{2d} 3^{2\delta+1} D$\,.
	\end{itemize}
\end{example}


% ------------------ %


\begin{example}
	La multiplication de deux \sftab[x] partiels de deux suites
	$( p_j )_{1 \leq j \leq r} \in \PPseq[r]$
	et
	$( q_j )_{1 \leq j \leq s} \in \PPseq[s]$
	d'intersection vide, c'est-à-dire sans nombre premier commun, est \enquote{naturelle}\,.
	Considérons les deux \sftab[x] partiels suivants où l'on note $2$ et $3$ au lieu de $(2)$ et $(3)$\,.

	\vspace{-1.5ex}
	\begin{multicols}{2}
	\begin{center}
		\begin{tblr}{
			colspec    = {Q[r,$]*{4}{Q[c,$]}},
			vline{2}   = {.95pt},
			vline{3-5} = {dashed},
			hline{2}   = {.95pt}
		}
			n + \bullet
				&  0  &  1  &  2  &  3
		\\
			2
				&  1  &  2  &  1  &  2
		\end{tblr}
	\end{center}

	\begin{center}
		\begin{tblr}{
			colspec    = {Q[r,$]*{4}{Q[c,$]}},
			vline{2}   = {.95pt},
			vline{3-5} = {dashed},
			hline{2}   = {.95pt}
		}
			n + \bullet
				&  0  &  1  &  2  &  3
		\\
			3
				&  3  &  1  &  1  &  3
		\end{tblr}
	\end{center}
	\end{multicols}


	\vspace{-1ex}
	La multiplication de ces \sftab[x] partiels est le \sftab\  suivant, partiel a priori, mais si l'on sait que $2$ et $3$ sont les seuls diviseurs premiers de $\consprod<4>$\,, alors le \sftab\ est non partiel.

	\begin{center}
		\begin{tblr}{
			colspec    = {Q[r,$]*{4}{Q[c,$]}},
			vline{2}   = {.95pt},
			vline{3-5} = {dashed},
			hline{2}   = {.95pt}
		}
			n + \bullet
				&  0  &  1  &  2  &  3
		\\
			(2, 3)
				&  3  &  2  &  1  &  6
		\end{tblr}
	\end{center}

	Ceci résume la situation suivante qui est équivalente à ce que donne la conjonction des deux premiers \sftab[x] partiels (les abus de notations sont évidents).


	\vspace{-1ex}
	\begin{multicols}{2}
	\begin{itemize}
		\item $A \wedge 6 = 1$
		      et
		      $n     = 2^{2a}   3^{2\alpha+1} A$\,.

		\item $B \wedge 6 = 1$
		      et
		      $n + 1 = 2^{2b+1} 3^{2\beta}    B$\,.

		\item $C \wedge 6 = 1$
		      et
		      $n + 2 = 2^{2c}   3^{2\gamma}   C$\,.

		\item $D \wedge 6 = 1$
		      et
		      $n + 3 = 2^{2d+1} 3^{2\delta+1} D$\,.
	\end{itemize}
	\end{multicols}
\end{example}


%
%
%
%% ------------------ %
%
%
%\newpage
%\bigskip
%\medskip
\section{Structure des \sftab[x]} \label{sftab-struct}

% ------------------ %


\subsection{A propos des \sftab[x] partiels} \label{sftable-constraint}


\begin{fact} \label{sftable-multiple}

	Dans la deuxième ligne d'un \sftab\ partiel d'ordre $p$\,, les positions des valeurs $p$ sont congrues modulo $p$\,.
\end{fact}


\begin{proof}
	Penser aux multiples de $p$\,.
\end{proof}


% ------------------ %


\begin{fact} \label{sftable-parity-square}
	$\forall (n, k, p) \in ( \NNs )^2 \times \PP$\,,
	si $\consprod \in \NNsquare$\,,
	alors dans le \sftab\ partiel d'ordre $p$ associé à $\consprod$\,, le nombre de valeurs $p$ est forcément pair.
\end{fact}


\begin{proof}
	Évident, mais très pratique, comme nous le verrons dans la suite.
\end{proof}


% ------------------ %


\subsection{A propos des \sftab[x] non partiels} \label{sftab-illegal}


\begin{fact} \label{sftab-recu}
	Dans les tableaux ci-dessous, où $k \geq 2$\,, les puces $\bullet$ indiquent des valeurs quelconques.

	\begin{enumerate}
		\item Si nous avons un \sftab\ du type suivant, alors $\consprod<k-1> \in \NNssquare$\,.
	\end{enumerate}

	\begin{center}
		\begin{tblr}{
			colspec     = {Q[r,$]*{5}{Q[c,$]}},
			vline{2}    = {.95pt},
			vline{3-6}  = {dashed},
			hline{2}    = {.95pt},
			column{2-Z} = {2em},
			% Subsquare
			cell{2}{6} = {blue!15},
		}
			n + \bullet
				&  0
				&  1
				&  \dots
				&  k - 1
				&  k
		\\
				&  \bullet
				&  \bullet
				&  \dots
				&  \bullet
				&  1
		\end{tblr}
	\end{center}


	\begin{enumerate}[start=2]
		\item Si nous avons un \sftab\ du type suivant, alors $\consprod[n+1]<k-1> \in \NNssquare$\,.
	\end{enumerate}

	\begin{center}
		\begin{tblr}{
			colspec     = {Q[r,$]*{5}{Q[c,$]}},
			vline{2}    = {.95pt},
			vline{3-6}  = {dashed},
			hline{2}    = {.95pt},
			column{2-Z} = {2em},
			% Subsquare
			cell{2}{2} = {blue!15},
		}
			n + \bullet
				&  0
				&  1
				&  \dots
				&  k - 1
				&  k
		\\
				&  1
				&  \bullet
				&  \dots
				&  \bullet
				&  \bullet
		\end{tblr}
	\end{center}
\end{fact}


\begin{proof}
	Immédiat via le fait \ref{facto-square}, car nous avons soit $n + k \in \NNssquare$\,, soit $n \in \NNssquare$\,.
\end{proof}


% ------------------ %


\begin{fact} \label{sftable-illegal-0-sol}
	Soit le \sftab\ généralisé ci-après où
	$r \in \NN_{\geq 2}$\,,
	$( n_i )_{1 \leq i \leq r} \in \NNseq[r]$
	et
	$d \in \NNsf$\,.

    \begin{center}
    	\begin{tblr}{
    		colspec     = {Q[r,$]*{3}{Q[c,$]}},
    		vline{2}    = {.95pt},
    		vline{3-Y}  = {dashed},
   			hline{2}    = {.95pt},
			column{2-Z} = {2em},
			% Rejected
			cell{2}{2,Z} = {red!15},
    	}
    		\bullet
   				&  n_1
    			&  \dots
    			&  n_r
    	\\
    			&  d
    			&  \dots
				&  d
    	\end{tblr}
    \end{center}

	Ce \sftab\ est impossible si l'une des deux conditions suivantes est validée.
	
	\begin{enumerate}
		\item $\dfrac{n_r - n_1}{d} \notin \NN$\,.

		\item $\dfrac{n_r - n_1}{d} \in \{ 1, 2, 4, 6, 10, 14, 18, 22, 26, 30, 34, 38, 42, 46, 50, 54, 58, 62, 66, 70, 74, 78, 82,$ \\ $86, 90, 94, 98 \}$\,.
	\end{enumerate}
\end{fact}


\begin{proof}
	$n_1 = d A^2$ et $n_r = d B^2$ nous donnent $d (B^2 - A^2) = n_r - n_1$\,. On conclut directement pour le premier cas, et via le fait \ref{diff-square-ko} dans le second.
\end{proof}





% ------------------ %


%\newpage
\section{Premières applications}

\foreach \k in {2,...,5} {
%\foreach \k in {5} {
%	%\newpage
%	\medskip
	\subsection{Le cas de \k\ facteurs} \label{apply-\k}

	\leavevmode
	\smallskip
	
	\input{\contentdir/case-\k}
}



% ------------------ %


%\newpage
%\medskip
\section{Et après ?}

\subsection{La méthode via le cas de 6 facteurs} \label{apply-6}

\leavevmode
\smallskip

La méthode présentée ci-dessus permet de faire appel à des programmes informatiques pour limiter les traitements à la main, et à la sueur des neurones, de \sftab[x] problématiques comme nous avons dû le faire dans la section \ref{apply-4}.
Expliquons cette tactique semi-automatique en traitant le cas de $6$ facteurs.

%\newpage
\begin{enumerate}
	\item On raisonne par l'absurde en supposant que $\consprod<6> \in \NNssquare$\,.


	\item Comme $\forall p \in \PP_{\geq 6}$\,, $p$ divise au maximum un seul des facteurs $(n + i)$ de $\consprod<6>$\,,
	nous avons juste besoin de considérer l'ensemble $\setproba{P} = \setgene{2, 3, 5}$ des diviseurs premiers stricts de $6$\,.


	\item Pour chaque élément $p$ de $\setproba{P}$\,, on construit la liste $\setproba*{V}{p}$ des \sftab[x] partiels relatifs à $p$ et $\consprod<6> \in \NNssquare$ en s'appuyant sur la section \ref{sftable-constraint}.


	\item Via les listes $\setproba*{V}{p}$\,, on calcule toutes les multiplications de tous les \sftab[x] partiels relatifs à des nombres $p$ différents, et pour chacune d'elles, on ne la garde que si elle ne vérifie aucune des conditions suivantes, celles du dernier cas devant être indiquées à la main au programme qui va donc évoluer au gré des démonstrations faites par un humain (démonstrations que l'on espère le plus rare possible).
	%
	\begin{enumerate}
		\item Le tableau commence, ou se termine, par la valeur $1$\,. Dans ce cas, on sait par récurrence que le tableau produit n'est pas possible (voir le fait \ref{sftab-recu}).

		\item Le tableau est rejeté par le fait \ref{sftable-illegal-0-sol}.

		\item Le tableau \enquote{produit} contient un sous-tableau que nous savons impossible suite à un raisonnement humain fait \emph{localement}\,, c'est-à-dire que seul les facteurs indiqués dans le sous-tableau, et le sous-tableau lui-même sont utilisés pour raisonner.
		C'est le cas des \sftab[x] du fait \ref{no-sftab-6.1.2.3}.
	\end{enumerate}
\end{enumerate}


Dans le dépôt en ligne associé à ce document sont placés des fichiers \verb#Pyhon#
\footnote{
	L'emploi de scripts codés rapidement est totalement fonctionnel ici.
}
qui nous amènent à analyser les deux \sftab[x] problématiques suivants pour lesquels nous allons justifier que les valeurs $1$ posent problème
\footnote{
	Toutes les règles 4-a, 4-b et 4-c sont utilisées pour n'arriver qu'aux deux \sftab[x] à analyser à la main.
}.
%
\begin{center}
	\begin{tblr}{
		colspec    = {Q[r,$]*{6}{Q[c,$]}},
		vline{2}   = {.95pt},
		vline{3-Y} = {dashed},
		hline{2}   = {.95pt},
		% Rejected
		cell{2,3}{3,Y} = {red!15},
	}
		n + \bullet  
			&  0   &  1  &  2  &  3  &  4  &  5
	\\  
			&  30  &  1  &  2  &  3  &  1  &  5
	\\  
			&  5   &  1  &  3  &  2  &  1  &  30
	\end{tblr}
\end{center}


Ces deux cas sont rapides à gérer puisque, d'après le fait \ref{diff-square-ko}, $1$ et $4$ sont les seuls carrés distants de $3$\,, d'où $n+1 = 1$\,, mais ceci contredit $n \in \NNs$. Nous savons donc que $\consprod<6> \in \NNssquare$ sans effort.
Notons au passage un nouveau cas problématique \enquote{local} pour nos futures recherches (le fait suivant généralise la technique que nous venons d'utiliser).


\newpage
\begin{fact} \label{sftable-illegal-1-sol}
	Soit le \sftab\ généralisé ci-après où
	$r \in \NN_{\geq 2}$\,,
	$( n_i )_{1 \leq i \leq r} \in \NNseq[r]$
	et
	$d \in \NNsf$\,.

    \begin{center}
    	\begin{tblr}{
    		colspec     = {Q[r,$]*{3}{Q[c,$]}},
    		vline{2}    = {.95pt},
    		vline{3-Y}  = {dashed},
   			hline{2}    = {.95pt},
			column{2-Z} = {2em},
			% Rejected
			cell{2}{2,Z} = {red!15},
    	}
    		\bullet
   				&  n_1  &  \dots  &  n_r
    	\\
    			&  d    &  \dots  &  d
    	\end{tblr}
    \end{center}

	Ce \sftab\ est impossible si $n_1 \geq d+1$
	et
	$\frac{n_r - n_1}{d} \in \setgene{3, 8}$\,.
\end{fact}


\begin{proof}
	Ceci vient des équivalences logiques suivantes en posant $n_1 = d A^2$ et $n_r = d B^2$ avec $(A, B) \in ( \NNs )^2$.
	
	\medskip
	\begin{stepcalc}[style=ar*, ope={\iff}]
		\dfrac{n_r - n_1}{d} \in \setgene{3, 8}
	\explnext{}
		B^2 -A^2 \in \setgene{3, 8}
	\explnext*{Voir le fait \ref{diff-square-ko}.}{}
		(A, B) \in \setgene{(1, 2), (1, 3)}
	\explnext{}
		(n_1, n_r) \in \setgene{(d, 4 d), (d, 9 d)}
	\end{stepcalc}

	\vspace{-2ex}	
	\leavevmode
\end{proof}


\begin{remark}
	On peut gérer les cas problématiques du cas $6$ via des manipulations algébriques similaires à celles qui avaient donné le fait \ref{no-sftab-6.1.2.3}.
	En effet, $x = n + \frac52$ nous donne ce qui suit avec un abus de notation évident.
	\begin{center}
	\begin{tblr}{
		colspec     = {Q[r,$]*{6}{Q[c,$]}},
		vline{2}    = {.95pt},
		vline{3-Y}  = {dashed},
		hline{2}    = {.95pt},
		column{2-Z} = {1.25em},
		% Focus
		cell{1-3}{2,7} = {green!15},
		cell{1-3}{5,4} = {orange!15},
	}
		x + \bullet  
			&  -\frac52  &  -\frac32  &  -\frac12  &  \frac12  &  \frac32  &  \frac52
	\\  
			&  30        &  1         &  2         &  3        &  1        &  5
	\\  
			&  5         &  1         &  3         &  2        &  1        &  30
	\end{tblr}
	\end{center}
	
	La multiplication des colonnes $1$ et $6$\,, ainsi que celle des colonnes $3$ et $4$\,, nous amènent au même \sftab\ généralisé suivant après avoir noté que $5 \times 30 = 6 \times 5^2$\,.
	\begin{center}
	\begin{tblr}{
		colspec     = {Q[r,$]*{2}{Q[c,$]}},
		vline{2}    = {.95pt},
		vline{3-Y}  = {dashed},
		hline{2}    = {.95pt},
		column{2-Z} = {3.25em},
		% Focus
		cell{1-3}{2} = {green!15},
		cell{1-3}{3} = {orange!15},
	}
		\bullet
			&  x^2 - \frac{25}{4}  &  x^2 - \frac14
	\\  
			&  6                   &  6
	\end{tblr}
	\end{center}
	
	
	Comme $x^2 - \frac14 - \big( x^2 - \frac{25}{4} \big) = 6$\,, le fait \ref{sftable-illegal-0-sol} nous permet de conclure.
\end{remark}



\subsection{Au-delà de 6 facteurs ?}

\leavevmode
\smallskip

Voici ce que donnent nos programmes \verb#Pyhon# sans trop d'efforts, mais avec du temps de calcul 
\footnote{
	Nous commençons à entrer dans un monde à la combinatoire élevée.
}.
Rappelons que chaque nouveau cas problématique est indiqué au programme qui évolue donc au gré de l'intervention humaine.


% ------------------ %


\nopb{k \in \setgene{7, 8}}


% ------------------ %


\byhand{9}
%
\begin{center}
	\begin{tblr}{
		colspec    = {Q[r,$]*{9}{Q[c,$]}},
		vline{2}   = {.95pt},
		vline{3-Y} = {dashed},
		hline{2}   = {.95pt},
	}
		n + \bullet
			&  0   &  1  &  2  &  3  &  4  &  5  &  6  &  7  &  8
	\\
			&  14  &  1  &  6  &  5  &  2  &  3  &  1  &  7  &  10
	\\
			&  10  &  7  &  1  &  3  &  2  &  5  &  6  &  1  &  14
	\end{tblr}
\end{center}

Extrayons du premier \sftab[], le \sftab\ généralisé suivant.
%
\begin{center}
	\begin{tblr}{
		colspec    = {Q[r,$]*{9}{Q[c,$]}},
		vline{2}   = {.95pt},
		vline{3-Y} = {dashed},
		hline{2}   = {.95pt},
	}
		\bullet
			&  n+1  &  n+2  &  n+4  &  n+5
	\\
			&  1    &  6    &  2    &  3
	\end{tblr}
\end{center}

%\newpage
En posant $m = n + 3$\,, nous obtenons le tableau ci-après.
%
\begin{center}
	\begin{tblr}{
		colspec    = {Q[r,$]*{9}{Q[c,$]}},
		vline{2}   = {.95pt},
		vline{3-Y} = {dashed},
		hline{2}   = {.95pt},
		% Focus
		cell{1-2}{2,5} = {green!15},
		cell{1-2}{3,4} = {orange!15},
	}
		\bullet
			&  m-2  &  m-1  &  m+1  &  m+2
	\\
			&  1    &  6    &  2    &  3
	\end{tblr}
\end{center}


En multipliant les colonnes $1$ et $4$\,, et aussi la $2$ et la $3$\,, nous obtenons le \sftab\ généralisé ci-dessous après avoir noté que $6 \times 2 = 3 \times 2^2$.

\begin{center}
	\begin{tblr}{
		colspec     = {Q[r,$]*{2}{Q[c,$]}},
		vline{2}    = {.95pt},
		vline{3-Y}  = {dashed},
		hline{2}    = {.95pt},
		column{2-Z} = {3em},
		% Focus
		cell{1-3}{2} = {green!15},
		cell{1-3}{3} = {orange!15},
	}
		\bullet
			&  m^2 - 4  &  m^2 - 1
	\\
			&  3        &  3
	\end{tblr}
\end{center}


Comme $m^2 - 4 - ( m^2 - 1 ) = 3$\,, le fait \ref{sftable-illegal-0-sol} nous montre que le premier \sftab[], celui commençant par $14$, est impossible.
Le cas du deuxième se traite de façon analogue, d'où finalement $\consprod<9> \notin \NNsquare$\,. Notons au passage un nouveau fait.


\newpage
\begin{fact} \label{no-sftab-1.6.*.2.3}
	Aucun \sftab\ ne peut contenir l'un des deux \sftab[x] généralisés suivants.
	\begin{center}
	\begin{tblr}{
		colspec    = {Q[r,$]*{5}{Q[c,$]}},
		vline{2}   = {.95pt},
		vline{3-5} = {dashed},
		hline{2}   = {.95pt}
	}
		m + \bullet
			&  0  &  1  &  3  &  4
	\\
			&  1  &  6  &  2  &  3
	\\
			&  3  &  2  &  6  &  1
	\end{tblr}
	\end{center} 
\end{fact}


% ------------------ %


\nopb{k \in \ZintervalC{10}{17}}
Au-delà, un programme basique n'est plus utilisable car il y a trop de tableaux à construire...




% ------------------ %
		

\section{Sources utilisées} \label{sources}

% ------------------ %


\bigskip
\textbf{Fait \ref{case-4}.}
	
%\smallskip
%\noindent
%Voir la source du fait  \ref{case-7}.

\smallskip
\noindent
\emph{La démonstration non algébrique a été impulsée par la source du fait \ref{case-7} donnée plus bas.}


% ------------------ %


\bigskip
\textbf{Fait \ref{case-5}.}
	
\begin{itemize}
	\item Un échange consulté le 28 janvier 2024, et titré 
	\emph{\enquote{\href{https://les-mathematiques.net/vanilla/discussion/comment/351293}{n(n+1)...(n+k) est un carré ?}}} 
	sur le site \url{lesmathematiques.net}\,.

    \smallskip
    \noindent
    \emph{La démonstration via le principe des tiroirs trouve sa source dans cet échange.}


	\item Un échange consulté le 12 février 2024, et titré 
	\emph{\enquote{\href{https://artisticmathematics.quora.com/Is-there-an-easier-way-of-proving-the-product-of-any-5-consecutive-positive-integers-is-never-a-perfect-square}{Is there an easier way of proving the product of any 5 consecutive positive integers is never a perfect square?}}} 
	sur le site \url{www.quora.com/}\,.

    \smallskip
    \noindent
    \emph{La démonstration \enquote{élémentaire} sans le principe des tiroirs vient de cet échange.}


	\item L'article \emph{\enquote{Le produit de 5 entiers consécutifs n'est pas le carré d'un entier.}} de T. Hayashi, Nouvelles Annales de Mathématiques, est consultable via \href{https://numdam.org}{Numdam}\,, la bibliothèque numérique française de mathématiques.
	
	\smallskip
	\noindent
	\emph{Cet article a fortement inspiré la longue preuve.}
\end{itemize}
\vspace{-1ex}


% ------------------ %


\bigskip
\textbf{Fait \ref{case-6}.}
	
\begin{itemize}
	\item Un échange consulté le 28 janvier 2024, et titré
\emph{\enquote{\href{https://math.stackexchange.com/q/90894/52365}{product of six consecutive integers being a perfect numbers}}} 
sur le site \url{https://math.stackexchange.com}\,.
	
	\smallskip
	\noindent
	\emph{La courte démonstration est donnée dans cet échange. Vous y trouverez aussi un très joli argument basé sur les courbes elliptiques rationnelles.}


	\item Une discussion archivée consultée le 28 janvier 2024 : 
	
	\noindent
	\url{https://web.archive.org/web/20171110144534/http://mathforum.org/library/drmath/view/65589.html}\,.
	
	\smallskip
	\noindent
	\emph{Cette discussion a impulsé la preuve fastidieuse, mais facile d'accès, via des tableaux.}
\end{itemize}
\vspace{-1ex}


% ------------------ %


\bigskip
\textbf{Fait \ref{case-7}.}
	
\smallskip
\noindent
Un échange consulté le 3 février 2024, et titré
\emph{\enquote{\href{https://math.stackexchange.com/q/2334887/52365}{Proof that the product of 7 successive positive integers is not a square}}} 
sur le site \url{https://math.stackexchange.com}\,.
	
\smallskip
\noindent
\emph{La courte démonstration est donnée dans cet échange, mais certaines justifications manquent.}


% ------------------ %


\bigskip
\textbf{Fait \ref{case-8}.}
	
\begin{itemize}
	\item Le document \emph{\enquote{Products of consecutive Integers}} de Vadim Bugaenko, Konstantin Kokhas, Yaroslav Abramov et Maria Ilyukhina obtenu via un moteur de recherche le 28 février 2024.


	\item Un échange consulté le 4 février 2024, et titré \emph{\enquote{\href{https://math.stackexchange.com/a/2271715/52365}{How to prove that the product of eight consecutive numbers can't be a number raised to exponent 4?}}} sur le site \url{https://math.stackexchange.com}\,.

    \smallskip
    \noindent
    \emph{La démonstration astucieuse vient de l'une des réponses de cet échange, mais la justification des deux inégalités n'est pas donnée.}
\end{itemize}
\vspace{-1ex}






\smallskip
\noindent



% ------------------ %


\bigskip
\textbf{Fait \ref{case-10}.}
	
\smallskip
\noindent
Un échange consulté le 13 février 2024, et titré
\emph{\enquote{\href{https://math.stackexchange.com/q/2361670/52365}{Product of 10 consecutive integers can never be a perfect square}}} 
sur le site \url{https://math.stackexchange.com}\,.

\smallskip
\noindent
\emph{La démonstration vient d'une source Wordpress donnée dans une réponse de cet échange, mais cette source est très expéditive...}




% ------------------ %


%\bigskip
\newpage

\hrule

\section{AFFAIRE À SUIVRE...}

\bigskip

\hrule

\end{document}