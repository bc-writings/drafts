\documentclass[12pt]{amsart}
\usepackage[T1]{fontenc}
\usepackage[utf8]{inputenc}

\usepackage[top=1.8cm, bottom=1.8cm, left=2.35cm, right=2.35cm]{geometry}

\usepackage{hyperref}
\usepackage{enumitem}
\usepackage{tcolorbox}
\usepackage{float}
\usepackage{cleveref}
\usepackage{multicol}
\usepackage{fancyvrb}
\usepackage{enumitem}
\usepackage{amsmath}
\usepackage{textcomp}
\usepackage{numprint}
\usepackage{tabularray}
\usepackage[french]{babel}
\frenchsetup{StandardItemLabels=true}
\usepackage{csquotes}

\usepackage[
    type={CC},
    modifier={by-nc-sa},
	version={4.0},
]{doclicense}

\newcommand\floor[1]{\left\lfloor #1 \right\rfloor}

\usepackage{tnsmath}


\newtheorem{fact}{Fait}[section]
\newtheorem{defi}{Définition}[section]
\newtheorem{example}{Exemple}[section]
\newtheorem{remark}{Remarque}[section]
\newtheorem*{proof*}{Une preuve alternative}

\npthousandsep{.}
\setlength\parindent{0pt}

\floatstyle{boxed} 
\restylefloat{figure}


\DeclareMathOperator{\taille}{\text{\normalfont\texttt{taille}}}

\newcommand{\logicneg}{\text{\normalfont non \!}}

\newcommand\sqseq[2]{\fbox{$#1$}_{\,\,#2}}


\DefineVerbatimEnvironment{rawcode}%
	{Verbatim}%
	{tabsize=4,%
	 frame=lines, framerule=0.3mm, framesep=2.5mm}
	 
	 
\newcommand\contentdir{\jobname}

\newcommand\NNsquare{\seqsuprageo{\NN}{}{}{}{2}}
\newcommand\NNssquare{\seqsuprageo{\NN}{}{*}{}{2}}

\newcommand\NNsf{\NN_{\kern-1pt s\kern-1pt f}}

\NewDocumentCommand\padicval{ O{p} m }{v_{#1}(#2)}

\NewDocumentCommand\consprod{ O{n} D<>{k} }{\pi_{#1}^{#2}}

\NewDocumentCommand\alt{ m }{\textbf{[A\kern1pt#1]}}

\newcommand\mycheckmark{{\color{green!60!black} \checkmark}}
\newcommand\myboxtimes{{\color{red!80!black} \boxtimes}}



\begin{document}

\title{Parler de tab réduit ou non càd sans factauer carré, du coup multuiplication mécanique et on réduit BROUILLON - Carrés parfaits et produits d'entiers consécutifs -- Une méthode efficace}
\author{Christophe BAL}
\date{25 Jan. 2024 -- 31 Jan. 2024}

\maketitle

\begin{center}
	\itshape
	Document, avec son source \LaTeX, disponible sur la page
	
	\url{https://github.com/bc-writing/drafts}.
\end{center}


\bigskip


\begin{center}
	\hrule\vspace{.3em}
	{
		\fontsize{1.35em}{1em}\selectfont
		\textbf{Mentions \enquote{légales}}
	}
			
	\vspace{0.45em}
	\small
	\doclicenseThis
	\hrule
\end{center}


\setcounter{tocdepth}{2}
\tableofcontents


% ------------------ %


\newpage
\section{Ce qui nous intéresse}

Dans l'article \enquote{Note on Products of Consecutive Integers}
\footnote{
	J. London Math. Soc. 14 (1939).
},
Paul Erdos démontre que pour tout couple $(n, k) \in \NNs \times \NNs$\,, le produit d'entiers consécutifs $\dprod_{i = 0}^{k} (n + i)$ n'est jamais le carré d'un entier. Dans ce court document, nous commençons par étudier quelques cas particuliers de façon \enquote{adaptative}\,, puis nous proposons ensuite une méthode efficace
\footnote{
	Cette méthode s'appuie sur une représentation trouvée dans \href{https://web.archive.org/web/20171110144534/http://mathforum.org/library/drmath/view/65589.html}{un message archivé} que l'auteur a consulté le 28 janvier 2024.
	Voir \url{https://web.archive.org/web/20171110144534/http://mathforum.org/library/drmath/view/65589.html}\,.
},
et semi-automatisable, pour gérer tous ces premiers cas, ainsi que d'autres. 


% ------------------ %


\section{Notations utilisées}

Dans la suite, nous utiliserons les notations suivantes.
\begin{itemize}
	\item $\PP$ désigne l'ensemble des nombres premiers.
	
%	\item Pour $p \in \PP$\,, $\primefield$ désigne le corps fini à $p$ éléments.
	
	\item $\forall (p ; n) \in \PP \times \NNs$\,, $\padicval{n}$ est la valuation $p$-adique de $n$\,.

	\item $2\,\NN$ désigne l'ensemble des nombres naturels pairs.
	
	\item $2\,\NN + 1$ désigne l'ensemble des nombres naturels impairs.
	
	\item $\forall (n , m) \in \NN^2$, $n \vee m$ désigne le PPCM de $n$ et $m$.

	\item $\forall (n , m) \in \NN^2$, $n \wedge m$ désigne le PGCD de $n$ et $m$.

	\item $a \strictdivides b$ signifie que $a \divides b$ et $a \neq b$ (division stricte).
\end{itemize}


% ------------------ %


\section{Prenons du recul}

\subsection{Tableaux de Vogler}

\leavevmode
\smallskip

L'idée de départ est simple : d'après le fait \ref{facto-square}, il semble opportun de se concentrer sur les diviseurs sans facteur carré des facteurs $(n + i)$ de $\consprod = n (n + 1) \cdots (n + k)$\,. 


% ------------------ %


\begin{defi}
	Considérons $(n, k) \in \NNs \times \NN$\,,
	$( a_i )_{0 \leq i \leq k} \subseteq \NNsf$
	et
	$( s_i )_{0 \leq i \leq k} \subseteq \NNssquare$
	tels que 
	$\forall i \in \ZintervalC{0}{k}$\,, $n + i = a_i s_i$\,.
	%
	Cette situation est résumée par le tableau suivant que nous nommerons tableau de Vogler en référence à la discussion où l'auteur a rencontré ce concept.

	\begin{center}
		\begin{tblr}{
			colspec    = {Q[r,$]*{5}{Q[c,$]}},
			vline{2}   = {.95pt},
			vline{3-6} = {dashed},
			hline{2}   = {.95pt}
		}
			n + \bullet
				&  0  
				&  1  
				&  2  
				&  \dots
				&  k
		\\
				&  a_0
				&  a_1
				&  a_2
				&  \dots  
				&  a_k
		\end{tblr}
	\end{center}
%	Si de plus $( a_i )_{0 \leq i \leq k} \subseteq \NNsf$\,, le tableau de Vogler sera dit réduit.
\end{defi}


% ------------------ %


%\newpage
\begin{example}
	Supposons avoir le tableau de Vogler suivant où $n \in \NNs$.

	\begin{center}
		\begin{tblr}{
			colspec    = {Q[r,$]*{4}{Q[c,$]}},
			vline{2}   = {.95pt},
			vline{3-5} = {dashed},
			hline{2}   = {.95pt}
		}
			n + \bullet
				&  0  
				&  1  
				&  2  
				&  3
		\\
				&  2
				&  5
				&  6
				&  1
		\end{tblr}
	\end{center}
	
	Ceci résume la situation suivante. 
	
	\vspace{-1ex}
	\begin{multicols}{2}
	\begin{itemize}
		\item $\exists A \in \NNs$ tel que $n     = 2 A^2$\,.

		\item $\exists B \in \NNs$ tel que $n + 1 = 5 B^2$\,.

		\item $\exists C \in \NNs$ tel que $n + 2 = 6 C^2$\,.

		\item $\exists D \in \NNs$ tel que $n + 3 =   D^2$\,.
	\end{itemize}
	\end{multicols}
\end{example}


% ------------------ %


\begin{fact} \label{vogler-sub-square}
	Dans les tableaux ci-dessous, les puces $\bullet$ indiquent des valeurs quelconques.
	
	\begin{enumerate}
		\item Si nous avons un tableau de Vogler du type suivant, alors $\consprod<k-1> \in \NNssquare$\,.
	\end{enumerate}

	\begin{center}
		\begin{tblr}{
			colspec     = {Q[r,$]*{5}{Q[c,$]}},
			vline{2}    = {.95pt},
			vline{3-6}  = {dashed},
			hline{2}    = {.95pt},
			column{2-Z} = {2em},
			% Subsquare
			cell{2}{6} = {blue!15},
		}
			n + \bullet
				&  0  
				&  1  
				&  \dots
				&  k - 1
				&  k
		\\
				&  \bullet
				&  \bullet
				&  \dots  
				&  \bullet
				&  1
		\end{tblr}
	\end{center}


	\begin{enumerate}[start=2]
		\item Si nous avons un tableau de Vogler du type suivant, alors $\consprod[n+1]<k-1> \in \NNssquare$\,.
	\end{enumerate}

	\begin{center}
		\begin{tblr}{
			colspec     = {Q[r,$]*{5}{Q[c,$]}},
			vline{2}    = {.95pt},
			vline{3-6}  = {dashed},
			hline{2}    = {.95pt},
			column{2-Z} = {2em},
			% Subsquare
			cell{2}{2} = {blue!15},
		}
			n + \bullet
				&  0  
				&  1  
				&  \dots
				&  k - 1
				&  k
		\\
				&  1
				&  \bullet
				&  \dots  
				&  \bullet
				&  \bullet
		\end{tblr}
	\end{center}
\end{fact}


\begin{proof}
	Immédiat via le fait \ref{facto-square}, car nous avons soit $n + k \in \NNssquare$\,, soit $n \in \NNssquare$\,.
\end{proof}


% ------------------ %


\begin{fact} \label{vogler-illegal}
	Soit $(n, d, a) \in ( \NNs )^3$ et $i \in \NN$\,.
	Les tableaux de Vogler ci-après sont impossibles
	(les puces $\bullet$ indiquent des valeurs quelconques). 

	\begin{enumerate}
		\item Pas de facteurs carrés trop près.
	\end{enumerate}
		
    \begin{center}
    	\begin{tblr}{
    		colspec     = {Q[r,$]*{5}{Q[c,$]}},
    		vline{2}    = {.95pt},
    		vline{3-6}  = {dashed},
   			hline{2}    = {.95pt},
			column{2-Z} = {3em},
			% Rejected
			cell{2}{2,6} = {red!15},
    	}
    		n + \bullet
   				&  i 
    			&  i + 1
    			&  \dots
    			&  i + d - 1
    			&  i + d
    	\\
    			&  ad
				&  \bullet
    			&  \dots
				&  \bullet
				&  ad
    	\end{tblr}
    \end{center}


	\begin{enumerate}[start=2]
		\item Pas de facteurs carrés pas trop loin.
	\end{enumerate}
		
    \begin{center}
    	\begin{tblr}{
    		colspec     = {Q[r,$]*{5}{Q[c,$]}},
    		vline{2}    = {.95pt},
    		vline{3-6}  = {dashed},
   			hline{2}    = {.95pt},
			column{2-Z} = {3.75em},
			% Rejected
			cell{2}{2,6} = {red!15},
    	}
    		n + \bullet
   				&  i 
    			&  i + 1
    			&  \dots
    			&  i + 2d - 1
    			&  i + 2d
    	\\
    			&  ad
				&  \bullet
    			&  \dots
				&  \bullet
				&  ad
    	\end{tblr}
    \end{center}
\end{fact}


\begin{proof}
	Tout est contenu dans le fait \ref{diff-square-ko}.
	
	\begin{enumerate}
		\item $n + i = ad A^2$ et $n + i + d = ad B^2$ donnent $ad (B^2 - A^2) = d$\,, puis $a (B^2 - A^2) = 1$, d'où $B^2 - A^2 = 1$ qui ne se peut pas car $B^2 > A^2 \geq 1$\,.

		\item $n + i = ad A^2$ et $n + i + 2 d = ad B^2$ donnent $ad (B^2 - A^2) = 2d$\,, \emph{i.e.} $a (B^2 - A^2) = 2$\,, d'où $B^2 - A^2 \in \setgene{1, 2}$ qui est impossible.
	\end{enumerate}
\end{proof}


% ------------------ %


\subsection{Construire des tableaux de Vogler}

\leavevmode

\smallskip
Pour fabriquer des tableaux de Vogler, nous allons \enquote{multiplier} des tableaux de Vogler partiels.
	

\begin{defi}
	Soient $(n, k, r) \in \NNs \times \NN \times \NNs$\,,
	$( p_j )_{1 \leq j \leq r} \subseteq \PPseq$\,,
	$( \epsilon_{i,j} )_{0 \leq i \leq k, 1 \leq j \leq r} \subseteq \setgene{0, 1}$
	et
	$( f_i )_{0 \leq i \leq k} \subseteq \NNs$
	vérifiant les conditions suivantes.
	%
	\begin{itemize}
		\item $\forall i \in \ZintervalC{0}{k}$\,, 
		$n + i = f_i \cdot \dprod_{j = 1}^{r} p_j^{\,\padicval[p_j]{n+i}}$\,.
		Noter que 
		$\forall i \in \ZintervalC{0}{k}$\,, 
		$\forall j \in \ZintervalC{1}{r}$\,, 
		$\GCD{f_i}{p_j} = 1$\,. 

		\item $\forall i \in \ZintervalC{0}{k}$\,, 
		$\forall j \in \ZintervalC{1}{r}$\,,
		$\padicval[p_j]{n+i} \equiv \epsilon_{i,j}$ modulo $2$\,.
	\end{itemize}
	
	Cette situation est résumée par le tableau suivant qui sera nommé tableau de Vogler partiel
	\footnote{
		Noter que $\forall i \in \ZintervalC{0}{k}$\,, $\forall j \in \ZintervalC{1}{r}$\,, $p_j^{\,\epsilon_{i,j}} \in \setgene{1, p_j}$\,.
	}.

	\begin{center}
		\begin{tblr}{
			colspec    = {Q[r,$]*{5}{Q[c,$]}},
			vline{2}   = {.95pt},
			vline{3-6} = {dashed},
			hline{2}   = {.95pt},
			column{2-Z} = {2.75em},
		}
			n + \bullet
				&  0  
				&  1  
				&  2  
				&  \dots
				&  k
		\\
			( p_j )_{1 \leq j \leq r}
				&  \dprod_{j = 1}^{r} p_j^{\,\epsilon_{0,j}}
				&  \dprod_{j = 1}^{r} p_j^{\,\epsilon_{1,j}}
				&  \dprod_{j = 1}^{r} p_j^{\,\epsilon_{2,j}}
				&  \dots  
				&  \dprod_{j = 1}^{r} p_j^{\,\epsilon_{k,j}}
		\end{tblr}
	\end{center}
\end{defi}


% ------------------ %


\begin{example}
	Supposons avoir le tableau de Vogler partiel suivant où $n \in \NNs$.

	\begin{center}
		\begin{tblr}{
			colspec    = {Q[r,$]*{4}{Q[c,$]}},
			vline{2}   = {.95pt},
			vline{3-5} = {dashed},
			hline{2}   = {.95pt},
%			column{2-Z} = {1.75em},
		}
			n + \bullet
				&  0  
				&  1  
				&  2  
				&  3
		\\
			(2, 3)
				&  2
				&  6
				&  1
				&  3
		\end{tblr}
	\end{center}
	
	Ceci résume la situation suivante. 
	%
	\begin{itemize}
		\item $\exists (a, \alpha, A) \in \NN^3 \times \NNs$
		      tel que $A \wedge 6 = 1$ 
		      et
		      $n     = 2^{2a+1} 3^{2\alpha} A$\,.
		
		\item $\exists (b, \beta, B) \in \NN^3 \times \NNs$
		      tel que $B \wedge 6 = 1$
		      et
		      $n + 1 = 2^{2b+1} 3^{2\beta+1} B$\,.
		
		\item $\exists (c, \gamma, C) \in \NN^3 \times \NNs$
		      tel que $C \wedge 6 = 1$
		      et
		      $n + 2 = 2^{2c} 3^{2\gamma} C$\,.
		
		\item $\exists (d, \delta, D) \in \NN^3 \times \NNs$
		      tel que $D \wedge 6 = 1$
		      et
		      $n + 3 = 2^{2d} 3^{2\delta+1} D$\,.
	\end{itemize}
\end{example}


% ------------------ %


\begin{example}
	La multiplication de deux tableaux de Vogler partiels est \enquote{naturelle} lorsqu'elle porte sur des suites 
	$( p_j )_{1 \leq j \leq r} \subseteq \PPseq$ 
	et
	$( q_j )_{1 \leq j \leq r} \subseteq \PPseq$
	d'intersection vide, c'est-à-dire sans nombre premier commun.
	Considérons les deux tableaux de Vogler partiels suivants où l'on note $2$ et $3$ au lieu de $(2)$ et $(3)$\,.
	
	\vspace{-1.5ex}
	\begin{multicols}{2}
	\begin{center}
		\begin{tblr}{
			colspec    = {Q[r,$]*{4}{Q[c,$]}},
			vline{2}   = {.95pt},
			vline{3-5} = {dashed},
			hline{2}   = {.95pt}
		}
			n + \bullet
				&  0  
				&  1  
				&  2  
				&  3
		\\
			2
				&  1
				&  2
				&  1
				&  2
		\end{tblr}
	\end{center}

	\begin{center}
		\begin{tblr}{
			colspec    = {Q[r,$]*{4}{Q[c,$]}},
			vline{2}   = {.95pt},
			vline{3-5} = {dashed},
			hline{2}   = {.95pt}
		}
			n + \bullet
				&  0  
				&  1  
				&  2  
				&  3
		\\
			3
				&  3
				&  1
				&  1
				&  3
		\end{tblr}
	\end{center}
	\end{multicols}


	\vspace{-1ex}
	La multiplication de ces tableaux de Vogler partiels est le tableau de Vogler partiel suivant.

	\begin{center}
		\begin{tblr}{
			colspec    = {Q[r,$]*{4}{Q[c,$]}},
			vline{2}   = {.95pt},
			vline{3-5} = {dashed},
			hline{2}   = {.95pt}
		}
			n + \bullet
				&  0  
				&  1  
				&  2  
				&  3
		\\
			(2, 3)
				&  3
				&  2
				&  1
				&  6
		\end{tblr}
	\end{center}
	
	Ceci résume la situation suivante, avec des notations \enquote{évidentes}\,, qui est équivalente à ce que donnent les deux premiers tableaux de Vogler partiels. 
	
	
	\vspace{-1ex}
	\begin{multicols}{2}
	\begin{itemize}
		\item $A \wedge 6 = 1$
		      et
		      $n     = 2^{2a}   3^{2\alpha+1} A$\,.

		\item $B \wedge 6 = 1$
		      et
		      $n + 1 = 2^{2b+1} 3^{2\beta}    B$\,.

		\item $C \wedge 6 = 1$
		      et
		      $n + 2 = 2^{2c}   3^{2\gamma}   C$\,.

		\item $D \wedge 6 = 1$
		      et
		      $n + 3 = 2^{2d+1} 3^{2\delta+1} D$\,.
	\end{itemize}
	\end{multicols}
\end{example}


% ------------------ %


\begin{fact} \label{vogler-multiple}
	Dans la deuxième ligne d'un tableau de Vogler partiel relatif à un unique nombre premier $p$\,, les valeurs $p$ sont séparées par exactement $(p-1)$ valeurs $1$\,.
\end{fact}


\begin{proof}
	Penser aux multiples de $p$\,.
\end{proof}


% ------------------ %


\begin{fact} \label{vogler-parity-square}
	$\forall (n, k, p) \in \NNs \times \NN \times \PP$\,,
	si $\consprod \in \NNsquare$\,, 
	alors dans le tableau de Vogler partiel relatif uniquement au nombre premier $p$\,, et associé à $\consprod$\,, le nombre de valeurs $p$ est forcément pair.
\end{fact}


\begin{proof}
	Évident, mais très pratique, comme nous le verrons dans la suite.
\end{proof}




\foreach \k in {2,...,5} {
%\foreach \k in {5} {
	\section{Application au cas de \k\ facteurs} \label{apply-\k}

	\input{\contentdir/case-\k}
}


\section{Et après ?}

La méthode présentée ci-dessus permet de faire appel à un programme pour n'avoir à traiter à la main, et aux neurones, que certains tableaux de Vogler problématiques comme nous avons dû le faire dans la section \ref{apply-4}.
Expliquons cette tactique semi-automatique en traitant le cas de $6$ facteurs.


\begin{enumerate}
	\item On raisonne par l'absurde en supposant que $\consprod<6> \in \NNssquare$\,.
	
 
	\item On fabrique la liste $\setproba{P}$ des diviseurs premiers stricts de $6$ : nous avons juste $2$\,, $3$ et $5$\,.
	Notons qu'avec $7$ facteurs, nous n'aurions pas garder $7$ car il est forcément de valuation paire dans chaque facteur $(n + i)$ de $\consprod<7>$ si $\consprod<7> \in \NNssquare$\,.
	
 
	\item Pour chaque élément $p$ de $\setproba{P}$\,, on construit la liste $\setproba*{V}{p}$ des $p$-tableaux de Vogler possibles relativement à $\consprod<6>$\,.
	
 
	\item Via les listes $\setproba*{V}{p}$\,, on calcule toutes les multiplications de $p$-tableaux de Vogler, et pour chacune d'elles on ne la garde que si elle ne vérifie aucune des conditions suivantes, celles du dernier cas devant être indiquées à la main au programme.
	%
	\begin{enumerate}
		\item Le tableau \enquote{produit} commence, ou se termine, par la valeur $1$\,. Dans ce cas, on sait par récurrence que le tableau produit n'est pas possible (voir le fait \ref{vogler-sub-square}). 

		\item L'une des interdictions du fait \ref{illegal-vogler} est validée par le tableau \enquote{produit}\,.
		
		\item Le tableau \enquote{produit} contient un sous-tableau que nous savons impossible suite à un raisonnement humain fait \emph{localement}\,, c'est-à-dire que seul les facteurs indiqués dans le sous-tableau, et le sous-tableau lui-même sont utilisés pour raisonner.
		Comme c'est ce qui a été fait en fin de section \ref{apply-4}, nous pouvons indiquer les deux sous-tableaux impossibles suivants.
	\end{enumerate}
\end{enumerate}

\begin{center}
	\begin{tblr}{
		colspec    = {Q[r,$]*{4}{Q[c,$]}},
		vline{2}   = {.95pt},
		vline{3-5} = {dashed},
		hline{2}   = {.95pt}
	}
		m + \bullet
			&  0  
			&  1 
			&  2 
			&  3
	\\
			&  6
			&  1
			&  2
			&  3
	\\
			&  3
			&  2
			&  1
			&  6
	\end{tblr}
	
	\smallskip
	
	\emph{\small Deux sous-tableaux de Vogler impossibles.}
\end{center}


{\Huge YAPLUKA !}



% ------------------ %


\section{Sources utilisées} \label{sources}

% ------------------ %


\bigskip
\textbf{Faits \ref{case-5}, \ref{case-7}, \ref{case-9}, \ref{case-10}, \ref{case-11}, \ref{case-12} et \ref{case-13}.}
	
\smallskip
\noindent
Un échange consulté le 13 février 2024, et titré
\emph{\enquote{\href{https://math.stackexchange.com/q/2361670/52365}{Product of 10 consecutive integers can never be a perfect square}}} 
sur le site \url{https://math.stackexchange.com}\,.

\smallskip
\noindent
\emph{La démonstration indiquée est celle du fait \ref{case-10}, une preuve venant d'une source Wordpress donnée dans une réponse de cet échange, mais cette source est très expéditive...}


% ------------------ %


\bigskip
\textbf{Fait \ref{case-6}.}
	
\smallskip
\noindent
Un échange consulté le 28 janvier 2024, et titré
\emph{\enquote{\href{https://math.stackexchange.com/q/90894/52365}{product of six consecutive integers being a perfect numbers}}} 
sur le site \url{https://math.stackexchange.com}\,.


% ------------------ %


\bigskip
\textbf{Fait \ref{case-7}.}
	
\smallskip
\noindent
Un échange consulté le 3 février 2024, et titré
\emph{\enquote{\href{https://math.stackexchange.com/q/2334887/52365}{Proof that the product of 7 successive positive integers is not a square}}} 
sur le site \url{https://math.stackexchange.com}\,.
	
\smallskip
\noindent
\emph{Il manque certaines justifications dans la démonstration donnée dans cet échange.}


% ------------------ %


\bigskip
\textbf{Fait \ref{case-8}.}
	
\smallskip
\noindent
Le document \emph{\enquote{Products of consecutive Integers}} de Vadim Bugaenko, Konstantin Kokhas, Yaroslav Abramov et Maria Ilyukhina obtenu via un moteur de recherche le 28 février 2024.






% ------------------ %


%\bigskip
\newpage

\hrule

\section{AFFAIRE À SUIVRE...}

\bigskip

\hrule

\end{document}
