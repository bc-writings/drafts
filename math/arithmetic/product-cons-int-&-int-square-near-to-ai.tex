\documentclass[12pt]{amsart}
%\usepackage[T1]{fontenc}
%\usepackage[utf8]{inputenc}

\usepackage[top=1.95cm, bottom=1.95cm, left=2.35cm, right=2.35cm]{geometry}

\usepackage{hyperref}
\usepackage{enumitem}
\usepackage{tcolorbox}
\usepackage{float}
\usepackage{cleveref}
\usepackage{multicol}
\usepackage{fancyvrb}
\usepackage{enumitem}
\usepackage{amsmath}
\usepackage{textcomp}
\usepackage{numprint}
\usepackage{tabularray}
\usepackage[french]{babel}
\frenchsetup{StandardItemLabels=true}
\usepackage{csquotes}
\usepackage{scalerel}
\usepackage{piton}

\NewPitonEnvironment{Python}{}
  {\begin{tcolorbox}}
  {\end{tcolorbox}}
  
\SetPitonStyle{
 	Number = ,
    String = \itshape ,
    String.Doc = \color{gray} \slshape ,
    Operator = ,
    Operator.Word = \bfseries ,
    Name.Builtin = ,
    Name.Function = ,
    Comment = \color{gray} ,
    Comment.LaTeX = \normalfont \color{gray},
    Keyword = \bfseries ,
    Name.Namespace = ,
    Name.Class = ,
    Name.Type = ,
    InitialValues = \color{gray}
}

\usepackage[
    type={CC},
    modifier={by-nc-sa},
	version={4.0},
]{doclicense}

\newcommand\floor[1]{\left\lfloor #1 \right\rfloor}

\usepackage{tnsmath}
  

\newtheorem{fact}{Fait}[section]
\newtheorem{defi}{Définition}[section]
\newtheorem{example}{Exemple}[section]
\newtheorem{remark}{Remarque}[section]
\newtheorem*{proof*}{Une preuve alternative}

\npthousandsep{.}
\setlength\parindent{0pt}

\floatstyle{boxed} 
\restylefloat{figure}


\DeclareMathOperator{\taille}{\text{\normalfont\texttt{taille}}}

\newcommand{\logicneg}{\text{\normalfont non \!}}

\newcommand\sqseq[2]{\fbox{$#1$}_{\,\,#2}}


\DefineVerbatimEnvironment{rawcode}%
	{Verbatim}%
	{tabsize=4,%
	 frame=lines, framerule=0.3mm, framesep=2.5mm}
	 
	 
\newcommand\contentdir{\jobname}

\newcommand\NNsf{\NN_{\kern-1pt s\kern-1pt f}}

\newcommand\NNsquare{\seqsuprageo{\NN}{}{}{}{2}}
\newcommand\NNssquare{\seqsuprageo{\NN}{*}{}{}{2}}

\NewDocumentCommand\PPseq{ O{n} }{\seqsuprageo{\PP}{}{sc}{#1}{}}

\NewDocumentCommand\GCD{ m  m }{#1 \wedge #2}

\NewDocumentCommand\padicval{ O{p} m }{v_{#1}(#2)}
\NewDocumentCommand\consprod{ O{n} D<>{k} }{\pi_{#1}^{#2}}

\NewDocumentCommand\alt{ m }{\textbf{[A\kern1pt#1]}}

\newcommand\mycheckmark{{\color{green!60!black} \checkmark}}
\newcommand\myboxtimes{{\color{red!80!black} \boxtimes}}



\begin{document}

\title{BROUILLON - Carrés parfaits et produits d'entiers consécutifs -- Une méthode efficace}
\author{Christophe BAL}
\date{25 Jan. 2024 -- 7 Fév. 2024}

\maketitle

\begin{center}
	\itshape
	Document, avec son source \LaTeX, disponible sur la page
	
	\url{https://github.com/bc-writing/drafts}.
\end{center}


\bigskip


\begin{center}
	\hrule\vspace{.3em}
	{
		\fontsize{1.35em}{1em}\selectfont
		\textbf{Mentions \enquote{légales}}
	}
			
	\vspace{0.45em}
	\small
	\doclicenseThis
	\hrule
\end{center}


\setcounter{tocdepth}{2}
\tableofcontents


% ------------------ %


\newpage
\section{Ce qui nous intéresse}

Dans l'article \enquote{Note on Products of Consecutive Integers}
\footnote{
	J. London Math. Soc. 14 (1939).
},
Paul Erdos démontre que pour tout couple $(n, k) \in \NNs \times \NNs$\,, le produit de $(k+1)$ entiers consécutifs $n (n + 1) \cdots (n + k)$ n'est jamais le carré d'un entier. 

\smallskip

Il est facile de trouver sur le web des preuves à la main de $n(n+1) \cdots (n + k) \notin \NNssquare$ pour $k \in \ZintervalC{1}{7}$\,.
Bien que certaines de ces preuves soient très sympathiques, leur lecture ne fait pas ressortir de schéma commun de raisonnement.
%
Dans ce document, nous allons tenter de limiter au maximum l'emploi de fourberies déductives en présentant une méthode très élémentaire
\footnote{
	Cette méthode s'appuie sur une représentation trouvée dans \href{https://web.archive.org/web/20171110144534/http://mathforum.org/library/drmath/view/65589.html}{un message archivé} : voir la section \ref{sources}.
},
efficace, et semi-automatisable, pour démontrer, avec peu d'efforts cognitifs, les premiers cas d'impossibilité.




% ------------------ %


%%\bigskip
%\section{Notations utilisées}
%
%Dans la suite, nous utiliserons les notations suivantes.
\begin{itemize}
	\item $2\,\NN$ désigne l'ensemble des nombres naturels pairs.
	
	\item $2\,\NN + 1$ désigne l'ensemble des nombres naturels impairs.
	
	\item $\forall (n , m) \in \NN^2$, $n \vee m$ désigne le PPCM de $n$ et $m$.

	\item $\forall (n , m) \in \NN^2$, $n \wedge m$ désigne le PGCD de $n$ et $m$.

	\item $a \strictdivides b$ signifie que $a \divides b$ et $a \neq b$ (division stricte).

	\item $\PP$ désigne l'ensemble des nombres premiers.
	
	\item $\forall (p ; n) \in \PP \times \NNs$\,, $\padicval{n} \in \NN$ est la valuation $p$-adique de $n$\,, c'est-à-dire $p^{\padicval{n}} \divides n$\,, mais $p^{\padicval{n} + 1} \ndivides n$\,.
\end{itemize}
%
%
%% ------------------ %
%
%
%%\newpage
%\section{Les carrés parfaits}
%
%\leavevmode
\smallskip

Via $N^2 - M^2 = (N - M)(N + M)$\,, il est immédiat de noter que 
$\forall (N, M) \in \NNs \times \NNs$\,, si $N > M$\,, alors $N^2 - M^2 \geq 3$\,. Le fait suivant précise ceci.


\begin{fact} \label{dist-square}
	$\forall (N, M) \in \NNs \times \NNs$, 
	si $N > M$\,, alors $N^2 - M^2 = \dsum_{k=M+1}^{N} (2 k - 1)$\,.
\end{fact}


% ------------------ %


\begin{proof}
	Il suffit d'utiliser $N^2 = \dsum_{k=1}^{N} (2 k - 1)$\,.
\end{proof}

%
%
%
%% ------------------ %
%
%
%\newpage
%\section{Prenons du recul}
%
%\leavevmode
\smallskip

Les exemples proposés précédemment sont tous sympathiques, mais, malheureusement, ils ne nous ont pas permis de noter un schéma général de raisonnement autre que de découvrir des carrés mal espacés : par exemple, tomber sur $N^2 - M^2 = 4$ lève une contradiction du fait \ref{dist-square}.
Tentons donc de limiter au maximum l'emploi de fourberies déductives ; pour cela, commençons par noter le fait suivant.


\begin{fact} \label{facto-square}
	$\forall n \in \NNssquare$\,, s'il existe $m \in \NNssquare$ tel que $n =  f m$ alors $f  \in \NNssquare$\,.
\end{fact}


\begin{proof}
	Il suffit de passer via les décompositions en facteurs premiers de $n$\,, $m$ et $f$\,.
\end{proof}


% ------------------ %


Concentrons-nous sur les diviseurs sans facteur carré des facteurs $(n + i)$ de $\consprod\ = \dprod_{i = 0}^{k} (n + i)$\,. 


\begin{defi}
	Soient $(n, k) \in \big( \NNs \big)^2$\,,
	$( a_i )_{0 \leq i \leq k} \subseteq \NNs$
	et
	$( s_i )_{0 \leq i \leq k} \subseteq \NNssquare$
	tels que 
	$\forall i \in \ZintervalC{0}{k}$\,, $n + i = a_i s_i$\,.
	%
	Ce type de situation sera résumé par le tableau suivant que nous nommerons tableau de Vogler en référence à la discussion où l'auteur a découvert ce concept.

	\begin{center}
		\begin{tblr}{
			colspec    = {Q[r,$]*{5}{Q[c,$]}},
			vline{2}   = {.95pt},
			vline{3-6} = {dashed},
			hline{2}   = {.95pt}
		}
			n + \bullet
				&  0  
				&  1  
				&  2  
				&  \dots
				&  k
		\\
				&  a_0
				&  a_1
				&  a_2
				&  \dots  
				&  a_k
		\end{tblr}
	\end{center}
\end{defi}


% ------------------ %


\begin{example}
	Supposons avoir le tableau de Vogler suivant où $n \in \NNs$.

	\begin{center}
		\begin{tblr}{
			colspec    = {Q[r,$]*{4}{Q[c,$]}},
			vline{2}   = {.95pt},
			vline{3-5} = {dashed},
			hline{2}   = {.95pt}
		}
			n + \bullet
				&  0  
				&  1  
				&  2  
				&  3
		\\
				&  2
				&  5
				&  6
				&  1
		\end{tblr}
	\end{center}
	
	Ceci résume la situation suivante. 
	
	\begin{itemize}
		\item $\exists A \in \NNs$ tel que $n     = 2 A^2$\,.

		\item $\exists B \in \NNs$ tel que $n + 1 = 5 B^2$\,.

		\item $\exists C \in \NNs$ tel que $n + 2 = 6 C^2$\,.

		\item $\exists D \in \NNs$ tel que $n + 3 = D^2  $\,.
	\end{itemize}
\end{example}


% ------------------ %


\begin{fact} \label{illegal-vogler}
	Soit $(n, d, i, a) \in \big( \NNs \big)^4$.
	Les tableaux de Vogler ci-après sont impossibles. 


	\begin{enumerate}
		\item Pas de facteurs carrés consécutifs.
	\end{enumerate}
		
    \begin{center}
    	\begin{tblr}{
    		colspec  = {Q[r,$]*{2}{Q[c,$]}},
    		vline{2} = {.95pt},
    		vline{3} = {dashed},
   			hline{2} = {.95pt}
    	}
    		n + \bullet
   				&  i 
    			&  i + 1
    	\\
    			&  a
				&  a
    	\end{tblr}
    \end{center}


	\begin{enumerate}[start=2]
		\item Pas de facteurs carrés trop près (les puces $\bullet$ indiquent des valeurs inconnues).
	\end{enumerate}
		
    \begin{center}
    	\begin{tblr}{
    		colspec    = {Q[r,$]*{5}{Q[c,$]}},
    		vline{2}   = {.95pt},
    		vline{3-6} = {dashed},
   			hline{2}   = {.95pt}
    	}
    		n + \bullet
   				&  i 
    			&  i + 1
    			&  \dots
    			&  i + d - 1
    			&  i + d
    	\\
    			&  ad
				&  \bullet
    			&  \dots
				&  \bullet
				&  ad
    	\end{tblr}
    \end{center}


	\begin{enumerate}[start=3]
		\item Pas de facteurs carrés pas trop loin.
	\end{enumerate}
		
    \begin{center}
    	\begin{tblr}{
    		colspec    = {Q[r,$]*{5}{Q[c,$]}},
    		vline{2}   = {.95pt},
    		vline{3-6} = {dashed},
   			hline{2}   = {.95pt}
    	}
    		n + \bullet
   				&  i 
    			&  i + 1
    			&  \dots
    			&  i + 2d - 1
    			&  i + 2d
    	\\
    			&  ad
				&  \bullet
    			&  \dots
				&  \bullet
				&  ad
    	\end{tblr}
    \end{center}
\end{fact}


\begin{proof}
	Tout est contenu dans le fait \ref{dist-square}.
	
	\begin{enumerate}
		\item Ici, $n + i = a A^2$ et $n + i + 1 = a B^2$ donnent $a (B^2 - A^2) = 1$\,, d'où $B^2 - A^2 = 1$ qui ne se peut pas car nous sommes dans $\NNs$.

		\item Ici, $n + i = ad A^2$ et $n + i + d = ad B^2$ donnent $ad (B^2 - A^2) = d$\,, \emph{i.e.} $a (B^2 - A^2) = 1$ qui a été rejeté dans l'explication précédente.

		\item Ici, $n + i = ad A^2$ et $n + i + 2 d = ad B^2$ donnent $ad (B^2 - A^2) = 2d$\,, \emph{i.e.} $a (B^2 - A^2) = 2$\,, d'où $B^2 - A^2 \in \setgene{1, 2}$ qui est impossible.
	\end{enumerate}
\end{proof}


% ------------------ %


Pour fabriquer des tableaux de Vogler, nous allons \enquote{multiplier} des $d$-tableaux de Vogler qui sont moins précis et définis comme suit.
	

\begin{defi}
	Soient $(n, k, d) \in \big( \NNs \big)^3$\,,
	$( q_i )_{0 \leq i \leq k} \subseteq \NN$\,,
	$( \epsilon_i )_{0 \leq i \leq k} \subseteq \setgene{0, 1}$
	et
	$( f_i )_{0 \leq i \leq k} \subseteq \NNs$
	tels que 
	$\forall i \in \ZintervalC{0}{k}$\,, $n + i = d^{\,2q_i + \epsilon_i} f_i$ avec $f_i \wedge d = 1$\,.
	%
	Ce type de situation sera résumé par le tableau suivant que nous nommerons $d$-tableau de Vogler (noter que $d^{\,\epsilon_i} \in \setgene{1, d}$).

	\begin{center}
		\begin{tblr}{
			colspec    = {Q[r,$]*{5}{Q[c,$]}},
			vline{2}   = {.95pt},
			vline{3-6} = {dashed},
			hline{2}   = {.95pt}
		}
			n + \bullet
				&  0  
				&  1  
				&  2  
				&  \dots
				&  k
		\\
			d
				&  d^{\,\epsilon_0}
				&  d^{\,\epsilon_1}
				&  d^{\,\epsilon_2}
				&  \dots  
				&  d^{\,\epsilon_k}
		\end{tblr}
	\end{center}
\end{defi}


% ------------------ %


\begin{example}
	Supposons avoir le $5$-tableau de Vogler suivant où $n \in \NNs$.

	\begin{center}
		\begin{tblr}{
			colspec    = {Q[r,$]*{4}{Q[c,$]}},
			vline{2}   = {.95pt},
			vline{3-5} = {dashed},
			hline{2}   = {.95pt}
		}
			n + \bullet
				&  0  
				&  1  
				&  2  
				&  3
		\\
			5
				&  1
				&  5
				&  1
				&  1
		\end{tblr}
	\end{center}
	
	Ceci résume la situation suivante. 
	
	\begin{itemize}
		\item $\exists (a, A) \in \NN \times \NNs$
		      tel que $A \wedge 5 = 1$ et $n     = 5^{\,2a} A$\,.
		
		\item $\exists (b, B) \in \NN \times \NNs$
		      tel que $B \wedge 5 = 1$ et $n + 1 = 5^{\,2b + 1} B$\,.
		
		\item $\exists (c, C) \in \NN \times \NNs$
		      tel que $C \wedge 5 = 1$ et $n + 2 = 5^{\,2c} C$\,.
		
		\item $\exists (d, D) \in \NN \times \NNs$
		      tel que $D \wedge 5 = 1$ et $n + 3 = 5^{\,2d} D$\,.
	\end{itemize}
\end{example}


% ------------------ %


\begin{example}
	La multiplication de deux $d$-tableaux de Vogler est \enquote{naturelle}\,. Considérons le $2$-tableau de Vogler et le $3$-tableau de Vogler suivants.
	
	\begin{multicols}{2}
	\begin{center}
		\begin{tblr}{
			colspec    = {Q[r,$]*{4}{Q[c,$]}},
			vline{2}   = {.95pt},
			vline{3-5} = {dashed},
			hline{2}   = {.95pt}
		}
			n + \bullet
				&  0  
				&  1  
				&  2  
				&  3
		\\
			2
				&  1
				&  2
				&  1
				&  2
		\end{tblr}
	\end{center}

	\begin{center}
		\begin{tblr}{
			colspec    = {Q[r,$]*{4}{Q[c,$]}},
			vline{2}   = {.95pt},
			vline{3-5} = {dashed},
			hline{2}   = {.95pt}
		}
			n + \bullet
				&  0  
				&  1  
				&  2  
				&  3
		\\
			3
				&  3
				&  1
				&  1
				&  3
		\end{tblr}
	\end{center}
	\end{multicols}


	La multiplication de ces $d$-tableaux de Vogler est le $6$-tableau de Vogler suivant.

	\begin{center}
		\begin{tblr}{
			colspec    = {Q[r,$]*{4}{Q[c,$]}},
			vline{2}   = {.95pt},
			vline{3-5} = {dashed},
			hline{2}   = {.95pt}
		}
			n + \bullet
				&  0  
				&  1  
				&  2  
				&  3
		\\
			6
				&  3
				&  2
				&  1
				&  6
		\end{tblr}
	\end{center}
	
	Ceci résume la situation suivante avec des notations \enquote{évidentes}\,. 
	
	\begin{itemize}
		\item $A \wedge 6 = 1$
		      et
		      $n     = 2^{\,2a}   3^{\,2\alpha+1} A$\,.

		\item $B \wedge 6 = 1$
		      et
		      $n + 1 = 2^{\,2b+1} 3^{\,2\beta}    B$\,.

		\item $C \wedge 6 = 1$
		      et
		      $n + 2 = 2^{\,2c}   3^{\,2\gamma}   C$\,.

		\item $D \wedge 6 = 1$
		      et
		      $n + 3 = 2^{\,2d+1} 3^{\,2\delta+1} D$\,.
	\end{itemize}
\end{example}


% ------------------ %


\begin{fact}
	Dans la deuxième ligne d'un $d$-tableau de Vogler, les valeurs $d$ sont séparées par exactement $(d-1)$ valeurs $1$\,.
\end{fact}


\begin{proof}
	Penser aux multiples de $d$\,.
\end{proof}


% ------------------ %


\begin{fact} \label{vogler-parity-square}
	$\forall p \in \PP$\,, si $\consprod\ \in \NNsquare$\,, alors dans un $p$-tableau de Vogler, le nombre de valeurs $p$ est forcément pair.
\end{fact}


\begin{proof}
	Évident, mais très pratique, comme nous le verrons dans la suite.
\end{proof}


%
%
%
%% ------------------ %
%
%
%\foreach \k in {2,...,5} {
%%\foreach \k in {5} {
%%	\newpage
%	\section{Application au cas de \k\ facteurs} \label{apply-\k}
%
%	\input{\contentdir/case-\k}
%}



% ------------------ %


\newpage
\section{Et après ?}

La méthode présentée ci-dessus permet de faire appel à un programme pour n'avoir à traiter à la main, et aux neurones, que certains tableaux de Vogler problématiques comme nous avons dû le faire dans la section \ref{apply-4}.
Expliquons cette tactique semi-automatique en traitant le cas de $6$ facteurs.


\begin{enumerate}
	\item On raisonne par l'absurde en supposant que $\consprod<6> \in \NNssquare$\,.
	
 
	\item On fabrique la liste $\setproba{P}$ des diviseurs premiers stricts de $6$ : nous avons juste $2$\,, $3$ et $5$\,.
	Notons qu'avec $7$ facteurs, nous n'aurions pas garder $7$ car il est forcément de valuation paire dans chaque facteur $(n + i)$ de $\consprod<7>$ si $\consprod<7> \in \NNssquare$\,.
	
 
	\item Pour chaque élément $p$ de $\setproba{P}$\,, on construit la liste $\setproba*{V}{p}$ des $p$-tableaux de Vogler possibles relativement à $\consprod<6>$\,.
	
 
	\item Via les listes $\setproba*{V}{p}$\,, on calcule toutes les multiplications de $p$-tableaux de Vogler, et pour chacune d'elles on ne la garde que si elle ne vérifie aucune des conditions suivantes, celles du dernier cas devant être indiquées à la main au programme.
	%
	\begin{enumerate}
		\item Le tableau \enquote{produit} commence, ou se termine, par la valeur $1$\,. Dans ce cas, on sait par récurrence que le tableau produit n'est pas possible (voir le fait \ref{vogler-sub-square}). 

		\item L'une des interdictions du fait \ref{illegal-vogler} est validée par le tableau \enquote{produit}\,.
		
		\item Le tableau \enquote{produit} contient un sous-tableau que nous savons impossible suite à un raisonnement humain fait \emph{localement}\,, c'est-à-dire que seul les facteurs indiqués dans le sous-tableau, et le sous-tableau lui-même sont utilisés pour raisonner.
		Comme c'est ce qui a été fait en fin de section \ref{apply-4}, nous pouvons indiquer les deux sous-tableaux impossibles suivants.
	\end{enumerate}
\end{enumerate}

\begin{center}
	\begin{tblr}{
		colspec    = {Q[r,$]*{4}{Q[c,$]}},
		vline{2}   = {.95pt},
		vline{3-5} = {dashed},
		hline{2}   = {.95pt}
	}
		m + \bullet
			&  0  
			&  1 
			&  2 
			&  3
	\\
			&  6
			&  1
			&  2
			&  3
	\\
			&  3
			&  2
			&  1
			&  6
	\end{tblr}
	
	\smallskip
	
	\emph{\small Deux sous-tableaux de Vogler impossibles.}
\end{center}


{\Huge YAPLUKA !}

Dans le dépôt en ligne où se trouve ce document est placé un programme nommé \verb#vogler-6.py# qui nous fournit les informations suivantes.



% ------------------ %
		

\section{Sources utilisées} \label{sources}

% ------------------ %


\bigskip
\textbf{Fait \ref{case-4}.}
	
%\smallskip
%\noindent
%Voir la source du fait  \ref{case-7}.

\smallskip
\noindent
\emph{La démonstration non algébrique a été impulsée par la source du fait \ref{case-7} donnée plus bas.}


% ------------------ %


\bigskip
\textbf{Fait \ref{case-5}.}
	
\begin{itemize}
	\item Un échange consulté le 28 janvier 2024, et titré 
	\emph{\enquote{\href{https://les-mathematiques.net/vanilla/discussion/comment/351293}{n(n+1)...(n+k) est un carré ?}}} 
	sur le site \url{lesmathematiques.net}\,.

    \smallskip
    \noindent
    \emph{La démonstration via le principe des tiroirs trouve sa source dans cet échange.}


	\item Un échange consulté le 12 février 2024, et titré 
	\emph{\enquote{\href{https://artisticmathematics.quora.com/Is-there-an-easier-way-of-proving-the-product-of-any-5-consecutive-positive-integers-is-never-a-perfect-square}{Is there an easier way of proving the product of any 5 consecutive positive integers is never a perfect square?}}} 
	sur le site \url{www.quora.com/}\,.

    \smallskip
    \noindent
    \emph{La démonstration \enquote{élémentaire} sans le principe des tiroirs vient de cet échange.}


	\item L'article \emph{\enquote{Le produit de 5 entiers consécutifs n'est pas le carré d'un entier.}} de T. Hayashi, Nouvelles Annales de Mathématiques, est consultable via \href{https://numdam.org}{Numdam}\,, la bibliothèque numérique française de mathématiques.
	
	\smallskip
	\noindent
	\emph{Cet article a fortement inspiré la longue preuve.}
\end{itemize}
\vspace{-1ex}


% ------------------ %


\bigskip
\textbf{Fait \ref{case-6}.}
	
\begin{itemize}
	\item Un échange consulté le 28 janvier 2024, et titré
\emph{\enquote{\href{https://math.stackexchange.com/q/90894/52365}{product of six consecutive integers being a perfect numbers}}} 
sur le site \url{https://math.stackexchange.com}\,.
	
	\smallskip
	\noindent
	\emph{La courte démonstration est donnée dans cet échange. Vous y trouverez aussi un très joli argument basé sur les courbes elliptiques rationnelles.}


	\item Une discussion archivée consultée le 28 janvier 2024 : 
	
	\noindent
	\url{https://web.archive.org/web/20171110144534/http://mathforum.org/library/drmath/view/65589.html}\,.
	
	\smallskip
	\noindent
	\emph{Cette discussion a impulsé la preuve fastidieuse, mais facile d'accès, via des tableaux.}
\end{itemize}
\vspace{-1ex}


% ------------------ %


\bigskip
\textbf{Fait \ref{case-7}.}
	
\smallskip
\noindent
Un échange consulté le 3 février 2024, et titré
\emph{\enquote{\href{https://math.stackexchange.com/q/2334887/52365}{Proof that the product of 7 successive positive integers is not a square}}} 
sur le site \url{https://math.stackexchange.com}\,.
	
\smallskip
\noindent
\emph{La courte démonstration est donnée dans cet échange, mais certaines justifications manquent.}


% ------------------ %


\bigskip
\textbf{Fait \ref{case-8}.}
	
\begin{itemize}
	\item Le document \emph{\enquote{Products of consecutive Integers}} de Vadim Bugaenko, Konstantin Kokhas, Yaroslav Abramov et Maria Ilyukhina obtenu via un moteur de recherche le 28 février 2024.


	\item Un échange consulté le 4 février 2024, et titré \emph{\enquote{\href{https://math.stackexchange.com/a/2271715/52365}{How to prove that the product of eight consecutive numbers can't be a number raised to exponent 4?}}} sur le site \url{https://math.stackexchange.com}\,.

    \smallskip
    \noindent
    \emph{La démonstration astucieuse vient de l'une des réponses de cet échange, mais la justification des deux inégalités n'est pas donnée.}
\end{itemize}
\vspace{-1ex}






\smallskip
\noindent



% ------------------ %


\bigskip
\textbf{Fait \ref{case-10}.}
	
\smallskip
\noindent
Un échange consulté le 13 février 2024, et titré
\emph{\enquote{\href{https://math.stackexchange.com/q/2361670/52365}{Product of 10 consecutive integers can never be a perfect square}}} 
sur le site \url{https://math.stackexchange.com}\,.

\smallskip
\noindent
\emph{La démonstration vient d'une source Wordpress donnée dans une réponse de cet échange, mais cette source est très expéditive...}




% ------------------ %


%\bigskip
\newpage

\hrule

\section{AFFAIRE À SUIVRE...}

\bigskip

\hrule

\end{document}