\documentclass[12pt]{amsart}
\usepackage[T1]{fontenc}
\usepackage[utf8]{inputenc}

\usepackage[top=1.95cm, bottom=1.95cm, left=2.35cm, right=2.35cm]{geometry}

\usepackage{hyperref}
\usepackage{enumitem}
\usepackage{tcolorbox}
\usepackage{float}
\usepackage{cleveref}
\usepackage{multicol}
\usepackage{fancyvrb}
\usepackage{enumitem}
\usepackage{amsmath}
\usepackage{textcomp}
\usepackage{numprint}
\usepackage[french]{babel}
\frenchsetup{StandardItemLabels=true}

\usepackage[
    type={CC},
    modifier={by-nc-sa},
	version={4.0},
]{doclicense}

\newcommand\floor[1]{\left\lfloor #1 \right\rfloor}

\usepackage{tnsmath}


\newtheorem{fact}{Fait}[section]
\newtheorem{example}{Exemple}[section]
\newtheorem{remark}{Remarque}[section]
\newtheorem*{proof*}{Preuve}

\npthousandsep{.}
\setlength\parindent{0pt}

\floatstyle{boxed} 
\restylefloat{figure}


\DeclareMathOperator{\taille}{\text{\normalfont\texttt{taille}}}

\newcommand{\logicneg}{\text{\normalfont non \!}}

\newcommand\sqseq[2]{\fbox{$#1$}_{\,\,#2}}


\DefineVerbatimEnvironment{rawcode}%
	{Verbatim}%
	{tabsize=4,%
	 frame=lines, framerule=0.3mm, framesep=2.5mm}
	 
	 
\newcommand\contentdir{\jobname}
\newcommand\ourset{\setproba{S}}
\newcommand\primeset{\setalge{P}}
	 
\begin{document}

\title{BROUILLON - n divise 2\^\,\!n + 1}
\author{Christophe BAL}
\date{16 Jan. 2024}

\maketitle

\begin{center}
	\itshape
	Document, avec son source \LaTeX, disponible sur la page
	
	\url{https://github.com/bc-writing/drafts}.
\end{center}


\bigskip


\begin{center}
	\hrule\vspace{.3em}
	{
		\fontsize{1.35em}{1em}\selectfont
		\textbf{Mentions \og légales \fg}
	}
			
	\vspace{0.45em}
	\doclicenseThis
	\hrule
\end{center}


\setcounter{tocdepth}{2}
\tableofcontents


\newpage
\section{Ce qui nous intéresse}

Dans l'article \enquote{Note on Products of Consecutive Integers}
\footnote{
	J. London Math. Soc. 14 (1939).
},
Paul Erdos démontre que pour tout couple $(n, k) \in \NNs \times \NNs$\,, le produit d'entiers consécutifs $\dprod_{i = 0}^{k} (n + i)$ n'est jamais le carré d'un entier. Dans ce court document, nous commençons par étudier quelques cas particuliers de façon \enquote{adaptative}\,, puis nous proposons ensuite une méthode efficace
\footnote{
	Cette méthode s'appuie sur une représentation trouvée dans \href{https://web.archive.org/web/20171110144534/http://mathforum.org/library/drmath/view/65589.html}{un message archivé} que l'auteur a consulté le 28 janvier 2024.
	Voir \url{https://web.archive.org/web/20171110144534/http://mathforum.org/library/drmath/view/65589.html}\,.
},
et semi-automatisable, pour gérer tous ces premiers cas, ainsi que d'autres. 


\section{Des résultats basiques}

\begin{fact}
	$1 \in \ourset$\,.
\end{fact}

\begin{proof}
	C'est clair.
\end{proof}


% -------------------- %


\begin{fact} \label{no-even}
	$\ourset \cap 2\,\NN = \emptyset$\,.
\end{fact}

\begin{proof}
	Notant $n = 2 k$\,, nous avons $n \in \ourset$ si, et seulement si, $2^{2 k} = 1 + 2 k r$ où $r \in \NN$\,; ceci permet de conclure.
\end{proof}


% -------------------- %


\begin{fact} \label{prime-sol}
	$\ourset \cap \PP = \setgene{3}$\,.
\end{fact}

\begin{proof}
	Travaillons modulo $p$\,.
	Comme $2^p \equiv - 1$
	implique
	$2^{2p} \equiv 1$\,,
	l'ordre $\sigma$ de $2$ vérifie $\sigma \neq 1$\,, et divise à la fois $2p$ et $p-1$\,.
	Ceci n'est possible que si $\sigma = 2$ , d'où $\ourset \cap \PP \subseteq \setgene{3}$\,.
	Comme clairement $3 \divides \big( 2^3 + 1 \big)$\,, on a aussi $3 \in \ourset$\,, d'où finalement $\ourset \cap \PP = \setgene{3}$\,.
\end{proof}


% -------------------- %


\begin{fact} \label{power-of-3}
	$\forall k \in \NNs$, $3^k \in \ourset$\,.
\end{fact}

\begin{proof}
	Nous allons raisonner par récurrence. Cette démonstration montre que le fait \ref{power-of-3} est immédiat à deviner.
    
    \medskip
    
    \textbf{Initialisation pour $k = 1$.}    
    Vu avant.
    
    
    \medskip
    
    \textbf{Étape de récurrence.}    
    On a les implications logiques suivantes.
    
    \medskip
    
    \begin{stepcalc}[style=ar*, ope=\implies]
    	( 3^k ) \divides 2^{( 3^k )} + 1
    \explnext{}
    	\exists m \in \ZZ \,.\, \Big[ 2^{( 3^k )} + 1 = m \cdot 3^k  \Big]
    \explnext{}
    	\exists m \in \ZZ \,.\, \Big[ 2^{( 3^k )} = - 1 + m \cdot 3^k  \Big]
    \explnext{}
    	\exists m \in \ZZ \,.\, \Big[ \big( 2^{( 3^k )} \big)^3 = \big( - 1 + m \cdot 3^k \big)^3  \Big]
    \explnext{}
    	\exists m \in \ZZ \,.\, \Big[ 2^{( 3^{k+1} )} = - 1 + 3 \cdot m \cdot 3^k - 3 \cdot \big( m \cdot 3^k \big)^2 + \big( m \cdot 3^k \big)^3  \Big]
    \explnext*{Besoin de \\ $k \neq 0$ ici.}{}
    	2^{( 3^{k+1} )} \equiv - 1 \mod\!( 3^{k+1} )
    \end{stepcalc}
    
    \smallskip
    
    En résumé, 
    $3^k \divides 2^{( 3^k )} + 1$ 
    implique
    $3^{k+1} \divides 2^{( 3^{k+1} )} + 1$\,.
    
    \medskip
    
    \textbf{Conclusion :} par récurrence sur $k \in \NNs$, nous savons que $3^k \in \ourset$\,.
\end{proof}


% -------------------- %


Finissons cette section par le fait suivant qui nous sera utile plus tard
\footnote{
	Voir le fait \ref{9-divisor}.
}.

\begin{fact} \label{not-7-divisor}
	Si $n \in \ourset$\,, alors $7 \ndivides n$\,.
\end{fact}

\begin{proof}
	Si $7$ divise $n \in \ourset$ alors $2^n \equiv -1$ modulo $7$\,.
	Le tableau suivant démontre que c'est impossible.
	\begin{center}
	\begin{tblr}{
		colspec  = {Q[r,$]*{6}{|Q[c,$]}},
		hline{2} = solid
	}
  		n          
			& 1 & 2 & 3 & 4 & 5 & 6
	\\
      	2^n \mod 7 
			& 2 & 4 & 1 & 2 & 4 & 1
    \end{tblr}
	\end{center}

    \leavevmode
\end{proof}






\section{Un peu de codage pour y voir plus clair}

Sur le site de téléchargement de ce document, dans le sous-dossier \verb+turing/palindrome+, se trouve le fichier \verb+palindrome.txt+ contenant un code utilisable
pour des tests manuels sur le site \url{https://turingmachinesimulator.com}.


\section{Structure de l'ensemble des solutions}

La preuve du fait \ref{power-of-3} amène naturellement au fait suivant.

\begin{fact} \label{prime-divisor}
	$\forall n \in \ourset$, $\forall p \in \PP$\,,
	si $p \divides n$ alors $p n \in \ourset$\,.
\end{fact}

\begin{proof}
	$2^n = -1 + k n$, où $k \in \ZZ$\,, donne :

    \medskip
    
    \begin{stepcalc}[style=sar]
    	2^{p n}
    \explnext{}
    	\big( 2^n \big)^p
    \explnext{}
    	\big( -1 + k n \big)^p
    \explnext{}
    	\dsum_{i=0}^p \binom{p}{i} \, (-1)^{p-i} \cdot (k n)^i
    \explnext*{$p \divides \binom{p}{i}$ si $0 < i < p$}{}
    	- 1 + \dsum_{i=1}^{p-1} p c_i \cdot (-1)^{p-i} \cdot (k n)^i + k^p \cdot n^p
    \explnext*{$n = p q$}{}
    	- 1 + pn \, \dsum_{i=1}^{p-1} c_i \cdot (-1)^{p-i} \cdot k^i n^{i-1} + pq \cdot n \cdot k^p \cdot n^{p-2}
    \end{stepcalc}

    \medskip

    On obtient finalement $2^{p n} = - 1 + pn \cdot r$ avec $r \in \ZZ$ comme souhaité.
\end{proof}


% -------------------- %


Notons au passage que ce qui précède et le fait \ref{prime-sol} donnent un exemple non trivial pour insister sur la nécessité de l'initialisation dans une preuve par récurrence car nous avons :
$\forall p \in \PP$\,, $p^k \divides 2^{( p^k )} + 1$ 
implique
$p^{k+1} \divides 2^{( p^{k+1} )} + 1$\,.


% -------------------- %


\begin{fact} \label{lcm}
	$\forall (n , m) \in \ourset^2$, $n \vee m \in \ourset$\,.
\end{fact}

\begin{proof}
	Soit $r \in \NN$ tel que $n \vee m = n r$\,. Rappelons que d'après le fait \ref{no-even}, aucun des entiers considérés ne peut être pair.
	
	\medskip
	
	Posons $d = 2^n$\,. Comme $r \in 2 \NN + 1$\,, nous avons :
	
	\medskip
	
	\begin{stepcalc}[style = ar*]
		2^{nr} + 1
	\explnext*{$r \in 2 \NN + 1$}{}
		1 - (-d)^r
	\explnext{}
		\big( 1 + d \big) \, \big( 1 + (-d) + \cdots + (-d)^{r-1} \big)
	\end{stepcalc}
	
	\medskip
	
	Comme $n \divides 2^n + 1$\,, c'est-à-dire $n \divides d + 1$\,, nous obtenons que $n \divides 2^{nr} + 1$\,, i.e. $n \divides 2^{n \vee m} + 1$\,.
	Par symétrie des rôles, nous avons aussi $m \divides 2^{n \vee m} + 1$\,.
	Finalement, $n \vee m \in \ourset$\,.
\end{proof}


Notons que la preuve précédente donne une démonstration alternative du fait \ref{prime-divisor} mais pour tout diviseur $p$ non trivial, premier ou non, de $n \in \ourset$\,.
En effet,
posons $d = 2^n$ et partons de nouveau de $2^{np} + 1 = \big( 1 + d \big) \, \big( 1 + (-d) + \cdots + (-d)^{p-1}  \big)$\,.
Comme $p \divides n \divides 2^n + 1$\,, nous avons modulo $p$ :

\medskip

\begin{stepcalc}[style = ar*, ope = \equiv]
	1 + (-d) + \cdots + (-d)^{p-1} 
\explnext*{$d \equiv 2^n \equiv - 1 \mod p$}{}
	1 + 1 + \cdots + 1^{p-1} 
\explnext{}
	p
\explnext{}
	0
\end{stepcalc}

\medskip 

Finalement,
$n \divides d + 1$ et $p \divides \big( 1 + (-d) + \cdots + (-d)^{p-1}  \big)$\, de sorte que $n p \divides 2^{n p} + 1$\,.


% -------------------- %


\begin{fact} \label{product}
	$\forall (n , m) \in \ourset^2$, $n m \in \ourset$\,.
\end{fact}

\begin{proof}
	Nous avons
	$n = \dprod_{p \divides n} p^{\padicval{n}}$
	et
	$m = \dprod_{p \divides m} p^{\padicval{m}}$
	où les produits sont finis.
	Les faits suivants permettent de conclure.
%
	\begin{itemize}
		\item $n \vee m = \dprod_{p \divides m} p^{\max ( \padicval{n} ; \padicval{m} )}$

		\item Si $\max ( \padicval{n} ; \padicval{m} ) < \padicval{n} + \padicval{m}$\,, alors le fait \ref{prime-divisor} donne que $p^\delta \cdot ( n \vee m ) \in \ourset$ où $\delta = \padicval{n} + \padicval{m} - \max ( \padicval{n} ; \padicval{m} )$\,.

		\item En répétant l'opération précédente autant de fois que nécessaire, on arrive à obtenir que $n m \in \ourset$\,.
	\end{itemize}
\end{proof}


% -------------------- %


\begin{fact} \label{gcd}
	$\forall (n , m) \in \ourset^2$, $n \wedge m \in \ourset$\,.
\end{fact}

\begin{proof}
	TODO
\end{proof}


% -------------------- %


\begin{fact}
	$\forall n \in \ourset$, $2^n + 1 \in \ourset$\,.
\end{fact}

\begin{proof}
	Le principe est similaire à la preuve du fait \ref{lcm}.
	Notant $M = 2^n + 1 = n k$ et $d = 2^n$\,, nous avons :

	\medskip
	
	\begin{stepcalc}[style = sar]
		2^M + 1
	\explnext{}
		2^{nk} + 1
	\explnext{}
		\big( 1 + d \big) \, \big( 1 + (-d) + \cdots + (-d)^{k-1} \big)
	\explnext{}
		M \, \big( 1 + (-d) + \cdots + (-d)^{k-1} \big)
	\end{stepcalc}
\end{proof}




\bigskip

\hrule

\section{AFFAIRE À SUIVRE...}

\bigskip

\hrule


% changer de base, essayer la base 3 pour rac 2 (idée de Voncent !)


%\subsection{L'arithmétique modulaire}
%
%\input{rationality-of-int-squares-n-decimal/general/modular}

\end{document}
