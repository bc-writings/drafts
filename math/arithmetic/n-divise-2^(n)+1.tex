\documentclass[12pt]{amsart}
\usepackage[T1]{fontenc}
\usepackage[utf8]{inputenc}

\usepackage[top=1.95cm, bottom=1.95cm, left=2.35cm, right=2.35cm]{geometry}

\usepackage{hyperref}
\usepackage{enumitem}
\usepackage{tcolorbox}
\usepackage{float}
\usepackage{cleveref}
\usepackage{multicol}
\usepackage{fancyvrb}
\usepackage{enumitem}
\usepackage{amsmath}
\usepackage{textcomp}
\usepackage{numprint}
\usepackage{tabularray}
\usepackage[french]{babel}
\frenchsetup{StandardItemLabels=true}
\usepackage{csquotes}

\usepackage[
    type={CC},
    modifier={by-nc-sa},
	version={4.0},
]{doclicense}

\newcommand\floor[1]{\left\lfloor #1 \right\rfloor}

\usepackage{tnsmath}


\newtheorem{fact}{Fait}[section]
\newtheorem{example}{Exemple}[section]
\newtheorem{remark}{Remarque}[section]
\newtheorem*{proof*}{Preuve}

\npthousandsep{.}
\setlength\parindent{0pt}

\floatstyle{boxed} 
\restylefloat{figure}


\DeclareMathOperator{\taille}{\text{\normalfont\texttt{taille}}}

\newcommand{\logicneg}{\text{\normalfont non \!}}

\newcommand\sqseq[2]{\fbox{$#1$}_{\,\,#2}}


\DefineVerbatimEnvironment{rawcode}%
	{Verbatim}%
	{tabsize=4,%
	 frame=lines, framerule=0.3mm, framesep=2.5mm}
	 
	 
\newcommand\contentdir{\jobname}
\newcommand\ourset{\setproba{N}}
%\NewDocumentCommand\primefield{ O{p} }{\setalge{F}_{#1}}
\NewDocumentCommand\padicval{ O{p} m }{v_{#1}(#2)}
\newcommand\strictdivides{\divides\kern.5pt\divides}
\newcommand\dinf{\inf_d}
\newcommand\dsup{\sup_d}

%\RenewDocumentEnvironment{proof}{}{}{}
	 
\begin{document}

\title{BROUILLON - n divise 2\^\,\!n + 1}
\author{Christophe BAL}
\date{16 Jan. 2024 -- 25 Jan. 2024}

\maketitle

\begin{center}
	\itshape
	Document, avec son source \LaTeX, disponible sur la page
	
	\url{https://github.com/bc-writing/drafts}.
\end{center}


\bigskip


\begin{center}
	\hrule\vspace{.3em}
	{
		\fontsize{1.35em}{1em}\selectfont
		\textbf{Mentions \og légales \fg}
	}
			
	\vspace{0.45em}
	\doclicenseThis
	\hrule
\end{center}


\setcounter{tocdepth}{2}
\tableofcontents


\newpage
\section{Ce qui nous intéresse}

Dans l'article \enquote{Note on Products of Consecutive Integers}
\footnote{
	J. London Math. Soc. 14 (1939).
},
Paul Erdos démontre que pour tout couple $(n, k) \in \NNs \times \NNs$\,, le produit de $(k+1)$ entiers consécutifs $n (n + 1) \cdots (n + k)$ n'est jamais le carré d'un entier. 

\smallskip

Il est facile de trouver sur le web des preuves à la main de $n(n+1) \cdots (n + k) \notin \NNssquare$ pour $k \in \ZintervalC{1}{7}$\,.
Bien que certaines de ces preuves soient très sympathiques, leur lecture ne fait pas ressortir de schéma commun de raisonnement.
%
Dans ce document, nous allons tenter de limiter au maximum l'emploi de fourberies déductives en présentant une méthode très élémentaire
\footnote{
	Cette méthode s'appuie sur une représentation trouvée dans \href{https://web.archive.org/web/20171110144534/http://mathforum.org/library/drmath/view/65589.html}{un message archivé} : voir la section \ref{sources}.
},
efficace, et semi-automatisable, pour démontrer, avec peu d'efforts cognitifs, les premiers cas d'impossibilité.





\section{Notations utilisées}

Dans la suite, nous utiliserons les notations suivantes.
\begin{itemize}
	\item $2\,\NN$ désigne l'ensemble des nombres naturels pairs.
	
	\item $2\,\NN + 1$ désigne l'ensemble des nombres naturels impairs.
	
	\item $\forall (n , m) \in \NN^2$, $n \vee m$ désigne le PPCM de $n$ et $m$.

	\item $\forall (n , m) \in \NN^2$, $n \wedge m$ désigne le PGCD de $n$ et $m$.

	\item $a \strictdivides b$ signifie que $a \divides b$ et $a \neq b$ (division stricte).

	\item $\PP$ désigne l'ensemble des nombres premiers.
	
	\item $\forall (p ; n) \in \PP \times \NNs$\,, $\padicval{n} \in \NN$ est la valuation $p$-adique de $n$\,, c'est-à-dire $p^{\padicval{n}} \divides n$\,, mais $p^{\padicval{n} + 1} \ndivides n$\,.
\end{itemize}


\section{Des résultats basiques}

On peut supposer que $a = 1$ i.e. $P(X) = X^4 + b X^3 + c X^2 + b X + 1$.

Dès lors si $P(r) = 0$ alors $r \neq 0$ et $P\left( \dfrac1r \right) = 0$ \emph{(voir ci-dessous)}.

En fait, nous avons :

\medskip

$P(X) = X^4 P\left( \dfrac1X \right)$ : caractérisation des polynômes symétriques de degré $4$

\medskip

$P\,^{\prime}(X) = 4 X^3 P\left( \dfrac1X \right) 
            - X^2 P\,^{\prime}\left( \dfrac1X \right)$

%\medskip
%
%$P\,^{\prime\prime}(X) = 12 X^2 P\left( \dfrac1X \right)  - 4 X P\,^{\prime}\left( \dfrac1X \right)
%		    - 2 X P\,^{\prime}\left( \dfrac1X \right) + P\,^{\prime\prime}\left( \dfrac1X \right)$
%
%$P\,^{\prime\prime}(X) = 12 X^2 P\left( \dfrac1X \right)  
%            - 6 X P\,^{\prime}\left( \dfrac1X \right) 
%            + P\,^{\prime\prime}\left( \dfrac1X \right)$

On en déduit que si $r$ est une racine d'ordre au moins $2$, il en est de même pour $\dfrac1r$.


%
%\section{Un peu de codage pour y voir plus clair}
%
%Un code utilisable pour des tests sur le site \url{https://turingmachinesimulator.com} est disponible sur le site de téléchargement de ce document :
voir dans le sous-dossier \verb+turing/black-and-white-leapfrog+ le fichier \verb+bw-leapfrog.txt+.


\section{Comportement des solutions}

La preuve du fait \ref{power-of-3} amène naturellement au fait suivant.

\begin{fact} \label{prime-divisor}
	$\forall n \in \ourset$, $\forall p \in \PP$\,,
	si $p \divides n$ alors $p n \in \ourset$\,.
\end{fact}

\begin{proof}
	$2^n = -1 + k n$, où $k \in \ZZ$\,, donne :

    \medskip
    
    \begin{stepcalc}[style=sar]
    	2^{p n}
    \explnext{}
    	\big( 2^n \big)^p
    \explnext{}
    	\big( -1 + k n \big)^p
    \explnext{}
    	\dsum_{i=0}^p \binom{p}{i} \, (-1)^{p-i} \cdot (k n)^i
    \explnext*{$p \divides \binom{p}{i}$ si $0 < i < p$}{}
    	- 1 + \dsum_{i=1}^{p-1} p c_i \cdot (-1)^{p-i} \cdot (k n)^i + k^p \cdot n^p
    \explnext*{$n = p q$}{}
    	- 1 + pn \, \dsum_{i=1}^{p-1} c_i \cdot (-1)^{p-i} \cdot k^i n^{i-1} + pq \cdot n \cdot k^p \cdot n^{p-2}
    \end{stepcalc}

    \medskip

    On obtient finalement $2^{p n} = - 1 + pn \cdot r$ avec $r \in \ZZ$ comme souhaité.
\end{proof}


% -------------------- %


Notons au passage que ce qui précède et le fait \ref{prime-sol} donnent un exemple non trivial pour insister sur la nécessité de l'initialisation dans une preuve par récurrence car nous avons :
$\forall p \in \PP$\,, $p^k \divides 2^{( p^k )} + 1$ 
implique
$p^{k+1} \divides 2^{( p^{k+1} )} + 1$\,.


% -------------------- %


\begin{fact} \label{lcm}
	$\forall (n , m) \in \ourset^2$\,, $n \vee m \in \ourset$\,.
\end{fact}

\begin{proof}
	Soit $r \in \NN$ tel que $n \vee m = n r$\,. Rappelons que d'après le fait \ref{no-even}, aucun des entiers considérés ne peut être pair.
	Posant $d = 2^n$\,, nous avons :
	
	\medskip
	
	\begin{stepcalc}[style = ar*]
		2^{nr} + 1
	\explnext*{$r \in 2 \NN + 1$}{}
		1 - (-d)^r
	\explnext{}
		\big( 1 + d \big) \, \big( 1 + (-d) + \cdots + (-d)^{r-1} \big)
	\end{stepcalc}
	
	\medskip
	
	Comme $n \divides 2^n + 1$\,, c'est-à-dire $n \divides d + 1$\,, nous obtenons que $n \divides 2^{nr} + 1$\,, c'est-à-dire $n \divides 2^{n \vee m} + 1$\,.
	Par symétrie des rôles, nous avons aussi $m \divides 2^{n \vee m} + 1$\,.
	Finalement, $n \vee m \in \ourset$\,.
\end{proof}


Notons que la preuve précédente donne une démonstration alternative du fait \ref{prime-divisor} mais pour tout diviseur $p$ non trivial, premier ou non, de $n \in \ourset$\,.
En effet,
posons $d = 2^n$ et partons de nouveau de $2^{np} + 1 = \big( 1 + d \big) \, \big( 1 + (-d) + \cdots + (-d)^{p-1}  \big)$\,.
Comme $p \divides n \divides 2^n + 1$\,, nous avons modulo $p$ :

\medskip

\begin{stepcalc}[style = ar*, ope = \equiv]
	1 + (-d) + \cdots + (-d)^{p-1} 
\explnext*{$d \equiv 2^n \equiv - 1 \mod p$}{}
	1 + 1 + \cdots + 1^{p-1} 
\explnext{}
	p
\explnext{}
	0
\end{stepcalc}

\medskip 

Finalement,
$n \divides d + 1$ et $p \divides \big( 1 + (-d) + \cdots + (-d)^{p-1}  \big)$\, de sorte que $n p \divides 2^{n p} + 1$\,.


% -------------------- %


\begin{fact} \label{product}
	$\forall (n , m) \in \ourset^2$, $n m \in \ourset$\,.
\end{fact}

\begin{proof}
	Nous avons
	$n = \dprod_{p \divides n} p^{\padicval{n}}$
	et
	$m = \dprod_{p \divides m} p^{\padicval{m}}$
	où les produits sont finis.
	Les faits suivants permettent de conclure.

	\begin{itemize}
		\item $n \vee m = \dprod_{p \divides m} p^{\max ( \padicval{n} ; \padicval{m} )}$

		\item Si $\max ( \padicval{n} ; \padicval{m} ) < \padicval{n} + \padicval{m}$\,, alors le fait \ref{prime-divisor} donne que $p^\delta \cdot ( n \vee m ) \in \ourset$ où $\delta = \padicval{n} + \padicval{m} - \max ( \padicval{n} ; \padicval{m} )$\,.

		\item En répétant l'opération précédente autant de fois que nécessaire, on arrive à obtenir que $n m \in \ourset$\,.
	\end{itemize}
\end{proof}


% -------------------- %


\begin{fact} \label{gcd}
	$\forall (n , m) \in \ourset^2$, $n \wedge m \in \ourset$\,.
\end{fact}

\begin{proof}
	TODO
	
	Clairement, $n \wedge m \divides {(2^n + 1) \wedge (2^m + 1)}$\,.
	
	????
	
	$(2^n + 1) \wedge (2^m + 1) \divides ? = ? 2^{n \wedge m} + 1$ lorsque $(m ; n) \in (2 \NN + 1)^2$
\end{proof}


% -------------------- %


\begin{fact}
	$\forall n \in \ourset$, $2^n + 1 \in \ourset$\,.
\end{fact}

\begin{proof}
	Le principe est similaire à la preuve du fait \ref{lcm}.
	Notant $M = 2^n + 1 = n k$ et $d = 2^n$\,, nous avons :

	\medskip
	
	\begin{stepcalc}[style = sar]
		2^M + 1
	\explnext{}
		2^{nk} + 1
	\explnext{}
		\big( 1 + d \big) \, \big( 1 + (-d) + \cdots + (-d)^{k-1} \big)
	\explnext{}
		M \, \big( 1 + (-d) + \cdots + (-d)^{k-1} \big)
	\end{stepcalc}
	
	\leavevmode
\end{proof}




\section{Structure de l'ensemble des solutions}

Un ensemble $\mathcal{T}$ est appelé treillis s'il vérifie les conditions suivantes.
%
	\begin{itemize}
		\item $\big( \mathcal{T} ; \leq \big)$ est un ensemble ordonné.

		\item $\forall (a ; b) \in \mathcal{T}^2$\,, l'ensemble $\setgene{a ; b}$ possède un minimum et un maximum.
	\end{itemize}

\begin{fact}
	La relation de divisibilité ordonne l'ensemble $\ourset$ via $n  \leq m$ si, et seulement si, $n \divides m$\,.
	
	\medskip
	
	Muni de cet ordre, $\ourset$ est un treillis.
\end{fact}

\begin{proof}
	Voir les faits \ref{lcm} et \ref{gcd}.
\end{proof}


% -------------------- %







\bigskip
%\newpage

\hrule

\section{AFFAIRE À SUIVRE...}

\bigskip

\hrule


% changer de base, essayer la base 3 pour rac 2 (idée de Voncent !)


%\subsection{L'arithmétique modulaire}
%
%????


\end{document}
