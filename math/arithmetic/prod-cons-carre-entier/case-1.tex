\begin{fact} \label{case-1}
	 $\forall n \in \NNs$\,, $n(n+1) \notin \NNsquare$\,.
\end{fact}


% ------------------ %


\begin{proof}
    Supposons que $\consprod<1> = n(n+1) \in \NNssquare$\,.
    %
    Clairement $\forall p \in \PP$\,, $\padicval{\consprod<1>} \in 2 \NN$\,.
    %
    Or $p \in \PP$ ne pent diviser à la fois $n$ et $n+1$\,.
    %
    Nous savons donc que $\forall p \in \PP$\,, 
    $\padicval{n} \in 2 \NN$ et $\padicval{n+1} \in 2 \NN$\,,
    autrement dit 
    $(n, n + 1) \in \NNsquare \times \NNsquare$\,.
    D'après le fait \ref{dist-square}, nous savons que ceci est impossible.
\end{proof}


% ------------------ %


\begin{proof}[Une preuve alternative]
     Supposons que $\consprod<1> = n(n+1) = N^2$ où $N \in \NNs$.
     Rappelons les deux identités classiques suivantes.
     %
     \begin{itemize}
     	\item $n(n+1) = 2 \dsum_{k=1}^{n} k$

     	\item $N^2 = \dsum_{k=1}^{N} (2 k - 1)$
     \end{itemize}
     %
     Dès lors, après avoir noté que $N > n$\,, les équivalences suivantes donnent une contradiction.
	
	\medskip
	
	\begin{stepcalc}[style = sar, ope = \iff]
		n(n+1) = N^2
	\explnext{}
		2 \dsum_{k=1}^{n} k = \dsum_{k=1}^{N} (2 k - 1)
	\explnext{}
		\dsum_{k=1}^{n} 2k = \dsum_{k=1}^{N} 2 k - N
	\explnext{}
		\dsum_{k=n+1}^{N} 2k = N
	\explnext*{Se souvenir que $N > 0$\,.}{}
		\dsum_{k=n+1}^{N-1} 2k + N = 0
	\end{stepcalc}

	\vspace{-2ex}	
	\leavevmode
\end{proof}
