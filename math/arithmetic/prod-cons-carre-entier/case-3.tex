\begin{fact} \label{case-3}
	 $\forall n \in \NNs$\,, $n(n+1)(n+2)(n+3) \notin \NNsquare$\,.
\end{fact}


% ------------------ %


\begin{proof}
    Nous pouvons ici faire les manipulations algébriques naturelles suivantes.
    
    \medskip
    
    \begin{stepcalc}[style = sar]
    	\consprod<3>
    \explnext{}
    	n(n+3) \cdot (n+1)(n+2)
    \explnext{}
    	(n^2 + 3n) \cdot (n^2 + 3n + 2)
    \explnext*{$m = n^2 + 3n + 1$}{}
    	(m - 1) (m + 1)
    \explnext{}
    	m^2 - 1
    \end{stepcalc}
    
    \medskip
    
    D'après le fait \ref{dist-square}, $m^2 - 1 \notin \NNsquare$\,, c'est-à-dire $\consprod<3> \notin \NNsquare$\,. 
\end{proof}


% ------------------ %


\begin{proof}[Une preuve alternative]
	Supposons que $\consprod<3> = n(n+1)(n+2)(n+3) \in \NNssquare$\,.
    %
    Comme $p \in \PP_{>3}$ ne peut diviser au maximum qu'un seul des quatre facteurs $n$\,, $(n+1)$\,, $(n+2)$ et $(n+3)$\,, nous savons que 
    $\forall p \in \PP_{>3}$\,, 
    $\big( \padicval{n} , \padicval{n + 1} , \padicval{n + 2} , \padicval{n + 3} \big) \in \big( 2 \NN \big)^4$\,.
    Mais que se passe-t-il pour $p = 2$ et $p = 3$ ?
    %
    Nous avons les alternatives suivantes pour chaque facteur $(n+i)$ de $\consprod<3>$\,.
    %
    \begin{itemize}
    	\item \alt{1}\,
		$\big( \padicval[2]{n + i} , \padicval[3]{n + i} \big) \in 2 \NN \times 2 \NN$

    	\item \alt{2}\,
		$\big( \padicval[2]{n + i} , \padicval[3]{n + i} \big) \in 2 \NN \times \big( 2 \NN + 1)$

    	\item \alt{3}\,
		$\big( \padicval[2]{n + i} , \padicval[3]{n + i} \big) \in \big( 2 \NN + 1 \big) \times 2 \NN$

    	\item \alt{4}\,
		$\big( \padicval[2]{n + i} , \padicval[3]{n + i} \big) \in \big( 2 \NN + 1 \big) \times \big( 2 \NN + 1)$
    \end{itemize}
    
    \medskip
    
    Notons que $n \notin \NNsquare$ car sinon $n(n+1)(n+2)(n+3) \in \NNsquare$ donnerait $(n+1)(n+2)(n+3) \in \NNsquare$\,, mais ceci ne se peut pas d'après le fait \ref{case-2}.
    De même, $n+3 \notin \NNsquare$\,.
    Ceci nous montre que $\padicval[2]{n} \in 2 \NN + 1$\,, ou $\padicval[3]{n} \in 2 \NN + 1$\,, mais aussi que $\padicval[2]{n+1} \in 2 \NN + 1$\,, ou $\padicval[3]{n+3} \in 2 \NN + 1$\,.
	
	\medskip
	
	Supposons d'abord que $\padicval[2]{n} \in 2 \NN + 1$\,.
	%
	\begin{itemize}
		\item Comme $2 \ndivides n+3$\,, nous avons $\padicval[3]{n+3} \in 2 \NN + 1$\,.
		Dès lors nous devons aussi avoir $\padicval[3]{n} \in 2 \NN + 1$ puisque ni $(n+1)$\,, ni $(n+2)$ n'est divisible par $3$\,.
		Donc $n = 6 A^2$ et $n+3 = 3 D^2$ avec $(A, D) \in \NN^2$.

		\item Comme $2 \ndivides n+1$ et $3 \ndivides n+1$\,, nous avons $n+1 = B^2$ avec $B \in \NN$\,.

		\item Comme $2 \divides n+2$ et $3 \ndivides n+2$\,, nous avons $n+2 = 2 C^2$ avec $C \in \NN$\,.

		\item Le tableau suivant montre que $n = 6 A^2 = B^2 - 1 = 3 D^2 - 3 = 2 C^2 - 2$ est impossible modulo $5$\,, et donc a fortiori dans $\NN$\,.  
	\end{itemize}

	\begin{center}
		\begin{tblr}{
			colspec={*{5}{Q[c,$]}},
			vline{2-5} = {},
			hline{2-6} = {}
		}
        	k   &  k^2  &  k^2 - 1  &  3 k^2 - 3  &  2 k^2 - 2
			\\
        	0   &  0    &  4        &  2          &  3
			\\
        	1   &  1    &  0        &  0          &  0
			\\
        	2   &  4    &  3        &  4          &  1
			\\
        	3   &  4    &  3        &  4          &  1
			\\
        	4   &  1    &  4        &  0          &  0
        \end{tblr}
	\end{center}
\end{proof}

