\begin{fact}
	 $\forall n \in \NNs$\,, $n(n+1)(n+2)(n+3) \notin \NNsquare$\,.
\end{fact}


% ------------------ %


\begin{proof}
    Nous pouvons ici faire les manipulations algébriques naturelles suivantes.
    
    \medskip
    
    \begin{stepcalc}[style = sar]
    	\consprod<3>
    \explnext{}
    	n(n+3) \cdot (n+1)(n+2)
    \explnext{}
    	(n^2 + 3n) \cdot (n^2 + 3n + 2)
    \explnext*{$m = n^2 + 3n + 1$}{}
    	(m - 1) (m + 1)
    \explnext{}
    	m^2 - 1
    \end{stepcalc}
    
    \medskip
    
    D'après le fait \ref{dist-square}, $m^2 - 1 \notin \NNsquare$\,, c'est-à-dire $\consprod<3> \notin \NNsquare$\,. 
\end{proof}


% ------------------ %

%
%\begin{proof}[Une preuve alternative]
%    Supposons que $\consprod<2> = n(n+1)(n+2)(n+3) \in \NNssquare$\,.
%    %
%    Comme $p \in \PP_{>3}$ ne peut diviser au maximum qu'un seul des quatre facteurs $n$\,, $(n+1)$\,, $(n+2)$ et $(n+3)$\,, nous savons que 
%    $\forall p \in \PP_{>3}$\,, 
%    $\big( \padicval{n} , \padicval{n + 1} , \padicval{n + 2} , \padicval{n + 3} \big) \in \big( 2 \NN \big)^4$\,.
%    Mais que se passe-t-il pour $p = 2$ et $p = 3$ ?
%    Nous avons les alternatives suivantes pour chaque facteur $(n+i)$ de $\consprod<2>$\,.
%    %
%    \begin{itemize}
%    	\item \alt{1} 
%		$\big( \padicval[2]{n + i} , \padicval[3]{n + i} \big) \in 2 \NN \times 2 \NN$
%
%    	\item \alt{2}
%		$\big( \padicval[2]{n + i} , \padicval[3]{n + i} \big) \in 2 \NN \times \big( 2 \NN + 1)$
%
%    	\item \alt{3}
%		$\big( \padicval[2]{n + i} , \padicval[3]{n + i} \big) \in \big( 2 \NN + 1 \big) \times 2 \NN$
%
%    	\item \alt{4}
%		$\big( \padicval[2]{n + i} , \padicval[3]{n + i} \big) \in \big( 2 \NN + 1 \big) \times \big( 2 \NN + 1)$
%    \end{itemize}
%    
%    Notons alors les obligations suivantes. 
%    %
%    \begin{itemize}
%    	\item Si $(n+i)$ vérifie \alt{1}\,, alors $n+i \in \NNsquare$\,.
%
%    	\item Si $(n+i)$ vérifie \alt{2}\,, alors il existe un autre facteur $(n+i^\prime)$ tel que $\padicval[3]{n + i^\prime} \in 2 \NN + 1$\,. Ceci montre que forcément le couple de facteurs est $(n , n+3)$\,.
%
%    	\item Si $(n+i)$ vérifie \alt{3}\,, alors forcément le couple de facteurs est soit $(n , n+2)$\,, soit $(n+1 , n+3)$\,.
%
%    	\item Si $(n+i)$ vérifie \alt{4}\,, alors XXX
%    \end{itemize}
%\end{proof}
