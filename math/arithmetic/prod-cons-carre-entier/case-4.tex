Nous allons démontrer le fait suivant de deux façons différentes, toutes les deux étant intéressantes dans leur approche.

\begin{fact}
	 $\forall n \in \NNs$\,, $n(n+1)(n+2)(n+3)(n+4) \notin \NNsquare$\,.
\end{fact}


% ----------------------- %


\begin{proof}[Démonstration 1]
	Supposons que $\consprod<4> = n(n+1)(n+2)(n+3)(n+4) \in \NNssquare$\,.

	\medskip

	Posant $m = n+2$\,, nous avons $\consprod<4> = (m-2)(m-1)m(m+1)(m+2) = m(m^2 - 1)(m^2 - 4)$ où $m \in \NN_{\geq 3}$\,.
	On notera dans la suite $u = m^2 - 1$ et $q = m^2 - 4$\,.
	
	\medskip
	
	Supposons d'abord que $m \in \NNssquare$\,.
	%
	\begin{itemize}
		\item De $muq \in \NNssquare$\,, nous déduisons $uq \in \NNssquare$\,.

		\item Comme $u - q = 3$\,, nous savons que $u \wedge q \in \setgene{1, 3}$\,.

		\item Si $u \wedge q = 1$\,, 
		alors $\forall p \in \PP$\,, 
		$\padicval{u} \in 2 \NN$ et $\padicval{q} \in 2 \NN$\,,
		d'où 
		$(u, q) \in \NNsquare \times \NNsquare$\,.
		Or ceci est impossible d'après le fait \ref{dist-square}
		\footnote{
			On peut aussi noter que le fait \ref{case-3} lève une contradiction car nous avons $m \in \NNsquare$ et $u \in \NNsquare$ qui donnent $(m-1)m(m+1) \in \NNsquare$
		}.

		\item Si $u \wedge q = 3$\,, 
		alors $\forall p \in \PP - \setgene{3}$\,, 
		$\padicval{u} \in 2 \NN$ et $\padicval{q} \in 2 \NN$\,,
		mais aussi $\padicval[3]{u} \in 2 \NN + 1$ et $\padicval[3]{q} \in 2 \NN + 1$\,.
		Donc 
		$u = 3 U^2$ et $q = 3 Q^2$ avec $(U, Q) \in \NN^2$\,.
		Or $u - q = 3$ donne $U^2 - Q^2 = 1$\,, et le fait \ref{dist-square} nous indique une contradiction.
	\end{itemize}
	
	\medskip
	
	Supposons maintenant que $m \notin \NNssquare$\,.
	%
	\begin{itemize}
		\item Nous avons vu ci-dessus que $u \notin \NNsquare$ et $q \notin \NNsquare$\,. On peut donc écrire $m = \alpha M^2$\,, $u = \beta U^2$\,, $q = \gamma Q^2$ où $(M, U, Q) \in \NN^3$, et $(\alpha, \beta, \gamma) \in \big( \NN_{>1} \big)^3$ forme un triplet de naturels sans facteur carré.


		\item Notons que $\beta \neq \gamma$ car, dans le cas contraire, nous aurions $3 = u - q = \beta \big( U^2 - Q^2 \big)$ qui fournirait $0 < \abs{U^2 - Q^2} < 3$\,, et ceci contredirait le fait \ref{dist-square}.


		\item Nous avons $m \wedge u = 1$\,, $m \wedge q \in \setgene{1, 2, 4}$ et $u \wedge q \in \setgene{1, 3}$
		avec $m \wedge u = m \wedge q = u \wedge q = 1$ impossible car sinon on aurait $(m, u, q) \in \big( \kern-1.5pt\NNsquare \big)^3$ via $muq \in \NNsquare$\,.
		Dès lors, $\forall p \in \PP_{>3}$\,, $\big( \padicval{m} , \padicval{u} , \padicval{q} \big) \in \big( 2 \NN \big)^3$.


		\item Les points précédents donnent 
		$\setgene{\alpha, \beta, \gamma} \subseteq \setgene{2, 3, 6}$\,,
		et aussi 
		$\alpha \wedge \beta = 1$\,, $\alpha \wedge \gamma \in \setgene{1, 2}$\,, $\beta \wedge \gamma \in \setgene{1, 3}$
		avec $\alpha \wedge \beta = \alpha \wedge \gamma = \beta \wedge \gamma = 1$ impossible. 
		Le tableau \enquote{mécanique} ci-après nous amène à juste considérer $(\alpha, \beta, \gamma) = (2, 3, 2)$ et  $(\alpha, \beta, \gamma) = (2, 3, 6)$\,.
	\end{itemize}

	\begin{center}
		\begin{tblr}{
			colspec={*{6}{Q[c,$]}c},
			vline{2-7} = {},
			hline{2-5} = {}
		}
        	\alpha & \beta & \gamma 
				& \alpha \wedge \beta & \alpha \wedge \gamma & \beta \wedge \gamma
				& Statut
			\\
        	2 & 3 & 2
				& 1 & 2 & 1
				& OK
%			\\
%        	2 & 3 & 3
%				& 1 & 1 & 3
%				& OK
			\\
        	2 & 3 & 6
				& 1 & 2 & 3
				& OK
%			\\
%        	3 & 2 & 2
%				& 1 & 1 & 2
%				& KO
			\\
        	3 & 2 & 3
				& 1 & 3 & 1
				& KO
			\\
        	3 & 2 & 6
				& 1 & 3 & 2
				& KO
        \end{tblr}
	\end{center}


	\begin{itemize}
		\item $(\alpha, \beta, \gamma) = (2, 3, 2)$ nous donne $m = 2 M^2$, $m^2 - 1 = 3 U^2$ et $m^2 - 4 = 2 Q^2$.


		\item $(\alpha, \beta, \gamma) = (2, 3, 6)$ nous donne $m = 2 M^2$, $m^2 - 1 = 3 U^2$ et $m^2 - 4 = 6 Q^2$.
		Modulo $3$\,, nous avons :
		%
		\begin{itemize}
			\item $m \equiv 0$ ou $m \equiv -1$ via $m = 2 M^2$.

			\item $m^2 \equiv 1$ via $m^2 - 1 = 3 U^2$.

			\item Nous en déduisons que $m \equiv -1$\,, puis $m - 2 \equiv 0$\,, soit $3 \divides m - 2$\,.
		\end{itemize}

		\smallskip
		
		\noindent
		Via $m = 2 M^2$, nous obtenons $2 \divides m - 2$\,, et donc $6 \divides m - 2$\,.
	\end{itemize}
\end{proof}


% ----------------------- %


\begin{proof}[Démonstration 2]

	 XXX
\end{proof}