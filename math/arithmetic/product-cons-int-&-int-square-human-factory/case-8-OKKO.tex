\begin{fact} \label{case-8}
	 $\forall n \in \NNs$\,, $\consprod<7> \notin \NNsquare$\,.
\end{fact}


% ------------------ %


%Cette preuve reprend l'idée de la démonstration courte du fait \ref{case-6}.
%
%\begin{proof}[Preuve]
%    Supposons que $\consprod<7> \in \NNsquare$\,.
%    
%    \smallskip
%    
%    Commençons par de petites manipulations algébriques avec une première étape faisant apparaître le même coefficient pour $n$ dans chaque parenthèse, et dans la dernière étape on fait apparaître des entiers avec de petits facteurs premiers
%    \footnote{
%    	Par exemple, $a = x + 10$ donne $(a - 10) (a - 4) a (a + 2)$ qui est moins intéressant que le choix retenu.
%    }.
%    
%    \medskip
%    \begin{stepcalc}[style = sar]
%    	\consprod<5>
%	\explnext{}
%		n (n+7) \cdot (n+1) (n+6) \cdot (n+2) (n+5) \cdot (n+3) (n+4)
%	\explnext{}
%		(n^2 + 7n) (n^2 + 7n + 6) (n^2 + 7n + 10) (n^2 + 7n + 12)
%	\explnext*{$x = n^2 + 7n \in \NN_{\geq 8}$}{}
%		x (x + 6) (x + 10) (x + 12)
%	\explnext*{$a = x + 6 \in \NN_{\geq 16}$}{}
%		(a - 6) a (a + 4) (a + 6)
%    \end{stepcalc}
%  
%    \medskip
%    Nous avons $a \in \NNs$ vérifiant $a (a + 4) (a^2 - 36) \in \NNssquare$\,. 
%    Posons alors $a = \alpha A^2$ où $(\alpha, A) \in \NNsf \times \NNs$\,.
%    Comme $\alpha \in \NNsf$\,, nous avons $\alpha \divides (\alpha A^2 + 4) (\alpha^2 A^4 - 36)$\,, d'où $\alpha \divides 2^4 \cdot 3^2$\,, et finalement $\alpha \in \setgene{1, 2, 3, 6}$\,.
%    Nous allons voir que ceci est impossible.
%
%    \medskip
%    
%    Supposons avoir $\alpha = 1$\,.
%    
%    \begin{itemize}
%    	\item Notons l'équivalence suivante.
%   
%%        \medskip
%        \noindent\kern-6pt%
%        \begin{stepcalc}[style=ar*, ope=\iff]
%        	(A^2 + 4) (A^2 + 6) (A^2 - 6) \in \NNssquare
%    	\explnext{}
%        	...
%        \end{stepcalc}
%
%		\item ????
%    \end{itemize}
%\end{proof}
%
