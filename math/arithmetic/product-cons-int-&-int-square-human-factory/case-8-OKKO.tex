\begin{fact} \label{case-8}
	 $\forall n \in \NNs$\,, $\consprod<7> \notin \NNsquare$\,.
\end{fact}


% ------------------ %


Cette preuve reprend l'idée de la démonstration courte du fait \ref{case-6}.

\begin{proof}
    Supposons que $\consprod<7> \in \NNsquare$\,.
    Commençons par de petites manipulations algébriques où l'on cherche à faire apparaître le même coefficient pour $n$ dans chaque parenthèse.
    
    \medskip
    \begin{stepcalc}[style = sar]
    	\consprod<5>
	\explnext{}
		n (n+7) \cdot (n+1) (n+6) \cdot (n+2) (n+5) \cdot (n+3) (n+4)
	\explnext{}
		(n^2 + 7n) (n^2 + 7n + 6) (n^2 + 7n + 10) (n^2 + 7n + 12)
	\explnext*{$x = n^2 + 7n \in \NN_{\geq 8}$}{}
		x (x + 6) (x + 10) (x + 12)
	\explnext*{$a = x + 10 \in \NN_{\geq 18}$}{}
		(a - 10) (a - 4) a (a + 2)
    \end{stepcalc}
  
    \medskip
    Nous avons donc $a \in \NNs$ vérifiant $a (a + 2) (a - 4) (a - 10) \in \NNssquare$\,. 
    Posons alors $a = \alpha A^2$ où $(\alpha, A) \in \big( \NNs \big)^2$ avec $\alpha$ sans facteur carré.
    Comme $a = \alpha (\alpha A^2 + 2) (\alpha A^2 - 4) (\alpha A^2 - 10) \in \NNssquare$\,,
    et comme de plus $\alpha$ est sans facteur carré, nous devons avoir $\alpha \divides (\alpha A^2 + 2) (\alpha A^2 - 4) (\alpha A^2 - 10)$\,, d'où $\alpha \divides 80$\,, et finalement $\alpha \in \setgene{1, 2, 5, 10}$\,.
    Nous allons voir que ceci est impossible.


    Supposons avoir $\alpha = 1$\,.
    
    \begin{itemize}
    	\item Notons les équivalences suivantes.
        
        \medskip
        \noindent\kern-6pt%
        \begin{stepcalc}[style=ar*, ope=\iff]
        	(A^2 + 2) (A^2 - 4) (A^2 - 10)\in \NNssquare
    	\explnext{$u = A^2 - 4$}
        	(u + 6) u (u - 6) \in \NNssquare
    	\explnext{}
        	u(u^2 - 36) \in \NNssquare
        \end{stepcalc}

		\item ????
    \end{itemize}
\end{proof}

