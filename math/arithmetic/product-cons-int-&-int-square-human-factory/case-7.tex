\begin{fact} \label{case-7}
	 $\forall n \in \NNs$\,, $\consprod<6> \notin \NNsquare$\,.
\end{fact}


% ------------------ %


La très jolie démonstration suivante vient d'un échange sur \url{https://math.stackexchange.com} (voir la section \ref{sources}). Nous avons juste comblé quelques rares oublis, et apporté de petites simplifications.
	

\begin{proof}[Preuve]
	Supposons que $\consprod<6> \in \NNssquare$\,.
	
	\smallskip
	
	Commençons par quelques observations immédiates.
	%
    \begin{itemize}
    	\item  
		$\forall p \in \PP_{>5}$\,, 
   		$\forall i \in \ZintervalC{0}{6}$\,, 
    	$\padicval{n + i} \in 2 \NN$\,.
	
	
		\item $\exists u \in \setgene{0, 1, 2}$ tel que $\setgene{u, u + 2, u + 4} \subset 2 \NN + 1$\,.
		
		\noindent
		Nous avons alors
		$\forall p \in \PP_{>5} - \setgene{2}$\,, 
   		$(\padicval{u}, \padicval{u+2}, \padicval{u+4}) \in ( 2 \NN )^3$.
		Donc, pour tout naturel $m \in \setgene{u, u + 2, u + 4}$\,, 
		il existe $M \in \NNs$ tel que 
		$m = M^2$, $m = 3 M^2$, $m = 5 M^2$ ou $m = 15 M^2$.
	
	
		\item Parmi les trois naturels $u$\,, $u + 2$\, et $u + 4$\,, ...
		%
		\begin{itemize}
    		\item il en existe un, et un seul, divisible par $3$\,, comme on le constate vite en raisonnant modulo $3$\,,

    		\item au plus un est divisible par $5$\,,

    		\item au plus un est un carrée parfait d'après le fait \ref{diff-square-ko}.
		\end{itemize}
    \end{itemize}

    \smallskip
	
	Donc, il existe $(M, P, Q) \in ( \NNs )^3$ tel que 
    $\setgene{u, u + 2, u + 4} = \setgene{M^2, 3P^2, 5Q^2}$\,.
    Ceci permet de considérer les trois cas suivants qui lèvent tous une contradiction.
	%
    \begin{itemize}
    	\item Supposons avoir $u = M^2$.
		%
		\begin{enumerate}
			\item Comme $\setgene{u + 2, u + 4} = \setgene{3P^2, 5Q^2}$\,, nous savons que $3 \ndivides (u + 3)$ et $5 \ndivides (u + 3)$\,, d'où $u + 3 = 2^a T^2$ avec $(a, T) \in \NN \times \NNs$.

			\item Modulo $4$\,, $u \equiv M^2 \equiv 1$ car $u \in 2 \NN + 1$\,,
			donc $u + 3 \equiv 0$\,, d'où $a \geq 2$\,.

			\item Modulo $8$\,, $u \equiv M^2 \equiv 1$ car $u \in 2 \NN + 1$\,,
			donc $u + 3 \equiv 4$\,, d'où $a = 2$\,.

			\item Dès lors, $u + 3 \in \NNsquare$\,, puis $(u + 3, u) = (4, 1)$ via le fait \ref{diff-square-ko}.

			\item Forcément $n = u = 1$\,, mais $\padicval[7]{\consprod[1]<6>} = 1$ contredit $\consprod<6> \in \NNssquare$\,.
		\end{enumerate}


    	\item Supposons maintenant que $u + 2 = M^2$.

		\smallskip
		\noindent
		Le fait d'avoir $\setgene{u, u + 4} = \setgene{3P^2, 5Q^2}$ demande un peu de prudence pour adapter la preuve précédente. 
		Il faut considérer $(u - 1, u + 2)$ si $u > n$\,, et $(u + 2, u + 5)$ sinon.


    	\item Supposons enfin que $u + 4 = M^2$.

		\smallskip
		\noindent
		Comme $\setgene{u, u + 2} = \setgene{3P^2, 5Q^2}$\,, il suffit de raisonner avec $(u + 1, u + 4)$\,.
    \end{itemize}
\end{proof}

 