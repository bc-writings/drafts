\begin{fact} \label{case-7}
	 $\forall n \in \NNs$\,, $\consprod<7> \notin \NNsquare$\,.
\end{fact}


% ------------------ %


La jolie démonstration suivante vient d'un échange sur \url{https://math.stackexchange.com} (voir la section \ref{sources}). Nous avons comblé les trous en apportant de petites simplifications.
	

\begin{proof}[Preuve 1]
	Supposons que $\consprod<7> \in \NNssquare$\,.
	
	\smallskip
	
	Commençons par quelques observations immédiates.
	%
    \begin{itemize}
    	\item  
		$\forall p \in \PP_{\geq 7}$\,, 
   		$\forall i \in \ZintervalC{0}{6}$\,, 
    	$\padicval{n + i} \in 2 \NN$\,.
	
	
		\item $\exists u \in \setgene{n, n+1, n+2}$ tel que $\setgene{u, u + 2, u + 4} \subset 2 \NN + 1$\,.
		
		\noindent
		Nous avons alors
		$\forall p \in \PP - \setgene{3, 5}$\,, 
   		$(\padicval{u}, \padicval{u+2}, \padicval{u+4}) \in ( 2 \NN )^3$.
		Cette astuce permet de passer de la gestion des trois nombres premiers $2$\,, $3$ et $5$ à celle de $3$ et $5$\,.
		Donc, pour tout naturel $m \in \setgene{u, u + 2, u + 4}$\,, 
		il existe $M \in \NNs$ tel que 
		$m = M^2$, $m = 3 M^2$, $m = 5 M^2$ ou $m = 15 M^2$.
	
	
		\item Parmi les trois naturels $u$\,, $u + 2$\, et $u + 4$\,, ...
		%
		\begin{itemize}
    		\item il en existe un, et un seul, divisible par $3$\,, comme on le constate vite en raisonnant modulo $3$\,,

    		\item au plus un est divisible par $5$\,,

    		\item au plus un est un carrée parfait d'après le fait \ref{diff-square-ko}.
		\end{itemize}
    \end{itemize}

    \smallskip
	
	Donc, il existe $(M, P, Q) \in ( \NNs )^3$ tel que 
    $\setgene{u, u + 2, u + 4} = \setgene{M^2, 3P^2, 5Q^2}$\,.
    Ceci permet de considérer les trois cas suivants qui lèvent tous une contradiction.
	%
    \begin{itemize}
    	\item Supposons avoir $u = M^2$.
		%
		\begin{enumerate}
			\item Comme $\setgene{u + 2, u + 4} = \setgene{3P^2, 5Q^2}$\,, nous savons que $3 \ndivides (u + 3)$ et $5 \ndivides (u + 3)$\,, d'où $u + 3 = 2^a A^2$ avec $(a, A) \in \NN \times (2\NN + 1)$\,.

			\item Modulo $8$\,, $u \equiv M^2 \equiv 1$ car $u \in 2 \NN + 1$\,,
			donc $u + 3 \equiv 4$\,, d'où $a = 2$\,.

			\item Dès lors, $u + 3 \in \NNsquare$\,, puis $(u, u + 3) = (1, 4)$ via le fait \ref{diff-square-ko}.

			\item Nous arrivons à $n = u = 1$\,, mais $\padicval[7]{\consprod[1]<7>} = 1$ contredit l'hypothèse $\consprod<7> \in \NNssquare$\,.
		\end{enumerate}


    	\item Supposons maintenant que $u + 2 = M^2$.

		\smallskip
		\noindent
		Ce qui suit démontre que $\setgene{u, u + 4} = \setgene{3P^2, 5Q^2}$ est impossible.
		%
		\begin{enumerate}
			\item Si $(u, u + 4) = (3P^2, 5Q^2)$\,, alors,
			comme $u \in 2 \NN + 1$\,, nous avons, modulo $4$\,,
			$u \equiv 3$
			et
			$u+4 \equiv 1$
			qui se contredisent.

			\item Si $(u, u + 4) = (5Q^2, 3P^2)$\,, on raisonne de même.
		\end{enumerate}


    	\item Supposons enfin que $u + 4 = M^2$.

		\smallskip
		\noindent
		Démontrons que $\setgene{u, u + 2} = \setgene{3P^2, 5Q^2}$ est impossible en raisonnant modulo $8$\,.
		%
		\begin{enumerate}
			\item $u + 4 \equiv M^2 \equiv 1$\, via $u \in 2 \NN + 1$\,,
			d'où $u + 3 \equiv 0$\,.

			\item $u + 2 \in \setgene{3P^2, 5Q^2}$ donne $u + 2 \equiv \text{$3$ ou $5$}$  via $u \in 2 \NN + 1$\,.

			\item Les deux points précédents se contredisent. \qedhere
		\end{enumerate}
    \end{itemize}
\end{proof}


% ------------------ %


Pour la preuve suivante, nous reprenons l'idée de la démonstration du cas \ref{case-10} ; nous indiquons juste les adaptations à faire en reprenant les notations de la preuve citée.


\begin{proof}[Preuve 2]%
    Ici nous avons au moins $5$ facteurs $(n + i)$ de $\consprod<7>$ de valuation $p$-adique paire dès que $p \in \PP_{\geq 5}$\,.
    Ceci nous amène aux cas suivants.
    %
    \begin{itemize}
    	\medskip
		\item Deux facteurs différents $(n+i)$ et $(n+i^\prime)$ vérifient \alt{1}\,.
		
		\smallskip
		\noindent
		Dans ce cas, $(n+i, n+i^\prime) = (M^2, N^2)$ avec $\abs{N^2 - M^2} \in \ZintervalC{1}{6}$\,, mais ce qui suit lève des contradictions.
		%
		\begin{enumerate}
			\item $\abs{N^2 - M^2} = 3$ donne $n = 1$\,, mais $\consprod[1]<7> = 7! \notin \NNsquare$ via $\padicval[7]{7!} = 1$\,.


			\item $\abs{N^2 - M^2} = 5$ donne $n \in \ZintervalC{2}{4}$\,, mais $\forall n \in \ZintervalC{2}{4}$\,, $\padicval[7]{\consprod[n]<7>} = 1$ donne $\consprod[n]<7> \notin \NNsquare$\,.
		\end{enumerate}


    	\medskip
		\item Deux facteurs différents $(n+i)$ et $(n+i^\prime)$ vérifient \alt{2}\,.
		
		\smallskip
		\noindent
		Dans ce cas, $(n+i, n+i^\prime) = (3 M^2, 3 N^2)$ avec $\abs{3(N^2 - M^2)} \in \ZintervalC{1}{6}$\,, mais c'est impossible d'après le fait \ref{diff-square-ko}.

    	\medskip
		\item Deux facteurs différents $(n+i)$ et $(n+i^\prime)$ vérifient \alt{3}\,.
		
		\smallskip
		\noindent
		Dans ce cas, $(n+i, n+i^\prime) = (2 M^2, 2 N^2)$ avec $\abs{2(N^2 - M^2)} \in \ZintervalC{1}{6}$\,, puis nécessairement $\abs{N^2 - M^2} = 3$ qui implique $n \in \ZintervalC{1}{2}$\,, mais on sait que cela est impossible.


    	\medskip
		\item Deux facteurs différents $(n+i)$ et $(n+i^\prime)$ vérifient \alt{4}\,.
		
		\smallskip
		\noindent
		Dans ce cas, $(n+i, n+i^\prime) = (6 M^2, 6 N^2)$ avec $\abs{6(N^2 - M^2)} \in \ZintervalC{1}{6}$\,, mais c'est impossible d'après le fait \ref{diff-square-ko}.
		%
		\qedhere
    \end{itemize}
\end{proof}

 