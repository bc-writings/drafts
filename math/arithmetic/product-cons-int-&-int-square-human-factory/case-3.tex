\begin{fact} \label{case-3}
	 $\forall n \in \NNs$\,, $n(n+1)(n+2) \notin \NNsquare$\,.
\end{fact}


% ------------------ %


\begin{proof}[Preuve 1]
    Supposons que $\consprod<3> \in \NNssquare$\,.

    \smallskip
    
    Posant $m = n+1$\,, nous avons $\consprod<3> = (m-1)m(m+1) = m(m^2 - 1)$ où $m \in \NN_{\geq 2}$\,.
    %
    Comme $\GCD{m}{(m^2 - 1)} = 1$\,, le fait \ref{prime-square} donne $(m, m^2 - 1) \in \NNssquare \times \NNssquare$\,.
    Or, $m^2 - 1 \in \NNssquare$ est impossible d'après le fait \ref{diff-square-ko}.
\end{proof}


% ------------------ %


\begin{proof}[Preuve 2]
    Supposons que $\consprod<3> \in \NNssquare$\,.

    \smallskip
    
    Comme $p \in \PP_{\geq 3}$ ne peut diviser au maximum qu'un seul des trois facteurs $n$\,, $(n+1)$ et $(n+2)$\,, nous savons que 
    $\forall p \in \PP_{\geq 3}$\,, 
    $\forall i \in \ZintervalC{0}{2}$\,, 
    $\padicval{n + i} \in 2 \NN$\,.
    Mais que se passe-t-il pour $p = 2$ ?
    
    \medskip
    
    Supposons d'abord $n \in 2 \NN$\,.
	%
	\begin{itemize}
		\item Posant $n = 2 m$\,, nous avons $\consprod<3> = 4 m(2m+1)(m+1)$\,, d'où $m(2m+1)(m+1) \in \NNssquare$\,.
		
		\item Comme $\padicval[2]{2m+1} = 0$\,, nous savons que $2m+1 \in \NNssquare$\,.
		
		\item Donc $\consprod[m]<2> = m(m+1)\in \NNssquare$ via le fait \ref{facto-square}, mais le fait \ref{case-2} interdit cela.
	\end{itemize}
    
    \medskip
    
    Supposons maintenant $n \in 2 \NN + 1$\,.
	%
	\begin{itemize}
		\item Nous savons que $n \in \NNssquare$ via $\padicval[2]{n} = 0$\,.

		\item On conclut comme dans le cas précédent mais en passant via $\consprod[n+1]<2>=(n+1)(n+2)$\,. \qedhere
	\end{itemize}
\end{proof}


% ------------------ %


Finissons avec une preuve dans l'esprit de la démonstration du cas \ref{case-10}.


\begin{proof}[Preuve 3]%
    Supposons que $\consprod<3> \in \NNssquare$\,.
    
    \smallskip
    
    Clairement, 
    $\forall p \in \PP_{\geq 3}$\,, 
    $\forall i \in \ZintervalC{0}{2}$\,, 
    $\padicval{n + i} \in 2 \NN$\,,
    donc pour les facteurs $(n + i)$ de $\consprod<3>$\,, nous avons l'une des deux alternatives suivantes, chacune d'elles levant une contradiction.
    %
    \begin{itemize}
    	\medskip
		\item Deux facteurs différents $(n+i)$ et $(n+i^\prime)$ sont de valuations $2$-adiques impairs.
		
		\smallskip
		\noindent
		Dans ce cas, $(n+i, n+i^\prime) = (2 M^2, 2 N^2)$ avec $\abs{2(N^2 - M^2)} \in \ZintervalC{1}{2}$\,, c'est-à-dire $\abs{N^2 - M^2} = 1$\,, mais c'est impossible d'après le fait \ref{diff-square-ko}.


    	\medskip
		\item Deux facteurs différents $(n+i)$ et $(n+i^\prime)$ sont de valuations $2$-adiques pairs.
		
		\smallskip
		\noindent
		Dans ce cas, $(n+i, n+i^\prime) = (M^2, N^2)$ avec $\abs{N^2 - M^2} \in \ZintervalC{1}{2}$\,, mais c'est impossible d'après le fait \ref{diff-square-ko}.
		%
		\qedhere
    \end{itemize}
\end{proof}

