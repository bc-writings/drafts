\begin{fact} \label{case-13}
	 $\forall n \in \NNs$\,, $\consprod<13> \notin \NNsquare$\,.
\end{fact}


% ------------------ %


\begin{proof}[Preuve]%
    Les arguments de la preuve du cas \ref{case-12} s'adaptent immédiatement.
\end{proof}


% ------------------ %


\begin{remark}
	Que donnerait l'analyse du cas suivant $\consprod<14> \notin \NNsquare$ ?
	Nous avons ce qui suit.
    %
    \begin{itemize}
		\item $\forall p \in \PP_{\geq 14}$\,, 
    $\forall i \in \ZintervalC{0}{13}$\,, 
    $\padicval{n + i} \in 2 \NN$\,.
		
		\item Au maximum trois facteurs $(n + i)$ de $\consprod<14>$ sont divisibles par $5$\,.

		\item Au maximum deux facteurs $(n + i)$ de $\consprod<14>$ sont divisibles par $7$\,.

		\item Au maximum deux facteurs $(n + i)$ de $\consprod<14>$ sont divisibles par $11$\,.

		\item Au maximum deux facteurs $(n + i)$ de $\consprod<14>$ sont divisibles par $13$\,. Un nouveau venu !

		\item Les points précédents nous donnent qu'au moins $5$ facteurs $(n + i)$ de $\consprod<14>$ sont de valu\-ation $p$-adique paire dès que $p \in \PP_{\geq 5}$\,. On peut donc tenter de mettre en route la même machinerie que pour le cas \ref{case-12}. 
    \end{itemize}
    
    Nous sentons donc ici la possibilité d'automatiser l'analyse de certaines situations. 
    Ceci a été fait dans mon document \emph{\enquote{Carrés parfaits et produits d'entiers consécutifs -- Jusqu'à 100 facteurs ?}} qui propose une approche informatique se basant principalement sur l'idée précédente afin de traiter les cas jusqu'à $100$ facteurs, c'est-à-dire ceux supposés connus dans la démonstration de Paul Erdős.
\end{remark}
