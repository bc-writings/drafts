\begin{fact} \label{case-13}
	 $\forall n \in \NNs$\,, $\consprod<13> \notin \NNsquare$\,.
\end{fact}

% ------------------ %


L'idée suivie est celle de la démonstration aux cas \ref{case-10} avec les changements suivants.


\begin{proof}[Preuve]%
    Les arguments de la preuve du cas \ref{case-12} s'adaptent immédiatement.
\end{proof}


% ------------------ %


\begin{remark}
	Que donnerait l'analyse du cas suivant $\consprod<14> \notin \NNsquare$ ?
	Nous avons ce qui suit.
    %
    \begin{itemize}
		\item $\forall p \in \PP_{\geq 14}$\,, 
    $\forall i \in \ZintervalC{0}{13}$\,, 
    $\padicval{n + i} \in 2 \NN$\,.
		
		\item Au maximum trois facteurs $(n + i)$ de $\consprod<14>$ sont divisibles par $5$\,.

		\item Au maximum deux facteurs $(n + i)$ de $\consprod<14>$ sont divisibles par $7$\,.

		\item Au maximum deux facteurs $(n + i)$ de $\consprod<14>$ sont divisibles par $11$\,.

		\item Au maximum deux facteurs $(n + i)$ de $\consprod<14>$ sont divisibles par $13$\,. Un nouveau venu !

		\item Les points précédents nous donnent qu'au moins $5$ facteurs $(n + i)$ de $\consprod<14>$ sont premiers relativement à tout premier $p \in \PP_{\geq 5}$\,.
    \end{itemize}
    
    À ce stade, il semble opportun de tenter une analyse informatique pour au moins traiter les cas jusqu'à $100$ facteurs, c'est-à-dire ceux supposés connus dans la démonstration de Paul Erdős. Nous ne le ferons pas dans ce document qui préfère laisser la place aux humains avant tout.
\end{remark}
