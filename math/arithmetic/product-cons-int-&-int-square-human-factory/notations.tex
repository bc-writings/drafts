Dans la suite, nous emploierons les notations suivantes.

\begin{itemize}
	\item $\NNsquare = \setgene{n^2, n \in \NN}$
	      et
	      $\NNssquare = \NNsquare \cap \NNs$\,.

	\item $\forall (n, k) \in \NNs \times \NN$\,, $\consprod = \dprod_{i = 0}^{k} (n + i)$\,. 
	Par exemple, nous avons $\consprod<0> = n$ et $\consprod<1> = n(n+1)$\,.

	\item $\PP$ désigne l'ensemble des nombres premiers.
	
	\item $\forall (p ; n) \in \PP \times \NNs$\,, $\padicval{n} \in \NN$ est la valuation $p$-adique de $n$\,,
	c'est-à-dire 
	$p^{\padicval{n}} \divides n$ et $p^{\padicval{n} + 1} \ndivides n$\,,
	autrement dit
	$p^{\padicval{n}}$ divise $n$\,, contrairement à $p^{\padicval{n} + 1}$\,.

	\item $\forall (n , m) \in \NN^2$, $n \wedge m$ désigne le PGCD de $n$ et $m$.
	
	\item $2\,\NN$ désigne l'ensemble des nombres naturels pairs.
	
	\item $2\,\NN + 1$ désigne l'ensemble des nombres naturels impairs.
\end{itemize}