\subsection{Structure}


\begin{fact} \label{facto-square}
	$\forall n \in \NNssquare$\,, s'il existe $m \in \NNssquare$ tel que $n =  f m$ alors $f  \in \NNssquare$\,.
\end{fact}


\begin{proof}
	Considérer les décompositions en facteurs premiers de $n$\,, $m$ et $f$\,.
\end{proof}


% ------------------ %


\begin{fact} \label{prime-square}
	$\forall (a, b) \in \NNs \times \NNs$, 
	si $\GCD{a}{b} = 1$ et $a b \in \NNssquare$\,,
	alors $a \in \NNssquare$ et $b \in \NNssquare$\,.
\end{fact}


\begin{proof}
	Clairement, $\forall p \in \PP$\,, nous avons $\padicval{ab} \in 2 \NN$\,.
    %
    Or $p \in \PP$ ne peut diviser à la fois $a$ et $b$\,, donc
    $\forall p \in \PP$\,, 
    $\padicval{a} \in 2 \NN$ et $\padicval{b} \in 2 \NN$\,,
    autrement dit 
    $(a, b) \in \NNssquare \times \NNssquare$\,.
\end{proof}


% ------------------ %


\begin{fact} \label{same-square-free}
	Soit $(a, b) \in \NNs \times \NNs$ tel que $a b \in \NNssquare$\,,
	ainsi que $(\alpha, \beta, A, B) \in \big( \NNsf \big)^2 \times \NN^2$ tel que $a = \alpha A^2$ et $b = \beta B^2$.
	Nous avons alors forcément $\alpha = \beta$\,.
\end{fact}


\begin{proof}
	Le fait \ref{facto-square} donne $\alpha \beta \in \NNssquare$\,.
	De plus, $\forall p \in \PP$\,, nous avons 
	$\padicval{\alpha} \in \setgene{0, 1}$
	et
	$\padicval{\beta} \in \setgene{0, 1}$\,.
	Finalement, $\forall p \in \PP$\,, $\padicval{\alpha} = \padicval{\beta}$\,, autrement dit $\alpha = \beta$\,.
\end{proof}


% ------------------ %


\subsection{Distance entre deux carrés parfaits}


$C^2 - A^2 = 6$ est impossible.


$B^2 - A^2 = 8$ avec $(B, A) = (3, 1)$


$C^2 - A^2 = 3$    n'est possible que si $(C, A) = (2, 1)$


Notant $m \in \NNs$ tel que $m^2 = w^2 - 16$\,, nous arrivons à $w^2 - m^2 = 16$\,.
		D'après le fait \ref{diff-square-ko}, $w^2 - m^2 = \dsum_{k=m+1}^{w} (2 k - 1)$\,.
		Ceci n'est possible que si $(w, m) = (5, 3)$
		\footnote{
			Noter que l'on doit avoir $2 w - 1 \leq 16$\,, d'où $w \in \ZintervalC{0}{8}$\,.
		}.
		
		
		
Ensuite, prenant $m \in \NNs$ tel que $m^2 = u^2 - 9$\,, comme $u^2 - m^2 = \dsum_{k=m+1}^{w} (2 k - 1)$\,, on a $(u, m) = (5, 4)$
		\footnote{
			Noter que $2 u - 1 \leq 9$ implique $u \in \ZintervalC{0}{5}$\,. Ceci permet d'analyser tous les cas mentalement.
		}.
		On aboutit alors à la contradiction suivante.
		
\begin{fact} \label{dist-square}
	$\forall (N, M) \in \NNs \times \NNs$, 
	si $N > M$\,, alors $N^2 - M^2 = \dsum_{k=M+1}^{N} (2 k - 1)$\,.
\end{fact}


\begin{proof}
	Il suffit d'utiliser $N^2 = \dsum_{k=1}^{N} (2 k - 1)$\,, une formule facile à démontrer algébriquement, et évidente à découvrir géométriquement.
\end{proof}


% ------------------ %


L'identité précédente permet d'éliminer beaucoup de situations en s'aidant, si besoin, d'un petit programme informatique (voir un peu plus bas).

\begin{fact} \label{diff-square-ko}
	Soit $(N, M) \in \NNs \times \NNs$ tel que $N > M$\,.
	%
	\begin{enumerate}
		\item $N^2 - M^2 \geq 2M + 1$\,.
		
		\item $N^2 - M^2 < 3$ est impossible.
	\end{enumerate}
\end{fact}


\begin{proof}
	\leavevmode
	
	\begin{enumerate}
		\item $N^2 - M^2 = \dsum_{k=M+1}^{N} (2 k - 1) \geq 2(M + 1) - 1 = 2M + 1$
		
		\smallskip
		\noindent
		On peut aussi juste procéder comme suit. 
		
		\smallskip
		\noindent
		$N^2 - M^2 = (N - M)(N + M) \geq 1 \cdot (M + 1 + M) = 2M + 1$



		\item Immédiat puisque $2M + 1 \geq 3$\,.
	\end{enumerate}

\end{proof}


