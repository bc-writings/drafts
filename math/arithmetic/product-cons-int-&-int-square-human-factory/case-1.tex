% ------------------ %


Bien que simple, le fait suivant va être régulièrement utilisé dans la suite.


\begin{fact} \label{facto-square}
	$\forall n \in \NNssquare$\,, s'il existe $m \in \NNssquare$ tel que $n =  f m$ alors $f  \in \NNssquare$\,.
\end{fact}


\begin{proof}
	Il suffit de passer via les décompositions en facteurs premiers de $n$\,, $m$ et $f$\,.
\end{proof}


% ------------------ %


Nous allons souvent être amené à étudier des différences de carrés parfaits.
Commençons par indiquer une jolie formule.

\begin{fact} \label{dist-square}
	$\forall (N, M) \in \NNs \times \NNs$, 
	si $N > M$\,, alors $N^2 - M^2 = \dsum_{k=M+1}^{N} (2 k - 1)$\,.
\end{fact}


\begin{proof}
	Il suffit d'utiliser $N^2 = \dsum_{k=1}^{N} (2 k - 1)$\,, une formule facile à démontrer algébriquement, et évidente à découvrir géométriquement.
\end{proof}


% ------------------ %


L'identité précédente permet d'éliminer beaucoup de situations en s'aidant, si besoin, d'un petit programme informatique comme celui donné à la fin de cette section.

\begin{fact} \label{diff-square-ko}
	Soit $(N, M) \in \NNs \times \NNs$ tel que $N > M$\,.
	%
	\begin{enumerate}
		\item $N^2 - M^2 \geq 2M + 1$\,.
		
		\item $N^2 - M^2 < 3$ est impossible.
	\end{enumerate}
\end{fact}


\begin{proof}
	\leavevmode
	
	\begin{enumerate}
		\item $N^2 - M^2 = \dsum_{k=M+1}^{N} (2 k - 1) \geq 2(M + 1) - 1 = 2M + 1$
		
		\smallskip
		\noindent
		On peut aussi juste procéder comme suit. 
		
		\smallskip
		\noindent
		$N^2 - M^2 = (N - M)(N + M) \geq 1 \cdot (M + 1 + M) = 2M + 1$



		\item Immédiat puisque $2M + 1 \geq 3$\,.
	\end{enumerate}

\end{proof}


