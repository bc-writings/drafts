Via $N^2 - M^2 = (N - M)(N + M)$\,, il est immédiat de noter que 
$\forall (N, M) \in \NNs \times \NNs$\,, si $N > M$\,, alors $N^2 - M^2 \geq 3$\,. Le fait suivant précise ceci.


\begin{fact} \label{dist-square}
	$\forall (N, M) \in \NNs \times \NNs$, 
	si $N > M$\,, alors $N^2 - M^2 = \dsum_{k=M+1}^{N} (2 k - 1)$\,.
	
	En particulier, $N^2 - M^2 \geq 3$\,.
\end{fact}


\begin{proof}
	Il suffit d'utiliser $N^2 = \dsum_{k=1}^{N} (2 k - 1)$\,.
\end{proof}


% ------------------ %


Bien que simple, le fait suivant va nous rendre de grands services dans la suite.


\begin{fact} \label{facto-square}
	$\forall n \in \NNssquare$\,, s'il existe $m \in \NNssquare$ tel que $n =  f m$ alors $f  \in \NNssquare$\,.
\end{fact}


\begin{proof}
	Il suffit de passer via les décompositions en facteurs premiers de $n$\,, $m$ et $f$\,.
\end{proof}
