Voici une approche la plus simple possible ne faisant pas appel au principe des tiroirs.
Nous commençons par supposer que $\consprod<5> \in \NNssquare$\,.
    
\smallskip

En notant $m = n+2$\,, nous avons $\consprod<5> = m (m \pm 2) (m \pm 1) = m (m^2 - 1) (m^2 - 4)$ où $m \in \NN_{\geq 3}$\,.

\medskip
%	\newpage

Démontrons que $m \in \NNssquare$\,.
%	
\begin{itemize}
	\item Si $m \in 2 \NN + 1$\,, nous avons clairement $\GCD{m}{(m^2 - 1)} = 1$ et $\GCD{m}{(m^2 - 4)} = 1$ (ici, la parité de $m$ doit être utilisée).
	Donc $\GCD{m}{\big( (m^2 - 1) (m^2 - 4) \big)} = 1$\,, puis $m \in \NNssquare$ selon le fait \ref{prime-square}.


	\item Si $m \in 2 \NN$\,, alors $\GCD{m}{(m^2 - 1)} = 1$ et $\GCD{m}{(m^2 - 4)} \in \setgene{1, 2, 4}$ (ici, la parité de $m$ ne limite pas les possibilités). Soyons plus fin.
	Notant $m - 2 = 2A$ et $m + 2 = 2B$\,, nous avons clairement $\GCD{m}{A} = 1 = \GCD{m}{B}$ car $\GCD{m}{(m - 2)} = 2 = \GCD{m}{(m + 2)}$\,.
	Comme $\consprod<5> = 4 m (m^2 - 1) A B$\,, nous avons aussi $m (m^2 - 1) A B \in \NNssquare$ via le fait \ref{facto-square}, et finalement $m \in \NNssquare$ selon le fait \ref{prime-square} et $\GCD{m}{\big( (m^2 - 1) A B \big)} = 1$\,. 
\end{itemize}

\medskip

Ce qui suit lève une contradiction.
%	
\begin{itemize}
	\item $m \in \NNssquare$ et $\consprod<5> \in \NNssquare$ donnent $(m^2 - 1) (m^2 - 4) \in \NNssquare$ via le fait \ref{facto-square}. 

	\item En posant $x = m^2 \in \NN_{\geq 9}$\,, nous arrivons à $(x - 1) (x - 4) = x^2 - 5 x + 4 \in \NNssquare$\,, mais ceci est impossible d'après l'implication suivante.
	
	\medskip	
	\noindent\kern-10pt%
	\begin{stepcalc}[style=ar*, ope={\implies}]
		x^2 - 5 x + 4 = (x-2)^2 - x
		\,\,\text{ et }\,\,
		x^2 - 5 x + 4 = (x-3)^2 + x - 5
	\explnext*{$x-5 >0$ et $x > 0$\,.}{}
		(x-3)^2 < x^2 - 5 x + 4 < (x - 2)^2
	\end{stepcalc}
\end{itemize}

\vspace{-1.5ex}
\qedhere