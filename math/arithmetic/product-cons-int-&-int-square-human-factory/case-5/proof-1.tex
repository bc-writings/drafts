Commençons par une idée simple consistant à se concentrer sur les nombres premiers de valuation impaire dans $\consprod<5>$ supposé être un carré parfait.
Nous commençons par supposer que $\consprod<5> \in \NNssquare$\,.
 
\smallskip

Clairement, 
$\forall p \in \PP_{\geq 5}$\,, 
$\forall i \in \ZintervalC{0}{4}$\,, 
$\padicval{n + i} \in 2 \NN$\,.
Pour $p = 2$ et $p = 3$\,, nous avons les alternatives suivantes pour chaque facteur $(n+i)$ de $\consprod<5>$\,.
%
\begin{itemize}
	\smallskip
	\item \alt{1}\,
	$\big( \padicval[2]{n + i} , \padicval[3]{n + i} \big) \in 2 \NN \times 2 \NN$

	\smallskip
	\item \alt{2}\,
	$\big( \padicval[2]{n + i} , \padicval[3]{n + i} \big) \in 2 \NN \times \big( 2 \NN + 1)$

	\smallskip
	\item \alt{3}\,
	$\big( \padicval[2]{n + i} , \padicval[3]{n + i} \big) \in \big( 2 \NN + 1 \big) \times 2 \NN$

	\smallskip
	\item \alt{4}\,
	$\big( \padicval[2]{n + i} , \padicval[3]{n + i} \big) \in \big( 2 \NN + 1 \big) \times \big( 2 \NN + 1)$
\end{itemize}

\medskip

Comme nous avons cinq facteurs pour quatre alternatives, ce bon vieux principe des tiroirs va nous permettre de lever des contradictions très facilement.
%
\begin{itemize}
	\medskip
	\item Deux facteurs différents $(n+i)$ et $(n+i^\prime)$ vérifient \alt{1}\,.
		
	\smallskip
	\noindent
	Dans ce cas, $(n+i, n+i^\prime) = (M^2, N^2)$ avec $(M, N) \in \NNs$.
	Par symétrie des rôles, on peut supposer $N > M$\,, de sorte que $N^2 - M^2 \in \setgene{1, 2, 3, 4}$\,. 
	Selon le fait \ref{diff-square-ko}, seul $N^2 - M^2 = 3$ avec $(M, N) = (1, 2)$ est possible, puis nécessairement $n = 1$\,, or $\consprod[1]<5> = 5 ! \in \NNsquare$ est faux car $\padicval[5]{5!} = 1$\,.
		
		
	\explainthis{Autre méthode : on note que $n \notin \NNssquare$ car sinon $n(n+1)(n+2)(n+3)(n+4) \in \NNssquare$ donne $(n+1)(n+2)(n+3)(n+4) \in \NNssquare$ via le fait \ref{facto-square}, mais ceci contredit le fait \ref{case-4}.%
	De même, $n+4 \notin \NNssquare$\,.%
	Dès lors, nous avons $\setgene{n+i, n+i^\prime} \subseteq \setgene{n+1, n+2, n+3}$\,, d'où l'existence de deux carrés parfaits non nuls éloignés de moins de $3$\,, et ceci contredit le fait \ref{diff-square-ko}.}


	\medskip
	\item Deux facteurs différents $(n+i)$ et $(n+i^\prime)$ vérifient \alt{2}\,.
		
	\smallskip
	\noindent
	Dans ce cas, le couple de facteurs est $(n, n + 3)$\,, ou $(n + 1, n + 4)$\,.    
	%
	\begin{enumerate}
		\item Supposons d'abord que $n$ et $(n+3)$ vérifient \alt{2}\,.
			
		\noindent
		Comme $\forall p \in \PP - \setgene{3}$\,, $\padicval{n} \in 2 \NN$ et $\padicval{n + 3} \in 2 \NN$\,,
		mais aussi $\padicval[3]{n} \in 2 \NN + 1$ et $\padicval[3]{n + 3} \in 2 \NN + 1$\,,
		nous avons $n = 3 M^2$ et $n+3 = 3 N^2$ où $(M, N) \in ( \NNs )^2$\,.
		Or, ceci donne $3 = 3 N^2 - 3 M^2$\,, puis $N^2 - M^2 = 1$ qui contredit le fait \ref{diff-square-ko}.

		\item De façon analogue, on ne peut pas avoir $(n+1)$ et $(n+4)$ vérifiant \alt{2}\,.
	\end{enumerate}


	\medskip
	\item Deux facteurs différents $(n+i)$ et $(n+i^\prime)$ vérifient \alt{3}\,.
		
	\smallskip
	\noindent
	Comme dans le point précédent, c'est impossible car on aurait $2 = 2 N^2 - 2 M^2$\,, ou $4 = 2 N^2 - 2 M^2$, mais ceci contredirait le fait \ref{diff-square-ko}. 
		
	\smallskip
		
	\noindent
	En effet, ici les couples possibles sont $(n, n + 2)$\,, $(n, n + 4)$\,,  $(n + 2, n + 4)$ et $(n + 1, n + 3)$
	\footnote{
		A priori, rien n'empêche d'avoir $n$\,, $(n + 2)$ et $(n + 4)$ vérifiant tous les trois \alt{3}\,.
	}.


	\medskip
	\item Deux facteurs différents $(n+i)$ et $(n+i^\prime)$ vérifient \alt{4}\,.
		
	\smallskip
	\noindent
	Ceci donne deux facteurs différents divisibles par $6$\,, mais c'est impossible. \qedhere
\end{itemize}