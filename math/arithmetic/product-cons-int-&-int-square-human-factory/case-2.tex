\begin{fact} \label{case-2}
	 $\forall n \in \NNs$, $n(n+1) \notin \NNsquare$\,.
\end{fact}


% ------------------ %


\begin{proof}[Preuve 1]
	Il suffit de noter que $\forall n \in \NNs$, $n^2 < n(n+1) < (n+1)^2$.
\end{proof}


% ------------------ %


\begin{proof}[Preuve 2]
    Supposons que $\consprod<2> = n(n+1) \in \NNssquare$\,.
    
    \smallskip
    
    Comme $\GCD{n}{(n + 1)} = 1$\,, le fait \ref{prime-square} donne $(n, n + 1) \in \NNssquare \times \NNssquare$\,, d'où l'existence de deux carrés parfaits non nuls distants de $1$\,.
    D'après le fait \ref{diff-square-ko}, ceci est impossible.
\end{proof}


% ------------------ %


\begin{proof}[Preuve 3]
	Supposons que $\consprod<2> = n(n+1) = N^2$ où $N \in \NNs$.
     
    \smallskip
    
    Nous obtenons une contradiction comme suit.
	
	\medskip
	
	\begin{stepcalc}[style = ar*, ope = \iff]
		n(n+1) = N^2
	\explnext*{$n(n+1) = 2 \dsum_{k=1}^{n} k$  et  $N^2 = \dsum_{k=1}^{N} (2 k - 1)$\,.}{}
		2 \dsum_{k=1}^{n} k = \dsum_{k=1}^{N} (2 k - 1)
	\explnext{}
		\dsum_{k=1}^{n} 2k = \dsum_{k=1}^{N} 2 k - N
	\explnext*{$N > n$ car $N^2 - n^2 = n > 0$\,.}{}
		\dsum_{k=n+1}^{N} 2k - N = 0
	\explnext*{$N > 0$ rend impossible la dernière égalité.}{}
		\dsum_{k=n+1}^{N-1} 2k + N = 0
	\end{stepcalc}

	\vspace{-2ex}	
	\leavevmode
\end{proof}

