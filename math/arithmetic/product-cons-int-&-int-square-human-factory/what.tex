Dans l'article \emph{\enquote{Note on Products of Consecutive Integers}}
\footnote{
	J. London Math. Soc. 14 (1939).
},
Paul Erdős démontre que pour tout couple $(n, k) \in \NNs \times \NNs$\,, le produit de $(k+1)$ entiers consécutifs $n (n + 1) \cdots (n + k)$ n'est jamais le carré d'un entier. 
Plus précisément, l'argument général de Paul Erdős est valable pour $k + 1 \geq 100$\,, soit à partir de $100$ facteurs.

\medskip

Dans ce document, nous proposons quelques cas particuliers résolus de façon \enquote{adaptative} à la sueur des neurones, le but recherché étant de fournir différentes approches même si parfois cela peut prendre du temps.


\begin{remark}
	Il arrivera parfois que certaine démonstration cite d'autres preuves données plus tard dans le texte. Ceci permet de respecter les sources qui ont été utilisées.
\end{remark}


\begin{remark}
	La majorité des preuves évitent l'emploi de récurrence.
\end{remark}


\begin{remark}
	Vous trouverez dans mon document \emph{\enquote{Carrés parfaits et produits d'entiers consécutifs -- Une méthode efficace}}\,, un moyen très efficace pour traiter sans effort les premiers cas à la main, mais via de la récurrence.
	L'existence de ce document justifie que nous ne parlions de cette méthode que pour $k = 6$ afin d'obtenir une preuve alternative.
\end{remark}
