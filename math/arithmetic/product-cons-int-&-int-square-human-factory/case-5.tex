\begin{fact} \label{case-5}
	 $\forall n \in \NNs$\,, $n(n+1)(n+2)(n+3)(n+4) \notin \NNssquare$\,.
\end{fact}


% ------------------ %


Commençons par une idée simple consistant à se concentrer sur les nombres premiers de valuation impaire dans $\consprod<5>$ supposé être un carré parfait.


\begin{proof}[Preuve 1]%
    Supposons que $\consprod<5> \in \NNssquare$\,.
    
    \smallskip
    
    Clairement, 
    $\forall p \in \PP_{\geq 5}$\,, 
    $\forall i \in \ZintervalC{0}{4}$\,, 
    $\padicval{n + i} \in 2 \NN$\,.
    Pour $p = 2$ et $p = 3$\,, nous avons les alternatives suivantes pour chaque facteur $(n+i)$ de $\consprod<5>$\,.
    %
    \begin{itemize}
    	\smallskip
		\item \alt{1}\,
		$\big( \padicval[2]{n + i} , \padicval[3]{n + i} \big) \in 2 \NN \times 2 \NN$

    	\smallskip
		\item \alt{2}\,
		$\big( \padicval[2]{n + i} , \padicval[3]{n + i} \big) \in 2 \NN \times \big( 2 \NN + 1)$

    	\smallskip
		\item \alt{3}\,
		$\big( \padicval[2]{n + i} , \padicval[3]{n + i} \big) \in \big( 2 \NN + 1 \big) \times 2 \NN$

    	\smallskip
		\item \alt{4}\,
		$\big( \padicval[2]{n + i} , \padicval[3]{n + i} \big) \in \big( 2 \NN + 1 \big) \times \big( 2 \NN + 1)$
    \end{itemize}
    
    \medskip
    
    Comme nous avons cinq facteurs pour quatre alternatives, ce bon vieux principe des tiroirs va nous permettre de lever des contradictions très facilement.
    %
    \begin{itemize}
    	\medskip
		\item Deux facteurs différents $(n+i)$ et $(n+i^\prime)$ vérifient \alt{1}\,.
		
		\smallskip
		\noindent
		Dans ce cas, $(n+i, n+i^\prime) = (M^2, N^2)$ avec $(M, N) \in \NNs$.
		Par symétrie des rôles, on peut supposer $N > M$\,, de sorte que $N^2 - M^2 \in \setgene{1, 2, 3, 4}$\,. 
		Selon le fait \ref{diff-square-ko}, seul $N^2 - M^2 = 3$ avec $(M, N) = (1, 2)$ est possible, puis nécessairement $n = 1$\,, or $\consprod[1]<5> = 5 ! \in \NNsquare$ est faux car $\padicval[5]{5!} = 1$\,.
		
		
		\explainthis{Autre méthode : on note que $n \notin \NNssquare$ car sinon $n(n+1)(n+2)(n+3)(n+4) \in \NNssquare$ donne $(n+1)(n+2)(n+3)(n+4) \in \NNssquare$ via le fait \ref{facto-square}, mais ceci contredit le fait \ref{case-4}.%
		De même, $n+4 \notin \NNssquare$\,.%
		Dès lors, nous avons $\setgene{n+i, n+i^\prime} \subseteq \setgene{n+1, n+2, n+3}$\,, d'où l'existence de deux carrés parfaits non nuls éloignés de moins de $3$\,, et ceci contredit le fait \ref{diff-square-ko}.}


    	\medskip
		\item Deux facteurs différents $(n+i)$ et $(n+i^\prime)$ vérifient \alt{2}\,.
		
		\smallskip
		\noindent
		Dans ce cas, le couple de facteurs est $(n, n + 3)$\,, ou $(n + 1, n + 4)$\,.    
		%
		\begin{enumerate}
			\item Supposons d'abord que $n$ et $(n+3)$ vérifient \alt{2}\,.
			
			\noindent
			Comme $\forall p \in \PP - \setgene{3}$\,, $\padicval{n} \in 2 \NN$ et $\padicval{n + 3} \in 2 \NN$\,,
			mais aussi $\padicval[3]{n} \in 2 \NN + 1$ et $\padicval[3]{n + 3} \in 2 \NN + 1$\,,
			nous avons $n = 3 M^2$ et $n+3 = 3 N^2$ où $(M, N) \in ( \NNs )^2$\,.
			Or, ceci donne $3 = 3 N^2 - 3 M^2$\,, puis $N^2 - M^2 = 1$ qui contredit le fait \ref{diff-square-ko}.

			\item De façon analogue, on ne peut pas avoir $(n+1)$ et $(n+4)$ vérifiant \alt{2}\,.
		\end{enumerate}


    	\medskip
		\item Deux facteurs différents $(n+i)$ et $(n+i^\prime)$ vérifient \alt{3}\,.
		
		\smallskip
		\noindent
		Comme dans le point précédent, c'est impossible car on aurait $2 = 2 N^2 - 2 M^2$\,, ou $4 = 2 N^2 - 2 M^2$, mais ceci contredirait le fait \ref{diff-square-ko}. 
		
		\smallskip
		
		\noindent
		En effet, ici les couples possibles sont $(n, n + 2)$\,, $(n, n + 4)$\,,  $(n + 2, n + 4)$ et $(n + 1, n + 3)$
		\footnote{
			A priori, rien n'empêche d'avoir $n$\,, $(n + 2)$ et $(n + 4)$ vérifiant tous les trois \alt{3}\,.
		}.


    	\medskip
		\item Deux facteurs différents $(n+i)$ et $(n+i^\prime)$ vérifient \alt{4}\,.
		
		\smallskip
		\noindent
		Ceci donne deux facteurs différents divisibles par $6$\,, mais c'est impossible. \qedhere
    \end{itemize}
\end{proof}


% ------------------ %


La preuve suivante s'inspire directement de la démonstration du cas \ref{case-10} : on utilise plus efficacement le principe des tiroirs en commençant par un raisonnement plus grossier a priori. Comme quoi certains détails ne comptent pas, ou au contraire comptent trop !

\begin{proof}[Preuve 2]%
    Supposons que $\consprod<5> \in \NNssquare$\,.
    
    \smallskip
    
    Clairement, 
    $\forall p \in \PP_{\geq 5}$\,, 
    $\forall i \in \ZintervalC{0}{4}$\,, 
    $\padicval{n + i} \in 2 \NN$\,.
    On doit donc s'intéresser à $p \in \setgene{2, 3}$\,, mais on peut observer très grossièrement qu'au maximum deux facteurs $(n + i)$ de $\consprod<5>$ sont divisibles par $3$\,, donc au moins $3$ facteurs sont de valuation $p$-adique paire dès que $p \in \PP_{\geq 3}$\,.
    Ces facteurs vérifient alors l'une des deux alternatives suivantes,
    chacune d'elles levant une contradiction.
    %
    \begin{itemize}
    	\medskip
		\item Deux facteurs différents $(n+i)$ et $(n+i^\prime)$ sont de valuations $2$-adiques impairs.
		
		\smallskip
		\noindent
		Dans ce cas, $(n+i, n+i^\prime) = (2 M^2, 2 N^2)$ avec $\abs{2(N^2 - M^2)} \in \ZintervalC{1}{4}$\,, c'est-à-dire $\abs{N^2 - M^2} \in \setgene{1, 2}$\,, mais c'est impossible d'après le fait \ref{diff-square-ko}.


    	\medskip
		\item Deux facteurs différents $(n+i)$ et $(n+i^\prime)$ sont de valuations $2$-adiques pairs.
		
		\smallskip
		\noindent
		Dans ce cas, $(n+i, n+i^\prime) = (M^2, N^2)$ avec $\abs{N^2 - M^2} \in \ZintervalC{1}{4}$\,, mais ceci n'est possible que si $\abs{N^2 - M^2} = 3$ d'après le fait \ref{diff-square-ko} qui donne aussi que soit $(M, N) = (1, 2)$\,, soit $(M, N) = (2, 1)$\,.
		Ceci impose d'avoir $n = 1$\,, mais $\consprod[1]<5> = 5! \notin \NNsquare$ car $\padicval[5]{5!} = 1$\,.
		%
		\qedhere
    \end{itemize}
\end{proof}


% ------------------ %


Voici une approche la plus simple possible ne faisant pas appel au principe des tiroirs.


\begin{proof}[Preuve 3]
	Supposons que $\consprod<5> \in \NNssquare$\,.
    
    \smallskip
    
	En notant $m = n+2$\,, nous avons $\consprod<5> = m (m \pm 2) (m \pm 1) = m (m^2 - 1) (m^2 - 4)$ où $m \in \NN_{\geq 3}$\,.
    
    \medskip
%	\newpage
    
    Démontrons que $m \in \NNssquare$\,.
	%	
	\begin{itemize}
		\item Si $m \in 2 \NN + 1$\,, nous avons clairement $\GCD{m}{(m^2 - 1)} = 1$ et $\GCD{m}{(m^2 - 4)} = 1$ (ici, la parité de $m$ doit être utilisée).
		Donc $\GCD{m}{\big( (m^2 - 1) (m^2 - 4) \big)} = 1$\,, puis $m \in \NNssquare$ selon le fait \ref{prime-square}.

		\item Si $m \in 2 \NN$\,, alors $\GCD{m}{(m^2 - 1)} = 1$ et $\GCD{m}{(m^2 - 4)} \in \setgene{1, 2, 4}$ (ici, la parité de $m$ ne limite pas les possibilités). Soyons plus fin.
		Notant $m - 2 = 2A$ et $m + 2 = 2B$\,, nous avons clairement $\GCD{m}{A} = 1 = \GCD{m}{B}$ car $\GCD{m}{(m - 2)} = 2 = \GCD{m}{(m + 2)}$\,.
		Comme $\consprod<5> = 4 m (m^2 - 1) A B$\,, nous avons aussi $m (m^2 - 1) A B \in \NNssquare$ via le fait \ref{facto-square}, et finalement $m \in \NNssquare$ selon le fait \ref{prime-square} et $\GCD{m}{\big( (m^2 - 1) A B \big)} = 1$\,. 
	\end{itemize}
    
    \medskip
    
    Ce qui suit lève une contradiction.
	%	
	\begin{itemize}
		\item $m \in \NNssquare$ et $\consprod<5> \in \NNssquare$ donnent $(m^2 - 1) (m^2 - 4) \in \NNssquare$ via le fait \ref{facto-square}. 

		\item En posant $x = m^2 \in \NN_{\geq 9}$\,, nous arrivons à $(x - 1) (x - 4) = x^2 - 5 x + 4 \in \NNssquare$\,, mais ceci est impossible d'après l'implication suivante.
		
		\noindent\kern-10pt%
		\begin{stepcalc}[style=ar*, ope={\implies}]
			x^2 - 5 x + 4 = (x-2)^2 - x
			\,\,\text{ et }\,\,
			x^2 - 5 x + 4 = (x-3)^2 + x - 5
		\explnext*{$x-5 >0$ et $x > 0$\,.}{}
			(x-3)^2 < x^2 - 5 x + 4 < (x - 2)^2
		\end{stepcalc}
	\end{itemize}

    \vspace{-1.5ex}
    \qedhere
\end{proof}


% ------------------ %


Voici une approche similaire à la dernière preuve du cas \ref{case-4}.
	
	
\begin{proof}[Preuve 4]
    Supposons que $\consprod<5> \in \NNssquare$\,.
    
    \smallskip
    
	Clairement, $\forall p \in \PP_{\geq 5}$\,, 
   	$\forall i \in \ZintervalC{0}{3}$\,, 
    $\padicval{n + i} \in 2 \NN$\,,
    ceci nous amène à considérer deux alternatives.
    
    \medskip
%    \newpage
    
    Supposons d'abord $\setgene{n, n + 2, n + 4} \subset 2 \NN + 1$\,.
	%	
	\begin{itemize}
		\item
		Nous avons alors
		$\forall p \in \PP - \setgene{3}$\,, 
   		$(\padicval{n}, \padicval{n+2}, \padicval{n+4}) \in ( 2 \NN )^3$,
		donc, pour tout naturel $m \in \setgene{n, n + 2, n + 4}$\,, 
		il existe $M \in \NNs$ tel que 
		$m = M^2$ ou $m = 3 M^2$\,.
	
		\item En raisonnant modulo $3$\,, on constate que $3$ divise au maximum un seul des trois éléments de $\setgene{n, n + 2, n + 4}$\,, donc nous avons au moins deux carrés parfaits dans $\setgene{n, n + 2, n + 4}$\,, mais ceci contredit le fait \ref{diff-square-ko} (deux carrés parfaits ne sont jamais distants de $2$ ou $4$).
    \end{itemize}
    
    \medskip
    
    Supposons maintenant $\setgene{n + 1, n + 3} \subset 2 \NN + 1$\,.
	%	
	\begin{itemize}
		\item
		Comme ci-dessus,
		soit $(n + 1, n + 3) = (A^2, 3 B^2)$\,,
		soit $(n + 1, n + 3) = (3 A^2, B^2)$\,,
		avec $(A, B) \in ( \NNs )^2$\,,
		car $(n + 1, n + 3) = (A^2, B^2)$ est impossible.
	
		\item Supposons $(n + 1, n + 3) = (A^2, 3 B^2)$\,. Ce qui suit lève alors une contradiction.
		%	
		\begin{itemize}
			\item Forcément, $n = 3 C^2$ ou $n = 6 C^2$ avec $C \in \NNs$.
			Le fait \ref{diff-square-ko} impose d'avoir $n = 6 C^2$.

			\item Donc 
			$\setgene{n + 2, n + 4} \subset 2 \NN - 3 \NN$\,, 
			puis, via le fait \ref{diff-square-ko},
			$\setgene{n + 2, n + 4} = \setgene{D^2, 2 E^2}$ 
			avec $(D, E) \in ( \NNs )^2$ nécessairement.

			\item $(n+1)$ et $(n+2)$ étant trop proches pour être tous les deux des carrés parfaits, nous arrivons à $(n + 2, n + 4) = (2 D^2, E^2)$\,.

			\item Or $n + 4 \in \NNssquare$ et $\consprod<5> \in \NNssquare$ donnent $\consprod<4> \in \NNssquare$ d'après le fait \ref{facto-square}, mais ceci contredit le fait \ref{case-4}.
		\end{itemize}
	
		\item Forcément, $(n + 1, n + 3) = (3 A^2, B^2)$\,, mais ce qui suit lève une nouvelle contradiction via une démarche similaire à la précédente.
		%	
		\begin{itemize}
			\item Forcément, $n + 4 = 6 C^2$ avec $C \in \NNs$. 

			\item Ensuite, $(n, n + 2) = (D^2, 2 E^2)$ avec $(D, E) \in ( \NNs )^2$.

			\item $n \in \NNssquare$ et $\consprod<5> \in \NNssquare$ donnent $\consprod[n+1]<4> \in \NNssquare$\,, ce qui est faux.\qedhere
		\end{itemize}
    \end{itemize}
\end{proof}


% ------------------ %


Bien que longue, la preuve suivante se comprend bien, car nous ne faisons qu'avancer à vue, mais avec rigueur.


\begin{proof}[Preuve 5]%
	Supposons que $\consprod<5> \in \NNssquare$\,.
	
    \smallskip
    
    Posant $m = n+2$\,, nous avons $\consprod<5> = m (m \pm 2) (m \pm 1) = m(m^2 - 1)(m^2 - 4)$ où $m \in \NN_{\geq 3}$\,.
	Pour la suite, on pose $u = m^2 - 1$ et $q = m^2 - 4$\,.

	\medskip
	
	Notons que $u \notin \NNssquare$ et $q \notin \NNssquare$\,.
		%
	\begin{itemize}
		\item $u \in \NNssquare$ donne $m^2 - 1 \in \NNssquare$ qui est impossible d'après le fait \ref{diff-square-ko}.

		\item $q \in \NNssquare$ donne $m^2 - 4 \in \NNssquare$ qui est impossible d'après le fait \ref{diff-square-ko}.
	\end{itemize}

	\medskip
	
	Supposons d'abord que $m \in \NNssquare$\,.
	%
	\begin{itemize}
		\item De $muq \in \NNssquare$\,, nous déduisons que $uq \in \NNssquare$ via le fait \ref{facto-square}.

		\item Comme $u - q = 3$\,, nous savons que $u \wedge q \in \setgene{1, 3}$\,.

		\item Si $u \wedge q = 1$\,, 
		alors $(u, q) \in \NNssquare \times \NNssquare$ d'après le fait \ref{prime-square}, mais ceci est impossible.

		\item Si $u \wedge q = 3$\,, 
		alors $\forall p \in \PP - \setgene{3}$\,, 
		$\padicval{u} \in 2 \NN$ et $\padicval{q} \in 2 \NN$\,,
		mais aussi $\padicval[3]{u} \in 2 \NN + 1$ et $\padicval[3]{q} \in 2 \NN + 1$\,, car  $u \notin \NNssquare$ et $q \notin \NNssquare$\,.
		Donc 
		$u = 3 U^2$ et $q = 3 Q^2$ avec $(U, Q) \in ( \NNs )^2$\,.
		Or $u - q = 3$ donne $U^2 - Q^2 = 1$\,, et le fait \ref{diff-square-ko} nous indique une contradiction.
	\end{itemize}
	
	\medskip
	
	Supposons maintenant que $m \notin \NNssquare$\,.
	%
	\begin{itemize}
		\item Ici, $m = \alpha M^2$\,, $u = \beta U^2$\,, $q = \gamma Q^2$ avec $(M, U, Q) \in ( \NNs )^3$ et $\setgene{\alpha, \beta, \gamma} \subset \NNsf \cap \NN_{>1}$\,.


		\item Notons que $\beta \neq \gamma$\,, car, dans le cas contraire, $3 = u - q = \beta \big( U^2 - Q^2 \big)$ fournirait $\beta = 3$\, puis $U^2 - Q^2 = 1$\,, et ceci contredirait le fait \ref{diff-square-ko}.


		\item Nous avons $m \wedge u = 1$\,, $m \wedge q \in \setgene{1, 2, 4}$ et $u \wedge q \in \setgene{1, 3}$
		avec $m \wedge u = m \wedge q = u \wedge q = 1$ impossible car sinon on aurait $(m, u, q) \in ( \NNssquare )^3$ via $muq \in \NNssquare$ et le fait \ref{prime-square}.


		\item Clairement, $\forall p \in \PP_{\geq 5}$\,, $\big( \padicval{m} , \padicval{u} , \padicval{q} \big) \in ( 2 \NN )^3$.


		\item Les points précédents donnent 
		$\setgene{\alpha, \beta, \gamma} \subseteq \setgene{2, 3, 6}$
		avec de plus
		$\beta \neq \gamma$\,,
		ainsi que 
		$\alpha \wedge \beta = 1$\,, $\alpha \wedge \gamma \in \setgene{1, 2}$ et $\beta \wedge \gamma \in \setgene{1, 3}$\,.
%		avec $\alpha \wedge \beta = \alpha \wedge \gamma = \beta \wedge \gamma = 1$ impossible.
		%
		Notons au passage que $\alpha \wedge \beta = 1$ implique $(\alpha, \beta) = (2, 3)$\,, ou $(\alpha, \beta) = (3, 2)$\,.
		%
		Via le tableau \enquote{mécanique} ci-après, nous obtenons que forcément $(\alpha, \beta, \gamma) = (2, 3, 2)$ ou $(\alpha, \beta, \gamma) = (2, 3, 6)$\,. Le plus dur est fait !
	\end{itemize}

	\begin{center}
		\begin{tblr}{
			colspec    = {*{7}{Q[c,$]}},
			vline{2-7} = {},
			hline{2-5} = {}
		}
        	  \alpha & \beta & \gamma 
			& \alpha \wedge \beta & \alpha \wedge \gamma & \beta \wedge \gamma
			& Statut
			\\
        	  2 & 3 & 2
			& 1 & 2 & 1
			& \mycheckmark
			\\
        	  2 & 3 & 6
			& 1 & 2 & 3
			& \mycheckmark
			\\
        	  3 & 2 & 3
			& 1 & 3 & 1
			& \myboxtimes
			\\
        	  3 & 2 & 6
			& 1 & 3 & 2
			& \myboxtimes
        \end{tblr}
	\end{center}


	\begin{itemize}
		\item $(\alpha, \beta, \gamma) = (2, 3, 2)$ nous donne $m = 2 M^2$, $u = 3 U^2$ et $q = 2 Q^2$\,, d'où la contradiction $3 \cdot4 M^2 U^2 Q^2 \in \NNssquare$.


		\item $(\alpha, \beta, \gamma) = (2, 3, 6)$ nous donne $m = 2 M^2$, $m^2 - 1 = 3 U^2$ et $m^2 - 4 = 6 Q^2$\,, mais ce qui suit lève une autre contradiction.
		%
		\begin{itemize}
			\item Travaillons modulo $3$\,.
			Nous avons $m \equiv 2 M^2 \equiv \text{$0$ ou $-1$}$\,. 
			Or $m^2 - 1 = 3 U^2$ donne $m^2 \equiv 1$\,, d'où $m \equiv -1$\,, puis $3 \divides m - 2$\,, et enfin $6 \divides m - 2$ puisque $m$ est pair.
			
			\item Posant $m - 2 = 6 r$ et notant $s = m + 2$\,, nous avons $6 r s = 6 Q^2$\,, puis $r s = Q^2$.
			
			\item $s \notin \NNssquare$\,. Sinon $(m-2)(m-1)m(m+1) \in \NNssquare$ via $(m-2)(m-1)m(m+1)(m+2)  \in \NNssquare$ et le fait \ref{facto-square}, mais ceci ne se peut pas d'après le fait \ref{case-4}.
			
			\item Les deux résultats précédents et le fait \ref{same-square-free} donnent $(\pi, R, S) \in \NNsf \times ( \NNs )^2$ tel que $r = \pi R^2$ et $s = \pi S^2$ avec  $\pi \in \NN_{>1}$\,.
			
			\item Dès lors, $4 = s - 6r = \pi (S^2 - 6 R^2)$ donne $\pi = 2$\,, d'où $m + 2 = 2 S^2$\,.
			
			\item Finalement, $m = 2 M^2$ et $m + 2 = 2 S^2$ donnent $2 = 2(S^2 - M^2)$\,, soit $1 = S^2 - M^2$, ce qui contredit le fait \ref{diff-square-ko}.
			%
			\qedhere
		\end{itemize}
	\end{itemize}
\end{proof}

