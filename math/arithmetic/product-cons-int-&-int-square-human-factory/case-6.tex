\begin{fact} \label{case-6}
	 $\forall n \in \NNs$\,, $\consprod<6> \notin \NNsquare$\,.
\end{fact}


% ------------------ %


La démonstration suivante se trouve dans l'article \enquote{Solution of a Problem}
\footnote{
	The Analyst (1874).
}
de G. W. Hill et J. E. Oliver.
Une petite simplification a été faite pour arriver à $\consprod<6> = (a - 4) a (a + 2)$\,.


\begin{proof}[Preuve 1]%
    Supposons que $\consprod<6> \in \NNsquare$\,.
    
    \smallskip
    
    Commençons par de petites manipulations algébriques où la première modification fait apparaître le même coefficient pour $n$ dans chaque parenthèse.
    
    \medskip
    \begin{stepcalc}[style = sar]
    	\consprod<6>
	\explnext{}
		n (n+5) \cdot (n+1) (n+4) \cdot (n+2) (n+3)
	\explnext{}
		(n^2 + 5n) (n^2 + 5n + 4) (n^2 + 5n + 6)
	\explnext*{$x = n^2 + 5n \in \NN_{\geq 6}$}{}
		x (x + 4) (x + 6)
	\explnext*{$a = x + 4 \in \NN_{\geq 10}$}{}
		(a - 4) a (a + 2)
    \end{stepcalc}
  
    \medskip
    Nous avons $a \in \NN_{\geq 10}$ vérifiant $a (a + 2) (a - 4) \in \NNssquare$\,. 
    Posons $a = \alpha A^2$ où $(\alpha, A) \in \NNsf \times \NNs$\,,
    de sorte que $\alpha (\alpha A^2 + 2) (\alpha A^2 - 4) \in \NNssquare$ via le fait \ref{facto-square}.
    Or $\alpha\in \NNsf$ donne $\alpha \divides (\alpha A^2 + 2) (\alpha A^2 - 4)$\,, 
    d'où $\alpha \divides 8$\,, et ainsi $\alpha \in \setgene{1, 2}$
    \footnote{
    	On comprend ici le choix d'avoir $\consprod<6> = (a - 4) a (a + 2)$\,.
    }.
    Nous allons voir que ceci est impossible.
    
    \medskip
    
    Supposons que $\alpha = 1$\,.
    %
    \begin{itemize}
    	\item Notons les équivalences suivantes.
        
%        \medskip
        \noindent\kern-6pt%
        \begin{stepcalc}[style=ar*, ope=\iff]
        	(A^2 + 2) (A^2 - 4) \in \NNssquare
    	\explnext{$u = A^2 - 1$ où $- 1 = \dfrac{2 - 4}{2}$\,.}
        	(u + 3) (u - 3) \in \NNssquare
    	\explnext{}
        	u^2 - 9 \in \NNssquare
        \end{stepcalc}

		\item Ensuite, prenant $m \in \NNs$ tel que $m^2 = u^2 - 9$\,, le fait \ref{diff-square-ko} donne $(u, m) = (5, 4)$\, d'où la contradiction suivante.
        
%        \medskip
        \noindent\kern-6pt%
        \begin{stepcalc}[style=sar, ope=\iff]
        	u = 5
    	\explnext{}
        	A^2 - 1 = 5
    	\explnext*{$6 \notin \NNsquare$\,.}{}
        	A^2 = 6
        \end{stepcalc}
    \end{itemize}
    
    \medskip
    
%   	\vspace{-1ex}
	Supposons que $\alpha = 2$\,.
    %
    \begin{itemize}
    	\item Notons l'équivalence suivante.
        
%        \medskip
        \noindent\kern-6pt%
        \begin{stepcalc}[style=ar*, ope=\iff]
        	2 (2 A^2 + 2) (2 A^2 - 4) \in \NNssquare
    	\explnext{Via $4 \cdot 2 (A^2 + 1) (A^2 - 2)$\,.}
        	2 (A^2 + 1) (A^2 - 2) \in \NNssquare
        \end{stepcalc}

		\item Ensuite, en travaillant modulo $3$\,, nous avons
		$2 (A^2 + 1) (A^2 - 2) \equiv -4 \equiv -1$ qui ne correspond pas à un carré modulo $3$\,.
		%
		\qedhere 
    \end{itemize}   
\end{proof}


% ------------------ %

	
\begin{proof}[Preuve 2]
	Se reporter à la preuve du cas \ref{case-7} qui s'adapte mot pour mot au cas présent mais en considérant $u \in \setgene{n, n+1}$ tel que $\setgene{u, u + 2, u + 4} \subset 2 \NN + 1$\,.
\end{proof}


% ------------------ %


Bien que très longue
\footnote{
	Ce sera notre dernière tentative de démonstration à faible empreinte cognitive.
},
la preuve suivante est simple à comprendre car elle ne fait que dérouler le fil des faits découverts.

\begin{proof}[Preuve 3]%
    Supposons que $\consprod<6> \in \NNsquare$\,.

    \medskip
    \begin{stepcalc}[style = ar*, ope = \iff]
    	\consprod<6> = 
			n (n+1) (n+2) (n+3) (n+4) (n+5)
    \explnext*{$x = n + 2 + \dfrac12$ (on symétrise la formule).}{}
    	\consprod<6> = 
			\big( x \pm \dfrac52 \big) \big( x \pm \dfrac32 \big) \big( x \pm \dfrac12 \big)
    \end{stepcalc}

    \begin{stepcalc}[style = ar*, ope = \iff]
    	\consprod<6> = 
			\big( x \pm \dfrac52 \big) \big( x \pm \dfrac32 \big) \big( x \pm \dfrac12 \big)
    \explnext*{$y = 2 x$ (on chasse les fractions).}{}
    	2^6 \consprod<6> = 
			(y \pm 5) (y \pm 3) (y \pm 1)
    \end{stepcalc}
    
    \begin{stepcalc}[style = ar*, ope = \iff]
    	2^6 \consprod<6> = 
			(y \pm 5) (y \pm 3) (y \pm 1)
    \explnext*{$z = y^2$}{}
    	2^6 \consprod<6> = 
			(z - 25) (z - 9) (z - 1) 
    \explnext*{$u = z - 17$ où $17 = \dfrac{25 + 9}{2}$\,.}{}
    	2^6 \consprod<6> = 
			(u - 8) (u + 8) (u + 16)
    \end{stepcalc}

    \medskip
    Notant $a = u - 8$\,, $b = u + 8$ et $c = u + 16$\,, où $u = (2 n + 5)^2 - 17 \in 2 \NN$\,, nous avons les faits suivants.
    
    \begin{itemize}
		\item $u \in \NN_{\geq 32}$ car $(2 + 5)^2 - 17 = 32$\,.

		\item $(a, b, c) \in \big( \NN_{\geq 24} \big)^3$ avec $abc \in \NNssquare$ puisque $2^6 \consprod<6> \in \NNssquare$\,.

		\item $a \wedge b \divides 16$ via $b - a = 16$\,.

		\item $a \wedge c \divides 24$ via $c - a = 24$\,.

		\item $b \wedge c \divides 8$  via $c - b = 8$\,.

		\item En particulier, 
		$\forall p \in \PP_{>3}$\,, 
		$(\padicval{a}, \padicval{b}, \padicval{c}) \in ( 2 \NN )^3$.
	\end{itemize}

%	\newpage
	\medskip
	
	Démontrons qu'aucun des trois entiers $a$\,, $b$ et $c$ ne peut être un carré parfait.
	%
	\begin{itemize}
		\medskip
		\item Commençons par supposer que $a \in \NNssquare$\,. 
		
		\smallskip
		\noindent
		Dans ce cas, $bc \in \NNssquare$ via le fait \ref{facto-square}, soit $(u + 8) (u + 16) \in \NNssquare$\,.
		En posant $w = u + 12$\,, on arrive à $(w - 4) (w + 4) \in \NNssquare$\,, soit $w^2 - 16 \in \NNssquare$\,, d'où $(w, m) = (5, 3)$ grâce au fait \ref{diff-square-ko}.
		Or $u \in \NN_{\geq 32}$ donne $w \in \NN_{\geq 20}$\,, d'où une contradiction.

		
		\medskip
		\item Supposons maintenant que $b \in \NNssquare$\,.
		
		\smallskip
		\noindent
		Dans ce cas, $ac \in \NNssquare$\,, soit $(u - 8) (u + 16) \in \NNssquare$\,.
		En posant $w = u + 4$\,, on arrive à $(w - 12) (w + 12) \in \NNssquare$\,, soit $w^2 - 144 \in \NNssquare$\,.
		Notant $m \in \NNs$ tel que $m^2 = w^2 - 144$\,, nous arrivons à $w^2 - m^2 = 144$\,, d'où $(w, m) \in \setgene{(13, 5) , (15, 9) , (20, 16) , (37, 35)}$
		\footnote{
			Le programme reproduit après la preuve du fait \ref{diff-square-ko} donne rapidement cet ensemble de couples.
		}.
		Comme $u \in 2 \NN$ donne $w \in 2 \NN$\,, nécessairement $(w, m) = (20, 16)$\,, mais les équivalences suivantes lèvent une contradiction.

%		\medskip
		\noindent\!\!%
   		\begin{stepcalc}[style = sar, ope = \iff]
			u + 4 = 20
		\explnext{}
			u = 16
		\explnext{}
			(2 n + 5)^2 - 17 = 16
		\explnext*{$33 \notin \NNsquare$}{}
			(2 n + 5)^2 = 33
		\end{stepcalc}	
		
		
		\medskip
		\item Supposons enfin que $c \in \NNssquare$\,.
		
		\smallskip
		\noindent
		Dans ce cas, $ab \in \NNssquare$\,, soit $(u - 8) (u + 8) \in \NNssquare$\,, c'est-à-dire $u^2 - 64 \in \NNssquare$\,.
		Notant $m \in \NNs$ tel que $m^2 = u^2 - 64$\,, nous arrivons à $u^2 - m^2 = 64$\,.
		Ceci n'est possible que si $(u, m) \in \setgene{(10, 6), (17, 15)}$\,.
		Or $u \in \NN_{\geq 32}$ donne une contradiction.
	\end{itemize}

	\medskip

	Donc 
	$a = \alpha A^2$\,, $b = \beta B^2$ et $c = \gamma C^2$ 
	avec $(A, B, C) \in ( \NNs )^3$
	et
	$\setgene{\alpha, \beta, \gamma} \subset \NNsf \cap \NN_{>1}$\,,
	ceci nous donnant les faits suivants.
    
    \begin{itemize}
		\item $\alpha \wedge \beta \in \setgene{1, 2}$
		d'après $a \wedge b \divides 16$\,.

		\item $\alpha \wedge \gamma \in \setgene{1, 2, 3}$
		d'après $a \wedge c \divides 24$\,.

		\item $\beta \wedge \gamma \in \setgene{1, 2}$
		d'après $b \wedge c \divides 8$\,.

		\item $\setgene{\alpha, \beta, \gamma} \subseteq \setgene{2, 3, 6}$
		car $\forall p \in \PP_{>3}$\,, $(\padicval{a}, \padicval{b}, \padicval{c}) \in ( 2 \NN )^3$\,. 
    \end{itemize}
		
	\medskip
%	\newpage

	En fait, $\alpha$\,, $\beta$ et $\gamma$ sont différents deux à deux.
	
    \begin{itemize}	
		\item Démontrons que $\alpha \neq \beta$\,. 
		
		\noindent
		Dans le cas contraire, $16 = b - a = \alpha(B^2 - A^2)$ et $\alpha > 1$ donnent $B^2 - A^2 \in \setgene{1, 2, 4, 8}$\,, puis forcément $B^2 - A^2 = 8$ avec $(B, A) = (3, 1)$ d'après le fait \ref{diff-square-ko}.
		Comme de plus, $\alpha = 2$\,, nous obtenons $a = 2$ qui contredit $a \in \NN_{\geq 24}$\,.		

		\item Nous avons aussi $\beta \neq \gamma$\,. 
		
		\noindent
		Dans le cas contraire, $8 = c - b = \beta(C^2 - B^2)$ et $\beta > 1$ donnent $C^2 - B^2 \in \setgene{1, 2, 4}$\,, mais c'est impossible d'après le fait \ref{diff-square-ko}.
		

		\item Enfin, $\alpha \neq \gamma$\,. 
		
		\noindent
		Dans le cas contraire,
		$C^2 - A^2 \in \setgene{1, 2, 3, 4, 6, 8, 12}$
		car 
		$24 = c - a = \alpha(C^2 - A^2)$ et $\alpha > 1$\,.
		Le fait \ref{diff-square-ko} ne laisse plus que les possibilités suivantes.
		%
		\begin{enumerate}
			\item $C^2 - A^2 = 3$ n'est possible que si $(C, A) = (2, 1)$\,.
			Comme de plus $\alpha = 8$\,, nous avons $a = 8$ qui contredit $a \in \NN_{\geq 24}$\,.
			
			
			\item $C^2 - A^2 = 8$ n'est possible que si $(C, A) = (3, 1)$.
			Comme de plus $\alpha = 3$\,, nous avons $a = 3$ qui contredit $a \in \NN_{\geq 24}$\,.
			
			
			\item $C^2 - A^2 = 12$ n'est possible que si $(C, A) = (4, 2)$\,.
			Comme de plus $\alpha = 2$\,, nous avons $a = 8$ qui contredit $a \in \NN_{\geq 24}$\,.
		\end{enumerate}
	\end{itemize}


	\medskip
	
	Comme
	$\setgene{\alpha, \beta, \gamma} \subseteq \setgene{2, 3, 6}$\,,
	$\alpha \wedge \beta \in \setgene{1, 2}$\,,
	$\alpha \wedge \gamma \in \setgene{1, 2, 3}$ et
	$\beta \wedge \gamma \in \setgene{1, 2}$\,,
	et comme de plus $\alpha$\,, $\beta$ et $\gamma$ sont différents deux à deux, il ne nous reste plus qu'à analyser les cas suivants.
	La lumière est proche...

	\begin{center}
		\begin{tblr}{
			colspec    = {*{7}{Q[c,$]}},
			vline{2-7} = {},
			hline{2-7} = {}
		}
        	  \alpha & \beta & \gamma 
			& \alpha \wedge \beta & \alpha \wedge \gamma & \beta \wedge \gamma
			& Statut
			\\
        	  2 & 3 & 6
			& 1 & 2 & 3 
			& \myboxtimes
			\\
        	  2 & 6 & 3
			& 2 & 1 & 3 
			& \myboxtimes
			%
			\\
        	  3 & 2 & 6
			& 1 & 3 & 2 
			& \mycheckmark
			\\
        	  3 & 6 & 2
			& 3 & 1 & 2
			& \myboxtimes
			%
			\\
        	  6 & 2 & 3
			& 2 & 3 & 1
			& \mycheckmark
			\\
        	  6 & 3 & 2
			& 3 & 2 & 1
			& \myboxtimes
        \end{tblr}
	\end{center}


	Traitons les deux cas restants en nous souvenant que $a = u - 8$\,, $b = u + 8$ et $c = u + 16$\,.
	
	\begin{itemize}
		\item Supposons $(\alpha, \beta, \gamma) = (3, 2, 6)$\,, 
		autrement dit 
		$a = 3 A^2$\,, $b = 2 B^2$ et $c = 6 C^2$\,.
		
		\smallskip
		\noindent
		Travaillons modulo $3$ afin de lever une contradiction.
		\begin{enumerate}			
			\item $a \equiv u - 2$ et $a \equiv 3 A^2 \equiv 0$ donnent $u \equiv 2$\,.
			
			\item D'autre part, $b \equiv 2 B^2 \equiv \text{$0$ ou $2$}$\,.
			Or $b \equiv u + 2 \equiv 1$ lève une contradiction.
		\end{enumerate}
		
		
		\item Supposons $(\alpha, \beta, \gamma) = (6, 2, 3)$\,, 
		autrement dit 
		$a = 6 A^2$\,, $b = 2 B^2$ et $c = 3 C^2$\,.
		
		\smallskip
		\noindent
		La preuve précédente s'adapte directement car $a \equiv 6 A^2 \equiv 0$ et $b \equiv 2 B^2$ modulo $3$\,.
	\end{itemize}
\end{proof}

