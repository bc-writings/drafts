\begin{fact} \label{case-6}
	 $\forall n \in \NNs$\,, $\consprod<5> \notin \NNsquare$\,.
\end{fact}


% ------------------ %

Bien que très longue, la preuve suivante est simple à comprendre car elle ne fait que dérouler le fil des faits découverts.

\begin{proof}
    Supposons que in $\consprod<5> \in \NNsquare$\,. Dans ce qui suit, nous utiliserons $(a \pm b)$ comme raccourci de $(a + b) (a - b)$\,.

    \medskip
    \begin{stepcalc}[style = ar*, ope = \iff]
    	\consprod<5> = 
			n (n+1) (n+2) (n+3) (n+4) (n+5)
    \explnext*{$x = n + 3 + \dfrac12$}{}
    	\consprod<5> = 
			\big( x \pm \dfrac52 \big) \big( x \pm \dfrac32 \big) \big( x \pm \dfrac12 \big) 
    \explnext*{$y = 2 x$}{}
    	2^6 \consprod<5> = 
			(y \pm 5) (y \pm 3) (y \pm 1)
    \explnext*{$z = y^2$}{}
    	2^6 \consprod<5> = 
			(z - 25) (z - 9) (z - 1) 
    \explnext*{$u = z - 17$ où $17 = \dfrac{25 + 9}{2}$\,.}{}
    	2^6 \consprod<5> = 
			(u - 8) (u + 8) (u + 16)
    \end{stepcalc}

    \medskip
    Notant $a = u - 8$\,, $b = u + 8$ et $c = u + 16$\,, où $u = (2 n + 7)^2 - 17 \in 2 \NN$\,, nous avons les faits suivants.
    
    \begin{itemize}
		\item $(a, b, c) \in \big( \NNs \big)^3$ et $abc = 2^6 \consprod<5>$ où $\consprod<5> \in \NNsquare$\,.

		\item $a \wedge b \divides 16$ via $b - a = 16$\,.

		\item $a \wedge c \divides 24$ via $c - a = 24$\,.

		\item $b \wedge c \divides 8$  via $c - b = 8$\,.

		\item En particulier, 
		$\forall p \in \PP_{>3}$\,, 
		$(\padicval{a}, \padicval{b}, \padicval{c}) \in \big( 2 \NN \big)^3$.
	\end{itemize}


	Démontrons qu'aucun des trois entiers $a$\,, $b$ et $c$ ne peut être un carré parfait.
	%
	\begin{itemize}
		\medskip
		\item Commençons par supposer que $(a, b, c) \in \NNssquare \times \NNs \times \NNs$. 
		
		\smallskip
		\noindent
		Dans ce cas, $bc \in \NNssquare$\,, soit $(u + 8) (u + 16) \in \NNssquare$\,.
		En posant $w = u + 12$\,, on arrive à $(w - 4) (w + 4) \in \NNssquare$\,, soit $w^2 - 16 \in \NNssquare$\,.
		Notant $m \in \NNs$ tel que $m^2 = w^2 - 16$\,, nous arrivons à $w^2 - m^2 = 16$\,.
		D'après le fait \ref{dist-square}, $w^2 - m^2 = \dsum_{k=m+1}^{w} (2 k - 1)$\,.
		Ceci n'est possible que si $(w, m) = (5, 3)$
		\footnote{
			Noter que l'on doit avoir $2 w - 1 \leq 16$\,, d'où $w \in \ZintervalC{0}{8}$\,.
		}.
		Or $u \in 2 \NN$ donne $w \in 2 \NN$\,, d'où une contradiction.		
		
		
		\medskip
		\item Supposons maintenant que $(a, b, c) \in \NNs \times \NNssquare \times \NNs$. 
		
		\smallskip
		\noindent
		Dans ce cas, $ac \in \NNssquare$\,, soit $(u - 8) (u + 16) \in \NNssquare$\,.
		En posant $w = u + 4$\,, on arrive à $(w - 12) (w + 12) \in \NNssquare$\,, soit $w^2 - 144 \in \NNssquare$\,.
		Notant $m \in \NNs$ tel que $m^2 = w^2 - 144$\,, nous arrivons à $w^2 - m^2 = 144$\,, d'où $w^2 - m^2 = \dsum_{k=m+1}^{w} (2 k - 1)$\,.
		Ceci n'est possible que si $(w, m) \in \setgene{(13, 5) , (15, 9) , (20, 16) , (37, 35)}$
		\footnote{
			Comme $2 w - 1 \leq 144$ donne $w \in \ZintervalC{0}{72}$\,, il suffit de faire appel à un petit programme pour obtenir brutalement toutes les valeurs possibles.
		}.
		Ici aussi, $u \in 2 \NN$ donne $w \in 2 \NN$\,, donc $(w, m) = (20, 16)$\,, mais les équivalences suivantes lèvent une contradiction.

		\medskip
		\noindent\!\!%
   		\begin{stepcalc}[style = sar, ope = \iff]
			u + 4 = 20
		\explnext{}
			(2 n + 7)^2 - 17 = 24
		\explnext*{$41 \notin \NNsquare$}{}
			(2 n + 7)^2 = 41
		\end{stepcalc}	
		
		
		\medskip
		\item Supposons enfin que $(a, b, c) \in \NNs \times \NNs \times \NNssquare$\,. 
		
		\smallskip
		\noindent
		Dans ce cas, $ab \in \NNssquare$\,, soit $(u - 8) (u + 8) \in \NNssquare$\,, c'est-à-dire $u^2 - 64 \in \NNssquare$\,.
		Notant $m \in \NNs$ tel que $m^2 = u^2 - 64$\,, nous arrivons à $u^2 - m^2 = 64$\,.
		Ceci n'est possible que si $(u, m) \in \setgene{(10, 6), (17, 15)}$\,.
		Comme $u \in 2 \NN$\,, forcément $(u, m) = (10, 6)$\,, mais les équivalences suivantes lèvent une contradiction.

		\medskip
		\noindent\!\!%
   		\begin{stepcalc}[style = sar, ope = \iff]
			u = 10
		\explnext{}
			(2 n + 7)^2 - 17 = 10
		\explnext*{$27 \notin \NNsquare$}{}
			(2 n + 7)^2 = 27
		\end{stepcalc}
	\end{itemize}


	Nous avons donc $a = \alpha A^2$\,, $b = \beta B^2$ et $c = \gamma C^2$ avec $(A, B, C) \in \big( \NNs \big)^3$, et $(\alpha, \beta, \gamma) \in \big( \NN_{>1} \big)^3$ un triplet d'entiers sans facteurs carrés.
	Nous avons les faits suivants.
    
    \begin{itemize}
		\item $\alpha \wedge \beta \in \setgene{1, 2}$
		d'après $a \wedge b \divides 16$\,.

		\item $\alpha \wedge \gamma \in \setgene{1, 2, 3}$
		d'après $a \wedge c \divides 24$\,.

		\item $\beta \wedge \gamma \in \setgene{1, 2}$
		d'après $b \wedge c \divides 8$\,.

		\item $\setgene{\alpha, \beta, \gamma} \subseteq \setgene{2, 3, 6}$
		car $\forall p \in \PP_{>3}$\,, $(\padicval{a}, \padicval{b}, \padicval{c}) \in \big( 2 \NN \big)^3$\,. 
		

		\item Démontrons que $\alpha \neq \beta$\,. 
		
		\noindent
		Dans le cas contraire, $16 = b - a = \alpha(B^2 - A^2)$ et $\alpha > 1$ donnent $B^2 - A^2 \in \setgene{1, 2, 4, 8}$\,.
		Or nous avons les impossibilités suivantes.
		%
		\begin{enumerate}
			\item $B^2 - A^2 = 1$  et $B^2 - A^2 = 2$ contredisent le fait \ref{dist-square}.
			\item $B^2 - A^2 = 4$ n'est possible que si $(B, A) = (2, 0)$\,.
			\item $B^2 - A^2 = 8$ n'est possible que si $(B, A) = (3, 1)$ et $\alpha = 2$\,.
			Ceci donne $a = 2$\,, puis $u = 10$\,, mais nous avons vu que ceci était impossible.
		\end{enumerate}
		

		\item Nous avons aussi $\beta \neq \gamma$\,. 
		
		\noindent
		Dans le cas contraire, $8 = c - b = \beta(C^2 - B^2)$ et $\beta > 1$ donnent $C^2 - B^2 \in \setgene{1, 2, 4}$\,, mais ce qui précède ne laisse aucun choix possible.
		

		\item Enfin, $\alpha \neq \gamma$\,. 
		
		\noindent
		Dans le cas contraire,
		$C^2 - A^2 \in \setgene{1, 2, 3, 4, 6, 8, 12}$
		car 
		$24 = c - a = \alpha(C^2 - A^2)$ et $\alpha > 1$\,.
		Nous obtenons alors les impossibilités suivantes.
		%
		\begin{enumerate}
			\item $C^2 - A^2 \in \setgene{1, 2, 4}$ est à rejeter comme précédemment.
			
			\item $C^2 - A^2 = 3$ n'est possible que si $(C, A) = (2, 1)$ et $\alpha = 8$\,, mais les équivalences suivantes lèvent une contradiction.

    		\smallskip
    		\noindent\!\!%
       		\begin{stepcalc}[style = sar, ope = \iff]
    			a = 8
    		\explnext{}
    			u = 16
    		\explnext{}
    			(2 n + 7)^2 - 17 = 16
    		\explnext*{$33 \notin \NNsquare$}{}
    			(2 n + 7)^2 = 33
    		\end{stepcalc}	
			\smallskip
	
			\item $C^2 - A^2 = 6$ est impossible.
			
			\item $C^2 - A^2 = 8$ n'est possible que si $(C, A) = (3, 1)$ et $\alpha = 3$\,, mais les équivalences suivantes lèvent une contradiction.

    		\smallskip
    		\noindent\!\!%
       		\begin{stepcalc}[style = sar, ope = \iff]
    			a = 3
    		\explnext{}
    			u = 11
    		\explnext{}
    			(2 n + 7)^2 - 17 = 11
    		\explnext*{$28 \notin \NNsquare$}{}
    			(2 n + 7)^2 = 28
    		\end{stepcalc}	
			\smallskip
			
			
			\item $C^2 - A^2 = 12$ n'est possible que si $(C, A) = (4, 2)$ et $\alpha = 2$\,, mais ceci donnerait $a = 8$\,, or nous savons que cela est impossible.
		\end{enumerate}
	\end{itemize}


	Comme
	$\setgene{\alpha, \beta, \gamma} \subseteq \setgene{2, 3, 6}$\,,
	$\alpha \wedge \beta \in \setgene{1, 2}$\,,
	$\alpha \wedge \gamma \in \setgene{1, 2, 3}$ et
	$\beta \wedge \gamma \in \setgene{1, 2}$\,,
	et comme de plus $\alpha$\,, $\beta$ et $\gamma$ sont différents deux à deux, il ne nous reste plus qu'à analyser les cas suivants.

	\begin{center}
		\begin{tblr}{
			colspec={*{7}{Q[c,$]}},
			vline{2-7} = {},
			hline{2-7} = {}
		}
        	\alpha & \beta & \gamma 
				& \alpha \wedge \beta & \alpha \wedge \gamma & \beta \wedge \gamma
				& Statut
			\\
        	2 & 3 & 6
				& 1 
				& 2 
				& 3
				& \myboxtimes
			\\
        	2 & 6 & 3
				& 2 
				& 1
				& 3
				& \myboxtimes
			%
			\\
        	3 & 2 & 6
				& 1 
				& 3
				& 2
				& \mycheckmark
			\\
        	3 & 6 & 2
				& 3
				& 1
				& 2
				& \myboxtimes
			%
			\\
        	6 & 2 & 3
				& 2
				& 3 
				& 1
				& \mycheckmark
			\\
        	6 & 3 & 2
				& 3 
				& 2
				& 1
				& \myboxtimes
        \end{tblr}
	\end{center}


	Traitons les deux cas restants en nous souvenant que $a = u - 8$\,, $b = u + 8$ et $c = u + 16$\,, où $u = (2 n + 7)^2 - 17 \in 2 \NN$\,.
	
	\begin{itemize}
		\item Supposons $(\alpha, \beta, \gamma) = (3, 2, 6)$\,, 
		autrement dit 
		$a = 3 A^2$\,, $b = 2 B^2$ et $c = 6 C^2$\,.
		
		\smallskip
		\noindent
		Travaillons modulo $3$ afin de lever une contradiction.
		\begin{enumerate}			
			\item $a \equiv u - 2$ et $a \equiv 3 A^2 \equiv 0$ donnent $u \equiv 2$\,.
			
			\item D'autre part, $b \equiv 2 B^2 \equiv \text{$0$ ou $2$}$ via les carrés modulo $3$\,.
			Or $b \equiv u + 2 \equiv 1$ lève une contradiction.
		\end{enumerate}
		
		
		\item Supposons $(\alpha, \beta, \gamma) = (6, 2, 3)$\,, 
		autrement dit 
		$a = 6 A^2$\,, $b = 2 B^2$ et $c = 3 C^2$\,.
		
		\smallskip
		\noindent
		La preuve précédente s'adapte sans difficulté puisque que $a \equiv 6 A^2 \equiv 0$ et $b \equiv 2 B^2$\,.
	\end{itemize}
\end{proof}

