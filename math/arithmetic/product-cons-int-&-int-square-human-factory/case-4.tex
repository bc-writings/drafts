\begin{fact} \label{case-4}
	 $\forall n \in \NNs$\,, $n(n+1)(n+2)(n+3) \notin \NNsquare$\,.
\end{fact}


% ------------------ %


\begin{proof}[Preuve 1]
    Nous pouvons ici faire les manipulations algébriques naturelles suivantes qui cherchent à obtenir le même coefficient pour $n$ dans chaque parenthèse.
    
    \medskip
    
    \begin{stepcalc}[style = sar]
    	\consprod<4>
    \explnext{}
    	n(n+3) \cdot (n+1)(n+2)
    \explnext{}
    	(n^2 + 3n) \cdot (n^2 + 3n + 2)
    \explnext*{$m = n^2 + 3n$}{}
    	m (m + 2)
    \explnext{}
    	m^2 + 2m
    \explnext{}
    	(m + 1)^2 - 1
    \end{stepcalc}
    
    \medskip
    
    Comme $m > 0$\,, $(m + 1)^2 - 1 \notin \NNsquare$ d'après le fait \ref{diff-square-ko}, donc $\consprod<4> \notin \NNsquare$\,. 
\end{proof}


% ------------------ %


\begin{proof}[Preuve 2]
	En \enquote{symétrisant} certaines expressions, nous obtenons d'autres manipulations algébriques qui permettent de conclure comme ci-dessus.
    
    \medskip
    
    \begin{stepcalc}[style = sar]
    	\consprod<4>
    \explnext{}
    	n(n+1)(n+2)(n+3)
    \explnext*{$x = n + \dfrac32$}{}
    	\big( x \pm \dfrac32 \big) \big( x \pm \dfrac12 \big)
    \explnext{}
    	\big( x^2 - \dfrac94 \big) \big( x^2 - \dfrac14 \big)
    \explnext*{$y = x^2  - \dfrac54$ où $\dfrac54 = \dfrac12 \big( \dfrac94+ \dfrac14 \big)$\,.}{}
    	(y \pm 1)
    \explnext{}
    	y^2 - 1
    \explnext{}
    	\Big( \big( n + \dfrac32 \big)^2 - \dfrac54 \Big)^2 - 1
    \explnext{}
    	\big( n^2 + 3n + 1 \big)^2 - 1
    \end{stepcalc}

    \vspace{-1.5ex}
    \qedhere
\end{proof}


% ------------------ %


Un échange sur \url{https://math.stackexchange.com} a inspiré la démonstration non algébrique suivante (voir la section \ref{sources}).	
	
	
\begin{proof}[Preuve 3]
    Supposons que $\consprod<4> \in \NNssquare$\,.
    
    \smallskip
    
	Clairement, nous avons les faits suivants.
	%
    \begin{itemize}
    	\item  
		$\forall p \in \PP_{>3}$\,, 
   		$\forall i \in \ZintervalC{0}{3}$\,, 
    	$\padicval{n + i} \in 2 \NN$\,.
	
	
		\item $\exists u \in \setgene{n, n+1}$ tel que $\setgene{u, u + 2} \subset 2 \NN + 1$\,.
		
		\noindent
		Nous avons alors
		$\forall p \in \PP - \setgene{3}$\,, 
   		$(\padicval{u}, \padicval{u+2}) \in ( 2 \NN )^2$,
		donc, pour tout naturel $m \in \setgene{u, u + 2}$\,, 
		il existe $M \in \NNs$ tel que 
		$m = M^2$ ou $m = 3 M^2$\,.
	
	
		\item Forcément, il existe $(A, B) \in ( \NNs )^2$ tel que 
        $\setgene{u, u + 2} = \setgene{A^2, 3 B^2}$\,. Voici pourquoi.
    	%
        \begin{itemize}
        	\item $\setgene{u, u + 2} = \setgene{A^2, B^2}$ donne deux carrés distants de $2$\,, ceci contredit le fait \ref{diff-square-ko}.

        	\item $\setgene{u, u + 2} = \setgene{3 A^2, 3 B^2}$ donne $3 A^2 - 3 B^2 = \pm 2$\,, ce qui est impossible.
        \end{itemize}
    \end{itemize}

    \smallskip
	
	Nous savons donc que l'un des facteurs $(n+i)$ de $\consprod<4>$ possède une valuation $3$-adique impaire. Ceci n'est possible que si $n$ et $(n+3)$ ont une valuation $3$-adique impaire.
	Dès lors, comme ci-dessus, nous avons $(Q, R) \in ( \NNs )^2$ tel que $\setgene{n+1, n+2} = \setgene{Q^2, 2 R^2}$\,.
	Ceci nous amène aux deux situations contradictoires suivantes où $(A, B, C, D) \in ( \NNs )^4$.
	
%    \newpage
    \begin{itemize}
    	\item Cas 1 : $(n, n+1, n+2, n+3) = (6A^2, B^2, 2C^2, 3D^2)$\,.
    	%
		\begin{itemize}
        	\item Posons $x = n + \frac32$\, de sorte que 
        	$x - \frac32 = 6 A^2$\,, $x - \frac12 = B^2$\,, $x + \frac12 = 2 C^2$ et $x + \frac32 = 3 D^2$\,.
        
        	\item Nous avons alors
        	$\big( x - \frac32 \big) \big( x + \frac32 \big) = 2 E^2$\,, c'est-à-dire $x^2 - \frac94 = 2 E^2$\,, avec $E \in \NNs$.
        
        	\item De même,
        	$x^2 - \frac14 = 2 F^2$ avec $F \in \NNs$.	
        
        	\item Par simple soustraction, nous obtenons $2 F^2 - 2 E^2 = 2$\,, puis $F^2 - E^2 = 1$\,, mais ceci contredit le fait \ref{diff-square-ko}.
        \end{itemize}
    	
	
		\item Cas 2 : $(n, n+1, n+2, n+3) = (3A^2, 2B^2, C^2, 6D^2)$\,.
		
		\smallskip
		\noindent
		Un raisonnement similaire au précédent montre que ce cas aussi est impossible. \qedhere
    \end{itemize}
\end{proof}

