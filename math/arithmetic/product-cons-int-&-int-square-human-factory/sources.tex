% ------------------ %


\bigskip
\textbf{Fait \ref{case-4}.}
	
%\smallskip
%\noindent
%Voir la source du fait  \ref{case-7}.

\smallskip
\noindent
\emph{La démonstration non algébrique a été impulsée par la source du fait \ref{case-7} donnée plus bas.}


% ------------------ %


\bigskip
\textbf{Fait \ref{case-5}.}
	
\begin{itemize}
	\item Un échange consulté le 28 janvier 2024, et titré 
	\emph{\enquote{\href{https://les-mathematiques.net/vanilla/discussion/comment/351293}{n(n+1)...(n+k) est un carré ?}}} 
	sur le site \url{lesmathematiques.net}\,.

    \smallskip
    \noindent
    \emph{La démonstration via le principe des tiroirs trouve sa source dans cet échange.}


	\item Un échange consulté le 12 février 2024, et titré 
	\emph{\enquote{\href{https://artisticmathematics.quora.com/Is-there-an-easier-way-of-proving-the-product-of-any-5-consecutive-positive-integers-is-never-a-perfect-square}{Is there an easier way of proving the product of any 5 consecutive positive integers is never a perfect square?}}} 
	sur le site \url{www.quora.com/}\,.

    \smallskip
    \noindent
    \emph{La démonstration \enquote{élémentaire} sans le principe des tiroirs vient de cet échange.}


	\item L'article \emph{\enquote{Le produit de 5 entiers consécutifs n'est pas le carré d'un entier.}} de T. Hayashi, Nouvelles Annales de Mathématiques, est consultable via \href{https://numdam.org}{Numdam}\,, la bibliothèque numérique française de mathématiques.
	
	\smallskip
	\noindent
	\emph{Cet article a fortement inspiré la longue preuve.}
\end{itemize}
\vspace{-1ex}


% ------------------ %


\bigskip
\textbf{Fait \ref{case-6}.}
	
\begin{itemize}
	\item Un échange consulté le 28 janvier 2024, et titré
\emph{\enquote{\href{https://math.stackexchange.com/q/90894/52365}{product of six consecutive integers being a perfect numbers}}} 
sur le site \url{https://math.stackexchange.com}\,.
	
	\smallskip
	\noindent
	\emph{La courte démonstration est donnée dans cet échange. Vous y trouverez aussi un très joli argument basé sur les courbes elliptiques rationnelles.}


	\item Une discussion archivée consultée le 28 janvier 2024 : 
	
	\noindent
	\url{https://web.archive.org/web/20171110144534/http://mathforum.org/library/drmath/view/65589.html}\,.
	
	\smallskip
	\noindent
	\emph{Cette discussion a impulsé la preuve fastidieuse, mais facile d'accès, via des tableaux.}
\end{itemize}
\vspace{-1ex}


% ------------------ %


\bigskip
\textbf{Fait \ref{case-7}.}
	
\smallskip
\noindent
Un échange consulté le 3 février 2024, et titré
\emph{\enquote{\href{https://math.stackexchange.com/q/2334887/52365}{Proof that the product of 7 successive positive integers is not a square}}} 
sur le site \url{https://math.stackexchange.com}\,.
	
\smallskip
\noindent
\emph{La courte démonstration est donnée dans cet échange, mais certaines justifications manquent.}


% ------------------ %


\bigskip
\textbf{Fait \ref{case-8}.}

\smallskip
\noindent
Un échange consulté le 4 février 2024, et titré \emph{\enquote{\href{https://math.stackexchange.com/a/2271715/52365}{How to prove that the product of eight consecutive numbers can't be a number raised to exponent 4?}}} sur le site \url{https://math.stackexchange.com}\,.

\smallskip
\noindent
\emph{La démonstration astucieuse vient de l'une des réponses de cet échange, mais la justification des deux inégalités n'est pas donnée.}


% ------------------ %


\bigskip
\textbf{Fait \ref{case-10}.}
	
\smallskip
\noindent
Un échange consulté le 13 février 2024, et titré
\emph{\enquote{\href{https://math.stackexchange.com/q/2361670/52365}{Product of 10 consecutive integers can never be a perfect square}}} 
sur le site \url{https://math.stackexchange.com}\,.

\smallskip
\noindent
\emph{La démonstration vient d'une source Wordpress donnée dans une réponse de cet échange, mais cette source est très expéditive...}

