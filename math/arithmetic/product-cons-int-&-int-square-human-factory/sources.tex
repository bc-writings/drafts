%Ce document n'aurait pas vu le jour sans les sources suivantes.


\begin{enumerate}
	\item Un échange consulté le 28 janvier 2024, et titré 
	\emph{\enquote{\href{https://les-mathematiques.net/vanilla/discussion/comment/351293}{n(n+1)...(n+k) est un carré ?}}} 
	sur le site \url{lesmathematiques.net}\,.
	
	\smallskip
	\noindent
	\emph{La démonstration du fait \ref{case-5}  via le principe des tiroirs trouve sa source dans cet échange.}


	\item L'article \emph{\enquote{Le produit de 5 entiers consécutifs n'est pas le carré d'un entier.}} de T. Hayashi, Nouvelles Annales de Mathématiques, est consultable via \href{https://numdam.org}{Numdam}\,, la bibliothèque numérique française de mathématiques.
	
	\smallskip
	\noindent
	\emph{Cet article a inspiré la preuve alternative du fait \ref{case-5}.}


% ------------------ %


	\item Un échange consulté le 28 janvier 2024, et titré
	\emph{\enquote{\href{https://math.stackexchange.com/q/90894/52365}{product of six consecutive integers being a perfect numbers}}} 
	sur le site \url{https://math.stackexchange.com}\,.
	
	\smallskip
	\noindent
	\emph{La démonstration courte du fait \ref{case-6} est donné dans cet échange. Vous y trouverez aussi un très joli argument basé sur les courbes elliptiques rationnelles.}


% ------------------ %


	\item Un échange consulté le 3 février 2024, et titré
	\emph{\enquote{\href{https://math.stackexchange.com/q/2334887/52365}{Proof that the product of 7 successive positive integers is not a square}}} 
	sur le site \url{https://math.stackexchange.com}\,.
	
	\smallskip
	\noindent
	\emph{La démonstration courte du fait \ref{case-7} est donné dans cet échange, mais certaines justifications manquent.}
\end{enumerate}
