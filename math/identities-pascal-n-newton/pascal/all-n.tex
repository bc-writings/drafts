Rien ne nous assure a priori de découvrir un moyen simple de développer $(a + b)^n$ mais soyons confiant et tentons l'aventure en nous appuyant sur notre notation simplifiée. Nous verrons alors si de nouveau les mathématiques nous révèleront une belle structure cachée.


\medskip

Considérons le cas de $(a + b)^3$ dont le moments importants du développement sont les suivants.


\medskip

\begin{stepcalc}[style = sar]
	(a + b)^3
	%
		\explnext{}
	%
	\dots
	%
		\explnext{}
	%
	\redit{a (a^2 + 2 a b + b^2)}
		\explnext[\hideit+]{}
	\phantom{x}\kern4pt%
	\blueit{{} + b (a^2 + 2 a b + b^2)}
	%
		\explnext{}
	%
	\redit{a^3 + 2 a^2 b + \phantom{2} a b^2}
		\explnext[\hideit+]{}
	\phantom{x}\kern4pt%
	\blueit{{} + \phantom{2} a^2 b + 2 a b^2 + b^3}
	%
		\explnext{}
	%
	\dots\end{stepcalc}


\medskip

Avec notre notation, $(a + b)^2 = a^2 + 2 a b + b^2$ s'écrit :


\medskip
\begin{center}
\begin{tabular}{|l*{3}{|c}|}
	\hline
	$2 \rightarrow$
		& $1$    &  $2$    &  $1$
	\\ \hline
		& $a^2$  &  $a b$  &  $b^2$
	\\ \hline
\end{tabular}

\smallskip
\itshape\small
Tableau pour $n = 2$
\end{center}


\medskip

Le tableau pour $n = 2$ donne par multiplication de chaque terme par $a$ :


\medskip
\begin{center}
\begin{tabular}{*{3}{|c}|}
	\hline
		$1$    &  $2$      &  $1$
	\\ \hline
		$a^3$  &  $a^2 b$  &  $a b^2$
	\\ \hline
\end{tabular}

\smallskip
\itshape\small
Distribution de $a$ sur $(a + b)^2$
\end{center}


\medskip

La multiplication de chaque terme par $b$ donne de façon analogue :

\medskip
\begin{center}
\begin{tabular}{*{3}{|c}|}
	\hline
		$1$      &  $2$      &  $1$
	\\ \hline
		$a^2 b$  &  $a b^2$  &  $b^3$
	\\ \hline
\end{tabular}

\smallskip
\itshape\small
Distribution de $b$ sur $(a + b)^2$
\end{center}


\medskip

Mis dans le tableau pour $n = 3$ , nous avons ce qui suit où les deux dernières lignes redonnent $(a + b)^3 = a^3 + 3 a^2 b + 3 a b^2 + b^3$ .


\medskip
\begin{center}
\begin{tabular}{*{4}{|c}|}
	\hline
		$1$    &  $2$      &  $1$      &
	\\ \hline
		$a^3$  &  $a^2 b$  &  $a b^2$  &
	\\ \hline\hline
		       &  $1$      &  $2$      &  $1$
	\\ \hline
		       &  $a^2 b$  &  $a b^2$  &  $b^3$
	\\ \hline\hline\hline\hline
		$1$    &  $3$      &  $3$      &  $1$
	\\ \hline
		$a^3$  &  $a^2 b$  &  $a b^2$  &  $b^3$
	\\ \hline
\end{tabular}

\smallskip
\itshape\small
Tableau pour $n = 3$
\end{center}


\medskip

On retrouve ce qui suit, l'une des étapes clés du développement de $(a + b)^3$.


\medskip

\begin{stepcalc}[style = sar]
	(a + b)^3
	%
		\explnext{}
	%
	\dots
	%
		\explnext{}
	%
	\redit{a^3 + 2 a^2 b + \phantom{2} a b^2}
		\explnext[\hideit+]{}
	\phantom{x}\kern4pt%
	\blueit{{} + \phantom{2} a^2 b + 2 a b^2 + b^3}
	%
		\explnext{}
	%
	\dots
\end{stepcalc}


\bigskip

Il est temps de faire un nouveau pas vers une belle abstraction en notant trois choses.

\begin{enumerate}
	\item La multiplication par $a$ revient à garder la ligne résumé du développement de $(a + b)^2$ .

	\item La multiplication par $b$ revient à décaler d'une case vers la droite la ligne résumé du développement de $(a + b)^2$ .

	\item La ligne résumé du développement de $(a + b)^3$ est la somme des deux lignes précédentes, une case vide valant zéro.
\end{enumerate}


\medskip

En appliquant les règles précédentes, dont il est clair qu'elles sont générales, nous retrouvons le développement $(a + b)^4 = a^4 + 4 a^3 b + 6 a^2 b^2 + 4 a b^3 + b^4$ à partir du tableau précédent.


\medskip
\begin{center}
\begin{tabular}{|l*{5}{|c}|}
	\hline
	$\redit{\times a} \rightarrow$	&
		$1$    &  $3$      &  $3$        &  $1$      &
	\\ \hline\hline
	$\blueit{\times b} \rightarrow$	&
		       &  $1$      &  $3$        &  $3$      &  $1$
	\\ \hline\hline\hline\hline
		&
		$1$    &  $4$      &  $6$        &  $4$      &  $1$
	\\ \hline
	    &
		$a^4$  &  $a^3 b$  &  $a^2 b^2$  &  $a b^2$  &  $b^4$
	\\ \hline
\end{tabular}

\smallskip
\itshape\small
Développement de $(a + b)^4$
\end{center}


\medskip

On reconnaît l'une des étapes importantes du développement de $(a + b)^4$.


\medskip

\begin{stepcalc}[style = sar]
	(a + b)^4
	%
		\explnext{}
	%
	\dots
	%
		\explnext{}
	%
	\redit{a^4 + 3 a^3 b + 3 a^2 b^2 + \phantom{3} a b^3}
		\explnext[\hideit+]{}
	\phantom{x}\kern4pt%
	\blueit{{} + \phantom{3} a^3 b + 3 a^2 b^2 + 3 a b^3 + b^4}
	%
		\explnext{}
	%
	\dots
\end{stepcalc}

\medskip

Poursuivons avec $(a + b)^5 = a^5 + 5 a^4 b + 10 a^3 b^2 + 10 a^2 b^3 + 5 a b^4 + b^5$ que nous avons admis précédemment.


\newpage
%\medskip
\begin{center}
\begin{tabular}{|l*{6}{|c}|}
	\hline
	$\redit{\times a} \rightarrow$	&
		$1$    &  $4$      &  $6$        &  $4$        &  $1$      &
	\\ \hline\hline
	$\blueit{\times b} \rightarrow$	&
		       &  $1$      &  $4$        &  $6$        &  $4$      &  $1$
	\\ \hline\hline\hline\hline
		&
		$1$    &  $5$      &  $10$       &  $10$       &  $5$      &  $1$
	\\ \hline
	    &
		$a^5$  &  $a^4 b$  &  $a^3 b^2$  &  $a^2 b^3$  &  $a b^4$  &  $b^5$
	\\ \hline
\end{tabular}

\smallskip
\itshape\small
Développement de $(a + b)^5$
\end{center}


\medskip

Nous voilà prêts à découvrir les développements de $(a + b)^6$ et $(a + b)^7$ sans trop nous fatiguer.


\medskip
\begin{center}
\begin{tabular}{|l*{7}{|c}|}
	\hline
	$\redit{\times a} \rightarrow$	&
		$1$    &  $5$      &  $10$       &  $10$       &  $5$        &  $1$      &
	\\ \hline\hline
	$\blueit{\times b} \rightarrow$	&
		       &  $1$      &  $5$        &  $10$       &  $10$       &  $5$      &  $1$
	\\ \hline\hline\hline\hline
		&
		$1$    &  $6$      &  $15$       &  $20$       &  $15$       &  $6$      &  $1$
	\\ \hline
	    &
		$a^6$  &  $a^5 b$  &  $a^4 b^2$  &  $a^3 b^3$  &  $a^2 b^4$  &  $a b^5$  &  $b^6$
	\\ \hline
\end{tabular}

\smallskip
\itshape\small
Développement de $(a + b)^6$
\end{center}


\medskip
\begin{center}
\begin{tabular}{|l*{8}{|c}|}
	\hline
	$\redit{\times a} \rightarrow$	&
		$1$    &  $6$      &  $15$       &  $20$       &  $15$       &  $6$      &  $1$      &
	\\ \hline\hline
	$\blueit{\times b} \rightarrow$	&
		       &  $1$    &  $6$      &  $15$       &  $20$       &  $15$       &  $6$      &  $1$
	\\ \hline\hline\hline\hline
		&
		$1$    &  $7$      &  $21$       &  $35$       &  $35$       &  $21$       &  $7$      &  $1$
	\\ \hline
	    &
		$a^7$  &  $a^6 b$  &  $a^5 b^2$  &  $a^4 b^3$  &  $a^3 b^4$    &  $a^2 b^5$  &  $a b^6$  &  $b^7$
	\\ \hline
\end{tabular}

\smallskip
\itshape\small
Développement de $(a + b)^7$
\end{center}


\medskip

Nous avons démontré les deux nouvelles identités remarquables suivantes.

\begin{itemize}[label = \small\textbullet]
	\item $(a + b)^6 = a^6 + 6 a^5 b + 15 a^4 b^2 + 20 a^3 b^3 + 15 a^2 b^4 + 6 a b^5 + b^6$

	\item $(a + b)^7 = a^7 + 7 a^6 b + 21 a^5 b^2 + 35 a^4 b^3 + 35 a^3 b^4 + 21 a^2 b^5 + 7 a b^6 + b^7$
\end{itemize}

\medskip

Auriez-vous eu le courage de trouver le dernier développement directement en calculant avec des $a$ et des $b$ ? L'auteur de ces lignes ne l'aura jamais !

