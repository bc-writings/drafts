Commençons par nous intéresser aux développements des expressions
$(a + b)^2$ ,
$(a + b)^3$ et
$(a + b)^4$ .


% -------------------- %


\subsubsection{Développement de $(a + b)^2$}

Rappelons que  $(a + b)^2 = a^2 + 2 a b + b^2$ .
Ceci se démontre facilement comme suit.

\medskip

\begin{stepcalc}[style = sar]
	(a + b)^2
		\explnext{}
	(\redit{a} + \blueit{b}) (a + b)
		\explnext{}
	\redit{a^2 + a b} + \blueit{b a + b^2}
		\explnext{}
	a^2 + 2 a b + b^2
\end{stepcalc}


% -------------------- %


\subsubsection{Développement de $(a + b)^3$}

Pour développer $(a + b)^3$ , nous allons bien entendu nous appuyer sur celui de $(a + b)^2$ . Allons-y.

\medskip

\begin{stepcalc}[style = sar]
	(a + b)^3
		\explnext{}
	(\redit{a} + \blueit{b}) (a + b)^2
		\explnext{}
	(\redit{a} + \blueit{b}) (a^2 + 2 a b + b^2)
		\explnext{}
	\redit{a^3 +a \cdot 2 a b + a \cdot b^2}
	+
	\blueit{b \cdot a^2 + b \cdot 2 a b + b^3}
		\explnext{}
	\redit{a^3 + 2 a^2 b + a b^2}
	+
	\blueit{b a^2 + 2 a b^2 + b^3}
		\explnext{}
	a^3 + 3 a^2 b + 3 a b^2 + b^3
\end{stepcalc}

\medskip

Sans trop de peine, nous avons obtenu :
$(a + b)^3 = a^3 + 3 a^2 b + 3 a b^2 + b^3$ . On peut simplifier la compréhension des calculs comme suit.

\medskip

\begin{stepcalc}[style = sar]
	(a + b)^3
		\explnext{}
	(\redit{a} + \blueit{b}) (a + b)^2
	%
		\explnext{}
	%
	\redit{a (a + b)^2}
		\explnext[\hideit+]{}
	\phantom{x}\kern4pt%
	\blueit{{} + b (a + b)^2}
	%
		\explnext{}
	%
	\redit{a (a^2 + 2 a b + b^2)}
		\explnext[\hideit+]{}
	\phantom{x}\kern4pt%
	\blueit{{} + b (a^2 + 2 a b + b^2)}
	%
		\explnext{}
	%
	\redit{a^3 + 2 a^2 b + \phantom{2} a b^2}
		\explnext[\hideit+]{}
	\phantom{x}\kern4pt%
	\blueit{{} + \phantom{2} a^2 b + 2 a b^2 + b^3}
	%
		\explnext{}
	%
	a^3 + 3 a^2 b + 3 a b^2 + b^3
\end{stepcalc}


% -------------------- %


\subsubsection{Développement de $(a + b)^4$}

Pour développer $(a + b)^4$ , nous allons nous inspirer de la présentation ci-dessus qui simplifie la compréhension. Une routine s'installe...

\medskip

\begin{stepcalc}[style = sar]
	(a + b)^4
		\explnext{}
	(\redit{a} + \blueit{b}) (a + b)^3
	%
		\explnext{}
	%
	\redit{a (a + b)^3}
		\explnext[\hideit+]{}
	\phantom{x}\kern4pt%
	\blueit{{} + b (a + b)^3}
	%
		\explnext{}
	%
	\redit{a (a^3 + 3 a^2 b + 3 a b^2 + b^3)}
		\explnext[\hideit+]{}
	\phantom{x}\kern4pt%
	\blueit{{} + b (a^3 + 3 a^2 b + 3 a b^2 + b^3)}
	%
		\explnext{}
	%
	\redit{a^4 + 3 a^3 b + 3 a^2 b^2 + \phantom{3} a b^3}
		\explnext[\hideit+]{}
	\phantom{x}\kern4pt%
	\blueit{{} + \phantom{3} a^3 b + 3 a^2 b^2 + 3 a b^3 + b^4}
	%
		\explnext{}
	%
	a^4 + 4 a^3 b + 6 a^2 b^2 + 4 a b^3 + b^4
\end{stepcalc}

\medskip

Sans trop de peine, nous avons obtenu :
$(a + b)^4 = a^4 + 4 a^3 b + 6 a^2 b^2 + 4 a b^3 + b^4$ .


% -------------------- %


\subsubsection{Aller plus loin}

Au moins en théorie, pour $n \in \NNs$ nous avons exhibé un procédé de développement de $(a + b)^n$ via celui de $(a + b)^{n - 1}$ en utilisant $(a + b)^n = (a + b) (a + b)^{n - 1}$ .
Ceci étant indiqué, même pour $n = 5$ , nous devinons qu'il va vite être pénible de rédiger à chaque fois les développements.
Il devient alors naturel de se demander si par hasard il n'y aurait pas un moyen efficace de développer par exemple $(a + b)^7$ . Nous allons voir que c'est bien le cas.


% -------------------- %


\subsubsection{Une notation efficace}

Retenir les trois identités remarquables suivantes n'est a priori pas simple.

\begin{itemize}[label = \small\textbullet]
	\item $(a + b)^2 = a^2 + 2 a b + b^2$

	\item $(a + b)^3 = a^3 + 3 a^2 b + 3 a b^2 + b^3$

	\item $(a + b)^4 = a^4 + 4 a^3 b + 6 a^2 b^2 + 4 a b^3 + b^4$
\end{itemize}

Un moyen efficace de le faire est de noter que les écritures standardisées utilisées sont toutes avec des puissances décroissantes de $a$ et croissantes de $b$ .
Ainsi dans $a^4 + 4 a^3 b + 6 a^2 b^2 + 4 a b^3 + b^4$ où l'exposant maximal est $4$ nous avons en se souvenant que $x^0 \eq[conv] 1$ par convention :

\medskip
\begin{center}
\begin{tabular}{c*{5}{|c}}
	$a^r$
		& $a^4$      &  $a^3$      &  $a^2$      &  $a^1 = a$  &  $a^0 = 1$
	\\ \hline
	$b^s$ 
		& $b^0 = 1$  &  $b^1 = b$  &  $b^2$      &  $b^3$      &  $b^4$
	\\ \hline
	Produit 
		& $a^4$      &  $a^3 b$    &  $a^2 b^2$  &  $a b^3$    &  $b^4$
\end{tabular}
\end{center}

\medskip
Avec cette écriture en tête, nous pouvons retenir plus simplement les développement de $(a + b)^n$ pour $n \in \ZintervalC{2}{5}$ comme suit où le cas admis pour $n = 5$ va nous permettre de vérifier la bonne compréhension de la convention utilisée.

\medskip
\begin{center}
\begin{tabular}{|l*{6}{|c}|}
	\hline
	$n$
		&      &       &        &        &       &
	\\ \hline
	$2 \rightarrow$ 
		& $1$  &  $2$  &  $1$   &        &       &
	\\ \hline
	$3 \rightarrow$ 
		& $1$  &  $3$  &  $3$   &  $1$   &       &
	\\ \hline
	$4 \rightarrow$ 
		& $1$  &  $4$  &  $6$   &  $4$   &  $1$  &
	\\ \hline
	$5 \rightarrow$ 
		& $1$  &  $5$  &  $10$  &  $10$  &  $5$  &  $1$
	\\ \hline
\end{tabular}
\end{center}

\medskip
Pour $(a + b)^5$ , l'exposant maximal sera $5$ et donc nous devons penser à ce qui suit.

\medskip
\begin{center}
\begin{tabular}{c*{6}{|c}}
	$a^r$
		& $a^5$  &  $a^4$    &  $a^3$      &  $a^2$      &  $a^1$    &  $a^0$
	\\ \hline
	$b^s$ 
		& $b^0$  &  $b^1$    &  $b^2$      &  $b^3$      &  $b^4$    &  $b^5$
	\\ \hline
	Produit 
		& $a^5$  &  $a^4 b$  &  $a^3 b^2$  &  $a^2 b^3$  &  $a b^4$  &  $b^5$
\end{tabular}
\end{center}

\medskip
Nous avons donc :

\medskip
\begin{center}
\begin{tabular}{|l*{6}{|c}|}
	\hline
	$5 \rightarrow$
		& $1$    &  $5$      &  $10$       &  $10$       &  $5$      &  $1$
	\\ \hline
		& $a^5$  &  $a^4 b$  &  $a^3 b^2$  &  $a^2 b^3$  &  $a b^4$  &  $b^5$
	\\ \hline
\end{tabular}
\end{center}

\medskip
Finalement $(a + b)^5 = a^5 + 5 a^4 b + 10 a^3 b^2 + 10 a^2 b^3 + 5 a b^4 + b^5$ . Le lecteur motivé pourra le vérifier via $(a + b)^5 = (a + b) (a + b)^4$ et le développement de $(a + b)^4$ démontré plus haut.

