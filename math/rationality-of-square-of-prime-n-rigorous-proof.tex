\documentclass[12pt]{amsart}
\usepackage[T1]{fontenc}
\usepackage[utf8]{inputenc}

\usepackage[top=1.95cm, bottom=1.95cm, left=2.35cm, right=2.35cm]{geometry}

\usepackage{hyperref}
\usepackage{enumitem}
\usepackage{tcolorbox}
\usepackage{float}
\usepackage{cleveref}
\usepackage{multicol}
\usepackage{fancyvrb}
\usepackage{enumitem}
\usepackage{amsmath}
\usepackage{textcomp}
\usepackage{numprint}
\usepackage[french]{babel}
\usepackage[
    type={CC},
    modifier={by-nc-sa},
	version={4.0},
]{doclicense}

\newcommand\floor[1]{\left\lfloor #1 \right\rfloor}

\usepackage{tnsmath}


\newtheorem{fact}{Fait}[section]
\newtheorem{example}{Exemple}[section]
\newtheorem{notation}{Notation}[section]
\newtheorem{remark}{Remarque}[section]
\newtheorem{unproved}{Non Prouvé}[section]
\newtheorem*{proof*}{Preuve}


\newcommand\seefact[1]{

	\smallskip

	\hfill {\footnotesize $\rightarrow$ Voir le fait \ref{#1}.}
}


\newcommand\seefactproof[2]{

	\smallskip

	\hfill {\footnotesize $\rightarrow$ Voir le fait \ref{#1} et la preuve dans la section \ref{#2}.}
}


\newcommand\seethreefacts[3]{

	\smallskip

	\hfill {\footnotesize $\rightarrow$ Voir les faits \ref{#1} et \ref{#2} ainsi que la section \ref{#3}.}
}


\npthousandsep{.}
\setlength\parindent{0pt}

\floatstyle{boxed}
\restylefloat{figure}


\DeclareMathOperator{\taille}{\text{\normalfont\texttt{taille}}}


\newcommand\sqrtp{\sqrt{p\,\vphantom{M}}}



\newcommand{\logicneg}{\text{\normalfont non \!}}

\newcommand\sqseq[2]{\fbox{$#1$}_{\,\,#2}}


\DefineVerbatimEnvironment{rawcode}%
	{Verbatim}%
	{tabsize=4,%
	 frame=lines, framerule=0.3mm, framesep=2.5mm}



\begin{document}

\title{BROUILLON - Une preuve rigoureuse (\!?) de l'irrationalité de la racine carrée d'un nombre premier}
\author{Christophe BAL}
\date{4 Avril 2019 - 4 Mai 2019}

\maketitle

\begin{center}
	\itshape
	Document, avec son source \LaTeX, disponible sur la page

	\url{https://github.com/bc-writing/drafts}.
\end{center}


\bigskip


\begin{center}
	\hrule\vspace{.3em}
	{
		\fontsize{1.35em}{1em}\selectfont
		\textbf{Mentions \og légales \fg}
	}

	\vspace{0.45em}
	\doclicenseThis
	\hrule
\end{center}


\bigskip
\setcounter{tocdepth}{2}
\tableofcontents


% --------------------- %


\section{Où allons-nous ?}

%\newpage
\section{????}

????


% --------------------- %


\section{Notations}

Dans la suite, nous utiliserons les notations suivantes.
\begin{itemize}
	\item $2\,\NN$ désigne l'ensemble des nombres naturels pairs.
	
	\item $2\,\NN + 1$ désigne l'ensemble des nombres naturels impairs.
	
	\item $\forall (n , m) \in \NN^2$, $n \vee m$ désigne le PPCM de $n$ et $m$.

	\item $\forall (n , m) \in \NN^2$, $n \wedge m$ désigne le PGCD de $n$ et $m$.

	\item $a \strictdivides b$ signifie que $a \divides b$ et $a \neq b$ (division stricte).

	\item $\PP$ désigne l'ensemble des nombres premiers.
	
	\item $\forall (p ; n) \in \PP \times \NNs$\,, $\padicval{n} \in \NN$ est la valuation $p$-adique de $n$\,, c'est-à-dire $p^{\padicval{n}} \divides n$\,, mais $p^{\padicval{n} + 1} \ndivides n$\,.
\end{itemize}


% --------------------- %


\section{\texorpdfstring{$\sqrt{p\,}$ n'est pas rationnel, une preuve très classique}%
		                {Racine carrée de p n'est pas rationnel, une preuve très classique}}

\begin{fact}
	$\forall p \in \PP, \sqrtp \not\in \QQ$.
\end{fact}

\begin{proof}
	Soit $p \in \PP$ quelconque mais fixé. L'irrationalité de $\sqrtp$ peut se démontrer très classiquement comme suit.
	
	\begin{itemize}[label=\small\textbullet]
		\item Regardons ce qu'il se passe si nous supposons l'existence de $(r ; s) \in \QQ \times \QQs$ tel que $\sqrtp = \dfrac{r}{s}$. On peut supposer que $(r ; s) \in \QQp \times \QQsp$ mais nous n'aurons pas besoin de supposer que $\pgcd(r ; s) = 1$.

	
		\item $\sqrtp = \dfrac{r}{s} \, \Leftrightarrow \, p \times s^2 = r^2$ car $r \geq 0$ et $s > 0$.
	
		\item Nous pouvons écrire les naturels
		$\displaystyle s = \prod_{j=1}^{n} p_j$
		et
		$\displaystyle r = \prod_{i=1}^{m} q_i$
		sous forme de produits de nombres premiers.
		Après simplification dans $p \times s^2 = r^2$ des nombres premiers communs entre les $p_j$ et les $q_i$ de part et d'autre, il restera un nombre impair de facteurs premiers égaux à $p$ à gauche, et un nombre pair à droite, éventuellement nul. 
		Ceci n'est clairement pas possible. 

	
		\item Comme nous obtenons quelque chose d'impossible, il ne peut pas exister $(r ; s) \in \QQ \times \QQs$ tel que $\sqrtp = \dfrac{r}{s}$.
	\end{itemize}
\end{proof}


\begin{unproved}
	Une première chose que nous avons admise très cavalièrement c'est la possibilité d'écrire un naturel comme un produit de facteurs premiers.
	
	\seefact{exists-decompo}
\end{unproved}


\begin{unproved}
	Un autre fait a été présenté comme immédiat à savoir l'impossibilité d'avoir une égalité entre deux produits de facteurs premiers dont l'un possède un nombre impair de facteurs premiers égaux à $p$, et l'autre en a un nombre pair éventuellement nul.
	
	\smallskip
	
	Ceci équivaut à l'impossibilité d'avoir $p \times a = b$ où soit $b = 1$, soit $b \neq 1$ s'écrit comme un produit de facteurs premiers tous différents de $p$ \emph{(pour se ramener à ce cas, il suffit de simplifier de part et d'autre des $p$ tant que c'est possible)}.
	
	\seefact{pseudo-prime-divisor}
\end{unproved}





% --------------------- %


\section{Décompositions en produit de facteurs premiers}

\begin{fact} \label{exists-decompo}
	$\forall a \in \NN - \setgene{0 ; 1}$, il existe au moins une suite finie de nombres premiers $(p_j)_{1 \leq j \leq n}$
	telle que $\displaystyle a = \prod_{j=1}^{n} p_j$. 
\end{fact}

\begin{proof}
	Considérons $a \in \NN - \setgene{0 ; 1}$.
	
	\begin{itemize}[label=\small\textbullet]
		\item Si $a$ premier il suffit de choisir $n = 1$ et $p_1 = a$.
	
	
		\item Dans le cas contraire, $a = b \, c$ où $(b ; c) \in \ZintervalC{2}{a-1}$ par définition d'un nombre premier.
		Il suffit alors de reprendre le même type de raisonnement à partir de $b$ et $c$ car l'on obtiendra de proche en proche des naturels de plus en plus petits et donc forcément il arrivera un moment où la décomposition en produit de deux naturels du type $b \, c$ ne sera plus possible.
	\end{itemize}
\end{proof}


\begin{unproved}
	Nous avons été bien cavaliers avec l'argument \emph{\og on obtiendra des naturels de plus en plus petits et donc forcément il arrivera un moment où... \fg}. Ce type d'argument se rédige proprement à l'aide du raisonnement par récurrence.

	\seefactproof{recursivity}{exists-decompo-clean}
\end{unproved}



\begin{fact} \label{pseudo-prime-divisor}
	Si $p \in \PP$ alors il n'existe pas $(a ; b) \in \NNs \times \NNs$ tel que
	$p \, a = b$ avec $b = 1$ ou $b \neq 1$ s'écrivant comme un produit de facteurs premiers tous différents de $p$.
\end{fact}
	

\begin{proof}
	Ceci découle directement du fait suivant plus facile à retenir.
\end{proof}



\begin{fact} \label{prime-divisor}
	Soit
	$(a ; b) \in \NNs \times \NNs$.
	%
	Si $p \in \PP$ vérifie $p \, a = b$ alors $b \neq 1$ et $p$ apparait dans toute décomposition en facteurs premiers de $b$.
\end{fact}
	

\begin{proof}
	$b \neq 1$ découle de $a \geq 1$ et $p \geq 2$. 
	Supposons avoir au moins une suite $(q_i)_{1 \leq i \leq m}$ de nombres premiers tous distincts de $p$ tels que $\displaystyle b = \prod_{i=1}^{m} q_i$ .
	Nous raisonnons alors comme suit.
	\begin{itemize}[label=\small\textbullet]
		\item Posant $q = q_1$ et $\displaystyle c = \prod_{i=2}^{m} q_i$ si $m \neq 1$ ou $c=1$ sinon, nous avons l'identité $p \, a = q \, c$ avec $p$ et $q$ deux nombres premiers distincts.
		
		
		\item Par définition des nombres premiers, $p$ et $q$ ont juste $1$ comme diviseur commun donc leur PGCD est $1$.

	
		\item Si $c \neq 1$, démontrons que $p$ divise $c$, autrement dit qu'il existe $k \in \NNs$ tel que $p \, k = c$ .
		L'algorithme d'Euclide nous donne par remontée des calculs l'existence de $(u ; v) \in \ZZ^2$ tel que $p \, u + q \, v = 1$. Ce résultat est appelé le théorème de Bachet-Bézout
		\footnote{
			À ne pas confondre avec une célèbre insulte du capitaine Haddock.
		}.
		
		\smallskip
		\noindent
		Nous avons alors sans effort :
		
		\smallskip
		\noindent
		$c = c(p \, u + q \, v)$
		
		\smallskip
		\noindent
		$c = p \, c \, u + q \, c \, v$
		
		\smallskip
		\noindent
		$c = p \, c \, u + p \, a \, v$ via $p \, a = q \, c$
		
		\smallskip
		\noindent
		$c = p(c \, u + a \, v)$
		
		\smallskip
		\noindent
		Donc $k = c \, u + a \, v$ convient.
		
		
		\item En répétant autant de fois que nécessaire ce qui précède, c'est à dire en isolant à chaque fois un facteur premier à droite, nous avons l'existence de $\widetilde{k} \in \NNs$ tel que $p \, \widetilde{k} = \widetilde{q}$ avec $\widetilde{q}$ un nombre premier distinct de $p$.

	
		\item Comme $\widetilde{q} \in \PP$, ses seuls diviseurs sont $1$ et lui-même. De $p \, \widetilde{k} = \widetilde{q}$ nous déduisons alors que $p$ est un diviseur de $\widetilde{q}$. Comme $p \neq 1$, nous avons alors $p = \widetilde{q}$ ce qui n'est pas possible car par construction $\widetilde{q} \neq p$.
	\end{itemize}
\end{proof}


\begin{unproved}
	L'algorithme d'Euclide, le théorème de Bachet-Bézout et l'existence d'une relation du type $p \, \widetilde{k} = \widetilde{q}$ ne peuvent être démontrés proprement que via un raisonnement par récurrence.

	\seethreefacts{euclide-algo}{bachet-bezout}{euclide-weak-lemma}
\end{unproved}


\begin{remark}
	Le fait \ref{prime-divisor} est une forme faible du lemme de divisibilité d'Euclide qui dit que si un nombre premier $p$ divise le produit de deux nombres entiers $b$ et $c$ alors $p$ divise $b$ ou $c$ \emph{(il suffit de considérer des décompositions en facteurs premiers de $b$ et $c$ en s'inspirant de la preuve précédente)}.  
\end{remark}


Notons au passage que le fait \ref{prime-divisor} implique l'unicité de la décomposition en facteurs premiers qui est indiquée dans le fait suivant.

\begin{fact} \label{prime-decompo}
	$\forall a \in \NN - \setgene{0 ; 1}$, il existe une et une seule suite finie croissante, non nécessairement strictement, de nombres premiers $(p_j)_{1 \leq j \leq n}$
	telle que $\displaystyle a = \prod_{j=1}^{n} p_j$. 
\end{fact}
	

\begin{proof}
	L'existence découlant directement du fait \ref{exists-decompo}, il nous reste à démontrer l'unicité.
	Pour cela considérons deux suites finies croissantes de nombres premiers
	$(p_j)_{1 \leq j \leq n}$
	et
	$(q_i)_{1 \leq i \leq m}$
	telles que $\displaystyle \prod_{j=1}^{n} p_j = \prod_{i=1}^{m} q_i$ .
	Nous raisonnons alors comme suit pour prouver que les deux suites sont indentiques.
	
	\begin{itemize}[label=\small\textbullet]
		\item Quitte à changer les noms des suites, on peut supposer que $p_1 \leq q_1$ .
		

		\item D'après le fait \ref{prime-divisor}, nous savons qu'il existe $i$ tel que $q_i = p_1$ .
	
		\item Par croissance de la suite $q$, nous avons $q_1 \leq q_i$.
		
		\item Nous avons alors $p_1 \leq q_1 \leq q_i = p_1$ puis $p_1 = q_1$ .
		
		\item D'après le point précédent nous pouvons réduire de un les tailles des suites.
	\end{itemize}
	
	On voit alors que l'on pourra ainsi répéter le raisonnement pour obtenir que les deux suites sont de même taille et identiques \emph{(une démonstration par récurrence trouverait sa place ici pour plus de rigueur mais nous ne la ferons pas dans ce document car le fait \ref{prime-decompo} est juste un petit bonus de notre exposé)}.
\end{proof}



% --------------------- %


\section{La démonstration par récurrence}

\documentclass[12pt]{amsart}
\usepackage[T1]{fontenc}
\usepackage[utf8]{inputenc}

\usepackage[top=1.95cm, bottom=1.95cm, left=2.35cm, right=2.35cm]{geometry}

\usepackage{hyperref}
\usepackage{enumitem}
\usepackage{tcolorbox}
\usepackage{float}
\usepackage{cleveref}
\usepackage{multicol}
\usepackage{fancyvrb}
\usepackage{enumitem}
\usepackage{amsmath}
\usepackage{textcomp}
\usepackage{numprint}
\usepackage[french]{babel}
\usepackage[
    type={CC},
    modifier={by-nc-sa},
	version={4.0},
]{doclicense}

\newcommand\floor[1]{\left\lfloor #1 \right\rfloor}

\usepackage{tnsmath}


\newtheorem{fact}{Fait}[section]
\newtheorem{example}{Exemple}[section]
\newtheorem{notation}{Notation}[section]
\newtheorem{remark}{Remarque}[section]
\newtheorem{unproved}{Non Prouvé}[section]
\newtheorem*{proof*}{Preuve}


\newcommand\seefact[1]{

	\smallskip

	\hfill {\footnotesize $\rightarrow$ Voir le fait \ref{#1}.}
}


\newcommand\seefactproof[2]{

	\smallskip

	\hfill {\footnotesize $\rightarrow$ Voir le fait \ref{#1} et la preuve dans la section \ref{#2}.}
}


\newcommand\seethreefacts[3]{

	\smallskip

	\hfill {\footnotesize $\rightarrow$ Voir les faits \ref{#1} et \ref{#2} ainsi que la section \ref{#3}.}
}


\npthousandsep{.}
\setlength\parindent{0pt}

\floatstyle{boxed}
\restylefloat{figure}


\DeclareMathOperator{\taille}{\text{\normalfont\texttt{taille}}}


\newcommand\sqrtp{\sqrt{p\,\vphantom{M}}}



\newcommand{\logicneg}{\text{\normalfont non \!}}

\newcommand\sqseq[2]{\fbox{$#1$}_{\,\,#2}}


\DefineVerbatimEnvironment{rawcode}%
	{Verbatim}%
	{tabsize=4,%
	 frame=lines, framerule=0.3mm, framesep=2.5mm}



\begin{document}

\title{BROUILLON - A propos de la récurrence}
\author{Christophe BAL}
\date{4 Avril 2019 - 4 Mai 2019}

\maketitle

\begin{center}
	\itshape
	Document, avec son source \LaTeX, disponible sur la page

	\url{https://github.com/bc-writing/drafts}.
\end{center}


\bigskip


\begin{center}
	\hrule\vspace{.3em}
	{
		\fontsize{1.35em}{1em}\selectfont
		\textbf{Mentions \og légales \fg}
	}

	\vspace{0.45em}
	\doclicenseThis
	\hrule
\end{center}


\bigskip
%\setcounter{tocdepth}{2}
%\tableofcontents


% --------------------- %


\begin{fact} \label{recursivity}
	La preuve par récurrence s'exprime comme suit où $\setproba{P}(k)$ désignera n'importe quelle proposition dépendant d'un paramètre naturel $k \in \NN$.

	\medskip

	On suppose avoir démontré les deux faits suivants.
	
	\begin{itemize}[label=\small\textbullet]
		\item \textbf{Initialisation :}
		      $\setproba{P}(0)$ est vraie.

		\item \textbf{Hérédité :}
		      $\forall k \in \NN$ , $\big[ \setproba{P}(k) \implies \setproba{P}(k+1) \big]$ .
	\end{itemize}

	Sous ces hypothèses, nous pouvons affirmer que $\forall n \in \NN$, $\setproba{P}(n)$ est vraie.
\end{fact}


\begin{proof}
	Par l'absurde, en considérant le plus petit naturel $n_0$ tel que $\setproba{P}(n_0)$ soit fausse, et en notant que $n_0 > 0$ . 
\end{proof}

\bigskip

Dès lors, le schéma de preuve suivant n'est pas une preuve par récurrence.
	
	\begin{itemize}[label=\small\textbullet]
		\item \textbf{Initialisation :}
		      $\setproba{P}(0)$ est vraie.

		\item \textbf{Hérédité bis :}
		      supposons qu'il existe $k \in \NN$ tel que $\setproba{P}(k)$ soit vraie, puis déduisons-en que $\setproba{P}(k+1)$ est vraie.
	\end{itemize}

En effet, l'hérédité bis est de la forme :
$\big[ \exists k \in \NN \text{ tel que } \setproba{P}(k) \text{ vraie} \big] \implies \big[ \setproba{P}(k+1) \text{ vraie} \big]$ .

\bigskip

Est-ce pour autant un raisonnement faux ? Non car une preuve similaire à celle indiquée ci-dessus permet d'avoir malgré tout que $\forall n \in \NN$, $\setproba{P}(n)$ est vraie.


\end{document}



% --------------------- %


%\section{Parlons un peu des nombres naturels}
%
%\subsection{Les naturels, c'est quoi en fait...}

????


\subsection{D'où vient la démonstrations par récurrence ?}

\input{rationality-of-square-of-prime-n-rigorous-proof/proof/integer/induction}



%
%





\bigskip

\hrule

\section{AFFAIRE À SUIVRE...}

\bigskip

\hrule

%
%
%\section{Et les nombres rationnels et les racines carrées dans tout cela ?}
%
%????





%
%
%
%\section{Raisonnement par l'absurde}
%
%\section{Mais finalement comment savons-nous que notre preuve est rigoureuse ?}







\end{document}
