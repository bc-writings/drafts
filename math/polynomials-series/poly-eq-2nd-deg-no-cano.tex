\documentclass[12pt]{amsart}
\usepackage[T1]{fontenc}
\usepackage[utf8]{inputenc}

\usepackage[top=1.95cm, bottom=1.95cm, left=2.35cm, right=2.35cm]{geometry}

\usepackage{amsmath}
\usepackage{amssymb}
\usepackage{enumitem}
\usepackage{multicol}
\usepackage[french]{babel}
\usepackage[
    type={CC},
    modifier={by-nc-sa},
	version={4.0},
]{doclicense}

\usepackage{tnsmath}

\DeclareMathOperator{\taille}{\tau}

\newtheorem{fact}{Fait}
\newtheorem*{proof*}{Preuve}

\newtheorem{remark}{Remarque}[section]

\setlength\parindent{0pt}


\newcommand\squote[1]{\og #1 \fg{}}


\begin{document}

\title{BROUILLON - En finir avec la forme canonique du trinôme... \\ MANQUE DES DESSINS !}
\author{Christophe BAL}
\date{19 Octobre 2020}
\maketitle


\begin{center}
	\hrule\vspace{.3em}
	{
		\fontsize{1.35em}{1em}\selectfont
		\textbf{Mentions \og légales \fg}
	}
			
	\vspace{0.45em}
	\doclicenseThis
	\hrule
\end{center}



\setcounter{tocdepth}{2}
\tableofcontents


% ------------- %


\newpage
%\section{Une convention utile}  $F$ poly à coef réel et $f$ la fonction de $\RR$ dans $\RR$   +   PARKER DE L'ASPECT PLUS GEOMÉTRIQUE QU'ALGÉBRIQUE DE L'APPROCHE


% ------------- %



\section{Critère pour la non existence d'une racine réelle}

Soient $f(x) = a \, x^2 + b \, x + c$ où $a \neq 0$ et $\setgeo*{C}{f}$ la représentation graphique de $f$ vue comme fonction de $\RR$ dans $\RR$.

\medskip

Graphiquement on constate vite que $\setgeo*{C}{f}$ possède un axe de symétrique $\setgeo{d}: x = m$ où $m$ se calcule come suit où $(\alpha ; \beta) \in \RR^2$.

\medskip

\begin{stepcalc}[style = sar, ope = \iff] 
	f(\alpha) = f(\beta)
		\explnext{}
	a \, \alpha^2 + b \, \alpha + c = a \, \beta^2 + b \, \beta + c
		\explnext{}
	a(\alpha^2 - \beta^2) + b(\alpha - \beta) = 0
		\explnext{}
	(\alpha - \beta) (a (\alpha + \beta) + b) = 0
									\quad\quad \fbox{\texttt{1}}
		\explnext{}
	\dfrac{\alpha + \beta}{2} = - \dfrac{b}{2a}
\end{stepcalc}

\medskip

Nécessairement $m = - \dfrac{b}{2a}$
\footnote{
	Nous n'avons pas prouver la propriété de symétrie car nous n'en aurons pas besoin.
	Ceci se fait en montrant que $\forall \delta \in \RR$, $f(m - \delta) = f(m + \delta)$.
	Cette égalité est évidente dès lors que l'on a $f(x) - f(m) = a(x - m)^2$, une identité que nous allons prouver bientôt.
}.
Il est aussi aisé de conjecturer que $f(m)$ est un extremum de $f$ vue comme fonction de $\RR$ dans $\RR$. Ceci rend naturel le calcul suivant qui part de la factorisation \fbox{\texttt{1}} précédente.

\medskip

\begin{stepcalc}[style = sar] 
	f(x) - f(m)
		\explnext{}
	(x - m) (a \, x + a \, m + b)
		\explnext{%
			\tiny%
			\begin{stepcalc}[style = sar, ope = \iff]
				m = - \dfrac{b}{2a}
					\explnext{}	
				2am = - b
					\explnext{}	
				am + b = - am
			\end{stepcalc}%
		}
	(x - m) (a \, x - a \, m)
		\explnext{}
	a(x - m)^2
\end{stepcalc}

\medskip

Ce qui précède montre aussi que 
si $a > 0$ alors $f(m)$ est un minimum et
si $a < 0$ alors $f(m)$ est un maximum. On peut maintenant savoir quand $f$ n'admet aucun zéro réel.

\begin{enumerate}
	\item \textbf{Cas 1 : $a > 0$}
	
		  \smallskip
		  
		  $f$ n'a pas de zéro si et seulement si $f(m) > 0$. Voyons ce que cela implique.
		  
		  \begin{stepcalc}[style = sar, ope = \iff]
		  		f(m) > 0
					\explnext{}
				a \, m^2 + b \, m + c > 0
					\explnext{}
				a \cdot \dfrac{b^2}{4 a^2} - b \cdot \dfrac{b}{2a} + c > 0
					\explnext{}
				- \dfrac{b^2}{4 a} + c > 0
					\explnext{\tiny$4 a > 0$}
				- b^2 + 4 a c > 0
		  \end{stepcalc}



	\item \textbf{Cas 2 : $a < 0$}
	
		  \smallskip
		  
		  $f$ n'a pas de zéro si et seulement si $f(m) < 0$. Cela implique ce qui suit.
		  
		  \begin{stepcalc}[style = sar, ope = \iff]
		  		f(m) < 0
					\explnext{}
				- \dfrac{b^2}{4 a} + c < 0
					\explnext{\tiny$4 a < 0$}
				- b^2 + 4 a c > 0
		  \end{stepcalc}
\end{enumerate}
 

Nous retombons sur le critère classique suivant : $a \, x^2 + b \, x + c$ n'a pas de zéro réel si et seulement si $- b^2 + 4 a c > 0$, soit de façon équivalente si et seulement si $b^2 - 4 a c < 0$.

\medskip

Notons que notre mode de raisonnement ne permet pas de privilégier le 2\ieme{} critère, ce dernier est en fait naturel uniquement si l'on passe via la forme canonique, ou bien lorsque l'on trouvera des formules calculant les zéros de $f$ lorsque ces derniers existent \emph{(c'est ce que nous allons faire dans la section suivante)}.





% ------------- %


\section{Quand au moins une racine réelle existe}

Soit $f(x) = a \, x^2 + b \, x + c$ où $a \neq 0$ tel qu'il existe $\alpha \in \RR$ annulant $f$.
La section précédente implique que nécessairement $b^2 - 4 ac \geq 0$ .

\medskip

La factorisation \fbox{\texttt{1}} de $f(\alpha) - f(\beta)$ vue dans la section précédente nous donne ici sans effort : 
$f(x) - f(\alpha) = (x - \alpha) (a (x + \alpha) + b)$ .
On en déduit l'existence d'un autre zéro $\beta \in \RR$, éventuellement égal à $\alpha$, tel que
$f(x) = a (x - \alpha) (x - \beta)$ .
Ceci implique, après développement, que
$\alpha   +   \beta = - \dfrac{b}{a}$
et
$\alpha \cdot \beta = \dfrac{c}{a}$ .


\medskip

Exploitons ici aussi l'usage de $m = - \dfrac{b}{2a}$ de sorte que
$\dfrac{\alpha + \beta}{2} = m$
\emph{(ceci est aussi une conséquence de l'égalité $f(\alpha) = f(\beta)$)}.
On va paramétrer notre problème via une seule inconnue
\footnote{
	Cette astuce permet en fait de diminuer par deux le nombre d'inconnues lorsque l'on cherche les racines d'un polynôme $p$, de degré pair forcément, tel que $\setgeo*{C}{p}$ ait un axe de symétrie.
}
grâce à $m$.
Pour cela posons
$\delta = \dfrac{\beta - \alpha}{2}$ de sorte que
$\alpha = m - \delta$ et $\beta = m + \delta$.
Nous obtenons :

\medskip

\begin{stepcalc}[style = sar, ope = \iff]
	\alpha \cdot \beta = \dfrac{c}{a}
		\explnext{}
	(m - \delta)(m + \delta) = \dfrac{c}{a}
		\explnext{}
	m^2 - \delta^2 = \dfrac{c}{a}
		\explnext{}
	\delta^2 = m^2 - \dfrac{c}{a}
		\explnext{}
	\delta^2 = \dfrac{b^2}{4a^2} - \dfrac{c}{a}
		\explnext{}
	\delta^2 = \dfrac{b^2 - 4 ac}{4a^2}
		\explnext{\tiny$b^2 - 4 ac \geq 0$}
	\delta = \pm \dfrac{\sqrt{b^2 - 4 ac}}{2 a}
\end{stepcalc}

\medskip

Nous avons donc établi que si $\alpha$ et $\beta$ sont deux zéros, éventuellement confondus, de $f$ alors $b^2 - 4 ac \geq 0$ et les zéros sont $\dfrac{-b \pm \sqrt{b^2 - 4 ac}}{2 a}$ .
La réciproque est immédiate.


\end{document}
