Nous allons voir que le passage au degré $4$ va faire exploser une vaine conjecture qui supposerait que pour un degré donné il n'y a qu'un nombre fini de types de courbe.
L'argument est simple car les polynômes $f_r(x) = x (x^2 - 4) (x - r)$ où $r \in \RR - \setgene{-1 ; 0 ; 1}$ ont des courbes $\setgeo*{C}{r}$ non similaires deux à deux via les transformations autorisées. Voici pourquoi où $r \neq 2$ par hypothèse.

\begin{enumerate}
	\item Supposons que $f_r$ et $f_2$ ont des courbes $\setgeo*{C}{r}$ et $\setgeo*{C}{2}$ similaires i.e. $f_r(x) = \lambda f_2(a x + b) + k$ avec nécessairement $\lambda \neq 0$ et $a \neq 0$.


	\item $f_r(x) = x^4 - r x^3 - 4 x^2 + 4 r x$


	\item $f_2(x) = x^4 - 2 x^3 - 4 x^2 + 8 x$


	\item valeurs sur $\intervalC{-1}{1}$ et valeur en 2 évolue tro p différemment !
\end{enumerate}
