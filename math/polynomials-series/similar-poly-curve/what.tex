\subsection{Une preuve visuelle ou presque}

\leavevmode
\smallskip


% ------------- %


Il est connu que les courbes des fonctions affines sont toutes des droites non verticales, et celles représentant des trinômes du 2\ieme{} degré sont toutes des paraboles
\footnote{
	La définition géométrique des grecques anciens restent la meilleure.
}.
Laissant de côté le cas des fonctions constantes, nous constatons plus précisément les propriétés suivantes.
%
\begin{enumerate}
	\item Au lycée, on explique que l'on peut passer de la représentation de la fonction carrée 
	$\funcdef[h]{f}{x}{x^2}{}{}$ 
	à celle du trinôme du 2\ieme\ degré
	$\funcdef[h]{g}{x}{a \, x^2 + b \, x + c}{}{}$
	via une translation, une dilatation verticale et/ou une dilatation horizontale.


	\item De même, on peut passer de la représentation de la fonction idendité
	$\funcdef[h]{f}{x}{x}{}{}$ 
	à celle d'une fonction affine non constante
	$\funcdef[h]{g}{x}{a \, x + b}{}{}$
	via une translation, une dilatation verticale et/ou une dilatation horizontale.
\end{enumerate}


Une fois ceci noté, il devient naturel de se poser les questions suivantes.
%
\begin{enumerate}	
	\item Peut-on passer de la courbe de la fonction cube
	$\funcdef[h]{f}{x}{x^3}{}{}$
	à celle du polynôme du 3\ieme\ degré
	$\funcdef[h]{g}{x}{a \, x^3 + b \, x^2 + c \, x + d}{}{}$
	via une translation, une dilation verticale et/ou une dilatation horizontale ?

	\item Que se passe-t-il plus généralement pour les polynômes de degré  $k \geq 4$ ?
\end{enumerate}

 