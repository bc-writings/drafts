Considérons $P$ un polynôme anti-symétrique de degré $4$ et de coefficient dominant $1$ \emph{(il est toujours possible de supposer ce second point)}.

\medskip

$P(X) = -X^4 P\left( \dfrac1X \right)$ : caractérisation des polynômes anti-symétriques de degré $4$

\medskip

$P\,^{\prime}(X) = - 4 X^3 P\left( \dfrac1X \right) 
            + X^2 P\,^{\prime}\left( \dfrac1X \right)$


\medskip

Nous avons de nouveau la clé de voûte des raisonnements précédents sur la multiplicité d'une racine et de son inverse.
Donc les raisonnements de la section 2 nous donnent :

\begin{enumerate}
	\item $P$ n'a que des racines entières si et seulement si $P(X) = (X - 1)^3 (X + 1)$ ou bien $P(X) = (X - 1) (X + 1)^3$.

	\item $P$ n'a que des racines rationnelles dont une au moins non entière si et seulement si $P(X) = (X - r) \left( X - \dfrac1r \right) (X - 1) (X + 1)$ où $r \in \QQ - \NN$.
\end{enumerate}



