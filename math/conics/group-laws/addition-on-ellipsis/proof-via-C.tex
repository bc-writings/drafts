Travaillons dans le plan complexe associé au repère $\paxes{O | I | J}$ .
Nous posons $z_A = \ee^{\ii a}$ et $z_B = \ee^{\ii b}$ avec $(a ; b) \in \RR^2$ .
Nous avons alors $z_S = \ee^{\ii (a + b)}$ .  


\medskip


\textbf{Cas 1.} \emph{Supposons que $a \not\equiv b \,\, [2\pi]$ , soit $A \neq B$ .}

\medskip

Tout d'abord, nous avons $z_{\vect{AB}} = \ee^{\ii b}- \ee^{\ii a}$ et $z_{\vect{IS}} = \ee^{\ii (a + b)} - 1$. Nous devons trouver un réel $k \in \RR$ tel que $z_{\vect{IS}} = k z_{\vect{AB}}$ . Dans la suite, nous noterons $s = a + b$ . Nous avons alors :
\begin{flalign*}
	z_{\vect{IS}} 
		&= \ee^{\ii s} - 1
		& \\
		&= \ee^{\ii s / 2} \left( \ee^{\ii s / 2} - \ee^{- \ii s / 2} \right)
		& \\
		&= \ee^{\ii s / 2} \cdot 2 \ii \sin(0,5 s)
		& \\
		&= 2 \ii \sin(0,5 s) \ee^{\ii s / 2}
		& \\
\end{flalign*}

\vspace{-1em}

A l'aide de la même technique classique précédente, notant $d = b - a$, nous obtenons :
\begin{flalign*}
	z_{\vect{AB}} 
		&= \ee^{\ii b}- \ee^{\ii a}
		& \\
		&= \ee^{\ii a} \left( \ee^{\ii d} - 1 \right)
		& \\
		&= \ee^{\ii a} \cdot 2 \ii \sin(0,5 d) \ee^{\ii d / 2}
		& \\
		&= 2 \ii \sin(0,5 d) \ee^{\ii s / 2}
		& \text{En effet, $d + 2 a = s$.} \\
\end{flalign*}

\vspace{-1em}


Comme $d \not\equiv 0 \,\, [2\pi]$ par hypothèse, soit de façon équivalente $0,5 d \not\equiv 0 \,\, [\pi]$ , nous avons finalement $z_{\vect{IS}} = k z_{\vect{AB}}$ où $k = \frac{\sin(0,5 s)}{\sin(0,5 d)} \in \RR$ .



\bigskip

\textbf{Cas 2.} \emph{Supposons que $a \equiv b \,\, [2\pi]$ , soit $A = B$ .}

\medskip

Nous devons prouver l'orthogonalité des vecteurs $\vect{OA}$ et $\vect{IS}$ avec ici $z_S = \ee^{2 \ii a}$ .
Ceci découle de la nullité de la partie réelle du produit suivant
\footnote{
	Se souvenir que 
	$(x + \ii y) \, \overline{(x' + \ii y')} = xx' + yy' +\ii (x'y - xy')$ .
	Notez que cette identité est un deux-en-un contenant le produit scalaire et le déterminant des vecteurs $\vect{u}$ et $\vect{v}$ d'affixes complexes respectives $(x + \ii y)$ et $(x' + \ii y')$ .
	Si $\vect{u} \neq \vect{0}$ et $\vect{v} \neq \vect{0}$ ,
	la notation exponentielle nous donne sans effort que
	$\cos \theta = \frac{\dotprod{\vect{u}}{\vect{v}}}{\norm{\vect{u}} \cdot \norm{\vect{v}}}$
	et
	$\sin \theta = \frac{\det\left ( \vect{u} ; \vect{v} \right)}{\norm{\vect{u}} \cdot \norm{\vect{v}}}$
	où $\theta$ est une mesure en radians de l'angle orienté $\angleorient{\vect{u}}{\vect{v}}$.
}.
\begin{flalign*}
	z_{\vect{OA}} \,\, \overline{z_{\vect{IS}} \vphantom{T}}
		&= \ee^{\ii a} \, \overline{\left( \ee^{2 \ii a} - 1 \right)}
		& \\
		&= \ee^{\ii a} \, \left( \ee^{- 2 \ii a} - 1 \right)
		& \\
		&= \ee^{- \ii a} - \ee^{\ii a}
		& \\
		&= 2 \ii \sin a
		& \\
\end{flalign*}

\vspace{-1em}
