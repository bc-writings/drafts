Dans le BAC S de Juin 2018 (Métropole), la partie A de l'exercice de Spécialité Mathématiques s'intéressait à l'équation diophantienne \textbf{[ED]} : $x^2 - 8 y^2 = 1$ sur $\NN^2$.


\medskip

On a une solution évidente $(x \,; y) = (3 \,; 1)$ puis l'exercice introduit une matrice \squote{magique}
$A =
\begin{pmatrix} 
  3 & 8  \\ 
  1 & 3 
\end{pmatrix}$
pour ensuite construite des solutions $(x_n \,; y_n)$ de façon récursive et linéaire comme suit :
$\begin{pmatrix} 
  x_{n+1} \\ 
  y_{n+1} 
\end{pmatrix}
=
A
\begin{pmatrix} 
  x_{n} \\ 
  y_{n} 
\end{pmatrix}
$ .


\medskip

Très bien mais mais comment peut-on découvrir la matrice \squote{magique} $A$ ?