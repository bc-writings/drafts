Le procédé de construction que nous venons de prouver dans les sections précédentes se \emph{\og conserve \fg} par translations, et aussi par dilatations verticales et horizontales.
Il se trouve que ce sont ces transformations qui permettent d'obtenir une parabole $\setgeo{P}\,^\prime : y = a x^2 + b x + c$ , où $a \neq 0$ , à partir de celle de la parabole $\setgeo{P} : y = x^2$ .
Nous pouvons donc munir toute parabole $\setgeo{P}\,^\prime : y = a x^2 + b x + c$ d'une structure de groupe isomorphe à celle de $(\RR ; +)$ , et ceci avec un procédé géométrique simple pour \emph{\og additionner \fg} sur $\setgeo{P}\,^\prime$ . Que c'est joli !
