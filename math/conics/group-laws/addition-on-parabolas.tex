\RequirePackage{snapshot}
\documentclass[12pt]{amsart}
\usepackage[T1]{fontenc}
\usepackage[utf8]{inputenc}

\usepackage[top=1.95cm, bottom=1.95cm, left=2.35cm, right=2.35cm]{geometry}

\usepackage{amsmath}
\usepackage{amssymb}
\usepackage{enumitem}
\usepackage{multicol}
\usepackage[french]{babel}
\usepackage[
    type={CC},
    modifier={by-nc-sa},
	version={4.0},
]{doclicense}

\usepackage{tnsmath}

\DeclareMathOperator{\taille}{\tau}

\newtheorem{fact}{Fait}
\newtheorem*{proof*}{Preuve}

\setlength\parindent{0pt}


\newcommand\squote[1]{\og #1 \fg{}}


\begin{document}

\title{BROUILLON - Faire des additions sur une parabole}
\author{Christophe BAL}
\date{17 Juillet 2019 - 27 Juillet 2019}
\maketitle


\vspace{-.9em}


\begin{center}
	\hrule\vspace{.3em}
	{
		\fontsize{1.35em}{1em}\selectfont
		\textbf{Mentions \og légales \fg}
	}
			
	\vspace{0.45em}
	\doclicenseThis
	\hrule
\end{center}



\setcounter{tocdepth}{2}
\tableofcontents



\section{\texorpdfstring{Comment additionner des nombres grâce à la parabole d'équation $y = x^2$}%
                        {Comment additionner des nombres grâce à la parabole d'équation y = x**2}}

Voici un procédé facile à faire à l'aide d'une calculatrice.
Considérons un entier naturel $n$, puis calculons la somme de ses chiffres élevés au carré. Ceci nous donne un nouveau naturel auquel on peut appliquer le même procédé. Voici deux exemples.


\begin{example}
	Pour $n = 19$, nous obtenons :
	\begin{itemize}[label=\textbullet]
		\item $1^2 + 9^2 = 82$
		\item $8^2 + 2^2 = 68$
		\item $6^2 + 8^2 = 100$
		\item $1^2 + 0^2 + 0^2 = 1$ $\rightarrow$ Rien de nouveau à attendre.
	\end{itemize}
\end{example}


\begin{example}
	Pour $n = 1\,234\,567\,890$, après $1^2 + 2^2 + 3^2 + 4^2 + 5^2 + 6^2 + 7^2 + 8^2 + 9^2 + 0^2 = 285$ nous obtenons :
	\vspace{-.7em}
	\begin{multicols}{2}
		\begin{itemize}[label=\textbullet]
			\item $2^2 + 8^2 + 5^2 = 93$
			\item $9^2 + 3^2 = 90$
			\item $9^2 + 0^2 = 81$
			\item $8^2 + 1^2 = 65$
			\item $6^2 + 5^2 = 61$
			\item $6^2 + 1^2 = 37$
			\item $3^2 + 7^2 = 58$
		\end{itemize}
		\columnbreak
		\begin{itemize}[label=\textbullet]
			\item $5^2 + 8^2 = 89$
			\item $8^2 + 9^2 = 145$
			\item $1^2 + 4^2 + 5^2 = 42$
			\item $4^2 + 2^2 = 20$
			\item $2^2 + 0^2 = 4$
			\item $4^2 = 16$ 
			\item $1^2 + 6^2 = 37$ $\rightarrow$ Dèjà rencontré.
		\end{itemize}
	\end{multicols}
\end{example}

Dans le 1\ier{} cas, au bout d'un moment le procédé ne produit que des $1$. Ce sera le cas dès que l'on commence avec une puissance de $10$. Le 2\ieme{} exemple montre que le mieux que l'on puisse espérer c'est que le procédé devienne périodique à partir d'un moment \emph{(on parle de phénomène ultimement périodique)}.


\medskip

On peut explorer le comportement de ce procédé sur plusieurs valeurs grâce à un programme. Voici un code possible écrit en Python qui prend un peu de temps pour vérifier que pour tous les naturels $n \in \ZintervalC{1}{10^6}$, le procédé devient ultimement périodique.

\begin{rawcode}
NMAX    = 10**6
MAXLOOP = 10**20

for n in range(1, NMAX + 1):
    nbloops = 0
    results = []

    while nbloops < MAXLOOP and n not in results:
        nbloops += 1
        results.append(n)
        n = sum(int(d)**2 for d in str(n))

    if n not in results:
        print(f"Test raté pour n = {n}.")

print(f"Tests finis.")
\end{rawcode}

\medskip

Il reste à voir ce qu'il se passe dans le cas général. La section qui suit démontre que pour tout naturel $n$, le procédé sera toujours ultimement périodique.





\section{Preuve de la validité de la conjecture} \label{proof}
\label{proof}

Pour un naturel 
$\displaystyle      n = \left[ c_{d-1} c_{d-2} \cdots c_1 c_0 \right]_{10} 
\stackrel{\text{def}}{=} \sum_{k=0}^{d-1} c_k 10^k$
avec $c_{d-1} \neq 0$,
on pose
$\displaystyle sq(n) = \sum_{k=0}^{d-1} (c_k)^2$
et
$\taille(n) = d$ qui sera appelé \emph{\og taille de $n$ \fg}.

\medskip

Pour $(n \,; k) \in \NN^2$, on définit 
$  \sqseq{n}{0} = n$
et
$  \sqseq{n}{k} = sq^k(n)
\stackrel{\text{def}}{=} sq \,\circ sq \,\circ \cdots \,\circ sq(n)$ avec $(k-1)$ compositions si $k > 0$.
Autrement dit,
$\sqseq{n}{k+1} = sq \left( \, \sqseq{n}{k} \right)$.

\medskip

On note enfin
$V_n = \geneset{ \, \sqseq{n}{k} \, | \, k \in \NN }$
l'ensemble des valeurs prises par la suite $\left( \, \sqseq{n}{k} \right)_k$.



\medskip

\begin{fact}
	$\forall n \in \NN$, $sq(n) \leqslant 81 d$ où $d = \taille(n)$.
\end{fact}

\begin{proof*}
	Si $n = [c_{d-1} c_{d-2} \cdots c_1 c_0]_{10}$
	alors 
	$\displaystyle sq(n) = \sum_{k=0}^{d-1} (c_k)^2 \leqslant \sum_{k=0}^{d-1} 9^2 = 81 d $.
\end{proof*}



\medskip

\begin{fact}
	$\forall n \in \NN$, notant $d = \taille(n)$, nous avons les résultats suivants :
	
	\begin{enumerate}
		\item Si $d \geqslant 4$ alors $\taille(sq(n)) < \taille(n)$.
		
		\item Si $d \leqslant 3$ alors $\taille(sq(n)) \leqslant 3$.
	\end{enumerate}
\end{fact}

\begin{proof*}
	Notons que $n \geqslant 10^{d-1}$.
	Le comportement des fonctions $10^{x-1}$ et $81x$ sur $\RRsp$ assure l'existence d'un naturel $D$ tel que $\forall d \in \NN$, $d \geqslant D$ implique $10^{d-1} > 81d \geqslant sq(n)$. On a même beaucoup mieux : si $10^{D-1} > 81D \geqslant sq(n)$ alors $d \geqslant D$ implique $10^{d-1} > 81d \geqslant sq(n)$.
	
	\medskip
	
	Comme $10^3 > 81 \times 4$, nous avons sans effort le 1er point \emph{(rappelons que $10^k > n$ implique que $n$ admet au plus $(k-1)$ chiffres)}.
	
	\medskip
	
	Pour $d \leqslant 3$, le 2nd point découle de $sq(999) = 243$, $sq(99) = 162$ et $sq(9) = 81$.
\end{proof*}



\medskip

\begin{fact}
	$\forall n \in \NN$, l'ensemble $V_n$ est fini et donc la suite $\left( \, \sqseq{n}{k} \right)_{k \in \NN}$ est ultimement périodique, i.e. périodique à partir d'un certain rang.
\end{fact}

\begin{proof*}
	Le 2nd point dépend directement du 1er point via le principe des tiroirs et la définition récursive de la suite $\left( \, \sqseq{n}{k} \right)_k$.
	
	\medskip
	
	Pour le 1er point, il suffit de montrer que $V_n \subset \intervalC{0}{10^{\taille(n)}}$ pour $n \geqslant 4$ via une petite récurrence descendante finie, et pour $n \leqslant 3$ on a directement $V_n \subset \intervalC{0}{10^3}$.
\end{proof*}



\section{\texorpdfstring{Toute parabole d'équation $y = a x^2 + b x + c$ a une structure de groupe}%
                        {Toute parabole d'équation y = a x**2 + b x + c a une structure de groupe}}
      
Le procédé de construction que nous venons de prouver dans les sections précédentes se \emph{\og conserve \fg} par translations, et aussi par dilatations verticales et horizontales.
Il se trouve que ce sont ces transformations qui permettent d'obtenir une parabole $\geoset{P}\,^\prime : y = a x^2 + b x + c$ , où $a \neq 0$ , à partir de celle de la parabole $\geoset{P} : y = x^2$ .
Nous pouvons donc munir toute parabole $\geoset{P}\,^\prime : y = a x^2 + b x + c$ d'une structure de groupe isomorphe à celle de $(\RR ; +)$ , et ceci avec un procédé géométrique simple pour \emph{\og additionner \fg} sur $\geoset{P}\,^\prime$ . Que c'est joli !




\section{Une caractérisation analytique des fonctions trinômes nulles en zéro}
      
Finissons ce document avec un résultat plus technique en nous demandant quelles peuvent être les fonctions $f$ définies et dérivables sur $\RR$ qui vérifient les propriétés suivantes utilisées dans la démonstration de la section \ref{proof}.
\begin{enumerate}
	\item $\forall (a ; b) \in \RR^2$ , si $a \neq b$ alors $\frac{f(a) - f(b)}{a - b} = \frac{f(a + b)}{a + b}$ .

	\item $f(0) = 0$ .

	\item $\forall a \in \RRs$ , $f\,'(a) = \frac{f(2 a)}{2 a}$ . Notons que cette propriété est en fait une conséquence de la première via un passage à la limite de $b$ vers $a$ .
\end{enumerate}


\medskip


La fonction $f$ doit être une solution sur $\RR$ de l'équation différentielle $2 x \, y\,'(x) = y(2x)$ qui n'est pas très sympathique car d'un côté il a $x$ comme variable et de l'autre il y a $2x$ !
Pour avancer, nous allons nous limiter au cas où $f$ est la restriction d'une fonction $\widetilde{f}$ définie et holomorphe sur $\CC$ , c'est à dire telle que $\forall \omega \in \CC$ , la limite $\displaystyle \lim_{\stackrel{\abs{z - \omega} \, \rightarrow \, 0}{z \neq \omega}} \frac{\widetilde{f}(z) - \widetilde{f}(\omega)}{z - \omega}$ existe dans $\CC$
\emph{(par exemple, les fonctions polynomiales vérifient cette hypothèse)}.


\medskip

On sait alors qu'il existe $R > 0$ tel que $\abs{z} < R$ implique
$\displaystyle \widetilde{f}(z) = \sum_{k = 0}^{+ \infty} c_k z^k$ , ce développement étant unique car $c_k = \frac{\der[e]{f}{x}{k}(0)}{k!}$ .
De plus, on a le résultat fort suivant : les éventuels zéros $\lambda$ de $\widetilde{f}$ sont isolés, c'est à dire qu'il existe un voisinage de $\lambda$ sur lequel $\widetilde{f}$ ne s'annule qu'en $\lambda$ . Nous admettrons ces deux faits de l'analyse complexe car les prouver nous amènerait trop loin.


\medskip

Nous admettrons aussi que $\forall x \in \RR$ , $2 x \, f\,'(x) = f(2x)$ implique $\forall z \in \CC$ , $2 z \, \widetilde{f}\,'(z) = \widetilde{f}(2z)$ \emph{(ceci vient principalement de la propriété des zéros isolés et du fait que toute fonction holomorphe l'est à tout ordre)}. 


\medskip

Dès que $\abs{z} < 0,5 R$ , nous avons alors :
\begin{flalign*}
	2 z \, \widetilde{f}\,'(z) = \widetilde{f}(2z)
		&\Longleftrightarrow
		2 z \sum_{k = 0}^{+ \infty} k c_k z^{k - 1}
		=
		\sum_{k = 0}^{+ \infty} c_k (2z)^k
		& \\
		&\Longleftrightarrow
		\sum_{k = 0}^{+ \infty} 2 k c_k z^k
		=
		\sum_{k = 0}^{+ \infty} c_k (2z)^k
		& \\
		&\Longleftrightarrow
		\forall k \in \NN \, , \, 2 k c_k = 2^k c_k
		& \\
		&\Longleftrightarrow
		c_1 
		\,\,
		\text{et}
		\,\,
		c_2
		\,\,
		\text{quelconques et}
		\,\,
		\forall k \in \NNs - \setgene{1 ; 2} \, , \, c_k = 0
		& \\
		&\Longleftrightarrow
		\widetilde{f}(z) = c_1 z + c_2 z^2
		& \\
\end{flalign*}

\vspace{-1em}


Le principe des zéros isolés et le fait que  $\widetilde{f}(z)$ et $c_1 z + c_2 z^2$ soient holomorphes sur $\CC$ nous donnent que $\widetilde{f}(z) = c_1 z + c_2 z^2$ sur $\CC$ tout entier, et donc $f(x) = c_1 x + c_2 x^2$ sur $\RR$ .


\medskip

Compte tenu de la section précédente, la condition nécessaire ci-dessus est aussi suffisante. Nous avons donc obtenu une caractérisation des fonctions trinômes nulles en zéro parmi les fonctions qui sont la restriction d'une fonction définie et holomorphe sur $\CC$ .


\end{document}
