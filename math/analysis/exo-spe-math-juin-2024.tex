\documentclass[12pt]{amsart}
\usepackage[T1]{fontenc}
\usepackage[utf8]{inputenc}

\usepackage[top=1.95cm, bottom=1.95cm, left=2.35cm, right=2.35cm]{geometry}

\usepackage{amsmath}
\usepackage{amssymb}
\usepackage{enumitem}
\usepackage{multicol}
\usepackage[french]{babel}
\usepackage[
    type={CC},
    modifier={by-nc-sa},
	version={4.0},
]{doclicense}

\usepackage{tnsmath}

\setlength\parindent{0cm}

\begin{document}

\title{À propos de l'exercice de SPÉ MATHS du BAC de Juin 2024 (Métropole)}
\author{Christophe BAL}
\date{19 Juin 2024}
\maketitle


\begin{center}
	\itshape
	Document, avec son source \LaTeX, disponible sur la page
	
	\url{https://github.com/bc-writing/drafts}.
\end{center}


\bigskip


\begin{center}
	\hrule\vspace{.3em}
	{
		\fontsize{1.35em}{1em}\selectfont
		\textbf{Mentions \og légales \fg}
	}
			
	\vspace{0.45em}
	\doclicenseThis
	\hrule
\end{center}

\bigskip

Sachant que $\ln \alpha = 2(2 - \alpha)$ où $\alpha \in [1 ; 2]$\,, nous allons calculer $\setproba{I} = \dintegrate*{x^2 \ln x}{x}{1 / \alpha}{1}$ sans faire d'intégration par parties.

\medskip

\begin{stepcalc}[style=sar]
	\setproba{I}
\explnext{}
	\dintegrate*{%	
		x^2  \, \dintegrate*{\dfrac{1}{t}}{t}{1}{x}%
	}{x}{1 / \alpha}{1}
\explnext*{On veut une intégrale positive pour faciliter la suite.}{}
	-
	\dintegrate*{%	
		\dintegrate*{\dfrac{x^2}{t}}{t}{x}{1}%
	}{x}{1 / \alpha}{1}
\explnext*{Faire un schéma pour la permutation des intégrales.}{}
	-
	\dintegrate*{%	
		\dintegrate*{\dfrac{x^2}{t}}{x}{1 / \alpha}{t}%
	}{t}{1 / \alpha}{1}
\explnext{}
	\dintegrate*{%	
		\hook*{\dfrac{x^3}{3 \, t}}{x}{1 / \alpha}{t}%
	}{t}{1}{1 / \alpha}
\explnext{}
	\dintegrate*{%	
		\Big( \dfrac{t^2}{3} - \dfrac{1}{3 \, \alpha^3 \, t} \Big)%
	}{t}{1}{1 / \alpha}
\explnext{}
	\hook*{%	
		\dfrac{t^3}{9} - \dfrac{1}{3 \, \alpha^3} \, \ln t%
	}{t}{1}{1 / \alpha}
\explnext{}
	\dfrac{1}{9 \, \alpha^3}
	-
	\dfrac{1}{3 \, \alpha^3} \, \ln \big( \dfrac{1}{\alpha} \big)%
	-
	\dfrac{1}{9}%
\explnext*{$\ln \alpha = 2(2 - \alpha)$}{}
	\dfrac{1}{9 \, \alpha^3}
	+
	\dfrac{2(2 - \alpha)}{3 \, \alpha^3}%
	-
	\dfrac{1}{9}%
\explnext{}
	\dfrac{1 + 6(2 - \alpha) - \alpha^3}{9 \, \alpha^3}%
\explnext{}
	\dfrac{13 - 6 \, \alpha - \alpha^3}{9 \, \alpha^3}%
\end{stepcalc}

\end{document}
