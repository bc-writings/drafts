Soit $\setgeo*{C}{g}$ la courbe de la fonction
$\funcdef[h]{g}{x}{a \, x^3 + b \, x^2 + c \, x + d}{}{}$
où $a\neq 0$.
Nous allons démontrer que $\setgeo*{C}{g}$ s'obtient à partir de l'une des courbes suivantes en utilisant une translation horizontale, une translation verticale, une dilatation verticale et/ou une dilatation horizontale.

\begin{enumerate}
	\item $\Gamma_1$ représente $\funcdef[h]{f_1}{x}{x^3}{}{}$.

	\item $\Gamma_2$ représente $\funcdef[h]{f_2}{x}{x^3 - x}{}{}$ de sorte que \txtfuncdef{f_2}{x}{x(x - 1)(x + 1)}{}{}.

	\item $\Gamma_3$ représente $\funcdef[h]{f_3}{x}{x^3 + x}{}{}$ de sorte que \txtfuncdef{f_3}{x}{x(x - \ii)(x + \ii)}{}{} où $\ii \in \CC$.
\end{enumerate}


% ------------- %


\begin{proof}
	\leavevmode
	\begin{enumerate}
		\item \textbf{On peut supposer que $(a ; b ; d) = (1 ; 0 ; 0)$.}

		      \begin{enumerate}
		      		\item Il est immédiat que l'on peut supposer que $a = 1$. 
				          Dans la suite, on supposera donc \txtfuncdef{g}{x}{x^3 + b \, x^2 + c \, x + d}{}{}.

		      		\item En considérant $\setgeo*{C}{g}$, on observe un centre de symétrie qui a pour abscisse $m$ celle de l'unique point d'inflexion de $\setgeo*{C}{g}$.
					      
					      \smallskip
					      
					      \noindent
					      \begin{stepcalc}[style = sar, ope = \iff]
					      		\der{g}{x}{2}(x) = 0
									\explnext{}
								6x + 2b = 0
									\explnext{}
								x = - \dfrac{b}{3}
					      \end{stepcalc}
					      
					      \smallskip
					      
					      \noindent
					      Il devient naturel de poser $x = m + t$ avec $m = - \dfrac{b}{3}$.
					      
					      \smallskip
					      
					      \noindent
					      \begin{stepcalc}[style = sar]
					      		g(x)
									\explnext{}
								g(m + t)
									\explnext{}
								(m + t)^3 + b \, (m + t)^2 + c \, (m + t) + d
									\explnext{}
								m^3 + 3 m^2 \, t + 3 m \, t^2 + t^3
									+ b \, m^2 + 2 b \, m \, t + b \, t^2 
									+ c \, m + c \, t + d
					      \end{stepcalc} 
					      
					      \smallskip
					      
					      \noindent
					      Le coefficient de $t^3$ reste égal à $1$ et celui de $t^2$ est $3m + b = 0$. Ceci montre que l'on peut supposer $(a ; b) = (1 ; 0)$.
				          Dans la suite, on supposera donc \txtfuncdef{g}{x}{x^3 + c \, x + d}{}{}.

		      		\item Il est immédiat que l'on peut supposer dans la suite que \txtfuncdef{g}{x}{x^3 + c \, x}{}{}.
		      \end{enumerate}


% ------------- %


		\item \textbf{Cas 1 : $c = 0$}

		      \smallskip

		      \noindent
		      Nous n'avons rien à faire de plus car ici $\setgeo*{C}{g} = \Gamma_1$.


% ------------- %


		\item \textbf{Cas 2 : $c = - k^2$ avec $k > 0$}

		      \smallskip

		      \noindent
		      Ici
		      \txtfuncdef{g}{x}{x^3 - k^2 \, x}{}{}
		      soit
		      \txtfuncdef{g}{x}{x(x - k)(x + k)}{}{}.

		      \smallskip

		      \noindent
		      Nous avons donc $g(k \, x) = k^3 \, x(x - 1)(x + 1) = k^3 \, f_2(x)$
		      puis
		      $f_2(x) = \dfrac{1}{k^3} g(k \, x)$.

		      \smallskip

		      \noindent
		      On peut ainsi passer de $\setgeo*{C}{g}$ à $\Gamma_2$, et donc aussi de $\Gamma_2$ à $\setgeo*{C}{g}$, à l'aide des transformations autorisées.


% ------------- %


		\item \textbf{Cas 3 : $c = k^2$ avec $k > 0$}

		      \smallskip

		      \noindent
		      Ici
		      \txtfuncdef{g}{x}{x^3 - (k \, \ii)^2 \, x}{}{}
		      soit
		      \txtfuncdef{g}{x}{x(x - k \, \ii)(x + k \, \ii)}{}{}.

		      \smallskip

		      \noindent
		      Nous avons donc $g(k \, x) = k^3 \, x(x - \ii)(x + \ii) = k^3 \, f_3(x)$
		      puis comme dans le cas précédent on peut passer de $\Gamma_3$ à $\setgeo*{C}{g}$ à l'aide des transformations autorisées.
	\end{enumerate}
\end{proof}


% ------------- %


On notera que la preuve précédente est constructive, autrement dit on peut donner les applications à appliquer en fonction des coefficients $a$, $b$, $c$, et $d$ de \txtfuncdef{g}{x}{a \, x^3 + b \, x^2 + c \, x + d}{}{}.


% ------------- %


\medskip

Il est évident qu'il n'est pas possible de passer de $\Gamma_i$  à $\Gamma_j$ à l'aide des transformations autorisées \emph{(penser à la conservation géométrique du nombre de tangentes horizontales)}.
On peut donc parler de trois types de courbe pour les polynômes de degré 3 contre un seul pour les fonctions affines, et aussi un seul pour les trinômes du 2\ieme{} degré.
Alors que se passe-t-il pour les polynômes de degré 4 et plus généralement pour ceux de degré $n \geq 5$ ?