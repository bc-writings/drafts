Le procédé de construction que nous venons de prouver dans les sections \ref{proof-via-C} et \ref{proof-rambo-style} se \emph{\og conserve \fg} par translations et dilatations verticales et horizontales.
Il se trouve que ce sont ces transformations qui à partir du cercle trigonométrique permettent d'avoir une ellipse d'équation paramétrique $(x(t) , y(t)) = (x_0 + a \cos t , y_0 + b \sin t)$ .
Nous pouvons donc munir toute ellipse d'équation paramétrique $(x(t) , y(t)) = (x_0 + a \cos t , y_0 + b \sin t)$ d'une structure de groupe isomorphe à celle de $(\RR / 2 \pi \ZZ ; +)$ , et ceci avec un procédé géométrique simple pour \emph{\og additionner \fg} sur l'ellipse. Que c'est joli !
