Daniel Bernoulli
\footnote{
	Il était le fils de Jean Bernoulli (1667-1748) et le neveu de Jacques Bernoulli (1654-1705) qui étaient tous les deux des mathématiciens et physiciens suisses.
}
(1700 -- 1782) était un médecin, physicien et mathématicien suisse. Il présente en 1760 à l'Académie des sciences de Paris un mémoire intitulé \emph{\og Essai d'une nouvelle analyse de la mortalité causée par la petite vérole
\footnote{
	La petite vérole était le nom que l'on donnait à la variole.
}
et des avantages de l'inoculation pour la prévenir \fg}
\footnote{
	Pour les aspects historiques, se reporter au livre \emph{\og Histoires de mathématiques et de populations \fg} de Nicolas Bacaër aux éditions CASSINI.
}.
Nous allons présenter son raisonnement.


% ---------- %


\medskip


Commençons par proposer une modélisation de la propagation de la variole. Pour cela nous aurons besoin des notations suivantes où l'unité de temps sera l'année.

\begin{itemize}[label=\small\textbullet]
	\item $q_I$ désigne la probabilité d'être infecté par la variole dans l'année.

	\item $q_V$ désigne la probabilité de mourrir lorsque l'on attrape la variole pour la 1\iere fois. Une personne ne mourant pas dans l'année sera considérée comme immunisée contre la variole.

	\item $m(t)$ désigne, à la fin de l'année $t$, le taux de mortalité \emph{\og normale \fg} sans tenir compte d'une épidémie de variole \emph{(l'année $0$ est celle du début de l'étude réelle)}. 
\end{itemize}

Les notations précédentes sous-entendent que les quantités nommées sont constantes. C'est une hypothèse de modélisation ! Continuons avec les fonctions suivantes où $t$ désigne une année.

\begin{itemize}[label=\small\textbullet]
	\item $S(t)$ est le nombre de personnes saines n'ayant pas la variole à la fin de l'année $t$.

	\item $I(t)$ est le nombre de personnes qui se sont immunisées contre la variole au cours de l'année $t$. Ces personnes sont celles qui ont attrapé la variole pendant l'année $t$ sans   mourrir.

	\item $P(t) = S(t) + I(t)$ est le nombre personnes toujours vivantes à la fin de l'année $t$. 
\end{itemize}


% ---------- %


\medskip


Faisons le bilan entre l'année $t$ et l'année $(t + 1)$.

\begin{itemize}[label=\small\textbullet]
	\item $S(t + 1) = S(t) - q_I S(t) - m(t) S(t)$
	\begin{itemize}[label=$\rightarrow$]	
		\item $q_I S(t)$ est le nombre de personnes mortes de la variole.
		
		\item $m(t) S(t)$ est le nombre des personnes saines mortes de cause \emph{\og normale \fg}.
	\end{itemize}


	\item $I(t + 1) = I(t) + (1 - q_V) q_I S(t) - m(t) I(t)$
	\begin{itemize}[label=$\rightarrow$]	
		\item Parmi les $q_I S(t)$ nouvelles personnes infectées, il y en a juste $(1 - q_V) q_I S(t)$ qui ne meurent pas de la variole.
		
		\item $m(t) I(t)$ est le nombre d'infectés morts d'autre chose que de la variole.
	\end{itemize}


	\item Donc $P(t + 1) = P(t) - q_V q_I S(t) - m(t) P(t)$
\end{itemize}


Notant $\deltaval{f}{a}{b} = \frac{f(a) - f(b)}{a - b}$ le taux de variation de la fonction $f$ entre $a$ et $b$ distincts, nous avons :

\begin{itemize}[label=\small\textbullet]
	\item $\deltaval{S}{t}{t+1} = - q_I S(t) - m(t) S(t)$

	\item $\deltaval{V}{t}{t+1} = - q_V q_I S(t) - m(t) P(t)$
\end{itemize}


% ---------- %


\medskip


Il n'est pas aisé d'étudier des système d'équations sur des suites comme celui ci-dessus. Nous allons donc émettre de nouvelles hypothèses de modélisation pour aboutir à un modèle utilisant des équations différentielles. L'idée simple, mais peut-être un peu trop, est de négliger les variabilités excessives pendant une année, autrement dit on suppose $S$ et $P$ définies sur $\RRp$ et affines entre $t$ et $t+1$. Sous ces hypothèses fortes, nous avons alors pour $\delta t \in \intervalC{0}{1}$ :

\begin{itemize}[label=\small\textbullet]
	\item $\deltaval{S}{t}{t + \delta t} = \deltaval{S}{t}{t+1} = - q_I S(t) - m(t) S(t)$

	\item $\deltaval{V}{t}{t + \delta t} = \deltaval{V}{t}{t+1} = - q_V q_I S(t) - m(t) P(t)$
\end{itemize}

Il ne reste plus qu'à sauter le cap en supposant carrément $S$ et $P$ dérivables en chaque valeur $t$. Dès lors par passage à la limite via $\delta t \rightarrow 0^+$ sur $\intervalC{t}{t+1}$  et $\delta t \rightarrow 0^-$ sur $\intervalC{t-1}{t}$ pour $t \in \NNs$ :

\begin{itemize}[label=\small\textbullet]
	\item $S\,'(t) = - q_I S(t) - m(t) S(t)$

	\item $P\,'(t) = - q_V q_I S(t) - m(t) P(t)$
\end{itemize}

Quid des valeurs sur $\RRp \backslash \, \NN$ ? On peut supposer que l'on étend le système précédent de deux équations aux valeurs de $\RRp$ tout entier
\footnote{
	Peut-être que le lecteur aura noté une très, très grosse arnaque ici. Nous en discuterons dans la section suivante.
}
et non seulement à celles de $\NN$ . Encore une hypothèse de modélisation... Notre sac d'hypothèses s'alourdit !

% ---------- %


\medskip


En résumé, nous devons trouver deux fonctions $S$ et $P$ dérivables sur $\RRp$ vérifiant les deux équations précédentes.
A priori, ce n'est pas simple comme problème : ce qui crée une grosse difficulté c'est la présence de $m(t)$ qui n'est pas constante. Étant bloqué, essayons d'obtenir des informations sur $R(t) = \frac{S(t)}{P(t)}$ le taux des personnes saines  
\footnote{
	Une astuce plus technique consiste à partir de $\dfrac{- S\,'(t) - q_I S(t)}{S(t)} = m(t) = \dfrac{- P\,'(t) - q_V q_I S(t)}{P(t)}$ afin d'éliminer la fonction gênante $m(t)$.
	C'est un peu moins immédiat...
}.
Au passage, nous avons besoin de supposer, de façon non abusive, que $S$ et $P$ ne s'annulent jamais.

\vspace{-1em}

\begin{flalign*}
	R\,'(t) &= \left( \frac{S(t)}{P(t)} \right)^\prime
		  & \\
	      &= \frac{1}{P^2(t)} \left( \, S\,'(t) P(t) - S(t)P\,'(t) \, \right) 
	      & \\
	      &= \frac{1}{P^2(t)} \left( \, [- q_I S(t) - m(t) S(t)] P(t) \,-\, S(t)[- q_V q_I S(t) - m(t) P(t)] \, \right) 
	      & \\
	      &= \frac{1}{P^2(t)} \left( \, - q_I S(t) P(t) - m(t) S(t) P(t) \,+\, q_V q_I S^2(t) + m(t) P(t) S(t) \, \right) 
	      & \\
	      &= \frac{1}{P^2(t)} \left( \, - q_I S(t) P(t) + q_V q_I S^2(t) \, \right) 
	      & \\
	      &=  - q_I \frac{S(t)}{P(t)} + q_V q_I \frac{S^2(t)}{P^2(t)}
	      & \\
	      &=  - q_I R(t) + q_V q_I R^2(t)
	      & \\
\end{flalign*}

\vspace{-1.5em}

Ce qui est beau, c'est que l'on tombe sur une équation différentielle peu agressive a priori : $R\,'(t) = - q_I R(t) + q_V q_I R^2(t)$ . Quiconque connaissant les formules générales de dérivation trouvent qu'il suffit de raisonner comme suit en se souvenant que $S$ et $P$ ne s'annulent jamais et donc $R$ non plus.

\vspace{-1em}

\begin{flalign*}
	R\,'(t) = - q_I R(t) + q_V q_I R^2(t)
		& \Longleftrightarrow  - \frac{R\,'(t)}{R^2(t)} = \frac{q_I}{R(t)} - q_V q_I
		& \\
		& \Longleftrightarrow  f\,'(t) = q_I f(t) - q_V q_I  
				\,\, \text{où on a posé $f(t) = \frac{1}{R(t)}$ .}
		& \\
\end{flalign*}

\vspace{-1em}

Nous aboutissons à une simple équation différentielle linéaire du 1\ier ordre et il est connu que nécessairement $f(t) = q_V + k \ee^{q_I t}$ où $k \in \RR$ est une constante dépendant des conditions réelles
\footnote{
	Notre équation différentielle est juste sur $\RRp$ mais cela n'est pas gênant ici. 
}.
Nous avons finalement :$\frac{S(t)}{P(t)} = \frac{1}{q_V + k \ee^{q_I t}}$ . Tout ceci a été obtenu via de bien jolis calculs...